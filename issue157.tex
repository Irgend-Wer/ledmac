\documentclass[a4paper, oneside, notitlepage, 12pt]{article}
\usepackage{fontspec}
\usepackage{libertineotf}
\usepackage{xunicode}
\usepackage{polyglossia}
\setmainlanguage{french}
%
\usepackage{eledmac}
\usepackage[shiftedpstarts]{eledpar}
\maxchunks{300}
\renewcommand*{\goalfraction}{0.922}
\renewcommand{\hangingsymbol}{[\,}
\title{}
\date{}
\author{}
%
\begin{document}
%
\begin{pages}
\begin{Rightside}
	\beginnumbering
	\setcounter{stanzaindentsrepetition}{2}
	\setstanzaindents{14,0,0}
	\pstart\section{Un titre}\pend
	\stanza
	\edtext{Ceci}{\Efootnote{Une note à la ligne 1}} est les premier vers, que je fais assez long pour qu'il y ait un rejet: lorem ipsum et caetera..&
	\edtext{Et voilà}{\Efootnote{Une note à la ligne 2}} le vers suivant&
	\edtext{Encore un vers}{\Efootnote{Une note à la ligne 3}}, que je fais assez long pour qu'il y ait un rejet: lorem ipsum et caetera...&
	\edtext{Et}{\Efootnote{Une note à la ligne 4}} un autre&
	Le vers cinq, pour que l'on voit la numérotation\&
	\endnumbering
\end{Rightside}
\begin{Leftside}
	\beginnumbering
	\setcounter{stanzaindentsrepetition}{2}
	\setstanzaindents{14,0,0}
	\pstart\section{Un titre}\pend
	\stanza
	\edtext{Ceci}{\Efootnote{Une note à la ligne 1}} est les premier vers, que je fais assez long pour qu'il y ait un rejet: lorem ipsum et caetera..&
	\edtext{Et voilà}{\Efootnote{Une note à la ligne 2}} le vers suivant&
	\edtext{Encore un vers}{\Efootnote{Une note à la ligne 3}}, que je fais assez long pour qu'il y ait un rejet: lorem ipsum et caetera...&
	\edtext{Et}{\Efootnote{Une note à la ligne 4}} un autre&
	Le vers cinq, pour que l'on voit la numérotation\&
	\endnumbering
\end{Leftside}
\Pages
\end{pages}
\end{document}
