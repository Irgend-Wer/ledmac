\documentclass[a4paper,oneside]{book}

\usepackage[no-math]{fontspec}
\usepackage{xltxtra,xunicode,amsmath}
\usepackage{ragged2e}
\usepackage{polyglossia} 
\usepackage{bidi}
\setdefaultlanguage{english}
\setotherlanguages{french}


\usepackage{eledmac}
\usepackage{eledpar}
\setlength{\columnrulewidth}{0.4pt}
\numberonlyfirstinline[A,B,C,D] 
\symlinenum{$\parallel$}

\footparagraph{A} 
\footparagraph{B}
\footparagraph{C}
\footparagraph{D}
\Xnotenumfont[A,B,C,D]{\bfseries}

\renewcommand{\Rlineflag}{*}
\let\oldDfootfmt\Dfootfmt
\renewcommand{\Dfootfmt}[3]{%
\let\printlines\printlinesR
\oldDfootfmt{#1}{#2}{#3}}

\newcommand{\up}[1]{\textsuperscript{#1}}

\begin{document}

\subsection{Parallel texts with RTL paragraphs}

\newpage
\begin{pairs}
\begin{Leftside}\sloppy 
    \beginnumbering
        \pstart
\subparagraph{1}\sloppy
London is the capital of Great Britain, its political, economic, and commercial centre. It is one of the largest cities in the world and the largest city in Europe. Its population is about 8 million.
\pend

        \pstart
\subparagraph{2}\sloppy
London is divided into several parts: the City, Westminster, the West End, and the East End.
        \pend

        \pstart
\subparagraph{3}\sloppy
The heart of London is the City, its \edtext{financial}{\Bfootnote{economical}} and business centre. Numerous banks, offices, and firms are situated there, including the Bank of England, the Stock Exchange, and the Old Bailey. Few people live here, but over a million people come to the City to work. There are some famous ancient buildings within the City. Perhaps the most striking of them is the St. Paul's Cathedral, the greatest of English churches. It was built in the 17th century by Sir Christopher Wren. The Tower of London was founded by Julius Caesar and in 1066 rebuilt by William the Conqueror. It was used as a fortress, a royal palace, and a prison. Now it is a museum.
        \pend   

        \pstart 
Westminster is the governmental part of London. Nearly all English kings and queens have been crowned in Westminster Abbey. Many outstanding statesmen, scientists, writers, poets, and painters are buried here: Newton, Darwin, Chaucer, Dickens, Tennyson, Kipling, etc.
        \pend

        \pausenumbering
    \end{Leftside} 

\begin{Rightside}\sloppy 
    \beginnumbering

        \pstart 
        Paris, la capitale de la France, est une de plus grandes et plus belles villes du monde. Sa population est 2,2 millions d’habitants et sa superficie est 105 km carrés. C’est le centre politique, \edtext{administratif}{\Dfootnote{bureauctatique}}, culturel et scientifique de la France. Cette ville se trouve dans la région d’Ile-de-France.
        \pend

        \pstart 
Paris a un climat de type océanique dégradé: les étés sont relativement frais (18 °C en moyenne), des hivers doux (6 °C en moyenne) avec des pluies fréquentes en toute saison et un temps changeant.
        \pend   

        \pstart 
L’île de la Cité est le coeur de la ville, la plus vieille partie de Paris qui se trouve sur la Seine au centre de la ville.
Les monuments les plus célèbres de Paris datent d’époques variées. Ils se trouvent souvent dans le centre et sur les rives de la Seine. On trouve sur l’île de la Cité des monuments anciens comme la cathédrale Notre-Dame, de style gothique, bâtie du XII\up{e} au XIII\up{e} siècle.
        \pend

        \pstart 
        \begin{RTL}
A Paris il y a beaucoup de monuments de style classique. La Sorbonne au cœur du quartier Latin, a été construite au début du XVII\up{e} siècle. Le Louvre, l’ancien résidence royale, est actuellement un grand musée. L’Hôtel des Invalides abrite depuis le 15 décembre 1840 les cendres de Napoléon Ier et son tombeau. Le Panthéon, édifié à la fin du XVIII\up{e} siècle à proximité de la Sorbonne, est devenu sous la Révolution un temple civil où des Français illustres sont enterrés.
        \end{RTL}
        \pend

        \endnumbering
\end{Rightside} 

\Columns

\end{pairs}

\resumenumbering
\pstart
In 1066 William from Normandy came with his people to England. They were French. William thought that he had right to become King of England. After the battle at Hastings he got the name of William the Conqueror and became King of England. The King \edtext{lived}{\Bfootnote{lived and died}} in London. A lot of his people lived in London too. But William was afraid of the English, of Londoners and he \edtext{built}{\Bfootnote{constructed}} the White Tower to live in it. It was the beginning of the Tower of London and now it is one of the most important and beautiful buildings in it 
\pend
\endnumbering

\end{document}
