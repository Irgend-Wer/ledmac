\documentclass{book}
\usepackage[xetex,a4paper,showframe]{geometry}
\usepackage{fontspec}
\usepackage{xunicode}
\usepackage{polyglossia}

\usepackage{eledmac}
\usepackage{eledpar}  
\usepackage{bidi}  
\setdefaultlanguage{russian}
\setotherlanguage{hebrew}    
\setmainfont[Ligatures=TeX]{Linux Libertine O}
\newfontfamily\hebrewfont[Script=Hebrew]{Linux Libertine O}
\newfontfamily\russianfont[Script=Cyrillic]{Linux Libertine O}    

\begin{document}
\numberlinefalse
\numberpstarttrue 
\sidepstartnumtrue 
\beforeeledchapter

\begin{pairs}

\begin{Rightside} %
\begin{hebrew}%
\begin{RTL}%

\beginnumbering%
\pstart
\eledchapter{המאמר השני}
\pend
\pstart
בפנות התוריות, ר״ל שהם יסודות ועמודים אשר בית אלהים נכון עליהם, ובמציאותם יציר מציאות התורה מסדרת ממנו     יתברך, ואלו יציר העדר אחת מהם תפל התורה בכללה חלילה.
\pend    
\endnumbering%

\end{RTL}%
\end{hebrew}%
\end{Rightside}%

\begin{Leftside} %
\begin{hebrew}
\begin{RTL}

\beginnumbering%
\pstart
\eledchapter{המאמר השני}
\pend
\pstart
בפנות התוריות, ר״ל שהם יסודות ועמודים אשר בית אלהים נכון עליהם, ובמציאותם יציר מציאות התורה מסדרת ממנו     יתברך, ואלו יציר העדר אחת מהם תפל התורה בכללה חלילה.
\pend    
\endnumbering%

\end{RTL}%
\end{hebrew}%
\end{Leftside}%

\Columns
\end{pairs}

\end{document}