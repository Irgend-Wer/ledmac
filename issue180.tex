\documentclass{book}
\usepackage[showframe, xetex,a4paper]{geometry}
\usepackage{ragged2e}% need to make eledmac works
\usepackage{fontspec}
\usepackage{xunicode}
\usepackage{polyglossia}




\usepackage{eledmac}
\usepackage{eledpar}  
 \usepackage{bidi} 
\setdefaultlanguage{russian}
\setotherlanguage{hebrew}    
\setmainfont[Ligatures=TeX]{Linux Libertine O}
\newfontfamily\hebrewfont[Script=Hebrew]{Linux Libertine O}
\newfontfamily\russianfont[Script=Cyrillic]{Linux Libertine O}    

\begin{document}
\numberlinefalse
\numberpstarttrue 
\sidepstartnumtrue 
\chapter{sncf}
\beforeeledchapter


\begin{pairs}

\begin{Rightside} %
\begin{RTL}%
\begin{hebrew}%
\beginnumbering%
\pstart
\eledchapter{המאמר השני}
\pend
\pstart
\eledsection{המאמר השני}
\pend
\pstart
בפנות התוריות, ר״ל שהם יסודות ועמודים אשר בית אלהים נכון עליהם, ובמציאותם יציר מציאות התורה מסדרת ממנו     יתברך, ואלו יציר העדר אחת מהם תפל התורה בכללה חלילה.
\pend    
\endnumbering%
\end{hebrew}%
\end{RTL}%
\end{Rightside}%



\begin{russian}%
\begin{Leftside} %
\beginnumbering%
\pstart
\eledchapter{Трактат Второй sncf sncf sncf}   
\pend 
\pstart
\eledsection{Трактат Второй sncf sncf sncf}   
\pend 
\pstart
О краеугольных [принципах] Торы, имеется ввиду, которые [есть] основы и столпы на которых дом Б-жий опирается/нахон, и с существованием их может быть представлено существование Торы упорядоченной от Него, благословенного, и если бы было представлено отсутствие одного из них — упадет Тора в общем, [Б-же] упаси.
\pend
\endnumbering%
\end{Leftside}%
\end{russian}%



\Columns
\end{pairs}

\beginnumbering%
\pstart
\eledchapter{Трактат Второй sncf sncf sncf}   
\pend 
\pstart
\eledsection{Трактат Второй sncf sncf sncf}   
\pend 
\pstart
О краеугольных [принципах] Торы, имеется ввиду, которые [есть] основы и столпы на которых дом Б-жий опирается/нахон, и с существованием их может быть представлено существование Торы упорядоченной от Него, благословенного, и если бы было представлено отсутствие одного из них — упадет Тора в общем, [Б-же] упаси.
\pend
\endnumbering%
\end{document}