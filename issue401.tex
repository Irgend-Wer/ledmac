
\documentclass{article}

\usepackage{reledmac}
   


\usepackage{fontspec}

\usepackage{polyglossia}
    \setmainlanguage{english}
    \setotherlanguage{sanskrit}
    \setotherlanguage{tibetan}
    \newfontfamily\devanagarifont{Nakula}
    \newfontfamily\tibetanfont{Jomolhari}
    
\usepackage{geometry}

    
\newcommand{\choiceline}[2]{\linenum{|#1|#2||#1|#2}}
    \firstlinenum{1}
    \linenumincrement{1}
    \linenumberstyle{alph}
    \sublinenumberstyle{alph}
  

\begin{document}


\setstanzaindents{0,0,0}

\beginnumbering
    \autopar

\tibetanfont
\begin{tibetan}

\noindent དེ་ནས་ཡང་བཅོམ་ལྡན་འདས་\edtext{ཀྱིས་}{\Bfootnote{A; ཀྱི་ B}}ཟླ་བ་གཞོན་ནུར་གྱུར་པ་ལ་བཀའ་སྩལ་པ། དེ་ལྟ་བས་ན་གཞོན་ནུ་བྱང་ཆུབ་སེམས་དཔའ་སེམས་དཔའ་ཆེན་པོ་ཆོས་ཐམས་ཅད་ལ་མངོན་པར་ཤེས་པའི་བྱི་དོར་བྱ་བ་སྦྱང་བར་འདོད་པས་ཏིང་ངེ་འཛིན་འདི་མཉན་པར་བྱ།

\begin{stanza}
།ཆོས་ཐམས་ཅད་ལ་མངོན་ཤེས་པ། །བརྩོད་པ་\edtext{མེད་པར་}{\choiceline{1}{2}\Bfootnote{A; མེད་པ་ B}}བསྟན་པ་སྟེ།&
།བསྩོད་པ་ལ་ནི་གང་གནས་པའི། །འཛིན་པ་དེ་ནི་མི་ཐར་རོ།\&
\end{stanza}

\begin{stanza}
།མངོན་ཤེས་\edtext{དེ་ཡི་}{\choiceline{2}{1}\Bfootnote{B; དེ་ཡིས་ A}}ཤེས་རབ་ཏེ། །སངས་རྒྱས་ཡེ་ཤེས་བསམ་མི་ཁྱབ།&
།འཛིན་པ་ལ་ནི་གང་གནས་པ། །དེ་ཡི་ཡེ་ཤེས་ཡོད་མ་ཡིན།\&
\end{stanza}

\end{tibetan}

\endnumbering



\end{document}
