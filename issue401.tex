
\documentclass{article}

\usepackage{reledmac}
   
\usepackage{reledpar}

\usepackage{fontspec}

\usepackage{polyglossia}
    \setmainlanguage{english}
    \setotherlanguage{sanskrit}
    \setotherlanguage{tibetan}
    \newfontfamily\devanagarifont{Nakula}
    \newfontfamily\tibetanfont{Jomolhari}
    
\usepackage{geometry}
    
\setlength{\Lcolwidth}{0.48\textwidth}
\setlength{\Rcolwidth}{0.48\textwidth} 
    
\newcommand{\choiceline}[2]{\linenum{|#1|#2||#1|#2}}

    \sublinenumberstyle*{latinalph}
    \let\fullstop\relax    
  \makeatletter	
  \let\@latinalph\@alph
	\Xnotenumfont{\ifledRcol@\tibetanfont\else\devanagarifont\fi}
  \makeatother
\begin{document}

\begin{pairs}

\begin{Leftside} 

\setstanzaindents{0,0,0}
   
\beginnumbering
    \autopar

\devanagarifont
\begin{sanskrit}
\noindent तत्र \edtext{खलु}{\Afootnote{A; एव B}} भगवान् पुनरपि चन्द्रप्रभं बोधिसत्वमामन्त्रायते स्म॥ तस्मात् तर्हि कुमार बोधिसत्वेन महासत्वेन सर्वधर्मानां महाभिज्ञापरिकर्म परिशोधयितुकेमनायं समाधिः श्रोतव्यः॥

\begin{stanza}
महाभिज्ञापरिकर्म \edtext{अविवादेन}{\choiceline{1}{2}\Afootnote{B; अविवदेन A}} देशितः।&
विवादे यस्तु चरति नोद्गृह्णन् स विमुच्यते॥\&
\end{stanza}

\begin{stanza}
अभिज्ञा तस्य सा प्रज्ञा बोद्धं ज्ञानम् अचिन्तियम्।&
उद्ग्रहे \edtext{य}{\choiceline{2}{3}\Afootnote{B; यो A}} स्थितो भोन्ति ज्ञानं तस्य न विद्यते॥\&
\end{stanza}

\end{sanskrit}

\endnumbering

\end{Leftside}


\begin{Rightside} 

\setstanzaindents{0,0,0}

\beginnumbering
    \autopar

\tibetanfont
\begin{tibetan}

\noindent དེ་ནས་ཡང་བཅོམ་ལྡན་འདས་\edtext{ཀྱིས་}{\Bfootnote{A; ཀྱི་ B}}ཟླ་བ་གཞོན་ནུར་གྱུར་པ་ལ་བཀའ་སྩལ་པ། དེ་ལྟ་བས་ན་གཞོན་ནུ་བྱང་ཆུབ་སེམས་དཔའ་སེམས་དཔའ་ཆེན་པོ་ཆོས་ཐམས་ཅད་ལ་མངོན་པར་ཤེས་པའི་བྱི་དོར་བྱ་བ་སྦྱང་བར་འདོད་པས་ཏིང་ངེ་འཛིན་འདི་མཉན་པར་བྱ།

\begin{stanza}
།ཆོས་ཐམས་ཅད་ལ་མངོན་ཤེས་པ། །བརྩོད་པ་\edtext{མེད་པར་}{\choiceline{1}{2}\Bfootnote{A; མེད་པ་ B}}བསྟན་པ་སྟེ།&
།བསྩོད་པ་ལ་ནི་གང་གནས་པའི། །འཛིན་པ་དེ་ནི་མི་ཐར་རོ།\&
\end{stanza}

\begin{stanza}
།མངོན་ཤེས་\edtext{དེ་ཡི་}{\choiceline{2}{1}\Bfootnote{B; དེ་ཡིས་ A}}ཤེས་རབ་ཏེ། །སངས་རྒྱས་ཡེ་ཤེས་བསམ་མི་ཁྱབ།&
།འཛིན་པ་ལ་ནི་གང་གནས་པ། །དེ་ཡི་ཡེ་ཤེས་ཡོད་མ་ཡིན།\&
\end{stanza}

\end{tibetan}

\endnumbering

\end{Rightside}

\end{pairs}

\Columns

\end{document}
