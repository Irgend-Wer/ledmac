\documentclass{article}
\usepackage{fontspec}
\usepackage{polyglossia}
\usepackage{setspace}
\setdefaultlanguage{sanskrit}
\newfontfamily\sanskritfont{Times New Roman}
\usepackage{eledmac}
\lineation{page}
\firstlinenum{1}
\linenumincrement{1}
\sidenotemargin{left}
\linenummargin{right}
\usepackage{microtype}
\title{dvitīyaḥ svārthānumānaparicchedaḥ}
\author{}
\date{}
\begin{document}
\maketitle
\section*{Abbreviation}
DP: Malvania, Dalsukhbhal, ed. \textit{Dharmottarapradīpa}. Patna: Kashiprasad Jayaswal Research Institute, 1971.\\
NB\d{T} : Stcherbatsky, ed. \textit{Nyāyabindu of Dharmakīrti with Dharmottara's Ṭikā}. Leningrad, 1918.
\vskip 2\baselineskip
\beginnumbering

\section{Definition and results}
\pstart
\ledsidenote{DP 87, NB\d{T} 17}evaṃ pratyakṣaṃ vyākhyāyānumānaṃ vyākhyātukāma āha. 

 \begin{quote}
 anumānaṃ dvidhā. 1 
 \end{quote}

anumānaṃ dvidhā, dviprakāram. athānumānalakṣaṇe vaktavye kim akasmāt prakārabhedaḥ kathyate. ucyate. parārthānumānaṃ śabdātmakam, svārthānumānaṃ tu jñānātmakam. tayor atyantabhedān naikam lakṣaṇam asti. tatas tayoḥ pratiniyataṃ lakṣaṇam ākhyātuṃ prakārabhedaḥ kathyate. prakārabhedo hi vyaktibhedaḥ. vyaktibhede ca kathite prativyaktiniyataṃ lakṣaṇaṃ śakyate vaktum. nānyathā. tato lakṣaṇanirdeśāṅgam eva prakārabhedakathanam. aśakyatāṃ ca prakārabhedakathanam antareṇa lakṣaṇanirdeśasya jñātvā prāk prakārabhedaḥ kathyata iti.
\pend
\pstart
\ledsidenote{DP 88}kiṃ punas tad dvaividhyam ity āha. 

\begin{quote}
svārthaṃ parārthaṃ ca. 2
\end{quote}

svasmāy idaṃ svārtham. yena svayaṃ pratipadyate tat svārtham. parasmāy idaṃ parārtham. yena paraṃ pratipādayati tat parārtham.
\pend
\pstart
tatra tayoḥ svārthaparārthānumānayor madhye svārthaṃ jñānaṃ kiṃviśiṣṭam ity āha. 

\begin{quote}
\ledsidenote{DP 89, NB\d{T} 17}tatra trirūpāl liṅgād yad anumeye jñānaṃ tad anumānam. 3
\end{quote}

trirūpād iti. trīṇi rūpāṇi yasya vakṣyamāṇalakṣaṇāni tat trirūpam. liṅgyate gamyate 'nenārtha iti liṅgam. tasmāt trirūpāl liṅgād yaj jātaṃ jñānam ity etad dhetudvāreṇa viśeṣaṇam. tat trirūpāc ca liṅgāt trirūpaliṅgālambhanam apy utpadyata iti viśinaṣṭi. anumeya ity etac ca viṣayadvāreṇa viśeṣaṇam.
\pend
\pstart
trirūpāl liṅgād yad utpannam anumeyālambhanaṃ jñānaṃ tat svārtham anumānam iti.
\pend
\pstart
 lakṣaṇavipratipattiṃ nirākṛtya phalavipratipattiṃ nirākartum āha. 

\begin{quote}
pramāṇaphalavyavasthātrāpi pratyakṣavat. 4
\end{quote} 

pramāṇasya yat phalaṃ tasya yā vyavasthātrānumāne 'pi pratyakṣavat pratyakṣa iva veditavyā. 
 \pend
 \pstart
\ledsidenote{DP 91}yathā hi nīlasarūpaṃ pratyakṣam anubhūyamānaṃ nīlabodharūpam avasthāpyate. tena nīlasārūpyaṃ vyavasthāpanahetuḥ pramāṇam. nīlabodharūpaṃ tu vyavasthāpyamānaṃ pramāṇaphalam. tadvad anumānaṃ nīlākāram utpadyamānaṃ nīlabodharūpam avasthāpyate. tena nīlasārūpyam asya pramāṇam. nīlavikalpanarūpaṃ tv asya pramāṇaphalam. sārūpyavaṣād dhi tan nīlapratītirūpaṃ sidhyati. nānyatheti.
\pend
\vskip 2\baselineskip
\section{Invariable concomitance or the three aspects of a valid logical mark}
\pstart
evam hi saṃkhyālakṣaṇaphalavipratipattayaḥ, pratyakṣaparicchede tu gocaravipratipattir nirākṛtā. lakṣaṇanirdeśaprasaṅgena tu trirūpaṃ liṅgaṃ prastutam. tad eva vyākhyātum āha. 

 \begin{quote}
trairūpyaṃ punar liṅgasyānumeye sattvam eva, sapakṣa eva sattvam, asapakṣe cāsattvam eva niścitam.\footnoteB{This sentences is divided as three seperate parts in NB\d{T}} 5
\end{quote}

trairūpyam ityādi. liṅgasya yat trairūpyaṃ yāni trīṇi rūpāṇi tad idam ucyata iti śeṣaḥ. kiṃ punas tat trairūpyam ity āha. anumeyaṃ vakṣyamāṇalakṣaṇam. tasmiṃ liṅgasya sattvam eva niścitam ekaṃ rūpam. yady api cātra niścitagrahaṇam na kṛtaṃ tathāpy ante kṛtaṃ, prakrāntayor dvayor api \ledsidenote{NB\d{T} 19}rūpayor apekṣaṇīyam. yato na yogyatayā liṅgaṃ parokṣajñānasya nimittam. yathā \ledsidenote{DP 92} bījam aṅkursya. adṛṣṭād dhūmād agner apratipatteḥ. nāpi svaviṣayajñānāpekṣaṃ parokṣārthaprakāśanam. yathā pradīpo ghaṭādeḥ. dṛṣṭād apy aniścitasambandhād apratipatteḥ. tasmāt parokṣārthanāntarīyakatayā niścayanam eva liṅgasya parokṣārthapratipādanavyāpāraḥ. nāparaḥ kaścit. ato 'nvayavyatirekapakṣadharmatvaniścayo liṅgavyāpārātmakatvād avaśyakartavya iti sarveṣu rūpeṣu niścitagrahaṇam apekṣaṇīyam.
\pend
\pstart
 tatra sattvavacanenāsiddhaṃ cākṣuṣatvādi nirastam. evakāreṇa pakṣaikadeśāsiddhaḥ nirasto hetuḥ. yathā cetanās taravaḥ svāpād iti. pakṣīkṛteṣu taruṣu pattrasaṃkocalakṣaṇaḥ svāpa ekadeśe na siddhaḥ. na hi sarve vṛkṣā rātrau pattrasaṃkocabhājaḥ, kiṃ tu kecid eva. sattvavacanasya \ledsidenote{DP 93}paścāt kṛtenaivakāreṇāsādhāraṇo dharmo nirastaḥ. yadi hy anumeya eva sattvam iti kuryāc chrāvaṇatvam eva hetuḥ syāt. niścitagrahaṇena saṃdigdhāsiddhaḥ sarvo nirastaḥ.
\pend
\pstart
 sapakṣo vakṣyamāṇalakṣaṇaḥ. tasminn eva sattvaṃ niścitaṃ dvitīyaṃ rūpam. ihāpi \ledsidenote{DP 94}sattvagrahaṇena viruddho nirastaḥ. sa hi nāsti sapakṣe. evakāreṇa sādhāraṇānaikāntikaḥ. sa hi na sapakṣa eva vartate, kiṃ tūbhayatrāpi. sattvagrahaṇāt pūrvāvadhāraṇavacanena sapakṣāvyāpisattākasyāpi prayatnānantarīyakasya hetutvaṃ kathitam. paścād avadhāraṇe tv ayam arthaḥ syāt. sapakṣe sattvam eva yasya sa hetur iti prayatnānantarīyakatvaṃ na hetuḥ syāt. niścitavacanena saṃdigdhānvayo 'naikāntiko nirastaḥ. yathā sarvajñaḥ kaścid vaktṛtvāt. vaktṛtvaṃ hi sapakṣe sarvajñe saṃdigdham. 
\pend
\pstart
 asapakṣo vakṣyamāṇalakṣaṇaḥ. tasminn asattvam eva niścitam tṛtīyaṃ rūpam. tatrāsattvagrahaṇena viruddhasya nirāsaḥ. viruddho hi vipakṣe 'sti. evakāreṇa sādhāraṇasya vipakṣaikadeśavṛtter nirāsaḥ. prayatnānantarīyakatve sādhye hy anityatvaṃ vipakṣaikadeśe vidyudādāv asty \ledsidenote{NB\d{T} 20}ākāśādau nāsti. tato niyamenāsya nirāsaḥ. asattvavacanāt pūrvasminn avadhāraṇe 'yam arthaḥ syāt. vipakṣa eva yo nāsti sa hetuḥ. tathā ca prayatnānantarīyakatvaṃ sapakṣe 'pi \ledsidenote{DP 95}nāsti. tato na hetuḥ syāt. tataḥ pūrvaṃ na kṛtam. niścitagrahaṇena saṃdigdhavipakṣavyāvṛttiko 'naikāntiko nirastaḥ. 
\pend
\pstart 
 nanu ca sapakṣa eva sattvam ity ukte vipakṣe 'sattvam eveti gamyata eva. tat kim arthaṃ punar ubhayor upādānaṃ kṛtam. tad ucyate. anvayo vyatireko vā niyamavān eva prayoktavyo nānyatheti darśayituṃ dvayor apy upādānaṃ kṛtam. aniyame hi dvayor api prayoge 'yam arthaḥ syāt. sapakṣe yo 'sti vipakṣe ca yo nāsti sa hetur iti. tathā ca sati sa śyāmas tatputratvāt dṛśyamānaputravad \ledsidenote{DP 96}iti tatputratvaṃ hetuḥ syāt. tasmān niyamavator evānvayavyatirekayoḥ prayogaḥ kartavyo, yena pratibandho gamyeta sādhanasya sādhyena. niyamavatoś ca prayoge 'vaśyakartavye dvayor eka eva prayoktavyo na dvāv iti niyamavān evānvayo vyatireko vā prayoktavya iti śikṣaṇārthaṃ dvayor upādānam iti.
\pend
\vskip 2\baselineskip
\section{Minor term. Introduction from similar and dissimilar instances}
\pstart 
 trairūpyakathanaprasaṅgenānumeyaḥ sapakṣo vipakṣaś coktaḥ. teṣāṃ lakṣaṇaṃ vaktavyam. tatra ko 'numeya ity āha. 

 \begin{quote}
anumeyo 'tra jijñāsitaviśeṣo dharmī. 6
\end{quote}

\ledsidenote{DP 97}anumeyo 'tretyādi. atra hetulakṣaṇe niścetavye dharmy anumeyaḥ. anyatra tu sādhyapratipattikāle samudāyo 'numeyaḥ. vyāptiniścayakāle tu dharmo 'numeya iti darśayitum atragrahanam. jijñāsito jñātum iṣṭo viśeṣo dharmo yasya dharmiṇaḥ sa tathoktaḥ.

kaḥ sapakṣaḥ.
\begin{quote}
sādhyadharmasāmānyena samāno 'rthaḥ sapakṣaḥ. 7
\end{quote}

samāno 'rthaḥ sapakṣaḥ. samānaḥ sadṛśo yo 'rthaḥ sa sapakṣa ukta upacārāt. samānaśabdena viśeṣyate. samānaḥ pakṣaḥ sapakṣaḥ. samānasya ca saśabda ādeśaḥ. 
 \pend
 \pstart
\ledsidenote{DP 98}syāt etat: \ledsidenote{NB\d{T} 21}kiṃ tat pakṣasapakṣayoḥ sāmānyaṃ yena samānaḥ sapakṣaḥ pakṣenety āha, sādhyadharmasāmānyeneti. sādhyaś cāsāv asiddhatvād dharmaś ca parāśritatvāt sādhyadharmaḥ. na ca viśeṣaḥ sādhyaḥ. api tu sāmānyaṃ. ata iha sāmānyaṃ sādhyam uktam. sādhyadharmaś cāsau sāmānyaṃ ceti sādhyadharmasāmānyena samānaḥ pakṣeṇa sapakṣa ity arthaḥ.
\pend
\pstart
ko 'sapakṣa ity āha: 

\begin{quote}
na sapakṣo 'sapakṣaḥ. 8
\end{quote}

na sapakṣo 'sapakṣaḥ. sapakṣo yo na bhavati so 'sapakṣaḥ.
\pend
\pstart
kaś ca sapakṣo na bhavati.

\begin{quote}
\ledsidenote{DP 99}tato 'nyas tadviruddhas tadabhāvaś ca. 9
\end{quote}

tataḥ sapakṣād anyaḥ, tena ca viruddhaḥ, tasya ca sapakṣasyābhāvaḥ. sapakṣād anyatvaṃ tadviruddhatvaṃ ca na tāvat pratyetuṃ śakyaṃ, yāvat sapakṣasvabhāvābhāvo na vijñātaḥ. tasmād anyatvaviruddhatvapratītisāmarthyāt sapakṣābhāvarūpau pratītāv anyaviruddhau.
\pend
\pstart
\ledsidenote{DP 100}tato 'bhāvaḥ sākṣāt sapakṣābhāvarūpaḥ pratīyate. anyaviruddhau tu sāmarthyād abhāvarūpau pratīyete. tatas trayāṇām apy asapakṣatvam. 
\pend
\vskip 2\baselineskip
\endnumbering
\end{document}