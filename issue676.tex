% !TeX program = xelatex
% !TeX encoding = UTF-8
% !TeX spellcheck = it_IT

\documentclass[b5paper]{book}

\usepackage{fontspec}
\setmainfont{Linux Libertine O}

\usepackage{polyglossia}
\setmainlanguage[babelshorthands=true]{italian}

\usepackage[bottom]{footmisc}

\usepackage[noeledsec,noledgroup,series={A,B,C}]{reledmac}
\labelpstarttrue
\lineation{page}
\linenummargin{inner}
\sidenotemargin{outer}

\begin{document}
Quel ramo del lago di Como, che volge a mezzogiorno,\footnote{blablabla} tra due catene non interrotte di monti, tutto a seni e a golfi, a seconda dello sporgere e del rientrare di quelli, vien, quasi a un tratto, a ristringersi, e a prender corso e figura di fiume,\footnote{blablabla} tra un promontorio a destra, e un’ampia costiera dall’altra parte; e il ponte, che ivi congiunge le due rive, par che renda ancor più sensibile all’occhio questa trasformazione, e segni il punto in cui il lago cessa, e l’Adda rincomincia, per ripigliar poi nome di lago dove le rive, allontanandosi di nuovo,\footnote{blablabla} lascian l’acqua distendersi e rallentarsi in nuovi golfi e in nuovi seni. La costiera,formata dal deposito di tre grossi torrenti, scende appoggiata a due monti contigui, l’uno detto di san Martino, l’altro, con voce lombarda, il Resegone, dai molti suoi cocuzzoli in fila, che in vero lo fanno somigliare a una sega: talché non è chi, al primo vederlo, purché sia di fronte, come per esempio di su le mura di Milano che guardano a settentrione, non lo discerna tosto, a un tal contrassegno, in quella lunga e vasta giogaia,\footnote{blablabla} dagli altri monti di nome più oscuro e di forma più comune. Per un buon pezzo, la costa sale con un pendii lento e continuo; poi si rompe in poggi e in valloncelli, in erte e in ispianate,secondo l’ossatura de’ due monti,\footnote{blablabla} e il lavoro dell’acque. Il lembo estremo, tagliato dalle foci de’ torrenti, è quasi tutto ghiaia e ciottoloni; il resto, campi e vigne, sparse di terre, di ville, di casali; in qualche parte boschi, che si prolungano su per la montagna. 

Quel ramo del lago di Como, che volge a mezzogiorno,\footnote{blablabla} tra due catene non interrotte di monti, tutto a seni e a golfi, a seconda dello sporgere e del rientrare di quelli, vien, quasi a un tratto, a ristringersi, e a prender corso e figura di fiume,\footnote{blablabla} tra un promontorio a destra, e un’ampia costiera dall’altra parte; e il ponte, che ivi congiunge le due rive, par che renda ancor più sensibile all’occhio questa trasformazione, e segni il punto in cui il lago cessa, e l’Adda rincomincia, per ripigliar poi nome di lago dove le rive, allontanandosi di nuovo,\footnote{blablabla} lascian l’acqua distendersi e rallentarsi in nuovi golfi e in nuovi seni. La costiera,formata dal deposito di tre grossi torrenti, scende appoggiata a due monti contigui, l’uno detto di san Martino, l’altro, con voce lombarda, il Resegone, dai molti suoi cocuzzoli in fila, che in vero lo fanno somigliare a una sega: talché non è chi, al primo vederlo, purché sia di fronte, come per esempio di su le mura di Milano che guardano a settentrione, non lo discerna tosto, a un tal contrassegno, in quella lunga e vasta giogaia,\footnote{blablabla} dagli altri monti di nome più oscuro e di forma più comune. Per un buon pezzo, la costa sale con un pendii lento e continuo; poi si rompe in poggi e in valloncelli, in erte e in ispianate,secondo l’ossatura de’ due monti,\footnote{blablabla} e il lavoro dell’acque. Il lembo estremo, tagliato dalle foci de’ torrenti, è quasi tutto ghiaia e ciottoloni; il resto, campi e vigne, sparse di terre, di ville, di casali; in qualche parte boschi, che si prolungano su per la montagna. 

Quel ramo del lago di Como, che volge a mezzogiorno,\footnote{blablabla} tra due catene non interrotte di monti, tutto a seni e a golfi, a seconda dello sporgere e del rientrare di quelli, vien, quasi a un tratto, a ristringersi, e a prender corso e figura di fiume,\footnote{blablabla} tra un promontorio a destra, e un’ampia costiera dall’altra parte; e il ponte, che ivi congiunge le due rive, par che renda ancor più sensibile all’occhio questa trasformazione, e segni il punto in cui il lago cessa, e l’Adda rincomincia, per ripigliar poi nome di lago dove le rive, allontanandosi di nuovo,\footnote{blablabla} lascian l’acqua distendersi e rallentarsi in nuovi golfi e in nuovi seni. La costiera,formata dal deposito di tre grossi torrenti, scende appoggiata a due monti contigui, l’uno detto di san Martino, l’altro, con voce lombarda, il Resegone, dai molti suoi cocuzzoli in fila, che in vero lo fanno somigliare a una sega: talché non è chi, al primo vederlo, purché sia di fronte, come per esempio di su le mura di Milano che guardano a settentrione, non lo discerna tosto, a un tal contrassegno, in quella lunga e vasta giogaia,\footnote{blablabla} dagli altri monti di nome più oscuro e di forma più comune. Per un buon pezzo, la costa sale con un pendii lento e continuo; poi si rompe in poggi e in valloncelli, in erte e in ispianate,secondo l’ossatura de’ due monti,\footnote{blablabla} e il lavoro dell’acque. Il lembo estremo, tagliato dalle foci de’ torrenti, è quasi tutto ghiaia e ciottoloni; il resto, campi e vigne, sparse di terre, di ville, di casali; in qualche parte boschi, che si prolungano su per la montagna. 

\chapter{lkjlkj}
%\raggedbottom %<=======
\beginnumbering
\numberpstarttrue
\pstart
Quel \edtext{ramo}{\Afootnote{blablabla blablablabla òsòlfk òlddkfàÒSLKFÒÀLK ÒLLKÀÒLSKFÀÒLK}} del lago di Como, che volge a mezzogiorno, tra due catene non interrotte di monti, tutto a seni e a golfi, a seconda dello sporgere e del rientrare di quelli, vien, quasi a un tratto, a ristringersi, e a prender corso e figura di fiume, tra un promontorio a \edtext{destra}{\Cfootnote{lkjlkjlkjlkjljlj}}, e un’ampia costiera dall’altra parte; e il ponte, che ivi congiunge le due rive, par che renda ancor più sensibile all’occhio questa trasformazione, e segni il punto in cui il lago cessa, e l’Adda rincomincia, per ripigliar poi nome di lago dove le \edtext{rive}{\Bfootnote{blablabla Quel ramo del lago di Como, che volge a mezzogiorno, tra due catene non interrotte di monti, tutto a seni e a golfi, a seconda dello sporgere e del rientrare di quelli, vien, quasi a un tratto, a ristringersi, e a prender corso e figura di fiume, tra un promontorio a destra, e un’ampia costiera dall’altra parte; e il ponte, che ivi congiunge le due rive, par che renda ancor più sensibile all’occhio questa trasformazione, e segni il punto in cui il lago cessa, e l’Adda rincomincia, per ripigliar poi nome di lago dove le rive, allontanandosi di nuovo, lascian l’acqua distendersi e rallentarsi in nuovi golfi e in nuovi seni. La costiera,formata dal deposito di tre grossi torrenti, scende appoggiata a due monti contigui, l’uno detto di san Martino, l’altro, con voce lombarda, il Resegone, dai molti suoi cocuzzoli in fila, che in vero lo fanno somigliare a una sega: talché non è chi, al primo vederlo, purché sia di fronte, come per esempio di su le mura di Milano che guardano a settentrione, non lo discerna tosto, a un tal contrassegno, in quella lunga e vasta giogaia, dagli altri monti di nome più oscuro e di forma più comune. Per un buon pezzo, la costa sale con un pendii lento e continuo; poi si rompe in poggi e in valloncelli, in erte e in ispianate,secondo l’ossatura de’ due monti, e il lavoro dell’acque. Il lembo estremo, tagliato dalle foci de’ torrenti, è quasi tutto ghiaia e ciottoloni; il resto, campi e vigne, sparse di terre, di ville, di casali; in qualche parte boschi, che si prolungano su per la montagna. 
}}, allontanandosi di nuovo, lascian l’acqua distendersi e rallentarsi in nuovi golfi e in nuovi seni. La costiera,formata dal deposito di tre grossi torrenti, scende appoggiata a due monti contigui, l’uno detto di san Martino, l’altro, con voce lombarda, il Resegone, dai molti suoi cocuzzoli in \edtext{fila}{\Cfootnote{blablabla}}, che in vero lo fanno somigliare a una sega: talché non è chi, al primo vederlo, purché sia di fronte, come per esempio di su le mura di Milano che guardano a settentrione, non lo discerna \edtext{tosto}{\Afootnote{blablabla}}, a un tal contrassegno, in quella lunga e vasta giogaia, dagli altri monti di nome più oscuro e di forma più comune. Per un buon pezzo, la costa sale con un pendii lento e continuo; poi si rompe in poggi e in valloncelli, in \edtext{erte}{\Afootnote{blablabla}} e in ispianate,secondo l’ossatura de’ due monti, e il lavoro dell’acque. Il lembo estremo, tagliato dalle foci de’ torrenti, è quasi tutto ghiaia e ciottoloni; il resto, campi e vigne, sparse di terre, di ville, di casali; in qualche parte boschi, che si prolungano su per la montagna. 
\pend

\pstart
Quel \edtext{ramo}{\Afootnote{blablabla}} del lago di Como, che volge a mezzogiorno, tra due catene non interrotte di monti, tutto a seni e a golfi, a seconda dello sporgere e del rientrare di quelli, vien, quasi a un tratto, a ristringersi, e a prender corso e figura di fiume, tra un promontorio a destra, e un’ampia costiera dall’altra parte; e il ponte, che ivi congiunge le due rive, par che renda ancor più sensibile all’occhio questa trasformazione, e segni il punto in cui il lago cessa, e l’Adda rincomincia, per ripigliar poi nome di lago dove le \edtext{rive}{\Bfootnote{blablabla Quel ramo del lago di Como, che volge a mezzogiorno, tra due catene non interrotte di monti, tutto a seni e a golfi, a seconda dello sporgere e del rientrare di quelli, vien, quasi a un tratto, a ristringersi, e a prender corso e figura di fiume, tra un promontorio a destra, e un’ampia costiera dall’altra parte; e il ponte, che ivi congiunge le due rive, par che renda ancor più sensibile all’occhio questa trasformazione, e segni il punto in cui il lago cessa, e l’Adda rincomincia, per ripigliar poi nome di lago dove le rive, allontanandosi di nuovo, lascian l’acqua distendersi e rallentarsi in nuovi golfi e in nuovi seni. La costiera,formata dal deposito di tre grossi torrenti, scende appoggiata a due monti contigui, l’uno detto di san Martino, l’altro, con voce lombarda, il Resegone, dai molti suoi cocuzzoli in fila, che in vero lo fanno somigliare a una sega: talché non è chi, al primo vederlo, purché sia di fronte, come per esempio di su le mura di Milano che guardano a settentrione, non lo discerna tosto, a un tal contrassegno, in quella lunga e vasta giogaia, dagli altri monti di nome più oscuro e di forma più comune. Per un buon pezzo, la costa sale con un pendii lento e continuo; poi si rompe in poggi e in valloncelli, in erte e in ispianate,secondo l’ossatura de’ due monti, e il lavoro dell’acque. Il lembo estremo, tagliato dalle foci de’ torrenti, è quasi tutto ghiaia e ciottoloni; il resto, campi e vigne, sparse di terre, di ville, di casali; in qualche parte boschi, che si prolungano su per la montagna. 
}}, allontanandosi di nuovo, lascian l’acqua distendersi e rallentarsi in nuovi golfi e in nuovi seni. La costiera,formata dal deposito di tre grossi torrenti, scende appoggiata a due monti contigui, l’uno detto di san Martino, l’altro, con voce lombarda, il Resegone, dai molti suoi cocuzzoli in \edtext{fila}{\Cfootnote{blablabla}}, che in vero lo fanno somigliare a una sega: talché non è chi, al primo vederlo, purché sia di fronte, come per esempio di su le mura di Milano che guardano a settentrione, non lo discerna \edtext{tosto}{\Afootnote{blablabla}}, a un tal contrassegno, in quella lunga e vasta giogaia, dagli altri monti di nome più oscuro e di forma più comune. Per un buon pezzo, la costa sale con un pendii lento e continuo; poi si rompe in poggi e in valloncelli, in \edtext{erte}{\Afootnote{blablabla}} e in ispianate,secondo l’ossatura de’ due monti, e il lavoro dell’acque. Il lembo estremo, tagliato dalle foci de’ torrenti, è quasi tutto ghiaia e ciottoloni; il resto, campi e vigne, sparse di terre, di ville, di casali; in qualche parte boschi, che si prolungano su per la montagna. 
\pend

\pstart
Quel \edtext{ramo}{\Afootnote{blablabla}} del lago di Como, che volge a mezzogiorno, tra due catene non interrotte di monti, tutto a seni e a golfi, a seconda dello sporgere e del rientrare di quelli, vien, quasi a un tratto, a ristringersi, e a prender corso e figura di fiume, tra un promontorio a destra, e un’ampia costiera dall’altra parte; e il ponte, che ivi congiunge le due rive, par che renda ancor più sensibile all’occhio questa trasformazione, e segni il punto in cui il lago cessa, e l’Adda rincomincia, per ripigliar poi nome di lago dove le \edtext{rive}{\Bfootnote{Quel ramo del lago di Como, che volge a mezzogiorno, tra due catene non interrotte di monti, tutto a seni e a golfi, a seconda dello sporgere e del rientrare di quelli, vien, quasi a un tratto, a ristringersi, e a prender corso e figura di fiume, tra un promontorio a destra, e un’ampia costiera dall’altra parte; e il ponte, che ivi congiunge le due rive, par che renda ancor più sensibile all’occhio questa trasformazione, e segni il punto in cui il lago cessa, e l’Adda rincomincia, per ripigliar poi nome di lago dove le rive, allontanandosi di nuovo, lascian l’acqua distendersi e rallentarsi in nuovi golfi e in nuovi seni. La costiera,formata dal deposito di tre grossi torrenti, scende appoggiata a due monti contigui, l’uno detto di san Martino, l’altro, con voce lombarda, il Resegone, dai molti suoi cocuzzoli in fila, che in vero lo fanno somigliare a una sega: talché non è chi, al primo vederlo, purché sia di fronte, come per esempio di su le mura di Milano che guardano a settentrione, non lo discerna tosto, a un tal contrassegno, in quella lunga e vasta giogaia, dagli altri monti di nome più oscuro e di forma più comune. Per un buon pezzo, la costa sale con un pendii lento e continuo; poi si rompe in poggi e in valloncelli, in erte e in ispianate,secondo l’ossatura de’ due monti, e il lavoro dell’acque. Il lembo estremo, tagliato dalle foci de’ torrenti, è quasi tutto ghiaia e ciottoloni; il resto, campi e vigne, sparse di terre, di ville, di casali; in qualche parte boschi, che si prolungano su per la montagna.}}, allontanandosi di nuovo, lascian l’acqua distendersi e rallentarsi in nuovi golfi e in nuovi seni. La costiera,formata dal deposito di tre grossi torrenti, scende appoggiata a due monti contigui, l’uno detto di san Martino, l’altro, con voce lombarda, il Resegone, dai molti suoi cocuzzoli in \edtext{fila}{\Cfootnote{blablabla Quel ramo del lago di Como, che volge a mezzogiorno, tra due catene non interrotte di monti, tutto a seni e a golfi, a seconda dello sporgere e del rientrare di quelli, vien, quasi a un tratto, a ristringersi, e a prender corso e figura di fiume, tra un promontorio a destra, e un’ampia costiera dall’altra parte; e il ponte, che ivi congiunge le due rive, par che renda ancor più sensibile all’occhio questa trasformazione, e segni il punto in cui il lago cessa, e l’Adda rincomincia, per ripigliar poi nome di lago dove le rive, allontanandosi di nuovo, lascian l’acqua distendersi e rallentarsi in nuovi golfi e in nuovi seni. La costiera,formata dal deposito di tre grossi torrenti, scende appoggiata a due monti contigui, l’uno detto di san Martino, l’altro, con voce lombarda, il Resegone, dai molti suoi cocuzzoli in fila, che in vero lo fanno somigliare a una sega: talché non è chi, al primo vederlo, purché sia di fronte, come per esempio di su le mura di Milano che guardano a settentrione, non lo discerna tosto, a un tal contrassegno, in quella lunga e vasta giogaia, dagli altri monti di nome più oscuro e di forma più comune. Per un buon pezzo, la costa sale con un pendii lento e continuo; poi si rompe in poggi e in valloncelli, in erte e in ispianate,secondo l’ossatura de’ due monti, e il lavoro dell’acque. Il lembo estremo, tagliato dalle foci de’ torrenti, è quasi tutto ghiaia e ciottoloni; il resto, campi e vigne, sparse di terre, di ville, di casali; in qualche parte boschi, che si prolungano su per la montagna.}}, che in vero lo fanno somigliare a una sega: talché non è chi, al primo vederlo, purché sia di fronte, come per esempio di su le mura di Milano che guardano a settentrione, non lo discerna \edtext{tosto}{\Afootnote{blablabla Quel ramo del lago di Como, che volge a mezzogiorno, tra due catene non interrotte di monti, tutto a seni e a golfi, a seconda dello sporgere e del rientrare di quelli, vien, quasi a un tratto, a ristringersi, e a prender corso e figura di fiume, tra un promontorio a destra, e un’ampia costiera dall’altra parte; e il ponte, che ivi congiunge le due rive, par che renda ancor più sensibile all’occhio questa trasformazione, e segni il punto in cui il lago cessa, e l’Adda rincomincia, per ripigliar poi nome di lago dove le rive, allontanandosi di nuovo, lascian l’acqua distendersi e rallentarsi in nuovi golfi e in nuovi seni. La costiera,formata dal deposito di tre grossi torrenti, scende appoggiata a due monti contigui, l’uno detto di san Martino, l’altro, con voce lombarda, il Resegone, dai molti suoi cocuzzoli in fila, che in vero lo fanno somigliare a una sega: talché non è chi, al primo vederlo}}, a un tal contrassegno, in quella lunga e vasta giogaia, dagli altri monti di nome più oscuro e di forma più comune. Per un buon pezzo, la costa sale con un pendii lento e continuo; poi si rompe in poggi e in valloncelli, in \edtext{erte}{\Afootnote{blablabla}} e in ispianate,secondo l’ossatura de’ due monti, e il lavoro dell’acque. Il lembo estremo, tagliato dalle foci de’ torrenti, è quasi tutto ghiaia e ciottoloni; il resto, campi e vigne, sparse di terre, di ville, di casali; in qualche parte boschi, che si prolungano su per la montagna. 
\pend

\pstart
Nel mezzo del cammin di nostra vita. Nel mezzo \edtext{del cammin}{\Afootnote{Nel mezzo del cammin di nostra vita.}} di nostra vita. Nel mezzo del cammin di nostra vita. Nel mezzo del cammin di nostra vita. Nel mezzo del cammin di nostra vita. Nel mezzo del cammin di nostra vita. Nel mezzo del cammin di nostra vita.
\pend

\pstart
Nel mezzo del cammin di nostra vita. Nel mezzo \edtext{del cammin}{\Afootnote{Nel mezzo del cammin di nostra vita.}} di nostra vita. Nel mezzo del cammin di nostra vita. Nel mezzo del cammin di nostra vita. Nel mezzo del cammin di nostra vita. Nel mezzo del cammin di nostra vita. Nel mezzo del cammin di nostra vita.
\pend

\pstart
Nel mezzo del cammin di nostra vita. Nel mezzo \edtext{del cammin}{\Afootnote{Nel mezzo del cammin di nostra vita.}} di nostra vita. Nel mezzo del cammin di nostra vita. Nel mezzo del cammin di nostra vita. Nel mezzo del cammin di nostra vita. Nel mezzo del cammin di nostra vita. Nel mezzo del cammin di nostra \edtext{vita}{\Afootnote{lkjlkjlj ljlasfkvjò jlkj lkj òkkj kkj òlkj òkjl jlòkj lkkj klj}}.
\pend

\pstart
Nel mezzo del cammin di nostra vita. Nel mezzo \edtext{del cammin}{\Afootnote{Nel mezzo del cammin di nostra vita.}} di nostra vita. Nel mezzo del cammin di nostra vita. Nel mezzo del cammin di nostra vita. Nel mezzo del cammin di nostra vita. Nel mezzo del cammin di nostra vita. Nel mezzo del cammin di nostra \edtext{vita}{\Afootnote{lkjlkjlj ljlasfkvjò jlkj lkj òkkj kkj òlkj òkjl jlòkj lkkj klj}}.
\pend
\endnumbering
\end{document}
