
\documentclass{article}
\usepackage[noend,noeledsec,series={A},noledgroup]{reledmac}
\usepackage{libertine}
\begin{document}
\beginnumbering
\pstart
\edtext{TOTO
  \edtext{TATA}{
    \lemma{ἐστι\textsuperscript{2}}
    \Afootnote[]{ἐναντίως ἔχουσι πρὸς ἀλλήλας, ἕτεραι γὰρ ἀλλήλων εἰσί V (153.24--25)}
  }
  {\lemma{διότι \ldots ἐστι}\Afootnote[]{ V \textit{ex homoeoteleuto}}}%HERE THE PROBLEM
  '';),
οὐδὲ ἡ ὑπόληψις ἡ καθ' ἑκάτερον τῶν μέσων
(\edtext{οἷον}{\Afootnote[]{D}} ``ὅτι πᾶσα ἡμίονος ἄτοκος'' καὶ ``ὅτι οὐδὲν ἐξωγκωμένην ἔχον τὴν γαστέρα ἄτοκόν ἐστι'' ἐναντίως ἔχουσι πρὸς ἀλλήλας· ἕτεραι γὰρ ἀλλήλων εἰσί).
τί γὰρ κωλύει τὸν εἰδότα ὅτι τὸ Α ὅλῳ ὑπάρχει τῷ Β καὶ τοῦτο τὸ Β ὑπάρχει τῷ Γ παντί
(ἰστέον ὅτι ἀντὶ τοῦ εἰπεῖν Δ εἶπε Γ), οἰηθῆναι αὐτὸν ὅτι τὸ Α οὐχ ὑπάρχει τῷ; οὐ γὰρ ἐπίσταται τὸ Α τῷ Γ·
οὐκ ἀκριβῶς οἶδεν ὅτι τὸ Α οὐχ ὑπάρχει τῷ ἐξωγκωμένην ἔχοντι τὴν γαστέρα, μὴ συνθεωρῶν καὶ συμμιγνύων τὸ καθ' ἑκάτερον, ἤγουν μὴ συλλαβὼν καὶ τὴν ἐλάττονα πρότασιν, τὴν ``ἥδε ἡ ἡμίονος ἐξώγκωται τὴν γαστέρα'', καὶ ἐν τῷ λαβεῖν τὰς δύο ταύτας προτάσεις ἐνεργείᾳ συλλογίσασθαι ὅτι ἥδε ἡ ἡμίονος ἄτοκος οὐκ ἔστι:--
}
{\lemma{διότι \ldots οὐκ ἔστι}\Afootnote{134}}
\pend

\endnumbering
\end{document}
