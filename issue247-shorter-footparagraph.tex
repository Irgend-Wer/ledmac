\documentclass[12pt]{article}

\usepackage{libertine}
\usepackage{eledmac}

\usepackage[showframe]{geometry}
\maxhXnotes{0.8\textheight}
\footparagraph{A}

\begin{document}
\beginnumbering

\pstart 
immobili, hoc esset aut propter indigentiam motoris aut mobilis. Si
propter indigentiam motoris, aut ergo \edtext{propter
  indigentiam}{\Afootnote{\emph{om.} M}} motoris primi aut
appropriati. Non propter indigentiam motoris simpliciter primi quia,
cum primus motor sit virtutis infinitae in motu suo, nullo indiget,
nec appropriati <quia>, cum \edtext{sit}{\Afootnote{sunt M}}
substantia separata, perfectionem suam
\edtext{recipit}{\Afootnote{recipiat MP}} a motore simpliciter
primo. Videtur igitur quod in motu suo <a> nullo inferiori dependeat,
quare etc. Nec propter indigentiam \edtext{mobilis}{\Afootnote{motoris
    M}} quia, cum mobile sit continuum et aeternum, a nihilo dependet,
et maxime non indigebit aliquo quod est infra eum et ab eo causatum
est.  
\pend

\pstart 
immobili, hoc esset aut propter indigentiam motoris aut mobilis. Si
propter indigentiam motoris, aut ergo \edtext{propter
  indigentiam}{\Afootnote{\emph{om.} M}} motoris primi aut
appropriati. Non propter indigentiam motoris simpliciter primi quia,
cum primus motor sit virtutis infinitae in motu suo, nullo indiget,
nec appropriati <quia>, cum \edtext{sit}{\Afootnote{sunt M}}
substantia separata, perfectionem suam
\edtext{recipit}{\Afootnote{recipiat MP}} a motore simpliciter
primo. Videtur igitur quod in motu suo <a> nullo inferiori dependeat,
quare etc. Nec propter indigentiam \edtext{mobilis}{\Afootnote{motoris
    M}} quia, cum mobile sit continuum et aeternum, a nihilo dependet,
et maxime non indigebit aliquo quod est infra eum et ab eo causatum
est.  
\pend

\pstart 
immobili, hoc esset aut propter indigentiam motoris aut mobilis. Si
propter indigentiam motoris, aut ergo \edtext{propter
  indigentiam}{\Afootnote{\emph{om.} M}} motoris primi aut
appropriati. Non propter indigentiam motoris simpliciter primi quia,
cum primus motor sit virtutis infinitae in motu suo, nullo indiget,
nec appropriati <quia>, cum \edtext{sit}{\Afootnote{sunt M}}
substantia separata, perfectionem suam
\edtext{recipit}{\Afootnote{recipiat MP}} a motore simpliciter
primo. Videtur igitur quod in motu suo <a> nullo inferiori dependeat,
quare etc. Nec propter indigentiam \edtext{mobilis}{\Afootnote{motoris
    M}} quia, cum mobile sit continuum et aeternum, a nihilo dependet,
et maxime non indigebit aliquo quod est infra eum et ab eo causatum
est.  
\pend

\pstart 
immobili, hoc esset aut propter indigentiam motoris aut mobilis. Si
propter indigentiam motoris, aut ergo \edtext{propter
  indigentiam}{\Afootnote{\emph{om.} M}} motoris primi aut
appropriati. Non propter indigentiam motoris simpliciter primi quia,
cum primus motor sit virtutis infinitae in motu suo, nullo indiget,
nec appropriati <quia>, cum \edtext{sit}{\Afootnote{sunt M}}
substantia separata, perfectionem suam
\edtext{recipit}{\Afootnote{recipiat MP}} a motore simpliciter
primo. Videtur igitur quod in motu suo <a> nullo inferiori dependeat,
quare etc. Nec propter indigentiam \edtext{mobilis}{\Afootnote{motoris
    M}} quia, cum mobile sit continuum et aeternum, a nihilo dependet,
et maxime non indigebit aliquo quod est infra eum et ab eo causatum
est.  
\pend

\pstart 
immobili, hoc esset aut propter indigentiam motoris aut mobilis. Si
propter indigentiam motoris, aut ergo \edtext{propter
  indigentiam}{\Afootnote{\emph{om.} XXM}} motoris primi aut
appropriati. Non propter indigentiam motoris simpliciter primi quia,
cum primus motor sit virtutis infinitae in motu suo, nullo indiget,
nec appropriati <quia>, cum \edtext{sit}{\Afootnote{sunt M}}
substantia separata, perfectionem suam
\edtext{recipit}{\Afootnote{recipiat MP}} a motore simpliciter
primo. Videtur igitur quod in motu suo <a> nullo inferiori dependeat,
quare etc. Nec propter indigentiam \edtext{mobilis}{\Afootnote{motoris
    M}} quia, cum mobile sit continuum et aeternum, a nihilo dependet,
et maxime non indigebit aliquo quod est infra eum et ab eo causatum
est.  
\pend

\pstart 
immobili, hoc esset aut propter indigentiam motoris aut mobilis. Si
propter indigentiam motoris, aut ergo \edtext{propter
  indigentiam}{\Afootnote{\emph{om.} M}} motoris primi aut
appropriati. Non propter indigentiam motoris simpliciter primi quia,
cum primus motor sit virtutis infinitae in motu suo, nullo indiget,
nec appropriati <quia>, cum \edtext{sit}{\Afootnote{sunt M}}
substantia separata, perfectionem suam
\edtext{recipit}{\Afootnote{recipiat MP}} a motore simpliciter
primo. Videtur igitur quod in motu suo <a> nullo inferiori dependeat,
quare etc. Nec propter indigentiam \edtext{mobilis}{\Afootnote{motoris
    M}} quia, cum mobile sit continuum et aeternum, a nihilo dependet,
et maxime non indigebit aliquo quod est infra eum et ab eo causatum
est.  
\pend

\pstart 
immobili, hoc esset aut propter indigentiam motoris aut mobilis. Si
propter indigentiam motoris, aut ergo \edtext{propter
  indigentiam}{\Afootnote{\emph{om.} M}} motoris primi aut
appropriati. Non propter indigentiam motoris simpliciter primi quia,
cum primus motor sit virtutis infinitae in motu suo, nullo indiget,
nec appropriati <quia>, cum \edtext{sit}{\Afootnote{sunt M}}
substantia separata, perfectionem suam
\edtext{recipit}{\Afootnote{recipiat MP}} a motore simpliciter
primo. Videtur igitur quod in motu suo <a> nullo inferiori dependeat,
quare etc. Nec propter indigentiam \edtext{mobilis}{\Afootnote{motoris
    M}} quia, cum mobile sit continuum et aeternum, a nihilo dependet,
et maxime non indigebit aliquo quod est infra eum et ab eo causatum
est.  
\pend

\pstart 
immobili, hoc esset aut propter indigentiam motoris aut mobilis. Si
propter indigentiam motoris, aut ergo \edtext{propter
  indigentiam}{\Afootnote{\emph{om.} M}} motoris primi aut
appropriati. Non propter indigentiam motoris simpliciter primi quia,
cum primus motor sit virtutis infinitae in motu suo, nullo indiget,
nec appropriati <quia>, cum \edtext{sit}{\Afootnote{sunt M}}
substantia separata, perfectionem suam
\edtext{recipit}{\Afootnote{recipiat MP}} a motore simpliciter
primo. Videtur igitur quod in motu suo <a> nullo inferiori dependeat,
quare etc. Nec propter indigentiam \edtext{mobilis}{\Afootnote{motoris
    M}} quia, cum mobile sit continuum et aeternum, a nihilo dependet,
et maxime non indigebit aliquo quod est infra eum et ab eo causatum
est.  
\pend

\pstart 
immobili, hoc esset aut propter indigentiam motoris aut mobilis. Si
propter indigentiam motoris, aut ergo \edtext{propter
  indigentiam}{\Afootnote{\emph{om.} M}} motoris primi aut
appropriati. Non propter indigentiam motoris simpliciter primi quia,
cum primus motor sit virtutis infinitae in motu suo, nullo indiget,
nec appropriati <quia>, cum \edtext{sit}{\Afootnote{sunt M}}
substantia separata, perfectionem suam
\edtext{recipit}{\Afootnote{recipiat MP}} a motore simpliciter
primo. Videtur igitur quod in motu suo <a> nullo inferiori dependeat,
quare etc. Nec propter indigentiam \edtext{mobilis}{\Afootnote{motoris
    M}} quia, cum mobile sit continuum et aeternum, a nihilo dependet,
et maxime non indigebit aliquo quod est infra eum et ab eo causatum
est.  
\pend


\endnumbering
\end{document}
