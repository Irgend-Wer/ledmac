\documentclass[a4paper,11pt]{article}

\usepackage{lscape}

\usepackage{polyglossia}
\setmainlanguage{german}
\setotherlanguage{sanskrit}

\usepackage{fontspec}
\setmainfont{TeX Gyre Termes}
\setmonofont{TeX Gyre Termes}
\newfontfamily\devanagarifont[Script=Devanagari]{Chandas}
\newfontfamily\devnag[Script=Devanagari]{Chandas}

\newfontface\dejlight{DejaVu Sans}

\DeclareTextFontCommand{\textding}{\dejlight}

\usepackage[hang,flushmargin]{footmisc}

\usepackage[noend,noeledsec]{reledmac}
\usepackage{libertineotf}
\arrangementX[A]{paragraph}
\Xarrangement[B]{paragraph}
\Xarrangement[A]{paragraph}

\setstanzaindents{3,1}
\setcounter{stanzaindentsrepetition}{1}
\linenumincrement{1}
\firstlinenum{1}

\begin{document}

\section{Edition}
\beginnumbering

\stanza
\edtext{\setline{1}\linenumannotation{a}\edtext{candraḥ}{\Bfootnote{em. : \textit{candra} Ms}} sūryas tathā \edtext{vahniḥ}{\Bfootnote{em. : \textit{vahni} Ms}} \linenumannotation{b}śarīre daśa nāḍikāḥ |&
\setline{1}\linenumannotation{c}\edtext{pañca sthā}{\Bfootnote{\textit{dehasthā} Pts}} \edtext{vāyavaḥ}{\Bfootnote{em. : \textit{vāyava} Ms}} pañca \linenumannotation{d}mano bindus tathaiva ca ||1||}{\lemma{\textbf{Ci}}\Afootnote[nosep]{Vgl. Pts S.84, Z. 5102-5104: candraḥ sūryas tathā vahniḥ śarīre daśa nāḍikāḥ | dehasthā vāyavaḥ pañca mano bindus tathaiva ca |}}\&

\endnumbering
\clearpage

\end{document}
