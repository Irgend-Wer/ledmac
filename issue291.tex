\documentclass[paper=A5]{scrbook}
\usepackage{eledmac}

\twolines{f.}
\morethantwolines{ff.}


\begin{document}
\beginnumbering\pstart
At the extremity of the \edtext{roadstead}{\Afootnote{road}} of Brest, in the \edtext{open space that lies stretched out between the Ile Longue and Point Kelerne}{\lemma{open \ldots Kelerne}\Bendnote{nothing much to add}}, may be seen two rocks crowned with massive granite buildings, and standing boldly up. On the former, the lazaretto of Trébéron has been established; the latter, which in other days was used as a burial-ground and thence took its name of the Ile des Morts, now contains the principal powder-magazine of the naval arsenal. The two rocks separated by an arm of the sea, are about six miles distant from Brest. In appearance these little islands are not unlike. \edtext{Beyond}{\Bendnote{Bbyond}} the ground occupied by the buildings upon them, they offer nothing to the eye save a succession of stony slopes, dotted here and there with coarse moss and prickly thorn-broom. Vainly there might you look \edtext{for any other shelter than that than that afforded}{\lemma{for \ldots afforded}\Afootnote{nothing much to say}} by the fissures of the rocks, for any other shade than that of the walls, for any other walk than the short terrace contrived in front of the buildings. Naked and sterile, the two isles remind you of a couple of immense sentry-boxes in stone, placed there for the purpose of keeping guard over the sea, which is roaring beneath them. But if the foot that treads them remains imprisoned within a narrow circle, the view from their summit extends over an infinite space. Here, you have the bay of Lanvoc, bordered by a dull-looking and stunted vegetation; there, \edtext{Roscanvel}{\Afootnote{Roscanvel}} with its shadows crossed by the graceful spire of its church; there, Spanish Point bristling with batteries; and lastly, close upon the horizon lies \edtext{Brest}{\Bendnote{Breast}}
\pend\endnumbering
\newpage
\doendnotes{B}

\end{document}
