\documentclass[b5paper,12pt]{book}
\usepackage{eledmac}	
\renewcommand{\notenumfont}{\bfseries\footnotesize} % bold line numbers
\renewcommand*{\symplinenum}{$|$$|$} % lemma separator
\numberonlyfirstinline
\numberonlyfirstintwolines
\footparagraph{B} %% apparatus fontium (\af)
\footparagraph{C} %% apparatus criticus (\ac) 
\begin{document}

\linenummargin{left}
\beginnumbering
\edlabel{II01s}

\pstart
   Item \edtext{Commentator}{\Bfootnote{AVERDEAN II.14 (Crawford 133.28--134.44).}} in isto secundo ponit 
   \edtext{differentiam}{\Afootnote{differentias S}} inter formas substantiales et accidentales per hoc quod
   forma \edtext{accidentalis habet}{\Afootnote{accidentales habent V}} subiectum in actu et
   \edtext{substantialis}{\Afootnote{substantiales V}} non. Et tamen dicitur \edtext{\edtext{quinto}{\Afootnote{septimo S}} 
   Metaphysicae}{\Bfootnote{IUNTINA VIII f. 118rb.}} quod anima est substantia in subiecto in
   actu, et econtra de formis \edtext{simplicium}{\Afootnote{simplicibus S}}, scilicet elementorum. Et hoc
   etiam habetur in \edtext{octavo Physicorum}{\Bfootnote{IUNTINA IV f. 365rb--va.}},
   ubi dicitur quod grave inanimatum non movetur ex se, sicut animal, quia non est divisibile in
   \edtext{partem per se moventem}{\lemma{partem \ldots\ moventem}\Afootnote{partes per se moventes S}}
   \edtext{et partem per se motam}{\lemma{et \ldots\ motam}\Afootnote{om T}}. Animal
   autem sic est divisibile, scilicet in 
   \edtext{animam, quae est}{\lemma{animam quae est}\Afootnote{partem T}} per se movens, et
   corpus, quod est per se motum. Ergo illud corpus est subiectum in actu, quia
   dicitur \edtext{\edtext{tertio}{\Afootnote{quinto AETV}} Physicorum}{\Bfootnote{PHYS
   III.1 201a10 [but some possible ref. in V.1, too].}} quod ens in potentia non movetur \edtext{per}{\Afootnote{ex se vel per A
   ex E}} se\edlabel{II01NN1}, sed ens in actu. Ergo anima est forma accidentalis.
\pend

\pstart  
   Item \edtext{ex}{\linenum{|\xlineref{II01NN1}}\lemma{se \ldots\ ex}\Afootnote{om E}} \edtext{septimo Metaphysicae}{\Bfootnote{META
   VII.5 1031a1.}} habetur quod substantiae non definiuntur per
   \edtext{additamenta}{\Afootnote{additamentum EST}}. Sed anima definitur per additamentum, scilicet per
   corpus, quod non est de eius essentia; ergo etc.
\pend

\edlabel{II01e}
\endnumbering
\end{document}
