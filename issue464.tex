\documentclass{book}

% \usepackage[utf8]{inputenc} 
% \usepackage[T1]{fontenc}
% \usepackage[greek,german]{babel}
\usepackage{fontspec}
\newfontfamily\greekfont[Mapping=tex-text,Script=Greek,Ligatures=Common]{GFS
Neohellenic}
\setromanfont[Mapping=tex-text,Numbers=Lowercase,Ligatures=Common]{Arial}
\usepackage{polyglossia}
\setdefaultlanguage{german}
\setotherlanguage[variant=ancient]{greek}
\usepackage[series={A,B},noeledsec,noend]{reledmac}
\usepackage{reledpar}

\renewcommand*{\thefootnoteA}{\alph{footnoteA}}
%\arrangementX[A]{paragraph}
\begin{document}

\begin{pairs}
\begin{Leftside}
\begin{greek}
%\selectlanguage{greek}
\beginnumbering
\pstart
Ἄνδρα μοι \edtext{ἔννεπε}{\Bfootnote{λέγε Q}}, Μοῦσα…
\pend
\endnumbering
\end{greek}
\end{Leftside}

\begin{Rightside} 
\beginnumbering
\pstart
Et quamquam Philon in tractatu suo de aqueductibus\footnoteA{Philon
  von Byzanz}. Magister Dominicus de Florentia\footnoteA{Dominicus von
  Florenz}
\pend
\endnumbering
\end{Rightside}
\end{pairs}
\Columns
\end{document}
