\documentclass[]{article}
\usepackage[noend,nocritical,nofamiliar,series={},noeledsec]{reledmac}
\usepackage{polyglossia}
\setmainlanguage{latin}
\setotherlanguage{english}
\usepackage{libertineotf}
\usepackage[widthliketwocolumns,continuousnumberingwithcolumns]{reledpar}
\setlength{\Lcolwidth}{0.47\textwidth}
\setlength{\Rcolwidth}{0.47\textwidth}
\columnsposition{c}
\setlength{\beforecolumnseparator}{0.03\textwidth}
\setlength{\aftercolumnseparator}{0.03\textwidth}

\AtBeginPairs{\sloppy}

\linenumberLevenifblanktrue
\linenumberRevenifblanktrue
\usepackage{lipsum}
\firstlinenum*{1}
\linenumincrement*{1}
\begin{document}

\begin{english}
\title{Mixing columns with not column, using continuous numbering}
\begin{abstract}
In this example, we use the \verb+widthliketwocolumns+ and \verb+continuousnumberingwithcolumns+ and \verb+\pausenumbering+ to alternate parallel typesetting in columns and normal typesetting, getting continuous line numbering between them.
\end{abstract}
\end{english}
\begin{pairs}  
\begin{Leftside} 
\beginnumbering  
\pstart  

\lipsum[23]
\pend  
\pausenumbering
\end{Leftside} 

\begin{Rightside}   
\beginnumbering  
\pstart  
\lipsum[25]
\pend  
\pausenumbering
\end{Rightside}  

\end{pairs}  
\Columns

\resumenumbering
\pstart  
\lipsum[23]
\pend  
\pausenumbering

\begin{pairs}  
 

\begin{Leftside} 
\resumenumbering  
\pstart  

\lipsum[23]
\pend  
\pausenumbering 
\end{Leftside}

\begin{Rightside}   
\resumenumbering  
\pstart  
\lipsum[25]
\pend  
\pausenumbering
\end{Rightside}  
\end{pairs}  
\Columns

\resumenumbering  
\pstart  
\lipsum[25]
\pend  
\pausenumbering

\begin{pairs}  
  

\begin{Leftside} 
\resumenumbering  
\pstart  

\lipsum[25]
\pend  
\pausenumbering 
\end{Leftside}

\begin{Rightside}   
\resumenumbering  
\pstart  
\lipsum[23]
\pend  
\endnumbering
\end{Rightside} 
\end{pairs}  

\Columns
\resumenumbering  
\pstart  
\lipsum[25]
\pend  
\pausenumbering

\end{document}
