\documentclass[12pt]{article}
\usepackage[series={A,B},noend,noledgroup,nopenalties]{reledmac}
\usepackage{reledpar}
\usepackage[paper=16cm:24cm]{typearea}
\usepackage[top=2.5cm,bottom=2.5cm,left=2.2cm,right=2.2cm]{geometry}
\usepackage[no-math]{fontspec}
\usepackage{perpage}
\usepackage{arabluatex}
\usepackage{luabidi}
\setmainfont{Amiri}
\newfontfamily\enfont[Ligatures=TeX]{Junicode}
\newfontfamily\ayafont[Script = Arabic, Mapping = arabicdigits]{Amiri}
\Xarrangement{paragraph}
\Xbhookgroup[A]{\pardir TRT\textdir TRT} %
\bhooknoteX[A]{\pardir TRT\textdir TRT} %
\newcommand{\n}[1]{\bgroup\textdir TLT #1\egroup}
\firstlinenum{1} \linenumincrement{1} \lineation{page}
\NewDocumentCommand{\arnote}{m}{\footnoteA{\txarb{#1}\hfill}}%\bodydir TRT \pardir TRT \textdir TRT

\NewDocumentCommand{\ennote}{m}{\footnoteB{#1}}
\NewDocumentCommand{\commentaire}{m}{\footnoteB{\txarb{#1}}}%\bodydir TRT \bgroup  \pardir TRT \textdir TRT \egroup
%Critical Notes
\newcommand{\variant}[3]{\edtext{\txarb{#1}}{\Afootnote{\txarb{#2 : #3}}}}
\newcommand{\om}[2]{\variant{\txarb{#1}}{\txarb{محذوفة من}}{\txarb{#2}}}
\newcommand{\add}[3]{\variant{\txarb{#1}}{\txarb{#3}}{\txarb{{زائدة في}} #2}}
\MakePerPage{footnoteA@typeset}
\MakePerPage{footnoteB@typeset}
\pretocmd\linenumrep{%
 \textdir TLT%
}{}{}
\linespread{1.325}

\sidenotemargin{{\protect}right\protect}
\renewcommand{\Leftpagehook}{%
\setRTL%
}

\renewcommand{\Rightpagehook}{%
\setLTR%
}
\begin{document}
\begin{pages}
\begin{Leftside}

\linenummargin{right}
\beginnumbering
\pstart[] %%\pardir TRT  \textdir TRT works as well...
\begin{txarab}
\flushright


الصلاة \om{والسلام}{ب} على رسول الله

اكترى؟ السيد بلغيث بن احمد

التواتى بيتا من عبد الله

عتيڧ مينصرعاما كاملا

بخمسة وعشرين الڢ \variant{مبداه}{ب}{ابتداءا}؟

ذو الڧعدة الذى ڧبل التاريخ

وڧبـ]ـض[ من الكراء المذكور

اعلاه المذكور اثنـ]ـى[ عشر الڢ

اعتراڢا وبڧى بذمة الكارى

ثلاثة عشر الڢ وكتب من حضر

لهما واشهداه وعرڢهما بتاريخ

اواخر؟ ڢاتح المحرم عام خمسة

واربعين ومايتين والڢ\arnote{\ 3 جويلية \n{1829}} عبد ربه

...حمد بن عبد الله التواتى \variant{تيب}{ب}{أتاب}

عليه امين

امين امين

]الحمد[ لله وحده

دڢع سيد بلغيث المذكور اعلاه

... ما عليه من الكراء لرب الدار عبد

الله المذكور وابراه ولا يبڧا بينهما

بعد ولا متبوعا؟ الا ان ثلاثة ءالڢ

... ماية اعتراڢا والباڧى

... وكتب من شهد بينهم

... يوم من جمادى الاول عا]م[...

... ... ... الڢ ... ... وعند ... ...



وستين الڢا وعند اخيه ... ابناء معلم عمر سبـ]ـعين[

الڢ وعند اريم عبد ابن امير زبير خمسة وتسعين الڢ

وعند زاكا الحر ستة وعشرين الڢا دڢع دنزنبدى

ثلاثة ءالڢ دڢع كنكر خمسة وعشرين

الڢا عند مكاج دنكام ثلاثة ءالڢ من ڧبال

الحرير وعند عثمان؟ ابن ميسكِڢ؟ ثلاثة ءالڢ من

ڧبال الحرير وعند الغيط \arnote{أبو الغيث.} الڢ وسبعة ما]يـ[ـة من ڧبال

الڧرْ وعند اُبَالْ دَوَكِ صاحب صبُن كَرِِ ءالڢ وخمسة

ما]يـ[ـة من ڧبال الڧرْدڢع ذاكا اربعة عاشر الڢا

ايضا عند الحاج مصطڢى خمسة عاشر وخمسة ما]يـ[ـة

اودع ايضا عند اخيه يحي اربعة ءالاڢ وعند برمن

بن برك ثلاث وعشرين الڢا غير مئتين واجل المذكو

رين؟؟ ربيع الثانى اشتروا السلعة خمسة عاشر

ڢى شهر الله محرم عام اربع وستين وميئتين

بعد الالڢ\arnote{ديسمبر  \n{1847}} وشهد عليهم الحاج اكلى بن

جبريل بن بتورو دڢع الحاج مصطڢى المذكور سبعة

ءالڢ

وعند باوا المذكور عشرة اجلود اصڢر واحمرامانة

عنده

وايضا عند باوا صاحب كلتدنك اخ لصوڢ بن

لحسن اثنا وخمسين الڢا وايضا عند معلم ...

جار لمعمم باك احدى عاشر الڢا وعند ابراهيم ابرنداوا

سبعة ءالاڢ غير ما]يـ[ـة وعند زا... صاحب كبكوا الڢين


وثلاث مايه واعند أدل معونر كنت سلڢ الجود والاحسان\arnote{قرض أو سلف الجود و الإحسان يعني أن القرض  غير ربوي ولكن حساب التضخم غير واضح.}

دڢع زكا ثلاث ءالڢ
\end{txarab}
\pend
\endnumbering
\end{Leftside}


\begin{Rightside}

\firstlinenum{1} \linenumincrement{1} \lineation{page}
\linenummargin{left}
\beginnumberingR
\pstart[]

 \parindent=0pt
[Praise and Peace be] upon the Prophet of God.

Bulghīth b. Ahmad al-Tuwātī rents a room [place]

 for an entire year from ʿAbd Allāh the manumitted slave of May Nasara,

 with 25.000, starting

from Dhū 'l-Qiʿda which is before this date.

He acknowledges that he got from

mentioned rent {\dots} 12,000.

Still remaning to pay by the renter

13,000. Written by whom he was present [during the transaction]

and taken as a witness, knowing both of them,

in the date of the end of Muḥarram of the year

1245.\ennote{July 1829.} The slave of his God

{\dots} Muḥammad b. ʿAbd Allāh al-Tuwātī.

May God forgive

him. Amen,

amen, amen.


Praise be to God alone.

Sīdī Bulghīth the aforementioned

paid \ {\dots} what he owes

from the rent to the owner of the house

{\dots} {\dots} Allah, mentioned above,

and he is innocent {\dots}

There is only {\dots}. {\dots} 3,000

{\dots} {\dots} recognized and the rest

{\dots} Written by whom he witnessed

{\dots} Jumādā I...[p. 69]


{\dots} 1,400. \ owes {\dots}

60,000. his brother {\dots} the sons of mʿallam ʿUmar 70.000

Arim\ennote{Probably  Yarima} the slave of the son of the Amir Zubayr 95.000;

Zaka [ZaK.] the free man, 26.000. Dan Zambadi

paid 3,000. Kankari\ennote{Most probalby Kankarau.} paid 25.000;

Magaji\ennote{Magaji is the governor, or leader in general, in Hausa.} Dan Kama owes 3,000 for

silk. {\dots} son of May SKF 3,000 for

silk. Ghayt owes 700 for

[Qar? ]. Ubal Dawaki, [governor] of Sabon Gari\ennote{Probably located in the outskirts of Zaria in Northen Nigeria.} 1,500

for [QR]. Dhaku paid 14,000.


Also Hajj Mustafa owes 15,500

cauri shells. His brother Yahya 4,000;  Barmini

b. [{\dots}.] 22.800 [litt. 23,000 -200] The due date is

{\dots} Rabīʿ II. They bought a merchandise 15

of the godly month of Muḥarram of 1264.\ennote{23 December 1847.}

Testified by al-Ḥājj Aklī

b. Jibrīl b. Batūri (Bitur?)... The aforementioned Ḥājj Muṣṭafā \edtext{paid
7.000.}{\Bfootnote{Paid 700?}} The aforementioned Bawa owes 10,000

price of yellow and red hides as a deposit in his custody.

He has

Also Bawa,  governor of Kiltadnak\ennote{Kel Denek Tuaregs, probably.} brother Sufi b.

 Laḥsan 52,000 Mallam {\dots}

 nieghbour of Mallam Baku owes 11,000 Ibrahim Abrandawa

 owes 6,900 [litt. 7000 -100]. Zā... the governor of Kabakawa\ennote{Kabakawam is an area located of southwest of Katsina.} 2,300

 Adli Maʿwanra [Kunta] a loan of grace

 and benevolence.

 [margin: Zaka paid 3,000]


\pend
\endnumbering
\end{Rightside}

\end{pages}
\Pages


\end{document}
