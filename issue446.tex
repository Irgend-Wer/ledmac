\documentclass[twoside, DIV14, 11pt]{scrbook}

\usepackage{fontspec}
\setmainfont{Linux Libertine O}
\usepackage{polyglossia}
\setdefaultlanguage{german}
\setotherlanguage[variant=ancient]{greek}
\newfontfamily\greekfont{Linux Libertine O}
\usepackage{reledmac}
\usepackage{reledpar}
\setstanzaindents{1,0}
\setcounter{stanzaindentsrepetition}{1}
\stanzaindentbase=1cm
\setlength{\Lcolwidth}{0.4\textwidth}
\setlength{\Rcolwidth}{0.55\textwidth}

\begin{document}
\begin{pairs}
\begin{Leftside}
    \selectlanguage{greek}
    \beginnumbering
    \begin{astanza}
        \textbf{Ἀντιγόνη}\skipnumbering&
        οὐ γὰρ τάφου νῷν τὼ κασιγνήτω Κρέων&
        τὸν μὲν προτίσας, τὸν δ' ἀτιμάσας ἔχει;&
        Ἐτεοκλέα μέν, ὡς λέγουσι, σὺν δίκης&
        χρήσει δικαίᾳ καὶ νόμου  κατὰ χθονὸς&
        ἔκρυψε τοῖς ἔνερθεν ἔντιμον νεκροῖς·&
        τὸν δ' ἀθλίως θανόντα Πολυνείκους νέκυν&
        ἀστοῖσί φασιν ἐκκεκηρῦχθαι τὸ μὴ&
        τάφῳ καλύψαι μηδὲ κωκῦσαί τινα,&
        ἐᾶν δ' ἄκλαυτον, ἄταφον, οἰωνοῖς γλυκὺν&
        θησαυρὸν εἰσορῶσι πρὸς χάριν βορᾶς.&
        τοιαῦτά φασι τὸν ἀγαθὸν Κρέοντα σοὶ&
        κἀμοί, λέγω γὰρ κἀμέ, κηρύξαντ' ἔχειν,&
        καὶ δεῦρο νεῖσθαι ταῦτα τοῖσι μὴ εἰδόσιν&
        σαφῆ προκηρύξοντα, καὶ τὸ πρᾶγμ' ἄγειν&
        οὐχ ὡς παρ' οὐδέν, ἀλλ' ὃς ἂν τούτων τι δρᾷ,&
        φόνον προκεῖσθαι δημόλευστον ἐν πόλει.&
        οὕτως ἔχει σοι ταῦτα, καὶ δείξεις τάχα&
        εἴτ' εὐγενὴς πέφυκας εἴτ' ἐσθλῶν κακή.\&
    \end{astanza}
    \endnumbering
\end{Leftside}
\begin{Rightside}
    \beginnumbering
    \selectlanguage{german}
    \begin{astanza}
        \textbf{ANTIGONE.}\skipnumbering&
        Hat unsrer Brüder Kreon den des Grabes nicht&
        Sogleich gewürdigt, aber den voll Schmach beraubt?&
        Denn zwar Eteokles hat er, wie man sagt, gerecht&
        Und guter Ordnung folgend, gleich in Erde wohl&
        Verhüllt, damit ihm Ehre bei den Toten sei;&
        Polyneikes' elend umgekommnen Leib jedoch&
        Verbeut er, heißt es, unter Heroldsruf der Stadt&
        Zu klagen noch ins Grab zu tun; nein, diesen soll&
        Grablos sie lassen, unbeweint, zum süßen Fund&
        Den Vögeln, welche Fraßbegier herniedertreibt.&
        Dergleichen habe der edle Kreon, heißt es, dir&
        Und mir, auch mir ja, sag ich, angekündiget &
        Und kehre nochmals, allen noch Unkundigen&
        Es deutlich auszurufen, und betreibe nicht&
        Für nichts die Sache; sondern wer dagegen tut,&
        Dem droht des Steinwurfs herber Tod in offner Stadt.&
        Nun, so verhält sich dieses, und du zeigest bald,&
        Ob selbst du edel sprossest, ob von Guten schlecht.\&
    \end{astanza}
    \endnumbering
\end{Rightside}
\end{pairs}
\Columns
\end{document}
