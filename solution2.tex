\documentclass{article}
%http://tex.stackexchange.com/a/196456/7712
\usepackage{eledmac,tikz,etoolbox}
\usetikzlibrary{calc}
\linenummargin{left}
\firstlinenum{1}

\linenumincrement{1}
\makeatletter
%%% This code will be in .sty file
\newcommand{\do@tikzline@numberalign}{%
  \newbox\@temp%
  \setbox\@temp=\hbox{X}%
  \hspace{-1\wd\@temp}%
  \hspace{-3\paperwidth}\unhbox\@temp\hspace{3\paperwidth}%
}

\newcommand{\do@tikzline}[1]{%
  \do@tikzline@numberalign%
  \edlabel{tikz:#1}%
  \l@demptyd@ta%
  \getline@num%
  \affixline@num%
  \affixside@note%
  \l@dld@ta%
  \l@dlsn@te%
  \new@line%
  \hfill%
  \l@drd@ta%
  \l@drsn@te%
}

\newcommand{\add@to@edtikzt}[2]{%
  \ifcsundef{edtikzt@#1}{\csdef{edtikzt@#1}{}}{}%
  \global\expandafter\appto\csname edtikzt@#1\endcsname{#2}%
}

\newcommand{\ledsidenotetikz}[2]{%
  \add@to@edtikzt{#2}{\ledsidenote{#1}}%
}%

\newcommand{\edtexttikz}[3]{%
  \showlemma{#1}%
  \add@to@edtikzt{#3}{\edtext{}{\lemma{#1}\xxref{tikz:#3}{tikz:#3}#2}}{}{}%
}

\newcommandx{\pstarttikz}[2][1,2,usedefault]{%
  \pstart[#2]%
  \noindent\begin{tikzpicture}[#1]%
}

\newcommandx{\pendtikz}[2][1,2,usedefault]{%
 \foreach \Node in {#1} {%\y1=y-coord of the node
    \path let \p1=($ (\Node) $) in node at (0,\y1)[inner sep=0,outer sep=0,text width=\textwidth]{\noindent\do@tikzline{\Node}};
    \csuse{edtikzt@\Node}%
  }%
 \end{tikzpicture}\skipnumbering
 \pend[#2]%
}

\makeatother
% And so, now, user code
\begin{document}
\beginnumbering
\pstart
\ledsidenote{first}Lorem ipsum dolor sit amet, consectetur adipiscing elit. Sed et ligula eros. Proin consequat ligula non lacus adipiscing, quis ornare leo tincidunt. Quisque quis molestie felis. Aliquam sapien quam, ultrices et magna a, dapibus iaculis dui. Mauris accumsan tristique est vitae ultricies. Donec ultricies magna quam. Nam imperdiet augue porttitor, dignissim sem non, tempus nisi. Donec vehicula ipsum eget ipsum ornare tempor. Phasellus enim justo, gravida laoreet vehicula cursus, tristique in libero. Lorem ipsum dolor sit amet, consectetur adipiscing elit. Integer non faucibus purus. Phasellus et convallis nulla. Fusce vestibulum tortor nisl, sit amet adipiscing turpis posuere vel. Morbi eget feugiat quam. Vestibulum eu lacinia urna. Maecenas eget consectetur ante.
\pend
\pstarttikz[sibling distance=4cm,
                    edge from parent/.style={draw,thick}]
   \node(A){Prova}
        child {node(B)[align=left] {Second line - \edtexttikz{left}{\Afootnote{gauche}}{B} } }
        child {node[align=right] {Second line - right}
            child {node (C)[align=left] {\ledsidenotetikz{side note}{C}Third line - left}}
            child {node {Third line }
              child {node(D){ourth line}}
              }
        };
\pendtikz[A,B,C,D][]
\pstart
\ledsidenote{second}Lorem ipsum dolor sit amet, consectetur adipiscing elit. Sed et ligula eros. Proin consequat ligula non lacus adipiscing, quis ornare leo tincidunt. Quisque quis molestie felis. Aliquam sapien quam, ultrices et magna a, dapibus iaculis dui. Mauris accumsan tristique est vitae ultricies. Donec ultricies magna quam. Nam imperdiet augue porttitor, dignissim sem non, tempus nisi. Donec vehicula ipsum eget ipsum ornare tempor. Phasellus enim justo, gravida laoreet vehicula cursus, tristique in libero. Lorem ipsum dolor sit amet, consectetur adipiscing elit. Integer non faucibus purus. Phasellus et convallis nulla. Fusce vestibulum tortor nisl, sit amet adipiscing turpis posuere vel. Morbi eget feugiat quam. Vestibulum eu lacinia urna. Maecenas eget consectetur ante.
\pend
\pstarttikz[grow=right,sibling distance=4cm,
                      edge from parent/.style={draw,thick}]
   \begin{scope}[xshift=-.42\textwidth]
   \node(A1)[align=left]{Prova}
        child {node(B1)[align=left] {Second line - \edtexttikz{left}{\Afootnote{gauche}}{B1} } }
        child {node(C1) {Second line - right}
            child {node  { Third line - left}}
            child {node {Third line }
              child[level distance=2cm] {node(D1){Fourth line}}
              }
        };
   \end{scope}
   
\pendtikz[D1,C1,A1,B1]
\pstart
Lorem ipsum dolor sit amet, consectetur adipiscing elit. Sed et ligula eros. Proin consequat ligula non lacus adipiscing, quis ornare leo tincidunt. Quisque quis molestie felis. Aliquam sapien quam, ultrices et magna a, dapibus iaculis dui. Mauris accumsan tristique est vitae ultricies. Donec ultricies magna quam. Nam imperdiet augue porttitor, dignissim sem non, tempus nisi. Donec vehicula ipsum eget ipsum ornare tempor. Phasellus enim justo, gravida laoreet vehicula cursus, tristique in libero. Lorem ipsum dolor sit amet, consectetur adipiscing elit. Integer non faucibus purus. Phasellus et convallis nulla. Fusce vestibulum tortor nisl, sit amet adipiscing turpis posuere vel. Morbi eget feugiat quam. Vestibulum eu lacinia urna. Maecenas eget consectetur ante.
\pend[]
\pstarttikz[thick,->]

\node(O) at (0,0){};
\node(N) at (0,3){\edtexttikz{Nord}{\Afootnote{Septentrion}}{N}};
\node(S) at (0,-3){\edtexttikz{Sud}{\Afootnote{Midi}}{S}};


\node(W) at (-3,0){\edtexttikz{Ouest}{\Afootnote{Couchant}}{O}};
\node(E) at (3,0){\edtexttikz{Est}{\Afootnote{Levant}}{O}};

\draw(O.center)--(N);
\draw(O.center)--(S);
\draw(O.center)--(E);
\draw(O.center)--(W);

\pendtikz[N,O,S]

\endnumbering


\end{document}