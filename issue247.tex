\documentclass{article}
\usepackage{xunicode,fontspec}
\makeatletter
\usepackage{eledmac}
\newinsert\Alongfootins

\newcommand*{\footlongparagraph}[1]{%
  \footparagraph{#1}%
  \csgdef{series@display#1}{longparagraph}%
  \count\csname #1footins\endcsname=0%
  \csxdef{default@#1footins}{0}%Use this to confine the  notes to one side only
  \count\csname #1longfootins\endcsname=1000%
  \dimen\csname #1longfootins\endcsname=\csuse{maxhXnotes@#1}%
  \skip\csname #1longfootins\endcsname=\csuse{beforeXnotes@#1}%
  \expandafter\let\csname v#1footnote\endcsname=\longpara@vfootnote
  \expandafter\let\csname #1footgroup\endcsname=\longpara@footgroup
}

\newcommand*{\longpara@vfootnote}[2]{%
    \csuse{bhookXnote@#1}
    \csuse{Xnotefontsize@#1}
    \footsplitskips
    \setbox0=\vbox{\hsize=\maxdimen
      \noindent\csname #1footfmt\endcsname#2[#1]}%
    \setbox0=\hbox{\unvxh0[#1]}%
    \dp0=0pt
    \ht0=\csname #1footfudgefactor\endcsname\wd0
  \insert\csname #1footins\endcsname
    \bgroup
      \if@RTL\noindent \leavevmode\fi\box0\penalty0
    \egroup
  \insert\csname #1longfootins\endcsname
    \bgroup
      \if@RTL\noindent \leavevmode\fi\unhbox0\penalty0
    \egroup
}


\newcommand*{\longpara@footgroup}[1]{%
  sss\unvbox\csname #1footins\endcsname
  \ifcsstring{Xragged@#1}{L}{\RaggedLeft}{}%
  \ifcsstring{Xragged@#1}{R}{\RaggedRight}{}%
  \makehboxofhboxes
  \setbox0=\hbox{{\csuse{Xnotefontsize@#1}\csuse{txtbeforeXnotes@#1}}\unhbox0 \removehboxes}%
  \csuse{Xnotefontsize@#1}
  \noindent\unhbox0\par%
  \global\hsize=\old@hsize%
  }%

\newcounter{notes}
\newcommand{\notes}{\textbf{\thenotes}}
\makeatother



\begin{document}
\footlongparagraph{A}
\beginnumbering
\pstart
\edtext{a}{\Afootnote{Muse, chante, et dis-nous la funeste colère
D'Achille, du héros dont Pelée est le père :
Aux Grecs elle causa des malheurs infinis,
Trancha de leurs guerriers les jours inaccomplis,
Et leurs corps, par milliers, privés de sépulture,
Des vautours et des chiens devinrent la pâture.
Ainsi s'accomplissait la volonté des Cieux,
Depuis qu'au camp des Grecs, un conflit malheureux
Vint diviser Achille et le chef de l'armée.
Mais par qui fut; entre eux, la discorde allumée ?
Quel en fut le motif ? Ce fut par Apollon :
De son prêtre Chrysès voulant venger l'affront,
Il lança sur les Grecs une peste cruelle ;
Du poison de ses traits l'atteinte était mortelle ;
Tous périssaient : c'est que le prêtre d'Apollon,
De Chryséis sa fille apportant la rançon,
Vint supplier les Grecs, et, du fils de Latone,
Il portait, avec lui, le sceptre et la couronne.
Aux Atrides surtout il adressait ses vœux :
« Puissiez-vous, leur dit-il, favorisés des Cieux,
Subjuguer les Troyens dans leur ville asservie
Et rentrant glorieux, revoir votre patrie !
Mais rendez-moi ma fille, acceptez sa rançon,
Et respectez en moi le prêtre d'Apollon ! »
Les Grecs auraient voulu qu'exauçant sa prière,
Agamemnon rendît Chryséis à son père,
Et reçût ses présents ; mais Atride irrité,
Apostrophant Chrysès d'un ton plein de fierté,
Le repousse, et joignant la menace à l'outrage :
« Fuis, vieillard, fuis, dit-il, fuis loin de ce rivage,
Et n'y reviens jamais ; précipite tes pas !
Le sceptre d'Apollon ne te sauverait pas ;
Je garderai pour moi ta fille, que tu pleures :
Elle ornera mon lit, dans mes riches demeures,
Tant qu'elle sera jeune, et loin de son pays,
Consacrera ses jours à tisser mes tapis. »
Chrysès est interdit, et ce malheureux père,
Avec effroi cédant à cet ordre sévère,
S'éloigne, et l’on entend ses sanglots et ses vœux,
Parmi le bruit confus des flots tumultueux;
Puis, s'adressant au Dieu dont il tient la couronne :
« Je t'invoque, dit-il, ô toi, fils de Latone,
Qu'adoré Ténédos ; toi dont le bras puissant
Arme de traits mortels ton arc retentissant :
Si mon zèle, à Chrysa, sut t'élever un temple,
À tes adorateurs si j'ai servi d'exemple,
Ecoute ma prière, et sensible à mes pleurs,
Venge-moi ; fais aux Grecs expier mes douleurs. »
Apollon, qui l'entend, exauce sa prière :
Des hauteurs de l'Olympe il descend vers la terre,
S'entoure d'un nuage, et dès qu'il est parti,
Du carquois agité les traits ont retenti.
Assis près des vaisseaux, d'une main invisible,
Il lance sur les Grecs une flèche terrible.
Les mulets et les chiens d'abord furent frappés ;
Mais bientôt, sous ses coups, les Grecs enveloppés
Furent atteints; la peste étendit son ravage,
Et de nombreux bûchers éclairaient le rivage.
Neuf jours, Phébus lança ses traits pernicieux.
Le dixième jour, Junon, du haut des cieux,
Junon, à qui des Grecs la nation est chère,
De Phébus-Apollon suspendit la colère.
Elle se sert d'Achille, et par elle inspiré,
Il assembla les Grecs dès le jour expiré.
Quand de tous les guerriers l'élite est assemblée,
Achille ainsi parla, devant cette assemblée :
« Aux combats si le Ciel joint la contagion,
Nous ne serons jamais les maîtres d'Ilion ;
Et quand il nous oppose un invincible obstacle,
Sans doute il nous convient d'invoquer un oracle.
Que, sans retard, par nous, un devin consulté
Nous explique pourquoi le Ciel s'est irrité,
Et pourquoi sous ses coups des Grecs le camp succombe.
Se plaindrait-il de nous ? Veut-il une hécatombe ? »
Achille alors s'assied, et Calchas s'est dressé :
Il connaît l'avenir, le présent, le passé.
Apollon fit de lui le premier des augures,
Et les choses qu'il dit sont toujours les plus sûres.
Se tournant vers Achille, il lui dit : « Tu veux donc
Connaître le motif du courroux d'Apollon :
Eh bien ! je le dirai ; mais fais-moi la promesse
Que quand j'aurai porté l'oracle qui me presse,
En tout temps je pourrai compter sur ton secours,
Car je comprends ici le danger que je cours :
D'Atride quand je viens contrister l’âme altière,
Je le sais, quand d'un roi l'on brave la colore,
Il sait dissimuler sa haine et son courroux,
Mais l'avenir prochain  lui réserve ses coups.
Ma voix, puisqu'il le faut, va donc se faire entendre,
Mais promets qu'au besoin tu sauras me défendre,
Et si de quelque chef j'excite le courroux,
Puis-je compter sur toi pour repousser ses coups ? »
}}
\edtext{a}{\Bfootnote{Muse, chante, et dis-nous la funeste colère
D'Achille, du héros dont Pelée est le père :
Aux Grecs elle causa des malheurs infinis,
Trancha de leurs guerriers les jours inaccomplis,
Et leurs corps, par milliers, privés de sépulture,
Des vautours et des chiens devinrent la pâture.
Ainsi s'accomplissait la volonté des Cieux,
Depuis qu'au camp des Grecs, un conflit malheureux
Vint diviser Achille et le chef de l'armée.
Mais par qui fut; entre eux, la discorde allumée ?
Quel en fut le motif ? Ce fut par Apollon :
De son prêtre Chrysès voulant venger l'affront,
Il lança sur les Grecs une peste cruelle ;
Du poison de ses traits l'atteinte était mortelle ;
Tous périssaient : c'est que le prêtre d'Apollon,
De Chryséis sa fille apportant la rançon,
Vint supplier les Grecs, et, du fils de Latone,
Il portait, avec lui, le sceptre et la couronne.
Aux Atrides surtout il adressait ses vœux :
« Puissiez-vous, leur dit-il, favorisés des Cieux,
Subjuguer les Troyens dans leur ville asservie
Et rentrant glorieux, revoir votre patrie !
Mais rendez-moi ma fille, acceptez sa rançon,
Et respectez en moi le prêtre d'Apollon ! »
Les Grecs auraient voulu qu'exauçant sa prière,
Agamemnon rendît Chryséis à son père,
Et reçût ses présents ; mais Atride irrité,
Apostrophant Chrysès d'un ton plein de fierté,
Le repousse, et joignant la menace à l'outrage :
« Fuis, vieillard, fuis, dit-il, fuis loin de ce rivage,
Et n'y reviens jamais ; précipite tes pas !
Le sceptre d'Apollon ne te sauverait pas ;
Je garderai pour moi ta fille, que tu pleures :
Elle ornera mon lit, dans mes riches demeures,
Tant qu'elle sera jeune, et loin de son pays,
Consacrera ses jours à tisser mes tapis. »
Chrysès est interdit, et ce malheureux père,
Avec effroi cédant à cet ordre sévère,
S'éloigne, et l’on entend ses sanglots et ses vœux,
Parmi le bruit confus des flots tumultueux;
Puis, s'adressant au Dieu dont il tient la couronne :
« Je t'invoque, dit-il, ô toi, fils de Latone,
Qu'adoré Ténédos ; toi dont le bras puissant
Arme de traits mortels ton arc retentissant :
Si mon zèle, à Chrysa, sut t'élever un temple,
À tes adorateurs si j'ai servi d'exemple,
Ecoute ma prière, et sensible à mes pleurs,
Venge-moi ; fais aux Grecs expier mes douleurs. »
Apollon, qui l'entend, exauce sa prière :
Des hauteurs de l'Olympe il descend vers la terre,
S'entoure d'un nuage, et dès qu'il est parti,
Du carquois agité les traits ont retenti.
Assis près des vaisseaux, d'une main invisible,
Il lance sur les Grecs une flèche terrible.
Les mulets et les chiens d'abord furent frappés ;
Mais bientôt, sous ses coups, les Grecs enveloppés
Furent atteints; la peste étendit son ravage,
Et de nombreux bûchers éclairaient le rivage.
Neuf jours, Phébus lança ses traits pernicieux.
Le dixième jour, Junon, du haut des cieux,
Junon, à qui des Grecs la nation est chère,
De Phébus-Apollon suspendit la colère.
Elle se sert d'Achille, et par elle inspiré,
Il assembla les Grecs dès le jour expiré.
Quand de tous les guerriers l'élite est assemblée,
Achille ainsi parla, devant cette assemblée :
« Aux combats si le Ciel joint la contagion,
Nous ne serons jamais les maîtres d'Ilion ;
Et quand il nous oppose un invincible obstacle,
Sans doute il nous convient d'invoquer un oracle.
Que, sans retard, par nous, un devin consulté
Nous explique pourquoi le Ciel s'est irrité,
Et pourquoi sous ses coups des Grecs le camp succombe.
Se plaindrait-il de nous ? Veut-il une hécatombe ? »
Achille alors s'assied, et Calchas s'est dressé :
Il connaît l'avenir, le présent, le passé.
Apollon fit de lui le premier des augures,
Et les choses qu'il dit sont toujours les plus sûres.
Se tournant vers Achille, il lui dit : « Tu veux donc
Connaître le motif du courroux d'Apollon :
Eh bien ! je le dirai ; mais fais-moi la promesse
Que quand j'aurai porté l'oracle qui me presse,
En tout temps je pourrai compter sur ton secours,
Car je comprends ici le danger que je cours :
D'Atride quand je viens contrister l’âme altière,
Je le sais, quand d'un roi l'on brave la colore,
Il sait dissimuler sa haine et son courroux,
Mais l'avenir prochain  lui réserve ses coups.
Ma voix, puisqu'il le faut, va donc se faire entendre,
Mais promets qu'au besoin tu sauras me défendre,
Et si de quelque chef j'excite le courroux,
Puis-je compter sur toi pour repousser ses coups ? »
}}

\pend
\endnumbering

\end{document}
