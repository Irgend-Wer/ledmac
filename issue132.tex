\documentclass[11pt]{article} 
\title{Turkish Embassy Letters} 
\author{ed. Rebecca Chung} 
\usepackage{times, eledmac, ulem, verse, setspace} 
\usepackage[protrusion=true,expansion=true]{microtype} 
\renewcommand{\rmdefault}{ptm} 

\newcommand{\A}{\Afootnote} 
\newcommand{\B}{\Bfootnote} 
\newcommand{\edt}{\edtext} 
\newcommand{\I}{\textit} 
\newcommand{\U}{\uline} 
\newcommand{\OED}{(\I{OED})} 
\newcommand{\CL}{\I{CL}} 
\newcommand{\CT}{(comma tail removed) H} 
\newcommand{\PE}{(paragraph ends) } 
\newcommand{\D}{\ldots} 

\setcounter{page}{8} 
\setcounter{firstlinenum}{0} 
\setlength\parindent{11pt} 
\renewcommand*{\thefootnoteA}{\alph{footnoteA}} 
\renewcommand{\notenumfont}{\bfseries\footnotesize} 
\renewcommand*{\thefootnoteB}{\alph{footnoteB}} 
\renewcommand{\notenumfont}{\bfseries\footnotesize} 
\footparagraph{A} 
\footparagraphX{B} 
\frenchspacing 
\noendnotes 

\begin{document} 
\beginnumbering 
\pstart 


\begin{center} 
\edt{Letter 1}{\A{1 H; LETTER I (Roman numerals throughout) F}} 
\linebreak\linebreak 
To the \edt{Countess of ------}{\B{Montagu's younger sister, Lady Frances 
Pierrepont (1690--1761) married John Erskine (1675--1732) 6th Earl of Mar 
in 1714. One year later, Mar led the failed 1715 Jacobite rebellion, 
which tried to depose George I (1660--1727), and restore exiled prince 
James (Jacobus) Stuart (1688--1766). Mar fled to France and was trying to 
raise another army. Lady Mar's interests were protected by Montagu 
(1689--1762) and her husband Edward Wortley Montagu (1678--1761), as well 
as by father Evelyn Pierrepont (1655--1726), Duke of Kingston-upon-Hull, 
Lord Privy Seal (keeper of the royal seal for personal signatures). By 
1728, Lady Mar was mentally ill. Montagu addresses Lady Mar in Letters 1, 
7, 9, 14, 15, 16, 19, 21, 23, 29, 33, 39, 45, and 46; see also Letter 50, 
l.7n. and 13n., for excerpts from Lady Mar's own letters to her 
husband.}}\raisebox{-1pt}{.}\linebreak\linebreak 
\end{center} 


\begin{flushright} 
\edt{\edt{\I{Rotterdam}}{\B{a prosperous Dutch port.}}, Friday, \edt{Aug. 
3}{\B{Plans kept changing. \I{The Flying Post or The Post Master}, 1716 
Apr. 14: ``Mr. Wortley Montague, late one of the Lords of the Treasury, is 
going Ambassador to Constantinople, in the Room of Colonel Sutton, who has 
obtain'd Leave to come Home." Jul. 3, office circular: ``Mr. Montague, 
being appointed to succeed Sir Robert Sutton as his Majesty's ambassador at 
the Ottoman Port, a man-of-war is ordered to convey him to Constantinople" 
(Polwarth, 1:36). Jul. 7: ``The honourable Edward Wortley, alias Montague, 
will set forward this Week on his Embassy to Constantinople"; ``Edward 
Wortley, alias Montague, Esq; Ambassador Extraordinary to the Ottoman 
Porte, sets out in a few Days for Constantinople, to relieve Sir Robert 
Sutton" (\I{Shift Shifted}; \I{Weekly Journal or British Gazetteer}). Jul. 
27, John Robethon (d. 1722, secretary to George I) to Alexander Hume 
Campbell, Lord Polwarth (1675--1740, ambassador-extraordinary to 
Copenhagen): ``The Danish ambassador left in one of the King's yachts for 
Holland on his way to Hanover this morning, and at the same time Mr. 
Wortley Montagu left in another yacht also for Holland, whence he proceeds 
by land to Constantinople" (\I{Polwarth}, 1:47). Jul. 28: ``Edward-Wortley 
Montague, Esq; set out Yesterday Morning on his Embassy to Constantinople" 
(\I{Weekly Packet}). Jul. 31, office circular enclosure: ``Mr. Wortley 
Montague, his Majesty's ambassador extraordinary to Constantinople, sets 
out to-morrow to begin his journey thither" (\I{Polwarth},1:49). Aug. 3, 
office circular: ``Last Wednesday being the anniversary of his Majesty's 
happy accession to the throne \ldots.The same day Mr. Wortley Montagu set 
out for his embassy \ldots " (\I{Polwarth}, 1:52). A man-of-war was the 
largest and most-heavily armed sailing ship in the British navy; for yacht, 
see l.8n.}}. \edt{O.S.}{\B{To convert from old-style British dates to 
new-style (European) Continental dates, add eleven days to the British 
date. Where needed for quick referencing, both dates are supplied [using 
brackets].}} 1716.}{\lemma{\I{Rotterdam}\D1716.}\A{Rotterdam, Friday 
Aug{\textsuperscript{t}} 3. O.S. H}}\end{flushright} 



\hspace{-11pt}I flatter myself (dear sister) that I shall give you some 
pleasure in letting you know that I \edt{am}{\A{have F}} safely passed the 
sea, though we had the ill fortune of a storm. We were persuaded by the 
captain of \edt{our}{\A{the F; our P}} \edt{yacht}{\A{yacht 1751; Yatcht 
HP; yatcht F}\B{In Dutch, \I{Jacht} means \I{hunt}; the Dutch used the 
light, fast-sailing yacht to combat pirates. In England, the yacht became 
a leisure ship after Charles II (1630--1685) returned from exile in one.}} 
to set out in a \edt{calm}{\B{absence of wind.}}, and he pretended there 
was nothing so easy as to tide it \edt{over;}{\A{over, L4}} 
\edt{but}{\A{but, F}} after two days slowly moving, the wind blew so 
\edt{hard}{\A{hard, F}} that none of the sailors could keep their feet, and 
we were all Sunday night tossed very handsomely. 

\edt{You}{\A{you H; You F L4}} see I have already \edt{learnt}{\A{learn't 
F}} to make a good bargain, and that it is not for 
\edt{nothing}{\A{nothing, P}} I will so much as tell \edt{you that}{\A{you, 
F: You P}} I \edt{am}{\A{am, Your affectionate sister. F; am------Your 
Affectionate Sister.------ P; am Your Affectionate Sister. L4}}\ 
\newline 
\begin{flushright} 
\edt{Your affectionate sister.}{\lemma{Your\D 
sister}\A{Y\textsuperscript{r} Affectionate Sister (right justified) 
(single rule follows) H; Your affectionate sister. F; ------Your 
Affectionate Sister.------P}}\end{flushright} 


\pend 
\endnumbering 
\end{document} 