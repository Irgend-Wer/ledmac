\documentclass[11pt,a4paper]{book}
\usepackage[utf8x]{inputenc}
\usepackage[T1]{fontenc}
\usepackage{typearea}
\usepackage[ngerman,latin]{babel}

%\renewcommand*{\chapterheadendvskip}{\vspace{0.725\baselineskip plus 0.115\baselineskip minus 0.192\baselineskip}}

\usepackage{imakeidx} 
\usepackage{eledmac}
\usepackage{eledpar}

\footparagraph{A}
\footparagraph{B}
\footparagraph{C}
%\makeindex[title=Index,columns=2]
\newcommand{\AAp}[4]{\edtext{#1}{\lemma{#2}\Afootnote{#3 \textit{#4}}}}
\newcommand{\BAp}[4]{\edtext{#1}{\lemma{#2}\Bfootnote{#3 \textit{#4}}}}
\newcommand{\CAp}[4]{\edtext{#1}{\lemma{#2}\Cfootnote{#3 \textit{#4}}}}

\makeindex[title=Allgemeinindex,columns=2]{}
\makeindex[name=persons, title=Index of names,columns=2]{}

\begin{document}
\begin{pages}
\begin{Leftside}
\beginnumbering
\pstart
\eledchapter*{blabla}
\pend\pstart
Manubiae Iovis\edindex[persons]{Manubiae Iovis} tres creduntur esse, quarum unae sint minimae, quae moneant placataeque sint. Alterae quae maiores\edindex{maiores} sint, ac veniant cum fragore, discutiantque aut divellant, quae a Iove \AAp{sint}{sint}{}{lacuna post sint ind. Porson\edindex{lacuna}}, et consilio deorum mitti existimentur.
\pend

\endnumbering
\end{Leftside}
\begin{Rightside}
\beginnumbering
\pstart
\eledchapter*{bla}
\pend\pstart
Maeson\edindex[persons]{Maeson} persona comica appellatur, aut coci, aut nautae, aut eius generis. Dici ab inventore eius Maesone comoedo\edindex[persons]{Maeson comoedus} , ut ait Aristophanes Grammaticus\edindex[persons]{Aristophanes Gram.}
\pend
\endnumbering
\end{Rightside}
\end{pages}
\Pages

%\printindex
%\printindex[persons]
\end{document}
