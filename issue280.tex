

% À compiler avec XeLaTeX
\documentclass{article}

\usepackage{fontspec}
\setmainfont{Linux Libertine O}
\usepackage{polyglossia}
\setmainlanguage{french}
\usepackage{setspace}
\onehalfspacing
\usepackage[xindy]{indextools}%On précise qu'on veut utiliser xindy


% Personnaliser l'apparence de lettergroup
\newcommand{\lettergroup}[1]{%
  {\centering\large---\emph{#1}---\par}%
}
\apptocmd{\subitem}{$\rightarrow$\,}{}{}
\usepackage[series={A},noeledsec,nofamiliar,noend,xindy,xindy+hyperref]{eledmac}
\usepackage[hyperindex=false]{hyperref}
\makeindex[name=animal,title=Animaux,columnseprule]
\begin{document}

\beginnumbering
\autopar

On va parler de 
chats sauvages\edindex[animal]{chat!sauvage|textbf} et domestiques\edindex[animal]{chat!domestique},
de tortues marines\edindex[animal]{tortue!marine} et terrestres\edindex[animal]{tortue!terrestre},
d'élèphant d'Asie\edindex[animal]{élèphant!d'Asie} et d'Afrique\edindex[animal]{élèphant!d'Afrique},
et soyons fou, d'élans de Sibérie\edindex[animal]{élan!de Sibérie} et d'Amérique\edindex[animal]{élan!d'Amérique}.


On va parler au paragraphe suivant des même animaux. C'est à dire : de 
chats sauvages\edindex[animal]{chat!sauvage} et domestiques\edindex[animal]{chat!domestique},
de tortues \edtext{marines}{\Afootnote{ninja\edindex[animal]{tortue!ninja|textbf}}}\edindex[animal]{tortue!marine} et terrestres\edindex[animal]{tortue!terrestre},
d'élèphant d'Asie\edindex[animal]{élèphant!d'Asie} et d'Afrique\edindex[animal]{élèphant!d'Afrique},
et soyons fou, d'élans de Sibérie\edindex[animal]{élan!de Sibérie} et d'Amérique\edindex[animal]{élan!d'Amérique}.

\endnumbering
\newpage
\beginnumbering

\autopar Et pour la bonne cause, on va aussi parler de chats sauvages\edindex[animal]{chat!sauvage|textbf} qui se disputent avec les chiens errants\edindex[animal]{chien!errant}.


\endnumbering

\printindex[animal]
\end{document}
