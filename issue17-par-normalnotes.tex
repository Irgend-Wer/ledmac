\documentclass{article}
\usepackage{fontspec,xunicode}
\usepackage{eledmac}
\usepackage[shiftedpstarts]{eledpar}

\usepackage{polyglossia}
\setmainlanguage{latin}
\setotherlanguage{french}
\renewcommand*{\goalfraction}{0.95}
%http://remacle.org/bloodwolf/historiens/dictys/troie.htm
\onlyXside[B]{L}
\onlyXside[C]{R}
\begin{document}
\maxhXnotes{0.25\textheight}
\begin{pages}

\begin{Leftside}
\begin{french}
\beginnumbering
\autopar

\edtext{Tous}{\Afootnote{Afootnote is for both left and right page. Afootnote is for both left and right page. Afootnote is for both left and right page. Afootnote is for both left and right page. Afootnote is for both left and right page. Afootnote is for both left and right page. Afootnote is for both left and right page. Afootnote is for both left and right page. Afootnote is for both left and right page. Afootnote is for both left and right page. Afootnote is for both left and right page. Afootnote is for both left and right page. Afootnote is for both left and right page. Afootnote is for both left and right page. Afootnote is for both left and right page. Afootnote is for both left and right page. Afootnote is for both left and right page. Afootnote is for both left and right page. Afootnote is for both left and right page. Afootnote is for both left and right page. Afootnote is for both left and right page. Afootnote is for both left and right page. Afootnote is for both left and right page. Afootnote is for both left and right page. Afootnote is for both left and right page. Afootnote is for both left and right page. Afootnote is for both left and right page. Afootnote is for both left and right page. Afootnote is for both left and right page. Afootnote is for both left and right page. Afootnote is for both left and right page. Afootnote is for both left and right page. Afootnote is for both left and right page. Afootnote is for both left and right page. Afootnote is for both left and right page. Afootnote is for both left and right page. Afootnote is for both left and right page. Afootnote is for both left and right page. Afootnote is for both left and right page. Afootnote is for both left and right page. Afootnote is for both left and right page. Afootnote is for both left and right page. Afootnote is for both left and right page. Afootnote is for both left and right page. Afootnote is for both left and right page. Afootnote is for both left and right page. Afootnote is for both left and right page. Afootnote is for both left and right page. Afootnote is for both left and right page. Afootnote is for both left and right page. Afootnote is for both left and right page. Afootnote is for both left and right page. Afootnote is for both left and right page. Afootnote is for both left and right page. Afootnote is for both left and right page. Afootnote is for both left and right page. Afootnote is for both left and right page. Afootnote is for both left and right page. Afootnote is for both left and right page. Afootnote is for both left and right page. Afootnote is for both left and right page. Afootnote is for both left and right page. Afootnote is for both left and right page. Afootnote is for both left and right page. Afootnote is for both left and right page. Afootnote is for both left and right page. Afootnote is for both left and right page. Afootnote is for both left and right page. Afootnote is for both left and right page. Afootnote is for both left and right page.}}
les 
\edtext{rois}{\Bfootnote{Bfootnote is only for left side. Bfootnote is only for left side. Bfootnote is only for left side. Bfootnote is only for left side. Bfootnote is only for left side. Bfootnote is only for left side. Bfootnote is only for left side. Bfootnote is only for left side. Bfootnote is only for left side. Bfootnote is only for left side. Bfootnote is only for left side. Bfootnote is only for left side. Bfootnote is only for left side. Bfootnote is only for left side. Bfootnote is only for left side. Bfootnote is only for left side. Bfootnote is only for left side. Bfootnote is only for left side. Bfootnote is only for left side. Bfootnote is only for left side. Bfootnote is only for left side. Bfootnote is only for left side. Bfootnote is only for left side. Bfootnote is only for left side. Bfootnote is only for left side. Bfootnote is only for left side. Bfootnote is only for left side. Bfootnote is only for left side. Bfootnote is only for left side. Bfootnote is only for left side. Bfootnote is only for left side. Bfootnote is only for left side. Bfootnote is only for left side. Bfootnote is only for left side. Bfootnote is only for left side. Bfootnote is only for left side. Bfootnote is only for left side. Bfootnote is only for left side. Bfootnote is only for left side. Bfootnote is only for left side. Bfootnote is only for left side. Bfootnote is only for left side. Bfootnote is only for left side. Bfootnote is only for left side. Bfootnote is only for left side. Bfootnote is only for left side. Bfootnote is only for left side. Bfootnote is only for left side. Bfootnote is only for left side. Bfootnote is only for left side. Bfootnote is only for left side. Bfootnote is only for left side. Bfootnote is only for left side. Bfootnote is only for left side. Bfootnote is only for left side. Bfootnote is only for left side. Bfootnote is only for left side. Bfootnote is only for left side. Bfootnote is only for left side. Bfootnote is only for left side. Bfootnote is only for left side. Bfootnote is only for left side. Bfootnote is only for left side. Bfootnote is only for left side. Bfootnote is only for left side. Bfootnote is only for left side. Bfootnote is only for left side. Bfootnote is only for left side. Bfootnote is only for left side. Bfootnote is only for left side. Bfootnote is only for left side. Bfootnote is only for left side. Bfootnote is only for left side. Bfootnote is only for left side. Bfootnote is only for left side. Bfootnote is only for left side. Bfootnote is only for left side. Bfootnote is only for left side. Bfootnote is only for left side. Bfootnote is only for left side. Bfootnote is only for left side. Bfootnote is only for left side. Bfootnote is only for left side. Bfootnote is only for left side. Bfootnote is only for left side. Bfootnote is only for left side. Bfootnote is only for left side. Bfootnote is only for left side. Bfootnote is only for left side. Bfootnote is only for left side. Bfootnote is only for left side. Bfootnote is only for left side. Bfootnote is only for left side. Bfootnote is only for left side. Bfootnote is only for left side. Bfootnote is only for left side. Bfootnote is only for left side. Bfootnote is only for left side. Bfootnote is only for left side. Bfootnote is only for left side. Bfootnote is only for left side. Bfootnote is only for left side. Bfootnote is only for left side. Bfootnote is only for left side. Bfootnote is only for left side. Bfootnote is only for left side. Bfootnote is only for left side. Bfootnote is only for left side. Bfootnote is only for left side. Bfootnote is only for left side. Bfootnote is only for left side. Bfootnote is only for left side. Bfootnote is only for left side. Bfootnote is only for left side. Bfootnote is only for left side. Bfootnote is only for left side. Bfootnote is only for left side. Bfootnote is only for left side. Bfootnote is only for left side. Bfootnote is only for left side.
}} de la Grèce qui descendaient de Minos, fils de Jupiter, vinrent en Crète pour y recueillir la riche succession de Crétéus. Ce prince, fils de Minos, avait réglé par son testament qu'il serait fait un partage égal de tout ce qu'il possédait d'or, d'argent et de troupeaux, entre les enfants de ses filles ; et il laissait son empire à Idoménée, fils de Deucalion, son frère et à Mérion , fils de Molus, son neveu, qui devaient gouverner chacun sa part avec un pouvoir indépendant. Entre les princes présents au partage, on distinguait Palamède, fils de Clymène et de Nauplius, et Oeax, appelés Crétéides, avec Ménélas, fils d'Aerope et de Plisthène, qu'Anaxibie, sa soeur, épouse de Nestor, et Agamemnon, son frère aîné, avaient chargé de les représenter dans l'assemblée des héritiers. On connaissait moins ces derniers comme fils de Plisthène, mort à la fleur de son âge et sans avoir rien fait de mémorable, que comme petits-fils d'Atrée. Ce prince, en effet, touché de compassion pour la faiblesse de leur âge, les avait recueillis au-près de lui, et s'était chargé de leur donner une éducation conforme à leur naissance. Ils se conduisirent tous dans cette occasion avec la grandeur et la générosité qu'on devait attendre de personnes de leur rang.

A la nouvelle de leur arrivée, tons les descendants d'Europe, dont le nom était en grande vénération dans l'île, se rendirent auprès d'eux, les saluèrent avec bonté et les conduisirent au temple. Là, après un sacrifice solennel où furent immolées suivant l'usage, nombre de victimes on leur servit un repas splendide, et on les traita avec autant d'abondance que de délicatesse. Les fêtes continuèrent les jours suivants. Les rois reçurent les témoignages de l'affection de leurs amis avec joie et reconnaissance; mais ils furent encore plus frappés de la magnificence du temple d'Europe. Ils ne pouvaient se laisser d'examiner, dans le plus grand détail, les riches présents envoyés de Sidon à cette princesse par son père Phénicie  et par ses nobles compagnes, et qui faisaient l'ornement de ce bel édifice.

Dans le même temps, Alexandre de Phrygie, fils de Priam, accompagné d'Énée  et de plusieurs de ses parents, se rendait coupable d'un grand attentat à Sparte et dans le palais de Ménélas, où il avait été reçu comme hôte, et traité tomme ami. Aussitôt après le départ du roi, épris d'amour pour Hélène, qui surpassait en beauté toutes les femmes de la Grèce, il l'enleva, et avec elle tous les trésors qu'il put emporter. Cette princesse fut accompagnée dans sa fuite par Aetra et Clymène, parentes de Ménélas, attachées à son service. La nouvelle du crime commis par Alexandre contre la maison de Ménélas parvint bientôt en Crète; et la renommée, qui se plaît ordinairement à grossir les objets, publia que le palais du roi avait été détruit, son empire renversé, et répandit d'autres bruits aussi funestes.

Ménélas, à cette nouvelle, quoique vive. meut affecté de l'enlèvement de son épouse, fut encore plus irrité de la connivence perfide qu'il crut apercevoir entre le ravisseur et ses parentes. Palamède, voyant ce prince indigné et furieux sortir du conseil sans proférer un seul mot, fait approcher de terre les vaisseaux et dispose tout pour le départ. Après quelques paroles consolantes adressées au roi, il embarque à la hâte tout ce qui provenait du partage, fait monter Ménélas avec lui sur la flotte, et, secondés d'un vent favorable, ils arrivent en peu de jours à Sparte. Déjà Agamemnon, Nestor , et tous les rois descendants de Pélops, y étaient accourus. A l'arrivée de Ménélas, ils s'assemblent ; et quoique l'atrocité de l'action leur inspiràt une profonde horreur et les portât â une prompte vengeance, cependant, après avoir délibéré mûrement, ils résolurent d'envoyer d'abord à Troie, en qualité de députés, Palamède, Ulysse et Ménélas, avec ordre de se plaindre de l'injure, et de redemander Hélène ainsi que tous les trésors enlevés.

Les députés arrivèrent bientôt à Troie et n'y trouvèrent point Alexandre. Ce prince qui, dans sa fuite précipitée, avait peu consulté les vents, s'était vu forcé de relâcher en Chypre. De là, après s'être saisi de quelques vaisseaux, il avait abordé sur la côte de Phénicie. Toujours tourmenté par cette même avidité qui l'avait accompagné à Sparte, il égorge de nuit, par trahison, le roi des Sidoniens, qui lui avait fait un accueil favorable. Tout ce que renferme le palais est le prix de son crime; toutes les richesses accumulées dans ce lieu, monuments de la grandeur royale, sont par son ordre injustement enlevées et portées sur ses vaisseaux. Cependant, aux cris lamentables de ceux qui avaient échappé aux ravisseurs, le peuple se soulève, se porte en foule au palais, et, dans le moment où Alexandre, après avoir pris tout ce qui était à sa convenance, se préparait à mettre à la voile, une troupe, armée à la hâte, se présente; le combat s'engage et se poursuit avec acharnement; nombre de combattants tombent de part et d'autre; les uns s'opiniâtrent à venger la mort de leur roi, les autres à conserver leur butin. Enfin les Troyens, après avoir eu deux de leurs vaisseaux brûlés, furent assez heureux pour sauver le reste, et échappèrent ainsi à la vengeance des Sidoniens déjà fatigués du carnage.

\endnumbering
\end{french}
\end{Leftside}

\begin{Rightside}
\beginnumbering
\autopar

\edtext{Cuncti}{\Cfootnote{%
Cfootnote is only for right side. Cfootnote is only for right side. Cfootnote is only for right side. Cfootnote is only for right side. Cfootnote is only for right side. Cfootnote is only for right side. Cfootnote is only for right side. Cfootnote is only for right side. Cfootnote is only for right side. Cfootnote is only for right side. Cfootnote is only for right side. Cfootnote is only for right side. Cfootnote is only for right side. Cfootnote is only for right side. Cfootnote is only for right side. Cfootnote is only for right side. Cfootnote is only for right side. Cfootnote is only for right side. Cfootnote is only for right side. Cfootnote is only for right side. Cfootnote is only for right side. Cfootnote is only for right side. Cfootnote is only for right side. Cfootnote is only for right side. Cfootnote is only for right side. Cfootnote is only for right side. Cfootnote is only for right side. Cfootnote is only for right side. Cfootnote is only for right side. Cfootnote is only for right side. Cfootnote is only for right side. Cfootnote is only for right side. Cfootnote is only for right side. Cfootnote is only for right side. Cfootnote is only for right side. Cfootnote is only for right side. Cfootnote is only for right side. Cfootnote is only for right side. Cfootnote is only for right side. Cfootnote is only for right side. Cfootnote is only for right side. Cfootnote is only for right side. Cfootnote is only for right side. Cfootnote is only for right side. Cfootnote is only for right side. Cfootnote is only for right side. Cfootnote is only for right side. Cfootnote is only for right side. Cfootnote is only for right side. Cfootnote is only for right side. Cfootnote is only for right side. Cfootnote is only for right side. Cfootnote is only for right side. Cfootnote is only for right side. Cfootnote is only for right side. Cfootnote is only for right side. Cfootnote is only for right side. Cfootnote is only for right side. Cfootnote is only for right side. Cfootnote is only for right side. Cfootnote is only for right side. Cfootnote is only for right side. Cfootnote is only for right side. Cfootnote is only for right side. Cfootnote is only for right side. Cfootnote is only for right side. Cfootnote is only for right side. Cfootnote is only for right side. Cfootnote is only for right side. Cfootnote is only for right side. Cfootnote is only for right side. Cfootnote is only for right side. Cfootnote is only for right side. Cfootnote is only for right side. Cfootnote is only for right side. Cfootnote is only for right side. Cfootnote is only for right side. Cfootnote is only for right side. Cfootnote is only for right side. Cfootnote is only for right side. Cfootnote is only for right side. Cfootnote is only for right side. Cfootnote is only for right side. Cfootnote is only for right side. Cfootnote is only for right side. Cfootnote is only for right side. Cfootnote is only for right side. Cfootnote is only for right side. Cfootnote is only for right side. Cfootnote is only for right side. Cfootnote is only for right side. Cfootnote is only for right side. Cfootnote is only for right side. Cfootnote is only for right side. Cfootnote is only for right side. Cfootnote is only for right side. Cfootnote is only for right side. Cfootnote is only for right side. Cfootnote is only for right side. Cfootnote is only for right side. Cfootnote is only for right side. Cfootnote is only for right side. Cfootnote is only for right side. Cfootnote is only for right side. Cfootnote is only for right side. Cfootnote is only for right side. Cfootnote is only for right side. Cfootnote is only for right side. Cfootnote is only for right side. Cfootnote is only for right side. Cfootnote is only for right side. Cfootnote is only for right side. Cfootnote is only for right side. Cfootnote is only for right side. Cfootnote is only for right side. Cfootnote is only for right side. Cfootnote is only for right side. Cfootnote is only for right side. Cfootnote is only for right side. Cfootnote is only for right side.%
}} reges, qui Minois Jove geniti, pronepotes, Graeciae imperitabant, ad dividendas inter se Cretei opes, Cretam convenere; Creteus namque ex Minoe, postrema sua ordinans, quidquid auri atque argenti, pecorum etiam fuit, nepotibus, quos filiae genuerant, ex aequo dividendum reliquerat, excepto civitatum terrarumque imperio; haec quippe Idomeneus cum Merione, Deucalionis Idomeneus, alter Moli, jussu ejus seorsum habuere. Convenere autem Clymenae et Nauplii Palamedes, et Oeax, dicti Creteidae : item Menelaus, Aeropa et Plisthene genitus, a quo Anaxibia soror, quae eo tempore Nestori denupta erat, et Agamemnon major frater, ut vice sua in divisione uteretur, petiverant. Sed hi non Plisthenis, ut erant, magis quam Atrei dicebantur; ob eam causam, quod quum Plisthenes, admodum parvus ipse agens in primis annis vita functus, nihil dignum ad memoriam nominis reliquisset, Atreus miseratione aetatis secum eos habuerat, neque minus quam regios educaverat. In qua divisione singuli pro nominis celebritate inter se quisque magnifice transegere.

Ad eos re cognita omnes ex origine Europae, quae in ea insula summa religione colitur, confluunt benigneque salutatos in templum deducunt. Ibi multarum hostiarum immolatione celebrata, exhibitisque epulis, largiter magnificeque eos habuere : itemque insecutis diebus. At reges Graeciae, etsi ea quae exhibebantur cum laetitia accipiebant, tamen multo magis templi ejus magnifica pulchritudine, pretiosaque exstructione operum afficiebantur, inspicientes repetentesque memoria, singula, quae ex Sidone a Phoenice patre ejus, atque nobilibus matronis transmissa magno tum decori erant.

Per idem tempus Alexander Phrygius, Priami filius, cum Aenea aliisque ex consanguinitate comitibus, Spartae in domum Menelai hospitio receptus, indignissimum facinus perpetraverat. Is namque ubi animadvertit regem abesse, quod erat Helena praeter caeteras Graeciae faeminas miranda specie, amore ejus captus, ipsamque et multas opes domo ejus aufert, Aethram etiam et Clymenam Menelai adfines, quae ob necessitudinem cum Helena agebant. Postquam Cretam nuncius venit, et cuncta quae ab Alexandro adversus domum Menelai commissa erant, aperuit, per omnem insulam, sicut in tali re fieri amat, fama in majus divulgatur: expugnatam quippe domum regis, eversumque regnum, et alia in talem modum singuli disserebant.


Quibus cognitis Menelaus, etsi abstractio conjugis animum permoverat, multo amplius tamen ob injuriam adfinium, quas supra memoravimus, consternabatur. At ubi animadvertit Palamedes, regem ira atque indignatione stupefactum, concilio excidisse, ipso naves parat, atque omni instrumento compositas terrae applicat. Dein pro temporo regem breviter consolatus, positis etiam ex divisione, quae in tali negotio tempus patiebatur, navem ascendere facit : atque ita ventis ex sententia flantibus, paucis diebus Spartam pervenere. Eo jam Agamemnon et Nestor, omnesque qui ex origine Pelopis in Graecia regnabant, cognitis rebus confluxerant. Igitur postquam Menelaum advenisse sciunt, in unum coeunt. Et quanquam atrocitas facti ad indignationem, ultumque injurias rapiebat, tamen ex consilii sententia legantur prius ad Troiam Palamedes, Ulysses et Menelaus ; hisque mandatur, ut conquesti iniurias, Helenam, et quae cum ea abrepta erant, repeterent.

Legati paucis diebus ad Trojiam veniunt, neque tum Alexandrum in loco offendere. Eum namque properatione navigii inconsulte usum venti ad Cyprum appulere. Unde sumptis aliquot navibus, Phoenicem delapsus, Sidoniorum regem, qui eum amice susceperat, noctu per insidias necat : eademque qua apud Lacedaomonam, cupiditate, universam domum ejus in scelus proprium convertit. Ita omnia quae ad ostentationem regiae magnificentiae fuere, indigne rapta, ad naves deferri jubet. Sed ubi ex lamentatione eorum qui casum domini deflentes, reliqui praedae aufugerant, tumultus ortus est, populus omnis ad regiam concurrit. Inde quod jam Alexander, abreptis quae cupiebat, ascensionem properabat, pro tempore armati ad naves veniunt; ortoque inter eos acri proelio, cadunt utrinque plurimi, quum obstinate hi regis necem defenderent, hi ne amitterent partam praedam summis opibus adniterentur. Incensis dein duabus navibus, Trojani reliquas strenue defensas liberant, atque ita fatigatis jam proelio hostibus evadunt.

\endnumbering
\end{Rightside}

\end{pages}
\Pages
\end{document}