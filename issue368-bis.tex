\documentclass[a4paper]{book}

\usepackage{fontspec}
\setmainfont{Linux Libertine O}

\usepackage{polyglossia}
\setmainlanguage[babelshorthands=true]{italian}

\usepackage[noledgroup,noeledsec,series={A,B,C}]{reledmac}
\usepackage[widthliketwocolumns,continuousnumberingwithcolumns]{reledpar}
\setlength{\Lcolwidth}{0.47\textwidth}
\setlength{\Rcolwidth}{0.47\textwidth}
\columnsposition{c}
\setlength{\beforecolumnseparator}{0.03\textwidth}
\setlength{\aftercolumnseparator}{0.03\textwidth}
\AtBeginPairs{\sloppy}

\lineation*{page}
\linenummargin{inner}
\sidenotemargin{outer}

\firstlinenum*{1}
\linenumincrement*{1}
\makeatletter
\renewcommand*{\ledlinenum}{%
  \bgroup%
  \ifluatex%
    \textdir TLT%
  \fi%
  \numlabfont(\the\page@num)\linenumrep{\line@num}%
  \ifsublines@
    \ifnum\subline@num>0\relax
      \unskip%
      \Xsublinesep@side%
      \sublinenumrep{\subline@num}%
    \fi
  \fi%
  \egroup%
}%

\makeatother
\begin{document}

\beginnumbering
\numberpstarttrue
\labelpstarttrue

\pstart\edlabel{test}%
    \edtext{}{\Bfootnote{{\textbf{\pstartref{test}}}\enspace nel mezzo del cammin}}%
    Scholion nr. 1%
\pend

\pausenumbering
\begin{pairs}
\begin{Leftside}
\resumenumbering
\numberpstartfalse
    \pstart\noindent Q. Mucius augur multa narrare de C. Laelio socero suo memoriter et iucunde solebat nec dubitare illum in omni sermone appellare sapientem; ego autem a patre ita eram deductus ad Scaevolam sumpta virili toga, ut, quoad possem et liceret, a senis latere numquam discederem; itaque multa ab eo prudenter disputata, multa etiam breviter et commode dicta memoriae mandabam fierique studebam eius prudentia doctior. Quo mortuo me ad pontificem Scaevolam contuli, quem unum nostrae civitatis et ingenio et iustitia praestantissimum audeo dicere. Sed de hoc alias; nunc redeo ad \edtext{augurem}{\Cfootnote{riprova}}. \textbf{A}\pend
\pausenumbering
\end{Leftside}

\begin{Rightside}
\beginnumbering
\numberpstartfalse
    \pstart\noindent Quinto Mucio l'augure raccontava spesso, a memoria e in modo piacevole, molti episodi della vita di Caio Lelio, suo suocero, e in ogni discorso non esitava a chiamarlo "il Saggio". A Scevola ero stato affidato da mio padre, quando presi la toga virile, perché non mi staccassi mai dal fianco del vecchio, nei limiti del possibile e del consentito. Perciò, fissavo nella mente molti dei suoi accorti ragionamenti e anche molte delle sue massime secche e gustose, e cercavo di migliorare la mia educazione facendo tesoro della sua esperienza di vita. Quando morì, passai alla scuola di Scevola il pontefice, l'uomo che oserei definire il più grande della nostra città per intelligenza e senso di giustizia. Ne parlerò un'altra volta: ora \edtext{ritorno}{\Cfootnote{prova}} all'augure. \textbf{B}\pend{}
\pausenumbering
\end{Rightside}
\end{pairs}
\Columns

\numberpstarttrue
\resumenumbering
\pstart%
    Cum saepe multa, tum memini domi in hemicyclio sedentem, ut solebat, cum et ego essem una et pauci admodum familiares, in eum sermonem illum incidere qui tum forte multis erat in ore. Meministi enim profecto, Attice, et eo magis, quod P. Sulpicio utebare multum, cum is tribunus plebis capitali odio a Q. Pompeio, qui tum erat consul, dissideret, quocum coniunctissime et amantissime vixerat, quanta esset hominum vel admiratio vel querella.%
\pend

\pstart%
    Itaque tum Scaevola cum in eam ipsam mentionem incidisset, euit nobis sermonem Laeli de amicitia habitum ab illo secum et cum altero genero, C. Fannio Marci filio, paucis diebus post mortem Africani. Eius disputationis sententias memoriae mandavi, quas hoc libro eui arbitratu meo; quasi enim ipsos induxi loquentes, ne 'inquam' et 'inquit' saepius interponeretur, atque ut tamquam a praesentibus coram haberi sermo videretur.%
\pend

\pstart%
    Cum enim saepe mecum ageres ut de amicitia scriberem aliquid, digna mihi res cum omnium cognitione tum nostra familiaritate visa est. Itaque feci non invitus ut prodessem multis rogatu tuo. Sed ut in Catone Maiore, qui est scriptus ad te de senectute, Catonem induxi senem disputantem, quia nulla videbatur aptior persona quae de illa aetate loqueretur quam eius qui et diutissime senex fuisset et in ipsa senectute praeter ceteros floruisset, sic cum accepissemus a patribus maxime memorabilem C. Laeli et P. Scipionis familiaritatem fuisse, idonea mihi Laeli persona visa est quae de amicitia ea ipsa dissereret quae disputata ab eo meminisset Scaevola. Genus autem hoc sermonum positum in hominum veterum auctoritate, et eorum inlustrium, plus nescio quo pacto videtur habere gravitatis; itaque ipse mea legens sic afficior interdum ut Catonem, non me loqui existimem.%
\pend

\pstart%
    Cum enim saepe mecum ageres ut de amicitia scriberem aliquid, digna mihi res cum omnium cognitione tum nostra familiaritate visa est. Itaque feci non invitus ut prodessem multis rogatu tuo. Sed ut in Catone Maiore, qui est scriptus ad te de senectute, Catonem induxi senem disputantem, quia nulla videbatur aptior persona quae de illa aetate loqueretur quam eius qui et diutissime senex fuisset et in ipsa senectute praeter ceteros floruisset, sic cum accepissemus a patribus maxime memorabilem C. Laeli et P. Scipionis familiaritatem fuisse, idonea mihi Laeli persona visa est quae de amicitia ea ipsa dissereret quae disputata ab eo meminisset Scaevola. Genus autem hoc sermonum positum in hominum veterum auctoritate, et eorum inlustrium, plus nescio quo pacto videtur habere gravitatis; itaque ipse mea legens sic afficior interdum ut Catonem, non me loqui existimem.%
\pend

\pstart%
    Cum enim saepe mecum ageres ut de amicitia scriberem aliquid, digna mihi res cum omnium cognitione tum nostra familiaritate visa est. Itaque feci non invitus ut prodessem multis rogatu tuo. Sed ut in Catone Maiore, qui est scriptus ad te de senectute, Catonem induxi senem disputantem, quia nulla videbatur aptior persona quae de illa aetate loqueretur quam eius qui et diutissime senex fuisset et in ipsa senectute praeter ceteros floruisset, sic cum accepissemus a patribus maxime memorabilem C. Laeli et P. Scipionis familiaritatem fuisse, idonea mihi Laeli persona visa est quae de amicitia ea ipsa dissereret quae disputata ab eo meminisset Scaevola. Genus autem hoc sermonum positum in hominum veterum auctoritate, et eorum inlustrium, plus nescio quo pacto videtur habere gravitatis; itaque ipse mea legens sic afficior interdum ut Catonem, non me loqui existimem.%
\pend

\labelpstartfalse
\endnumbering

\end{document}
