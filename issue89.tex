\documentclass{article}

\usepackage{polyglossia}
\setmainlanguage[locale=tunisia,numerals=maghrib]{arabic}
\setotherlanguage{french}
\newfontfamily\arabicfont[Script=Arabic]{XB Zar}

\usepackage{eledmac}
\footparagraph{A}

\title{القَوْلُ فِي السَّمْعِ}
\author{بن باجة}

\begin{document}
\maketitle

\beginnumbering
\pstart
وَالقُوَّةُ السَّامِعَةُ هِيَ اسْتِعْمَالُ حَاسَّةِ السَّمْعِ وَ\edtext{بعلها}{\Afootnote{غير واضح}} ادْرَاكٌ لِلْأَمْرِ الحَادِثِ فِي الهَوَاءِ عَنْ تَصَادُمِ جِسْمَيْنِ مُتَقَاوِمَيْنِ وَهَذِهِ الحَالُ هِيَ الَّتِي \edtext{تلور}{\Afootnote{غير واضح أيضا..}} بِهَا لِلشَّيْءِ مَسْمُوعًا وَإِحْسَاسُهَا هُوَ سَمْعٌ، وَذَلِكَ أَنَّ كُلَّ الأَجْسَامِ المُحْدِثَةِ لِلصَّوْتِ إِمَّا صَلْبَةٌ وَإِمَّا \edtext{رَطْبَةٌ}{\Afootnote{عَذْبَةٌ}}. فَإِنْ كَانَتْ صَلْبَةً فَإِذَا قَرَعَهُ قَارِعٌ حَدَثَ عَنْهُ صَوْتٌ وَأَمَّا إِنْ كَانَ رَطْبًا ِإنَّهُ لَا يَحْدِثُ عَنْهُ صَوْتٌ إِلَّا بِأَنْ تَكُونَ حَرَكَةُ القَارِعِ إِلَى المَقْرُوعِ أَسْرَعَ مِنِ انْخِرَاقِ ذَلِكَ الرَّطْبِ فَيُقَاوِمُهُ فَيَتَحَرَّكُ الجِسْمُ الَّذِي فِيهِ تِلْكَ الحَرَكَةُ وَينبعها وَيَنْدَفِعُ مِنْهُ إِلَى جَمِيعِ الجِهَاتِ الَّتِي تَلِي المَكَانَ الَّذِي يَلْتَقِي فِيهِ القَارِعُ وَالمَقْرُوعُ وَالهَوَاءُ مَعَ أَنَّهُ يَنْدَفِعُ عَنِ القَارِعِ وَيقبل عن القارع له خاصّا به كَمَا يَظْهَرُ ذَلِكَ مِنَ الأَجْسَامِ المُهْتَزَّةِ، وَبَيِّنٌ أَثَرُ ذَلِكَ الحِسِّ فِي أَوْتَارِ العُودِ فَإِنَّا نَجِدُهُ مَتَى حَرَّكْنَا البَمَّ فِي تَسْوِيَةِ المُطْلَقِ تَحَرَّكَ عَلَى المَثْنَى، كَمَا يَتَحَرَّكُ مَا عَلَى الزِّيرِ وَلَا مَا عَلَى المَثْلَثِ. \edtext{وَكَذَلِكَ إِذَا}{\Afootnote{وَكَذَلِكَ لَوْ}} اهْتَزَّ المَثْلَثُ لَمْ يَهْتَزَّ الزِّيرُ وَإِنْ وَضَعْنَا الإِصْبِعَ عَلَى سَبَّابَةِ الزِّيرِ تَحَرَّكَ مَا عَلَيْهِ. وَكَذَلِكَ يُعْرَضُ فِي المُتَسَاوِيَةِ الطَّبَقَةِ لِأَنَّهَا مُتَشَابِهَةٌ، وَكَذَلِكَ عُرِضَ الأَمْرُ بِعَيْنِهِ فِيمَا بِالكُلِّ الّذِي بِالكُلِّ مُتَشَابِهٌ وَلَيْسَ مُتَسَاوِي. وَالمَحْسُوسُ <الأَقْوَى> هُوَ ذَلِكَ الحِسُّ الَّذِي فِي الهَوَاءِ <..> الحَادِثِ عَنِ القَرْعِ لَكِنَّهُ مَقْرُونٌ بِحَرَكَةٍ <..> أَنْ يُحَسَّ دُونَ تَحَرُّكِ ذَلِكَ الهَوَاءِ فَلِذَلِكَ هُوَ أَثَرٌ مُقْتَرَنٌ بِهِ بِحَرَكَةٍ فِي الأَثَرِ فَلِذَلِكَ يُلْحِقُهُ عَنْ مَا يَرْجِعُ عَنْ جِسْمٍ لَنْ يَرْجِعَ بِعَيْنِهِ وَلَكِنْ لَا عَلَى تِلْكَ الحَالَةِ فَلِذَلِكَ يَلْزَمُ الصَّدُّ أَنْ يُغَيِّرَهَا لَكِنْ يَبْقَى الأَمْرُ وَاحِدًا بِعَيْنِهِ.

وَكَذَلِكَ فِي أُذُنِ الِإنْسَانِ خَاصَّةً لَمَّا كَانَتْ <.> التَّقَارُعُ <.> لِلْهَوَاءِ هُنَالِكَ أَصْنَافٌ مِنَ الرُّجُوعِ وَبَقِيَ الصَّوْتُ كَمَا يُعْرَضُ فِي الآلَاتِ المُصَوِّتَةِ كَالعُودِ وَبِذَلِكَ يَكُونُ لِلصَّوْتِ نَغْمَةٌ فَإِنَّ النَّغْمَةَ صَوْتٌ يَبْقَى زَمَانًا مَحْسُوسًا، وَكَذَلِكَ لَمْ يَكُنْ كُلُّ صَوْتٍ نَغْمَةً، فَلِذَلِكَ مَتَى رَدَعَهُ صَوْتٌ آخَرُ امْتَزَحَ الهَوَاءُ آنًا وَهُمَا بِأَحْوَالٍ مُخْتَلِفَةٍ فَتَحْدُثُ نَغْمَةٌ مُمْتَزِحَةٌ إِمًّا مَتَلَائِمَةٌ أَوْ مُتَنَافِرَةٌ. وَهَذَا هُوَ السَّبَبُ الَّذِي كَانَتْ لِلْإِيقَاعَاتِ \edtext{بصرة}{\Afootnote{كلمة غير واضحة.}} الملذّة مُتَنَافِرَةٌ وَالمُنَافِرَةٌ مُلَائِمَةً وَهَذَا هُوَ <.> بَيْنَهَا النَّغَمُ وَقَدْ فُصِّلَ ذَلِكَ كُلُّهُ فِي مَوَاضِعَ أُخَرَ وَلَمَّا كَانَ المَوْضِعُ الأَوَّلُ لِلسَّمْعِ هُوَ الهَوَاءُ لِأَنَّهُ القَابِلُ الأَوَّلُ لِلْصَّوْتِ كَذَلِكَ كَأَنَّ المُتَقَارِعَيْنِ مَحْسُوسَيْنِ بِالغَرَضِ وَكَذَلِكَ <.> الغَلَطُ لِلسَّمْعِ فَهُمَا كَمَا تَقَعُ لِلْبَصَرِ فِيمَا مَوْضُوعُهُ بِالغَرَضِ وَ<.> تَلَخَّصَ ذَلِكَ قَبْلُ، فَكَذَلِكَ قَدْ تُعْرَضُ أَصْوَاتٌ كَثِيرَةُ الأَجْسَامِ مُتَبَايِنَةٌ يُظَنُّ بِهَا أَنَّهَا وَاحِدَةٌ.
\pend
\endnumbering

\end{document}