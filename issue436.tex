\documentclass{book}

\usepackage[utf8]{inputenc}
\usepackage[T1]{fontenc}
\usepackage[german]{babel}


\usepackage[splitindex,innote]{indextools}%notenumber
\indexsetup{level=\section*,toclevel=section,noclearpage}

\makeindex[name=res,title=Index rerum]
\makeindex[name=pers,title=Index personarum]

\usepackage[series={A,B,C},noeledsec,noend,xindy,xindy+hyperref]{reledmac}
\usepackage{reledpar}
\usepackage[hyperindex=false]{hyperref}
\renewcommand{\thefootnote}{\roman{footnote}}

\begin{document}

\begin{ledgroup}
\begin{pairs}
\begin{Leftside}
\beginnumbering
\pstart
Et quamquam\footnote{toto} Philon in tractatu suo de aqueductibus\footnoteC{Philon von Byzanz\index[pers]{Philon von Byzanz}}. Magister\index[res]{Magister} Dominicus de Florentia\footnoteC{Dominicus von Florenz\index[pers]{Dominicus von Florenz}}
\pend
\endnumbering
\end{Leftside}

\begin{Rightside}
\beginnumbering
\pstart
Und obwohl ...
\pend
\endnumbering
\end{Rightside}

\end{pairs}
\Columns
\end{ledgroup}




\begin{ledgroup}
\setcounter{footnoteC}{0}
\setcounter{footnote}{0}
\begin{pairs}
\begin{Leftside}
\beginnumbering
\pstart
Et quamquam\footnote{toto} Heron\footnoteC{Heron von Alexandrien\index[pers]{Heron von Alexandrien}}. Magister\index[res]{Magister} Ptolemaeus\footnoteC{Ptolemaeus\index[pers]{Ptolemaeus}}.
\pend
\endnumbering
\end{Leftside}

\begin{Rightside}
\beginnumbering
\pstart
Und obwohl ...
\pend
\endnumbering
\end{Rightside}

\end{pairs}
\Columns
\end{ledgroup}


\newpage
\printindex[res]
\printindex[pers]

\end{document}

