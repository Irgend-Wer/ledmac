\listfiles
\documentclass{article}
\usepackage[no-math]{fontspec}
\usepackage{xltxtra,xunicode,amsmath}

\usepackage{polyglossia}
\setdefaultlanguage{french}

\usepackage[]{ledmac}
\footparagraph{B}

%% Au choix, on peut ici décommenter:
% 1. soit la ligne \setotherlanguage{arabic} (polyglossia)
% 2. soit la ligne \usepackage{arabxetex} (arabxetex)
% 3. soit les deux
% ... et cela donne des espaces énormes entre les notes.

\usepackage{bidi}

\makeatletter
\renewcommand*{\vBfootnote}[2]{%
  \insert\csname #1footins\endcsname
  \bgroup
    \notefontsetup
    \footsplitskips
    \setbox0=\vbox{\let\bidi@RTL@everypar\@empty
    \hsize=\maxdimen
      \noindent\csname #1footfmt\endcsname#2}%
    \setbox0=\hbox{\unvxh0}%
    \dp0=0pt
    \ht0=\csname #1footfudgefactor\endcsname\wd0
    \box0
    \penalty0
  \egroup}
\makeatother

\begin{document}

\beginnumbering

\pstart

\edtext{Jean}{\Bfootnote{Paul}} a vu \edtext{Jean}{\Bfootnote{Paul}}.
\edtext{Jean}{\Bfootnote{Paul}} a vu \edtext{Jean}{\Bfootnote{Paul}}.
\edtext{Jean}{\Bfootnote{Paul}} a vu \edtext{Jean}{\Bfootnote{Paul}}.
\edtext{Jean}{\Bfootnote{Paul}} a vu \edtext{Jean}{\Bfootnote{Paul}}.
\edtext{Jean}{\Bfootnote{Paul}} a vu \edtext{Jean}{\Bfootnote{Paul}}.
\edtext{Jean}{\Bfootnote{Paul}} a vu \edtext{Jean}{\Bfootnote{Paul}}.

\pend

\endnumbering

\end{document}