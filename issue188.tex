\documentclass{book}
\usepackage{geometry}
\usepackage[utf8]{inputenc}
\usepackage[T1]{fontenc}
\usepackage[german]{babel}
\usepackage{lmodern}
\usepackage{eledmac,eledpar} 
\firstlinenum{1}  \linenumincrement{1}%
\begin{document}
\beginnumbering
\pstart
\begin{edtabularl}
linea prima\ledouternote{nota1} & linea prima\ledouternote{nota2} \\
linea secunda\footnoteA{familiar footnote A1.} & linea secunda\footnoteA{familiar footnote A2.} \\
linea tertia & linea \edtext{tertia}{\Afootnote{critical note A}} \\
linea quarta & linea \edtext{quarta}{\Bfootnote{critical note B}} \\
linea quinta\ledouternote{nota3} & linea quinta\footnoteB{familiar footnote B.} \\
linea sexta & linea sexta
\end{edtabularl}
\pend
\pstart
\begin{edtabularl}
linea prima\ledouternote{nota1} & linea prima\ledouternote{nota2} \\
linea secunda\footnoteA{familiar footnote A1.} & linea secunda\footnoteA{familiar footnote A2.} \\
linea tertia & linea \edtext{tertia}{\Afootnote{critical note A}} \\
linea quarta & linea \edtext{quarta}{\Bfootnote{critical note B}} \\
linea quinta\ledouternote{nota3} & linea quinta\footnoteB{familiar footnote B.} \\
linea sexta & linea sexta
\end{edtabularl}
\pend
\endnumbering
\end{document}
