\documentclass{book}
\usepackage{geometry}
\usepackage[utf8]{inputenc}
\usepackage[T1]{fontenc}
\usepackage[german]{babel}
\usepackage{lmodern}
\usepackage{eledmac,eledpar}

\firstlinenum{1}
\linenumincrement{1}%

\begin{document}

\beginnumbering
\pstart
\begin{edtabularl}
linea prima & linea prima\ledouternote{sidenote} \\
linea secunda & linea secunda\footnoteA{familiar footnote A.} \\
linea tertia & linea \edtext{tertia}{\Afootnote{critical note A}} \\
linea quarta & linea \edtext{quarta}{\Bfootnote{critical note B}} \\
linea quinta & linea quinta\footnoteB{familiar footnote B.} \\
linea sexta & linea sexta 
\end{edtabularl}
\pend
\endnumbering

\end{document}
