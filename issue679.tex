% !TeX program = xelatex
% !TeX encoding = UTF-8
% !TeX spellcheck = it_IT

\documentclass[11pt,a4paper]{book}
\usepackage[outer=4.85cm,inner=4.85cm,top=5.8cm,bottom=5.8cm,headsep=4mm]{geometry}

\usepackage{fontspec}
\setmainfont{Linux Libertine O}

\usepackage{polyglossia}
\setmainlanguage[babelshorthands=true]{italian}
\setotherlanguage[variant=ancient]{greek}

\usepackage[noeledsec,noledgroup,series={A,B,C},noend]{reledmac}
\usepackage[widthliketwocolumns,continuousnumberingwithcolumns]{reledpar}
%\usepackage[continuousnumberingwithcolumns]{reledpar}
\firstlinenum*{1}
\linenumincrement*{1}
\AtBeginPairs{\sloppy}
\columnsposition{L}

\AtBeginDocument{%
\setlength{\Lcolwidth}{0.47\textwidth}
\setlength{\Rcolwidth}{0.47\textwidth}
\setlength{\beforecolumnseparator}{0.03\textwidth}
\setlength{\aftercolumnseparator}{0.03\textwidth}
}
\begin{document}
Quel ramo del lago di Como, che volge a mezzogiorno, tra due catene non interrotte di monti, tutto a seni e a golfi, a seconda dello sporgere e del rientrare di quelli, vien, quasi a un tratto, a ristringersi, e a prender corso e figura di fiume, tra un promontorio a destra, e un’ampia costiera dall’altra parte; e il ponte, che ivi congiunge le due rive, par che renda ancor più sensibile all’occhio questa trasformazione, e segni il punto in cui il lago cessa, e l’Adda rincomincia, per ripigliar poi nome di lago dove le rive, allontanandosi di nuovo, lascian l’acqua distendersi e rallentarsi in nuovi golfi e in nuovi seni.

\smallskip
\begin{small}
\numberlinefalse
\begin{pairs}
	\begin{Leftside}
	\beginnumbering
	\pstart
	\centerline{\textit{Euthyd.} 29}
	\pend
	\pstart
	\noindent\textgreek{παροιμία ἐπὶ τῶν τὰ αὐτὰ διὰ τῶν αὐτῶν δρώντων} P\textsuperscript{exc}W
	\pend
	\pstart
	\noindent\textgreek{Στράττις Ποταμοῖς}. W
	\pend
	\endnumbering
	\end{Leftside}

	\begin{Rightside}
	\beginnumbering
	\pstart
	\centerline{\textit{Euthyd.} 30}
	\pend
	\pstart
	\noindent\textgreek{πα. “λίνον λίνῳ συνάπτεις”· \textit{ἐπὶ τῶν τὰ αὐτὰ διὰ τῶν αὐτῶν} ἢ λεγόντων ἢ \textit{δρώντων}, ἢ τὰ ὅμοια εἰς φιλίαν συναπτόντων. μέμνηται δὲ αὐτῆς Ἀριστοτέλης ἐν τῷ \greeknumber{3} περὶ φυσικῆς ἀκροάσεως “οὐ γὰρ λίνον λίνῳ συνάπτειν ἔστι”}
	\pend
	\pstart
	\noindent\textgreek{καὶ \textit{Στράττις Ποταμοῖς} καὶ Πλάτων Εὐθυδήμῳ}
	\pend
	\endnumbering
	\end{Rightside}
\end{pairs}
\Columns
\numberlinetrue
\end{small}
\smallskip

Quel ramo del lago di Como, che volge a mezzogiorno, tra due catene non interrotte di monti, tutto a seni e a golfi, a seconda dello sporgere e del rientrare di quelli, vien, quasi a un tratto, a ristringersi, e a prender corso e figura di fiume, tra un promontorio a destra, e un’ampia costiera dall’altra parte; e il ponte, che ivi congiunge le due rive, par che renda ancor più sensibile all’occhio questa trasformazione, e segni il punto in cui il lago cessa, e l’Adda rincomincia, per ripigliar poi nome di lago dove le rive, allontanandosi di nuovo, lascian l’acqua distendersi e rallentarsi in nuovi golfi e in nuovi seni.


\beginnumbering
\numberpstarttrue
\pstart
Quel ramo del lago di Como, che volge a mezzogiorno, tra due catene non interrotte di monti, tutto a seni e a golfi, a seconda dello sporgere e del rientrare di quelli, vien, quasi a un tratto, a ristringersi, e a prender corso e figura di fiume, tra un promontorio a destra, e un’ampia costiera dall’altra parte; e il ponte, che ivi congiunge le due rive, par che renda ancor più sensibile all’occhio questa trasformazione, e segni il punto in cui il lago cessa, e l’Adda rincomincia, per ripigliar poi nome di lago dove le rive, allontanandosi di nuovo, lascian l’acqua distendersi e rallentarsi in nuovi golfi e in nuovi seni.
\pend

\pausenumbering
\begin{pairs}
\begin{Leftside}
\resumenumbering
\numberpstartfalse
	\pstart\noindent Quel ramo del lago di Como, che volge a mezzogiorno, tra due catene non interrotte di monti, tutto a seni e a golfi, a seconda dello sporgere e del rientrare di quelli, vien, quasi a un tratto, a ristringersi, e a prender corso e figura di fiume, tra un promontorio a destra, e un’ampia costiera dall’altra parte; e il ponte, che ivi congiunge le due rive, par che renda ancor più sensibile all’occhio questa trasformazione, e segni il punto in cui il lago cessa, e l’Adda rincomincia, per ripigliar poi nome di lago dove le rive, allontanandosi di nuovo, lascian l’acqua distendersi e rallentarsi in nuovi golfi e in nuovi seni.s\pend
\pausenumbering
\end{Leftside}

\begin{Rightside}
\beginnumbering
\numberpstartfalse
	\pstart\noindent Quel ramo del lago di Como, che volge a mezzogiorno, tra due catene non interrotte di monti, tutto a seni e a golfi, a seconda dello sporgere e del rientrare di quelli, vien, quasi a un tratto, a ristringersi, e a prender corso e figura di fiume, tra un promontorio a destra, e un’ampia costiera dall’altra parte; e il ponte, che ivi congiunge le due rive, par che renda ancor più sensibile all’occhio questa trasformazione, e segni il punto in cui il lago cessa, e l’Adda rincomincia, per ripigliar poi nome di lago dove le rive, allontanandosi di nuovo, lascian l’acqua distendersi e rallentarsi in nuovi golfi e in nuovi seni.\pend
\pausenumbering
\end{Rightside}
\end{pairs}
\Columns

\numberpstarttrue
\resumenumbering

\pstart
Quel ramo del lago di Como, che volge a mezzogiorno, tra due catene non interrotte di monti, tutto a seni e a golfi, a seconda dello sporgere e del rientrare di quelli, vien, quasi a un tratto, a ristringersi, e a prender corso e figura di fiume, tra un promontorio a destra, e un’ampia costiera dall’altra parte; e il ponte, che ivi congiunge le due rive, par che renda ancor più sensibile all’occhio questa trasformazione, e segni il punto in cui il lago cessa, e l’Adda rincomincia, per ripigliar poi nome di lago dove le rive, allontanandosi di nuovo, lascian l’acqua distendersi e rallentarsi in nuovi golfi e in nuovi seni. La costiera,formata dal deposito di tre grossi torrenti, scende appoggiata a due monti contigui, l’uno detto di san Martino, l’altro, con voce lombarda, il Resegone, dai molti suoi cocuzzoli in fila, che in vero lo fanno somigliare a una sega: talché non è chi, al primo vederlo, purché sia di fronte, come per esempio di su le mura di Milano che guardano a settentrione, non lo discerna tosto, a un tal contrassegno, in quella lunga e vasta giogaia, dagli altri monti di nome più oscuro e di forma più comune. Per un buon pezzo, la costa sale con un pendii lento e continuo; poi si rompe in poggi e in valloncelli, in erte e in ispianate,secondo l’ossatura de’ due monti, e il lavoro dell’acque. Il lembo estremo, tagliato dalle foci de’ torrenti, è quasi tutto ghiaia e ciottoloni; il resto, campi e vigne, sparse di terre, di ville, di casali; in qualche parte boschi, che si prolungano su per la montagna.
\pend
\endnumbering
\end{document}
