
% !TeX program = xelatex
% !TeX encoding = UTF-8
% !TeX spellcheck = it_IT

\documentclass[11pt,a4paper]{book}
\usepackage{libertine}

\usepackage[noend,noledgroup,series={A,B,C}]{reledmac}
\usepackage[shiftedpstarts]{reledpar}

\begin{document}
\part{Il \emph{Patroclus} di Coricio di Gaza}

\begin{pages}
\begin{Leftside}
\beginnumbering
\pstart{}
%%%%THIS APPARATUS NOTE DOESN'T APPEAR IN PDF:
\eledchapter[Patroclo]{\edtext{<Πάτροκλος>}{\lemma{}\Cfootnote[nosep]{Πάτροκλος ins, Πατρόκλου πρὸς Ἀχιλλέα μελέτη ReBois, Μελέτης γ´ ὑπόθεσις ante ἀφαιρεθεὶς Mor, ὑπόθεσις ReBois del}}}
\pend

\pstart
\eledsection{ὑπόθεσις}
\pend

\pstart
Ἀφαιρεθεὶς Ἀχιλλεὺς τῆς Βρισηίδος ἐπὶ τῆς οἰκείας διῆγε \edtext{σκηνῆς}{\Cfootnote{prova}} ὠργισμένος. πρεςβεύεται πρὸς αὐτὸν Ἀγαμέμνων μετὰ πλεί|\ledsidenote{434}στων τὴν κόρην ἀποδιδοὺς δωρεῶν. καὶ γέγονεν ἄπρακτος αὐτῷ ἡ πρεσβεία καὶ Βρισηίδι καὶ δώροις. χρόνος ἐν μέσῳ βραχὺς καὶ προσθήκη ταῖς Ἑλλήνων ἀτυχίαις πολλή. ᾔσθετο Πάτροκλος καὶ συνήλγησε καὶ τὰ μὲν δακρύων, τὰ δὲ νουθετῶν διαλλάξαι πειρᾶται τὸν Ἀχιλλέα τοῖς Ἕλλησιν. μελετῶμεν τὸν Πάτροκλον.
\pend

\endnumbering
\end{Leftside}

\begin{Rightside}
\beginnumbering
\numberpstartfalse

\pstart
\eledchapter{Patroclo}
\pend

\pstart
\eledsection{Argomento\footnoteA{Nota a}}
\pend

\pstart
Achille adirato, poiché era stato privato di Briseide, si ritira nella propria tenda. Agamennone allora gli manda ambasciatori per restituire la fanciulla con molti doni. La sua ambasciata coi doni e con Briseide tuttavia si rivela infruttuosa. Nel breve lasso di tempo intercorso (si viene a verificare) anche una crescita significativa delle sventure dei Greci. Patroclo ne viene informato e ne condivide il dolore e, sia con le lacrime sia con degli ammonimenti, cerca di riconciliare Achille coi Greci. Prendiamo dunque le parti di Patroclo.\footnoteA{Nota b.}
\pend

\endnumbering
\end{Rightside}
\end{pages}
\Pages
\end{document}
