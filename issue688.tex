% !TeX program = xelatex
% !TeX encoding = UTF-8
% !TeX spellcheck = it_IT

\documentclass[11pt,a4paper]{book}
\usepackage[%
	textwidth=12cm,textheight=18.7cm,left=4cm,%
	headsep=4mm,%
	]{geometry}

\usepackage{libertine}
\usepackage{polyglossia}
\setmainlanguage[babelshorthands=true]{italian}
\setotherlanguages{latin,greek,german}
	\setkeys{greek}{variant=ancient}
	\setkeys{latin}{variant=modern}
	\setkeys{german}{babelshorthands=true}

\usepackage[noeledsec,noledgroup,series={A,B,C}]{reledmac}
\usepackage[continuousnumberingwithcolumns]{reledpar}
	\lineation*{page}
	\linenummargin*{inner}
	\sidenotemargin*{outer}
	\AtBeginPairs{\sloppy}
	\linenummarginColumns{left}
	\linenummarginColumnsR{right}
	%\linenumOnlyPagesForColumns{right}
	%\linenumOnlyPagesForColumnsR{left}
	\linenumberLevenifblanktrue
	\linenumberRevenifblanktrue
	\firstlinenum*{1}
  \linenumincrement*{1}
\usepackage{lipsum}

\makeatletter
\renewcommand*{\ledlinenum}{%
   \bgroup%
   \ifluatex%
     \textdir TLT%
   \fi%
   \numlabfont\linenumrep{\line@num}%
   \ifsublines@
     \ifnum\subline@num>0\relax
       \unskip%
       \Xsublinesep@side%
       \sublinenumrep{\subline@num}%
     \fi
   \fi%
   (\the\page@num)
   \egroup%
 }%
 \renewcommand{\l@dlinenumR}{%
   \numlabfont\linenumrepR{\line@numR}\@Rlineflag%
   \ifsublines@R
     \ifnum\subline@numR>\z@
       \unskip\fullstop\sublinenumrepR{\subline@numR}%
     \fi
   \fi
   (\the\page@numR)
 }
\makeatother
\begin{document}
	
\begin{greek}
	
\beginnumbering
\numberpstarttrue
	
\pstart{}XXXXX\pend

\pstart{}XXXXX\pend

\pstart{}XXXXX\pend

\pstart{}XXXXX\pend

\pstart{}XXXXX\pend

\pstart{}XXXXX\pend

\pstart{}XXXXX\pend

\pstart{}XXXXX\pend

\pstart{}XXXXX\pend

\pstart{}XXXXX\pend

\pstart{}XXXXX\pend

\pstart{}XXXXX\pend

\pstart{}XXXXX\pend

\pstart{}XXXXX\pend

\pstart{}XXXXX\pend

\pstart{}XXXXX\pend

\pstart{}XXXXX\pend

\pstart{}XXXXX\pend

\pstart{}XXXXX\pend

\pstart{}XXXXX\pend

\pstart{}XXXXX\pend

\pstart{}XXXXX\pend

\pstart{}XXXXX\pend

\pstart{}XXXXX\pend

\pstart{}XXXXX\pend

\pstart{}XXXXX\pend

\pstart{}XXXXX\pend

\pausenumbering
\begin{pairs}
\begin{Leftside}
\resumenumbering
\numberpstartfalse
\pstart\noindent
\textbf{a)}
\edtext{Ῥαδαμάνθυος ὅρκος}%
	{\lemma{Ῥαδαμάνθυος ὅρκοι}\Cfootnote[nosep]{sch. \textit{Reip.}}}
\edtext{οὗτος}%
        {\Cfootnote[nosep]{om. P, scholl. \textit{Phaedr.} et \textit{Reip.}, Phot. Suid.}}
ὁ \edtext{κατὰ χηνὸς ἢ κυνὸς}%
        {\Cfootnote[nosep]{TW, Phot. Suid.: κατὰ κυνὸς ἢ χηνὸς
            scholl. \textit{Phaedr.} et \textit{Reip.}}}
\edtext{ἢ πλατάνου ἢ κριοῦ}%
        {\Cfootnote[nosep]{TW, sch. \textit{Phaedr.}, Phot. Suid.: ἢ
            κριοῦ ἢ πλατάνου sch. \textit{Reip.}}}
        ἤ τινος ἄλλου τοιούτου.
“\edtext{οἷς ἦν}%
        {\Cfootnote[]{οἷς οὖν ἦν sch. \textit{Phaedr.} (sed revera οἷς
            οὖν T: εἷς οὖν W)}}
μέγιστος ὅρκος
\edtext{ἅπαντι λόγῳ}%
        {\Cfootnote[nosep]{vix tetrametro iambico conveniens recepit
            Meineke, qui in commentario suo hos versus ex lyrico
            carmine ductos, posito prioris versus exitu post λόγῳ,
            censuit (huic opinioni non favit Kock): intra cruces
            Kassel--Austin: ἐν παντὶ λόγῳ sch. \textit{Phaedr.}}}
        κύων, ἔπειτα χήν,
\edtext{θεοὺς δ' ἐσίγων}{\Cfootnote[]{καὶ τἆλλα
    sch. \textit{Phaedr.}}}”· Κρατῖνος Χείρωσι (Cratinus, PCG IV
fr. 249).
\edtext{τοιοῦτοι δὲ καὶ οἱ Σωκράτους ὅρκοι}{\lemma{τοιοῦτοι~\dots{}~ὅρκοι}\Cfootnote[]{κατὰ τούτων
    δὲ νόμος ὀμνύναι, ἵνα μὴ κατὰ θεῶν οἱ ὅρκοι γίγνωνται
    sch. \textit{Phaedr.}}}. \textbf{TW}
\pend
\pausenumbering
\end{Leftside}

\begin{Rightside}
\beginnumbering
\numberpstartfalse
\pstart\noindent
\ledsidenote{Σ?}\textbf{b)} Ῥαδαμάνθυος ὅρκος ὁ κατὰ χηνὸς καὶ κυνὸς καὶ πλατάνου καὶ
τῶν τοιούτων. \textbf{P}
\pend
\pausenumbering
\end{Rightside}
\end{pairs}
\Columns

\numberpstarttrue
\resumenumbering

\pstart{}XXXXX\pend

\pstart{}XXXXX\pend

\pstart{}XXXXX\pend

\pstart{}XXXXX\pend

\pstart{}XXXXX\pend

\pstart{}XXXXX\pend

\pstart{}XXXXX\pend

\pstart{}XXXXX\pend

\pstart{}XXXXX\pend

\pstart{}XXXXX\pend

\pstart{}XXXXX\pend

\pstart{}XXXXX\pend

\pstart{}XXXXX\pend

\pstart{}XXXXX\pend

\pstart{}XXXXX\pend

\pstart{}XXXXX\pend

\pstart{}XXXXX\pend

\pstart{}XXXXX\pend

\pstart{}XXXXX\pend

\pstart{}XXXXX\pend

\pstart{}XXXXX\pend

\pausenumbering
\begin{pairs}
\begin{Leftside}
\resumenumbering
\numberpstartfalse
\pstart\noindent
\ledsidenote{Σ?}\textbf{a)}
\edtext{Ῥαδαμάνθυος ὅρκος}%
	{\lemma{Ῥαδαμάνθυος ὅρκοι}\Cfootnote[nosep]{sch. \textit{Reip.}}}
\edtext{οὗτος}%
        {\Cfootnote[nosep]{om. P, scholl. \textit{Phaedr.} et \textit{Reip.}, Phot. Suid.}}
ὁ \edtext{κατὰ χηνὸς ἢ κυνὸς}%
        {\Cfootnote[nosep]{TW, Phot. Suid.: κατὰ κυνὸς ἢ χηνὸς
            scholl. \textit{Phaedr.} et \textit{Reip.}}}
\edtext{ἢ πλατάνου ἢ κριοῦ}%
        {\Cfootnote[nosep]{TW, sch. \textit{Phaedr.}, Phot. Suid.: ἢ
            κριοῦ ἢ πλατάνου sch. \textit{Reip.}}}
        ἤ τινος ἄλλου τοιούτου.
“\edtext{οἷς ἦν}%
        {\Cfootnote[]{οἷς οὖν ἦν sch. \textit{Phaedr.} (sed revera οἷς
            οὖν T: εἷς οὖν W)}}
μέγιστος ὅρκος
\edtext{ἅπαντι λόγῳ}%
        {\Cfootnote[nosep]{vix tetrametro iambico conveniens recepit
            Meineke, qui in commentario suo hos versus ex lyrico
            carmine ductos, posito prioris versus exitu post λόγῳ,
            censuit (huic opinioni non favit Kock): intra cruces
            Kassel--Austin: ἐν παντὶ λόγῳ sch. \textit{Phaedr.}}}
        κύων, ἔπειτα χήν,
\edtext{θεοὺς δ' ἐσίγων}{\Cfootnote[]{καὶ τἆλλα
    sch. \textit{Phaedr.}}}”· Κρατῖνος Χείρωσι (Cratinus, PCG IV
fr. 249).
\edtext{τοιοῦτοι δὲ καὶ οἱ Σωκράτους ὅρκοι}{\lemma{τοιοῦτοι~\dots{}~ὅρκοι}\Cfootnote[]{κατὰ τούτων
    δὲ νόμος ὀμνύναι, ἵνα μὴ κατὰ θεῶν οἱ ὅρκοι γίγνωνται
    sch. \textit{Phaedr.}}}. \textbf{TW}
\pend
\pausenumbering
\end{Leftside}

\begin{Rightside}
\resumenumbering
\numberpstartfalse
\pstart\noindent
\textbf{b)} Ῥαδαμάνθυος ὅρκος ὁ κατὰ χηνὸς καὶ κυνὸς καὶ πλατάνου καὶ
τῶν τοιούτων. \textbf{P}
\pend
\pausenumbering
\end{Rightside}
\end{pairs}
\Columns

\numberpstarttrue
\resumenumbering

\pstart{}XXXXX\pend

\pstart{}XXXXX\pend

\pstart{}XXXXX\pend

\pstart{}XXXXX\pend

\pstart{}XXXXX\pend

\pstart{}XXXXX\pend

\pstart{}XXXXX\pend

\pstart{}XXXXX\pend

\pstart{}XXXXX\pend

\pstart{}XXXXX\pend

\pstart{}XXXXX\pend

\pstart{}XXXXX\pend

\pstart{}XXXXX\pend

\pstart{}XXXXX\pend

\pstart{}XXXXX\pend

\pstart{}XXXXX\pend

\pstart{}XXXXX\pend

\pstart{}XXXXX\pend

\pstart{}XXXXX\pend

\pstart{}XXXXX\pend

\pstart{}XXXXX\pend

\pstart{}XXXXX\pend

\pstart{}XXXXX\pend

\pstart{}XXXXX\pend

\pstart{}XXXXX\pend

\pstart{}XXXXX\pend

\pstart{}XXXXX\pend

\pstart{}XXXXX\pend

\pstart{}XXXXX\pend

\pstart{}XXXXX\pend

\pstart
25c6 \textit{ὦ τάν BP: ὦ τᾶν DTW}] %
\pend

\pausenumbering
\null
\begin{pairs}
\begin{Leftside}
\resumenumbering
\numberpstartfalse
\pstart
\noindent\textbf{a)} ὦ οὗτος, \textbf{TW} ⟦%
\edtext{ὦ ἑταῖρε}{\Cfootnote[nosep]{om. TP, sch. \textit{Ep.}, Suid.,
    Greene (habebat Hermann)}},
\textbf{W}⟧ ὦ τάλαν, καὶ \edtext{ὦ μέλε}{\Cfootnote[nosep]{TPW, sch. \textit{Ep.}: ὦ μέλεε Suid.}}. ταῦτα παρὰ τοῖς νεωτέροις ὑπὸ
        μόνων λέγεται γυναικῶν, παρὰ δὲ τοῖς παλαιοῖς καὶ ὑπ'
        ἀνδρῶν. πολλάκις δὲ καὶ ἐπὶ πλήθους φασὶ “%
        \edtext{ὦ τάν}{\Cfootnote[]{ὦτ' ἂν W}}”, ὡς παρὰ
        \edtext{Κτησιφῶντι}{\Cfootnote[]%
          {immo Νικοφῶντι, ut, collatis Ath. 7.523b, Poll. 7.33 et Suid. ν
406 (ubi autem Νικόφρων et in codd. legitur et A. Adler edidit), iam
viderat D. Ruhnken, \textit{Timaeus}, p. 282: παράχρησιν φιλοῦντες (=
παραχρώμενοι) pro παρὰ Κτησιφῶντι voluit Schmidt: παρὰ ∆ημοσθένει ἐν
τῷ ὑπὲρ Κ\textasciigrave{}οι\textasciiacute{} temptavit Adler: παρὰ
Κρατίνῳ ex Suid. ω 261 Bernhardy (in Suidae editione)}}
        (immo Nicophon, PCG VII fr. 30). οἱ γὰρ ἀττικοὶ τὴν
        πρώτην συλλαβὴν περισπῶσιν, τὴν δὲ δευτέραν
        \edtext{βραχύνουσιν}{\Cfootnote[nosep]{TW, Suid.: βαρύνουσι
            sch. \textit{Ep.}}}, καὶ
        βέλτιον· ἀδύνατον γὰρ εὑρεθῆναι μίαν λέξιν δύο ἔχουσαν
        περισπωμένας. ∆ίδυμος δὲ τὸ πλῆρες εἶναί φησιν “ὦ ἔταν”,
        ἀγνοῶν
        \edtext{ὡς †ἀπὸ†}{\lemma{ὡς ἀπὸ}\Cfootnote[nosep]{TPW: ὡς ἄρα
            sch. \textit{Ep.}, Suid.}} τοῦ “ἔτης” ἡ κλητική ἐστιν
        “ἔτα”, καὶ δωρικῶς “ἔταν” (Didym. fr. dubiae sedis 10, p. 403 Schmidt). \textbf{TW}
\pend
\pausenumbering
\end{Leftside}

\begin{Rightside}
\resumenumbering
\numberpstartfalse
\pstart
\noindent\ledsidenote{Σ}\textbf{b)} ὦ οὗτος, ὦ τάλαν, ὦ μέλε. τὸ δὲ
        πλῆρες “ὦ ἔταν”, †ἀπὸ† τοῦ “ἔτης” ἡ κλητικὴ “ἔτα” καὶ δωρικῶς
        “ἔταν”. \textbf{P}\pend
\pausenumbering
\end{Rightside}
\end{pairs}
\Columns

\numberpstarttrue
\resumenumbering

\pstart XXXXX\pend

\pstart{}XXXXX\pend

\pstart{}XXXXX\pend

\pstart{}XXXXX\pend

\pstart{}XXXXX\pend

\pstart{}XXXXX\pend


\numberpstartfalse
\endnumbering

\end{greek}
\end{document}