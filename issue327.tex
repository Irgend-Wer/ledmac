% !TeX program = xelatex
% !TeX encoding = UTF-8
% !BIB TS-program = biber

\documentclass[11pt,a4paper]{book}
%%%%% 
%%%%% If you comment only geometry, the compilation gives a warning but produce the expected pdf
%%%%% If you comment only tikz, the compilation is fine
%%%%% If you use both packages there are lots of issues and a bad pdf
%%%%% 
\usepackage[outer=4.85cm,inner=4.85cm,top=5.8cm,    bottom=5.8cm,headsep=4mm]{geometry}
\setlength{\parindent}{0.5cm}

\usepackage{fontspec}
    \setmainfont{Linux Libertine O}
\usepackage{polyglossia}
    \setmainlanguage[babelshorthands=true]{italian}
    \setotherlanguage[variant=polytonic]{greek}

\usepackage{longtable}
\usepackage{booktabs}
\usepackage{tikz}

%%%%%%%%%%%%%%%%%%%%%%%%%%%%%%%%%%%%%%%%
\usepackage{eledmac,eledpar}
\lineation{page}
\linenummargin{inner}
\sidenotemargin{outer}

\Xnotefontsize[A,B,C]{\footnotesize}

\renewcommand*{\ledlsnotefontsetup}%
    {\raggedleft\it\footnotesize}
\renewcommand*{\ledrsnotefontsetup}%
    {\raggedright\it\footnotesize}
\renewcommand*{\sidenotesep}%
    { \emph{|} }

\nonumberinfootnote[A,B]
\numberonlyfirstinline[C]
\numberonlyfirstintwolines[C]
\inplaceofnumber[A,B,C]{0em}
\nolemmaseparator[A,B]
\nonbreakableafternumber[C]
\inplaceoflemmaseparator[A,B]{0em}
\inplaceoflemmaseparator[C]{.5em}
\addtolength{\skip\Afootins}{2em plus.4em minus.4em}
\beforeXnotes[A]{2em plus.4em minus.4em}
\afternote[A,B,C]{2em plus.4em minus.4em}
\footparagraph{A}
\footparagraph{C}
%%%%%%%%%%%%%%%%%%%%%%%%%%%%%%%%%%%%%%%%

\usepackage{hyperref}

\begin{document}
\part{Prolegomena}
\chapter{Test1}
\section{Prova1}
Nel mezzo del cammin di nostra vita.\footnote{lkjlkj}

\bigskip

\begin{small}

\numberlinefalse
\begin{pairs}
    \begin{Leftside}
    \beginnumbering
    \pstart
    \noindent \textit{Resp.} III \pstartref{itm:Riii44} \textgreek{μύρον καταχέειν τῶν ἐν τοῖς ἁγιωτάτοις ἱεροῖς ἀγαλμάτων θέμις ἦν, \textit{ἐρίῳ} τε στέφειν αὐτά, καὶ τοῦτο \textit{κατά τινα ἱερατικὸν} νόμον, ὡς ὁ μέγας Πρόκλος φησίν}
    \pend
    \pstart
    \noindent \textit{Resp.} III \pstartref{itm:Riii49} \textgreek{ὁ Πρόκλος φησί· τὴν μὲν Δώριον ἁρμονίαν εἰς παιδείαν ἐξαρκεῖν ὡς \textit{καταστηματικήν}, τὴν δὲ Φρύγιον εἰς ἱερὰ καὶ ἐνθεασμοὺς (ἐκθεασμοὺς fort. A) ὡς \textit{ἐκστατικήν}}
    \pend
    \pstart
    \noindent \textit{Resp.} III \pstartref{itm:Riii55} \textgreek{ὁ ἐνόπλιος σύνθετός ἐστιν ἐξ ἰάμβου καὶ δακτύλου καὶ τῆς παριαμβίδος, ἀνδρικὸς πρὸς πράξεις ἀναγκαίας καὶ ἀκουσίους, \textit{ἐξορμητικὸς εἰς πόλεμον}. ὁ δὲ ἡρῷος δάκτυλος, ἁπλοῦς, κοσμιότητος ποιητικὸς καὶ ὁμαλότητος, παιδευομένοις προσήκων ὡς ἰσότητι κεκοσμημένος, ὡς ἐν τῷ εἰς ταῦτα ὑπομνήματι Πρόκλος φησίν}
    \pend
    \endnumbering
    \end{Leftside}

    \begin{Rightside}
    \beginnumbering
    \pstart
    \noindent Pr. \textit{Resp.} I 42.3-7 \textgreek{πρῶτον εἰπεῖν χρὴ καὶ διαπορῆσαι περὶ τῆς αἰτίας, δι’ ἣν οὐκ ἀποδέχεται τὴν ποιητικὴν ὁ Πλάτων, ἀλλὰ ἐξοικίζει τῆς ὀρθῆς πολιτείας, εἰ καὶ μύρον αὐτῆς καταχέας, ὡς τῶν ἐν τοῖς ἁγιωτάτοις ἱεροῖς ἀγαλμάτων θέμις, καὶ ὡς ἱερὰν στέψας αὐτήν, ὥσπερ καὶ ἐκεῖνα στέφειν ἦν νόμος}
    \pend
    \pstart
    \noindent Pr. \textit{Resp.} I 61.19-62.6 \textgreek{τὰς δὲ αὖ ἁρμονίας ἤδη μέν τινες τῶν θρηνοποιῶν καὶ συμποτικῶν, ὧν αἳ μὲν τὸ φιλήδονον χαλῶσιν, αἳ δὲ τὸ φιλόλυπον συντείνουσιν, τούτων δ’ οὖν ἐκβεβλημένων ἀξιοῦσιν τὰς λοιπάς, ὧν Δάμων ἐδίδασκεν, τήν τε Φρύγιον καὶ τὴν Δώριον αὐτὸν ὡς παιδευτικὰς παραδέχεσθαι· καὶ διαμφισβητοῦσι πρὸς ἀλλήλους, οἳ μὲν τὴν Φρύγιον εἰρηνικήν, τὴν δὲ Δώριον λέγοντες εἶναι κατ’ αὐτὸν πολεμικὴν ⟦εἶναι⟧, οἳ δὲ ἀνάπαλιν, τὴν μὲν Φρύγιον ὡς \textit{ἐκστατικὴν} εἶναι πολεμικήν, τὴν δὲ Δώριον \textit{καταστηματικὴν} καὶ εἰρηνικήν. \textit{ἡμεῖς} δὲ εὑρόντες ἐν Λάχητι (188d) […] οὐ φρυγιστὶ οὐδὲ αὖ ἰαστὶ ἢ λυδιστί, ἀλλὰ δωριστί, ἥπερ μόνη ἐστὶν ἁρμονία Ἑλληνική, ταύτην μὲν αὐτὸν ἡγούμεθα μόνην οἴεσθαι τῶν ἁρμονιῶν ἐν παιδείᾳ ἐξαρκεῖν, τὴν δὲ φρυγιστὶ πρὸς ἱερὰ καὶ ἐνθεασμοὺς ἐπιτηδείαν ὑπάρχειν}
    \pend
    \pstart
    \noindent Pr. \textit{Resp.} I I 61.2-11 + 62.13-14 \textgreek{τοὺς μὲν οὖν ῥυθμούς, ἐξ ὧν καὶ Δάμωνος ἀκοῦσαι λέγει καὶ ἀποδέχεται τοῦ λόγου, δῆλός ἐστιν τῶν μὲν συνθέτων τὸν ἐνόπλιον ἀποδεχόμενος, ὅς ἐστιν ἔκ τε ἰάμβου καὶ δακτύλου καὶ τῆς παριαμβίδος· τοῦτον γὰρ ἀνδρικὸν ἦθος ἐμποιεῖν καὶ παρατεταγμένον πρὸς πάσας τὰς ἀναγκαίας καὶ ἀκουσίους πράξεις· τῶν δὲ ἁπλῶν τὸν ἡρῷον δάκτυλον, περὶ οὗ καὶ λόγων φησὶν ἀκοῦσαι Δάμωνος καὶ δάκτυλόν γε καὶ ἡρῷον διακοσμοῦντος, ἐνδεικνύμενος ὡς ἄρα τὸν τοιοῦτον ῥυθμὸν ἡγεῖται κοσμιότητος εἶναι ποιητικὸν καὶ ὁμαλότητος καὶ τῶν τοιούτων ἀγαθῶν […] μόνον δὲ τὸν δάκτυλον καὶ ἡρῷον ἁρμόττειν παιδευομένοις καὶ ὅλως τὸν τῇ ἰσότητι κεκοσμημένον}
    \pend
    \endnumbering
    \end{Rightside}
\end{pairs}
\Columns
\numberlinetrue
\end{small}

\chapter{Test2}
\section{Prova2}

\begin{greek}
\beginnumbering
\numberpstarttrue
\labelpstarttrue

\pstart\edlabel{itm:Ri1}\textit{Titulus}] τὰ τοῦ διαλόγου πρόσωπα· \textbf{AT} Σωκράτης, Γλαύκων, Πολέμαρχος, Ἀδείμαν
τος, Κέφαλος. \textbf{A\super{2}T}\pend

\pstart \edlabel{itm:Ri2}\edtext{}{\Afootnote{{\textbf{\pstartref{itm:Ri2}}}\enspace ‒ Ἀττικῆς + λιμὴν, = \textit{Menex.} 243e4 n. 26 (TW); ἐπίνειον Ἀθηναίων, cf. \textit{Menex.} 245e8 n. 29 (TP\super{exc}W) … ἐπίνειον Κορινθίων ὥσπερ ὁ Πειραιεὺς Ἀθηναίων}}\edtext{}{\Bfootnote{{\textbf{\pstartref{itm:Ri2}}}\enspace ‒ Ἀττικῆς + λιμὴν, cf. Steph.Byz. 513.16-514.1, Lex.Rhet. 288.31, Suid. π 1455 ≈ Ps.-Hrd. \textit{Part.} p. 110, EGud. 457.24-25, Zon. 1527.11-12; ἐπίνειον Ἀθηναίων, vide ad \textit{Menex.} 29, cui adde sch. Thuc. 1.30.2 (p. 32.25-33.1; hinc Suid. ε 2489, p. 371.16-19); λιμὴν, cf. sch. Arist. \textit{Soph.El.} 1.20, p. 177b13 ἀθηναίων λιμὴν οὗτος ὁ Πειραιεύς (p. 95 Bülow-Jacobsen ‒ Ebbesen). aliter sch. (VEΓ\super{3}Θ) Ar. \textit{Eq.} 815C (II)}}327a1 \textit{Πειραιᾶ}] Πειραιεὺς πόλις Ἀττικῆς καὶ ἐπίνειον Ἀθηναίων ἤτοι λιμὴν ἢ ὁρμητήριον. \textbf{AT}\pend

\pstart\edlabel{Rvi510d5}%
    510d5 \textit{οὐκοῦν καὶ ὅτι (sine certo lemmate)}] %
    \edtext{}{\lemma{\textbf{\pstartref{Rvi510d5}}}\Cfootnote[nonum,nosep]{f. 72r A}}%
    schema \textbf{A}:
\begin{longtable}{l|l|l|l}
\multicolumn{2}{c |}{ὁρατά}   &   \multicolumn{2}{c}{νοητά} \\
\multicolumn{2}{c |}{δόξα}  &   \multicolumn{2}{c}{νόησις} \\
\multicolumn{2}{c |}{ἥλιος}   &   \multicolumn{2}{c}{τ’ ἀγαθόν} \\
\midrule
σκιῶδες & φυσικόν & μαθηματικόν & θεολογικόν \\
εἰκασία & πίστις   & διάνοια & ἐπιστήμη \\
\multicolumn{2}{c |}{γένεσις}    &   \multicolumn{2}{c}{οὐσία} 
\end{longtable}\pend

\labelpstartfalse
\numberpstartfalse
\endnumbering
\end{greek}
\end{document}
