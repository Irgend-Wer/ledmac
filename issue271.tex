\documentclass{article}
%For this case I think it's best if it ignores all the \Xend settings and just writes it all out.

\usepackage{fontspec}
\usepackage[series={A},nofamiliar,noend]{eledmac}
\usepackage{blindtext}
%\renewcommand*{\printnpnum}[1]{%
%p.~#1, l.}
\footparagraph{A}
\lineation{page}
\twolines{sq.}
%\let\Aendnote\Afootnote
%\Xendtwolines{sq.}
%\twolinesonlyinsamepage
%\Xendmorethantwolines{sqq.}
\twolinesbutnotmore
\begin{document}
\beginnumbering
\pstart
Les députés arrivèrent bientôt à Troie et n'y trouvèrent point Alexandre. Ce prince qui, dans sa fuite précipitée, avait peu consulté les vents, s'était vu forcé de relâcher en Chypre. De là, après s'être saisi de quelques vaisseaux, il avait abordé sur la côte de Phénicie. Toujours tourmenté par cette même avidité qui l'avait accompagné à Sparte, il égorge de nuit, par trahison, le roi des Sidoniens, qui lui avait fait un accueil favorable. Tout ce que renferme le palais est le prix de son crime; toutes les richesses accumulées dans ce lieu, monuments de la grandeur royale, sont par son ordre injustement enlevées et portées sur ses vaisseaux. Cependant, aux cris lamentables de ceux qui avaient échappé aux ravisseurs, le peuple se soulève, se porte en foule au palais, et, dans le moment où Alexandre, après avoir pris tout ce qui était à sa convenance, se préparait à mettre à la voile, une troupe, armée à la hâte, se présente; le combat s'engage et se poursuit avec acharnement; nombre de combattants tombent de part et d'autre; les uns s'opiniâtrent à venger la mort de leur roi, les autres à conserver leur butin. Enfin les Troyens, après avoir eu deux de leurs vaisseaux brûlés, furent assez heureux pour sauver le reste, et échappèrent ainsi à la vengeance des Sidoniens déjà fatigués du carnage.

Sur ces entrefaites, Palamède, un des députés qui s'étaient rendus à Troie, prince à qui sa valeur dans les combats et sa sagesse dans les conseils avaient mérité la plus grande confiance, se rend au palais de Priam. Là, devant le conseil assemblé, il se plaint du crime d'Alexandre, représente les droits de l'hospitalité indignement violés par lui, observe qu'une telle action est capable de réveiller la haine entre les deux nations, rappelle le souvenir des discordes qui, pour de semblables causes, divisèrent jadis les maisons d'llus et de Pélops, et d'autres familles encore, discordes qui ont entraîné les peuples dans des guerres désastreuses. Il met sous les yeux de Priam les dangers et l'incertitude des combats, les avantages et les douceurs de la paix, l'assure qu'un forfait aussi odieux ne manquera pas d'exciter l'indignation de toute la terre, de priver ses auteurs de tout secours humain, et de les conduire à une perte inévitable, digne récompense de leur détestable impiété. Il se préparait à continuer lorsque Priam l'interrompant, lui dit:
« Modérez-vous, je vous prie, Palamède; il n'est pas juste d'accuser un absent. Il peut bien arriver que ce grand crime dont on le charge soit suffisamment détruit dans sa réplique lorsqu'il sera présent. »
Sous ce prétexte et d'autres semblables, il ordonne de suspendre l'examen de l'affaire jusqu'à l'arrivée d'Alexandre. Il voyait bien, par l'effet du discours de Palamède sur chacun des conseillers, que l'on condamnait généralement, sans ces pendant oser rien dire, d'action de son fils. En effet, le prince grec avait exposé ses plaintes avec un art admirable; il avait répandu dans son discours un intérêt touchant bien capable de produire l'effet désiré. L'assemblée se sépara ainsi ce jour-là. Ensuite Anténor, homme généreux, et surtout ami de la justice et de la vertu, conduisit dans son palais les députés, qui l'y suivirent avec joie.

\edtext{Peu de jours après, le fils de Priam et ses compagnons arrivèrent, amenant avec eux la belle Hélène. Son retour mit la ville en mouvement. Les uns avaient l'action d'Alexandre en horreur ; \edtext{les autres s'attendrissaient sur Ménélas, qui en était la victime.}{\Afootnote{sur 2 lignes, en 2 pages ≠}} Tous étaient indignés, et personne ne cherchait à défendre le ravisseur. Priam, inquiet, appelle ses fils auprès de lui, les consulte sur ce qu'il doit faire dans une telle conjoncture : ils sont tous d'avis de ne point rendre Hélène. La vue des richesses qu'on avait enlevées avec elle les éblouissait, et ils n'ignoraient pas qu'il faudrait s'en dessaisir si on la rendait elle-même.}{\lemma{Peu\ldots elle-même}\Afootnote{sur plusieurs lignes}} Ils ne voyaient pas non plus avec indifférence les belles femmes de la suite d'Hélène, et se proposaient bien d'en faire leur conquête; car ces princes, dont les moeurs étaient aussi barbares que le langage, s'inquiétaient peu de ce qui était juste ou injuste, et ne voyaient dans cette affaire que deux objets qui partageaient également leur affection : le butin premièrement; ensuite le moyen d'assouvir leurs passions déréglées,

Priam, après cette réponse, les quitte, assemble les anciens, leur fait part de la résolution de ses fils et demande leur avis. Ceux-ci ne l'avaient pas encore donné, que les princes, sans garder aucune mesure, entrent tout-à-coup dans la salle du conseil, en menaçant chacun des assistants de leur vengeance, s'ils osent prendre le moindre arrêté contraitre à leurs intérêts. Cependant le peuple ne poutait retenir son indignation, et réclamait hautement contre l'injustice; il demandait satisfaction pour les députés, et pour lui-même la réparation des torts qu'il éprouvait journellement. Alexandre, toujours aveuglé par sa passion, et craignant tout d'un peuple irrité, sort accompagné de ses frères, les armes à la main, se jette au milieu de la multitude, et en fait un affreux carnage. Ce qui reste est sauvé par l'intervention des grands qui avaient assisté au conseil, et par Anténor, qui s'était mis â leur tête. Ainsi le peuple se retira méprisé, maltraité, et sans avoir rien obtenu.

Le lendemain, le roi, à la prière d'Hécube, se rend chez Hélène, la salue avec bonté, l'exhorte à prendre courage, et lui fait plusieurs questions sur son état et sur sa naissance. La princesse lui répondit que des liens de parenté l'unissaient â Alexandre, qu'elle appartenait plus à Priam et à Hécube qu'aux fils de Plisthène; et reprenant son origine de plus haut, elle dit que Danaüs et Agénor étaient leurs communs auteurs; que de Pléione, fille de Danaüs et d'Atlas, naquit Électre, qui, enceinte de Jupiter, avait mis au monde Dardanus, duquel sortirent Tros et les autres rois de Troie; que d'un autre côté, Taygète, fille d'Agénor, avait eu de Jupiter Lacédémon, père d'Amiclas ; que celui-ci donna le jour à Argalus, père d'Oebalus, qui engendra Tyndare, dont elle était la fille. Elle allégua aussi les liens qui l'unissaient à Hécube par Agénor, père de Phinée et de Phénice, aïeuls d'Hécube et de Léda, sa mère. Après avoir ainsi établi sa généalogie, elle conjura Priam et Hécube, les larmes aux yeux, de ne la point rendre aux Grecs après l'avoir prise sous leur protection. Elle ajouta que les richesses qui avaient été tirées du palais de Ménélas lui appartenaient, et qu'elle n'avait rien pris au-delà. On ne sait pas au juste si sa réponse lui fut inspirée par son amour pour Alexandre, ou par la crainte d'être punie un jour par son mari à cause de sa désertion.
\pend
\endnumbering

\newpage
%\doendnotes{A}
\end{document}
