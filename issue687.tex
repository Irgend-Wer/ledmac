% !TeX program = xelatex
% !TeX encoding = UTF-8
% !TeX spellcheck = it_IT

\documentclass[11pt,b5paper]{book}
\usepackage{libertine}

\usepackage[noeledsec,noledgroup,series={A,B,C}]{reledmac}
\usepackage[continuousnumberingwithcolumns]{reledpar}
	\lineation*{page}
	\linenummargin*{inner}
	\sidenotemargin*{outer}
	%\linenumOnlyPagesForColumns{right}
	%\linenumOnlyPagesForColumnsR{left}
\firstlinenum*{1}
\linenumincrement*{1}
\usepackage{lipsum}

\begin{document}
\beginnumbering
\numberpstarttrue
	
\pstart{}XXXXX\pend

\pstart{}XXXXX\pend

\pstart{}XXXXX\pend

\pstart{}XXXXX\pend

\pstart{}XXXXX\pend

\pstart{}XXXXX\pend

\pstart{}XXXXX\pend

\pstart{}XXXXX\pend

\pstart{}XXXXX\pend

\pstart{}XXXXX\pend

\pstart{}XXXXX\pend

\pstart{}XXXXX\pend

\pstart{}XXXXX\pend

\pstart{}XXXXX\pend

\pstart{}XXXXX\pend

\pstart{}XXXXX\pend

\pstart{}XXXXX\pend

\pstart{}XXXXX\pend

\pstart{}XXXXX\pend

\pstart{}XXXXX\pend

\pausenumbering
\begin{pairs}
\begin{Leftside}
\resumenumbering
\numberpstartfalse
\pstart\noindent
\textbf{a)}
\edtext{Ῥαδαμάνθυος ὅρκος}%
	{\lemma{Ῥαδαμάνθυος ὅρκοι}\Cfootnote[nosep]{sch. \textit{Reip.}}}
\edtext{οὗτος}%
        {\Cfootnote[nosep]{om. P, scholl. \textit{Phaedr.} et \textit{Reip.}, Phot. Suid.}}
ὁ \edtext{κατὰ χηνὸς ἢ κυνὸς}%
        {\Cfootnote[nosep]{TW, Phot. Suid.: κατὰ κυνὸς ἢ χηνὸς
            scholl. \textit{Phaedr.} et \textit{Reip.}}}
\edtext{ἢ πλατάνου ἢ κριοῦ}%
        {\Cfootnote[nosep]{TW, sch. \textit{Phaedr.}, Phot. Suid.: ἢ
            κριοῦ ἢ πλατάνου sch. \textit{Reip.}}}
        ἤ τινος ἄλλου τοιούτου.
“\edtext{οἷς ἦν}%
        {\Cfootnote[]{οἷς οὖν ἦν sch. \textit{Phaedr.} (sed revera οἷς
            οὖν T: εἷς οὖν W)}}
μέγιστος ὅρκος
\edtext{ἅπαντι λόγῳ}%
        {\Cfootnote[nosep]{vix tetrametro iambico conveniens recepit
            Meineke, qui in commentario suo hos versus ex lyrico
            carmine ductos, posito prioris versus exitu post λόγῳ,
            censuit (huic opinioni non favit Kock): intra cruces
            Kassel--Austin: ἐν παντὶ λόγῳ sch. \textit{Phaedr.}}}
        κύων, ἔπειτα χήν,
\edtext{θεοὺς δ' ἐσίγων}{\Cfootnote[]{καὶ τἆλλα
    sch. \textit{Phaedr.}}}”· Κρατῖνος Χείρωσι (Cratinus, PCG IV
fr. 249).
\edtext{τοιοῦτοι δὲ καὶ οἱ Σωκράτους ὅρκοι}{\lemma{τοιοῦτοι~\dots{}~ὅρκοι}\Cfootnote[]{κατὰ τούτων
    δὲ νόμος ὀμνύναι, ἵνα μὴ κατὰ θεῶν οἱ ὅρκοι γίγνωνται
    sch. \textit{Phaedr.}}}. \textbf{TW}
\pend
\pausenumbering
\end{Leftside}

\begin{Rightside}
\beginnumbering
\numberpstartfalse
\pstart\noindent
\ledsidenote{Σ?}\textbf{b)} Ῥαδαμάνθυος ὅρκος ὁ κατὰ χηνὸς καὶ κυνὸς καὶ πλατάνου καὶ
τῶν τοιούτων. \textbf{P}
\pend
\pausenumbering
\end{Rightside}
\end{pairs}
\Columns

\numberpstarttrue
\resumenumbering

\pstart{}XXXXX\pend

\pstart{}XXXXX\pend

\pstart{}XXXXX\pend

\pstart{}XXXXX\pend

\pstart{}XXXXX\pend

\pstart{}XXXXX\pend

\pstart{}XXXXX\pend

\pstart{}XXXXX\pend

\pstart{}XXXXX\pend

\pstart{}XXXXX\pend

\pstart{}XXXXX\pend

\pstart{}XXXXX\pend

\pstart{}XXXXX\pend

\pstart{}XXXXX\pend

\pstart{}XXXXX\pend

\pstart{}XXXXX\pend

\pstart{}XXXXX\pend

\pstart{}XXXXX\pend

\pstart{}XXXXX\pend

\pausenumbering
\begin{pairs}
\begin{Leftside}
\resumenumbering
\numberpstartfalse
\pstart\noindent
\ledsidenote{Σ?}\textbf{a)}
\edtext{Ῥαδαμάνθυος ὅρκος}%
	{\lemma{Ῥαδαμάνθυος ὅρκοι}\Cfootnote[nosep]{sch. \textit{Reip.}}}
\edtext{οὗτος}%
        {\Cfootnote[nosep]{om. P, scholl. \textit{Phaedr.} et \textit{Reip.}, Phot. Suid.}}
ὁ \edtext{κατὰ χηνὸς ἢ κυνὸς}%
        {\Cfootnote[nosep]{TW, Phot. Suid.: κατὰ κυνὸς ἢ χηνὸς
            scholl. \textit{Phaedr.} et \textit{Reip.}}}
\edtext{ἢ πλατάνου ἢ κριοῦ}%
        {\Cfootnote[nosep]{TW, sch. \textit{Phaedr.}, Phot. Suid.: ἢ
            κριοῦ ἢ πλατάνου sch. \textit{Reip.}}}
        ἤ τινος ἄλλου τοιούτου.
“\edtext{οἷς ἦν}%
        {\Cfootnote[]{οἷς οὖν ἦν sch. \textit{Phaedr.} (sed revera οἷς
            οὖν T: εἷς οὖν W)}}
μέγιστος ὅρκος
\edtext{ἅπαντι λόγῳ}%
        {\Cfootnote[nosep]{vix tetrametro iambico conveniens recepit
            Meineke, qui in commentario suo hos versus ex lyrico
            carmine ductos, posito prioris versus exitu post λόγῳ,
            censuit (huic opinioni non favit Kock): intra cruces
            Kassel--Austin: ἐν παντὶ λόγῳ sch. \textit{Phaedr.}}}
        κύων, ἔπειτα χήν,
\edtext{θεοὺς δ' ἐσίγων}{\Cfootnote[]{καὶ τἆλλα
    sch. \textit{Phaedr.}}}”· Κρατῖνος Χείρωσι (Cratinus, PCG IV
fr. 249).
\edtext{τοιοῦτοι δὲ καὶ οἱ Σωκράτους ὅρκοι}{\lemma{τοιοῦτοι~\dots{}~ὅρκοι}\Cfootnote[]{κατὰ τούτων
    δὲ νόμος ὀμνύναι, ἵνα μὴ κατὰ θεῶν οἱ ὅρκοι γίγνωνται
    sch. \textit{Phaedr.}}}. \textbf{TW}
\pend
\pausenumbering
\end{Leftside}

\begin{Rightside}
\resumenumbering
\numberpstartfalse
\pstart\noindent
\textbf{b)} Ῥαδαμάνθυος ὅρκος ὁ κατὰ χηνὸς καὶ κυνὸς καὶ πλατάνου καὶ
τῶν τοιούτων. \textbf{P}
\pend
\pausenumbering
\end{Rightside}
\end{pairs}
\Columns

\numberpstarttrue
\resumenumbering

\pstart{}XXXXX\pend

\pstart{}XXXXX\pend

\pstart{}XXXXX\pend

\pstart{}XXXXX\pend

\pstart{}XXXXX\pend

\numberpstartfalse
\endnumbering
\end{document}