\documentclass{article}
\usepackage{fontspec,xunicode}
\usepackage{libertineotf}

\usepackage{eledmac}
\usepackage[]{eledpar}
\renewcommand*{\goalfraction}{0.98}

\usepackage{polyglossia}
\setmainlanguage{latin}
\setotherlanguage{french}
\setotherlanguage{english}

%http://remacle.org/bloodwolf/historiens/dictys/troie.htm

\maxhXnotes{0.25\textwidth}
\onlyXside{L}
\begin{document}
\date{}
\begin{english}
\title{Multiple blank pages}
\maketitle
\begin{abstract}
If you have multiple \verb+\Pages+ with the latest version of eledmac (mac1.19.0-par1.13.0) you'll experience two blank pages at the end of the sections. With the older version (e.g. mac1.18.0-par1.12.1) there's no such a problem.
\end{abstract}

\end{english}



\begin{pages}
\begin{Leftside}
\begin{french}
\beginnumbering
\autopar

Peu de jours après, le fils de Priam et ses compagnons arrivèrent, amenant avec eux la belle Hélène. Son retour mit la ville en mouvement. Les uns avaient l'action d'Alexandre en horreur ; les autres s'attendrissaient sur Ménélas, qui en était la victime. Tous étaient indignés, et personne ne cherchait à défendre le ravisseur. Priam, inquiet, appelle ses fils auprès de lui, les consulte sur ce qu'il doit faire dans une telle conjoncture : ils sont tous d'avis de ne point rendre Hélène. La vue des richesses qu'on avait enlevées avec elle les éblouissait, et ils n'ignoraient pas qu'il faudrait s'en dessaisir si on la rendait elle-même. Ils ne voyaient pas non plus avec indifférence les belles femmes de la suite d'Hélène, et se proposaient bien d'en faire leur conquête; car ces princes, dont les moeurs étaient aussi barbares que le langage, s'inquiétaient peu de ce qui était juste ou injuste, et ne voyaient dans cette affaire que deux objets qui partageaient également leur affection : le butin premièrement; ensuite le moyen d'assouvir leurs passions déréglées,

Priam, après cette réponse, les quitte, assemble les anciens, leur fait part de la résolution de ses fils et demande leur avis. Ceux-ci ne l'avaient pas encore donné, que les princes, sans garder aucune mesure, entrent tout-à-coup dans la salle du conseil, en menaçant chacun des assistants de leur vengeance, s'ils osent prendre le moindre arrêté contraitre à leurs intérêts. Cependant le peuple ne poutait retenir son indignation, et réclamait hautement contre l'injustice; il demandait satisfaction pour les députés, et pour lui-même la réparation des torts qu'il éprouvait journellement. Alexandre, toujours aveuglé par sa passion, et craignant tout d'un peuple irrité, sort accompagné de ses frères, les armes à la main, se jette au milieu de la multitude, et en fait un affreux carnage. Ce qui reste est sauvé par l'intervention des grands qui avaient assisté au conseil, et par Anténor, qui s'était mis â leur tête. Ainsi le peuple se retira méprisé, maltraité, et sans avoir rien obtenu.

Le lendemain, le roi, à la prière d'Hécube, se rend chez Hélène, la salue avec bonté, l'exhorte à prendre courage, et lui fait plusieurs questions sur son état et sur sa naissance. La princesse lui répondit que des liens de parenté l'unissaient â Alexandre, qu'elle appartenait plus à Priam et à Hécube qu'aux fils de Plisthène; et reprenant son origine de plus haut, elle dit que Danaüs et Agénor étaient leurs communs auteurs; que de Pléione, fille de Danaüs et d'Atlas, naquit Électre, qui, enceinte de Jupiter, avait mis au monde Dardanus, duquel sortirent Tros et les autres rois de Troie; que d'un autre côté, Taygète, fille d'Agénor, avait eu de Jupiter Lacédémon, père d'Amiclas ; que celui-ci donna le jour à Argalus, père d'Oebalus, qui engendra Tyndare, dont elle était la fille. Elle allégua aussi les liens qui l'unissaient à Hécube par Agénor, père de Phinée et de Phénice, aïeuls d'Hécube et de Léda, sa mère. Après avoir ainsi établi sa généalogie, elle conjura Priam et Hécube, les larmes aux yeux, de ne la point rendre aux Grecs après l'avoir prise sous leur protection. Elle ajouta que les richesses qui avaient été tirées du palais de Ménélas lui appartenaient, et qu'elle n'avait rien pris au-delà. On ne sait pas au juste si sa réponse lui fut inspirée par son amour pour Alexandre, ou par la crainte d'être punie un jour par son mari à cause de sa désertion.

Hécube, qui connaissait son désir, et voyait en elle une parente, la tenait serrée contre son sein, et suppliait son époux de ne la point rendre. Cependant Priam et les princes étaient revenus à un meilleur avis ; ils insistaient pour qu'on renvoyât la députation avec une réponse favorable, et craignaient déjà de résister à la volonté du peuple : le seul Deiphobe appuyait Hécube, sans doute parce qu'il était épris de la même passion qu'Alexandre pour la beauté d'Hélène. Hécube, de son côté, s'adressait tantôt à Priam, tantôt à ses fils, et tantôt embrassant la princesse, elle jurait que rien ne pourrait l'en séparer. De cette manière, elle entraîna à son avis tous les assistants, et les caresses d'une mère triomphèrent enfin du bonheur public. Le jour suivant, Ménélas et ses collègues se rendirent à l'assemblée, redemandant Hélène, et avec elle tentes les richesses qui avaient été enlevées. Alors Priam, debout et entouré des princes ses fils, commande le silence; il prie Hélène, qui était présente, de choisir elle-même, et de déclarer si elle voulait retourner à Sparte ou demeurer à Troie. La princesse, dit-on, fit réponse qu'elle ne voulait ni revoir sa patrie, ni rester unie à Ménélas. Ainsi les princes sortent du conseil triomphants et joyeux de posséder \edtext{Hélène}{\Afootnote{{Hécube, qui connaissait son désir, et voyait en elle une parente, la tenait serrée contre son sein, et suppliait son époux de ne la point rendre. Cependant Priam et les princes étaient revenus à un meilleur avis ; ils insistaient pour qu'on renvoyât la députation avec une réponse favorable, et craignaient déjà de résister à la volonté du peuple : le seul Deiphobe appuyait Hécube xxxx, sans doute parce qu'il était épris de la même passion qu'Alexandre pour la beauté d'Hélène. Hécube, de son côté, s'adressait tantôt à Priam, tantôt à ses fils, et tantôt embrassant la princesse, elle jurait que rien ne pourrait l'en séparer. De cette manière, elle entraîna à son avis tous les assistants, et les caresses d'une mère triomphèrent enfin du bonheur public. Le jour suivant, Ménélas et ses collègues se rendirent à l'assemblée, redemandant Hélène, et avec elle tentes les richesses qui avaient été enlevées. Alors Priam, debout et entouré des princes ses fils, commande le silence; il prie Hélène, qui était présente, de choisir elle-même, et de déclarer si elle voulait retourner à Sparte ou demeurer à Troie. La princesse, dit-on, fit réponse qu'elle ne voulait ni revoir sa patrie, ni rester unie à Ménélas. Ainsi les princes sortent du conseil triomphants et joyeux de posséderHécube, qui connaissait son désir, et voyait en elle une parente, la tenait serrée contre son sein, et suppliait son époux de ne la point rendre. Cependant Priam et les princes étaient revenus à un meilleur avis ; ils insistaient pour qu'on renvoyât la députation avec une réponse favorable, et craignaient déjà de résister à la volonté du peuple : le seul Deiphobe XXXXX appuyait Hécube, sans doute parce qu'il était épris de la même passion qu'Alexandre pour la beauté d'Hélène. Hécube, de son côté, s'adressait tantôt à Priam, tantôt à ses fils, et tantôt embrassant la princesse, elle jurait que rien ne pourrait l'en séparer. De cette manière, elle entraîna à son avis tous les assistants, et les caresses d'une mère triomphèrent enfin du bonheur public. Le jour suivant, Ménélas et ses collègues se rendirent à l'assemblée, redemandant Hélène, et avec elle tentes les richesses qui avaient été enlevées. Alors Priam, debout et entouré des princes ses fils, commande le silence; il prie Hélène, qui était présente, de choisir elle-même, et de déclarer si elle voulait retourner à Sparte ou demeurer à Troie. La princesse, dit-on, fit réponse qu'elle ne voulait ni revoir sa patrie, ni rester unie à Ménélas. Ainsi les princes sortent du conseil triomphants et joyeux de posséderHécube, qui connaissait son désir, et voyait en elle une parente, la tenait serrée contre son sein, et suppliait son époux de ne la point rendre. Cependant Priam et les princes étaient revenus à un meilleur avis ; ils insistaient pour qu'on renvoyât la députation avec une réponse favorable, et craignaient déjà de résister à la volonté du peuple : le seul Deiphobe appuyait Hécube, sans doute parce qu'il était épris de la même passion qu'Alexandre pour la beauté d'Hélène. Hécube, de son côté, s'adressait tantôt à Priam, tantôt à ses fils, et tantôt embrassant la princesse, elle jurait que rien ne pourrait l'en séparer. De cette manière, elle entraîna à son avis tous les assistants, et les caresses d'une mère triomphèrent enfin du bonheur public. Le jour suivant, Ménélas et ses collègues se rendirent à l'assemblée, redemandant Hélène, et avec elle tentes les richesses qui avaient été enlevées. Alors Priam, debout et entouré des princes ses fils, commande le silence; il prie Hélène, qui était présente, de choisir elle-même, et de déclarer si elle voulait retourner à Sparte ou demeurer à Troie. La princesse, dit-on, fit réponse qu'elle ne voulait ni revoir sa patrie, ni rester unie à Ménélas. Ainsi les princes sortent du conseil triomphants et joyeux de posséderHécube, qui connaissait son désir, et voyait en elle une parente, la tenait serrée contre son sein, et suppliait son époux de ne la point rendre. Cependant Priam et les princes étaient revenus à un meilleur avis ; ils insistaient pour qu'on renvoyât la députation avec une réponse favorable, et craignaient déjà de résister à la volonté du peuple : le seul Deiphobe appuyait Hécube, sans doute parce qu'il était épris de la même passion qu'Alexandre pour la beauté d'Hélène. Hécube, de son côté, s'adressait tantôt à Priam, tantôt à ses fils, et tantôt embrassant la princesse, elle jurait que rien ne pourrait l'en séparer. De cette manière, elle entraîna à son avis tous les assistants, et les caresses d'une mère triomphèrent enfin du bonheur public. Le jour suivant, Ménélas et ses collègues se rendirent à l'assemblée, redemandant Hélène, et avec elle tentes les richesses qui avaient été enlevées. Alors Priam, debout et entouré des princes ses fils, commande le silence; il prie Hélène, qui était présente, de choisir elle-même, et de déclarer si elle voulait retourner à Sparte ou demeurer à Troie. La princesse, dit-on, fit réponse qu'elle ne voulait ni revoir sa patrie, ni rester unie à Ménélas. Ainsi les princes sortent du conseil triomphants et joyeux de posséderHécube, qui connaissait son désir, et voyait en elle une parente, la tenait serrée contre son sein, et suppliait son époux de ne la point rendre. Cependant Priam et les princes étaient revenus à un meilleur avis ; ils insistaient pour qu'on renvoyât la députation avec une réponse favorable, et craignaient déjà de résister à la volonté du peuple : le seul Deiphobe appuyait Hécube, sans doute parce qu'il était épris de la même passion qu'Alexandre pour la beauté d'Hélène. Hécube, de son côté, s'adressait tantôt à Priam, tantôt à ses fils, et tantôt embrassant la princesse, elle jurait que rien ne pourrait l'en séparer. De cette manière, elle entraîna à son avis tous les assistants, et les caresses d'une mère triomphèrent enfin du bonheur public. Le jour suivant, Ménélas et ses collègues se rendirent à l'assemblée, redemandant Hélène, et avec elle tentes les richesses qui avaient été enlevées. Alors Priam, debout et entouré des princes ses fils, commande le silence; il prie Hélène, qui était présente, de choisir elle-même, et de déclarer si elle voulait retourner à Sparte ou demeurer à Troie. La princesse, dit-on, fit réponse qu'elle ne voulait ni revoir sa patrie, ni rester unie à Ménélas. Ainsi les princes sortent du conseil triomphants et joyeux de posséder}}}.
Hécube, qui connaissait son désir, et voyait en elle une parente, la tenait serrée contre son sein, et suppliait son époux de ne la point rendre. Cependant Priam et les princes étaient revenus à un meilleur avis ; ils insistaient pour qu'on renvoyât la députation avec une réponse favorable, et craignaient déjà de résister à la volonté du peuple : le seul Deiphobe appuyait Hécube, sans doute parce qu'il était épris de la même passion qu'Alexandre pour la beauté d'Hélène. Hécube, de son côté, s'adressait tantôt à Priam, tantôt à ses fils, et tantôt embrassant la princesse, elle jurait que rien ne pourrait l'en séparer. De cette manière, elle entraîna à son avis tous les assistants, et les caresses d'une mère triomphèrent enfin du bonheur public. Le jour suivant, Ménélas et ses collègues se rendirent à l'assemblée, redemandant Hélène, et avec elle tentes les richesses qui avaient été enlevées. Alors Priam, debout et entouré des princes ses fils, commande le silence; il prie Hélène, qui était présente, de choisir elle-même, et de déclarer si elle voulait retourner à Sparte ou demeurer à Troie. La princesse, dit-on, fit réponse qu'elle ne voulait ni revoir sa patrie, ni rester unie à Ménélas. Ainsi les princes sortent du conseil triomphants et joyeux de posséder
Hécube, qui connaissait son désir, et voyait en elle une parente, la tenait serrée contre son sein, et suppliait son époux de ne la point rendre. Cependant Priam et les princes étaient revenus à un meilleur avis ; ils insistaient pour qu'on renvoyât la députation avec une réponse favorable, et craignaient déjà de résister à la volonté du peuple : le seul Deiphobe appuyait Hécube, sans doute parce qu'il était épris de la même passion qu'Alexandre pour la beauté d'Hélène. Hécube, de son côté, s'adressait tantôt à Priam, tantôt à ses fils, et tantôt embrassant la princesse, elle jurait que rien ne pourrait l'en séparer. De cette manière, elle entraîna à son avis tous les assistants, et les caresses d'une mère triomphèrent enfin du bonheur public. Le jour suivant, Ménélas et ses collègues se rendirent à l'assemblée, redemandant Hélène, et avec elle tentes les richesses qui avaient été enlevées. Alors Priam, debout et entouré des princes ses fils, commande le silence; il prie Hélène, qui était présente, de choisir elle-même, et de déclarer si elle voulait retourner à Sparte ou demeurer à Troie. La princesse, dit-on, fit réponse qu'elle ne voulait ni revoir sa patrie, ni rester unie à Ménélas. Ainsi les princes sortent du conseil triomphants et joyeux de posséder
Hécube, qui connaissait son désir, et voyait en elle une parente, la tenait serrée contre son sein, et suppliait son époux de ne la point rendre. Cependant Priam et les princes étaient revenus à un meilleur avis ; ils insistaient pour qu'on renvoyât la députation avec une réponse favorable, et craignaient déjà de résister à la volonté du peuple : le seul Deiphobe appuyait Hécube, sans doute parce qu'il était épris de la même passion qu'Alexandre pour la beauté d'Hélène. Hécube, de son côté, s'adressait tantôt à Priam, tantôt à ses fils, et tantôt embrassant la princesse, elle jurait que rien ne pourrait l'en séparer. De cette manière, elle entraîna à son avis tous les assistants, et les caresses d'une mère triomphèrent enfin du bonheur public. Le jour suivant, Ménélas et ses collègues se rendirent à l'assemblée, redemandant Hélène, et avec elle tentes les richesses qui avaient été enlevées. Alors Priam, debout et entouré des princes ses fils, commande le silence; il prie Hélène, qui était présente, de choisir elle-même, et de déclarer si elle voulait retourner à Sparte ou demeurer à Troie. La princesse, dit-on, fit réponse qu'elle ne voulait ni revoir sa patrie, ni rester unie à Ménélas. Ainsi les princes sortent du conseil triomphants et joyeux de posséder
Hécube, qui connaissait son désir, et voyait en elle une parente, la tenait serrée contre son sein, et suppliait son époux de ne la point rendre. Cependant Priam et les princes étaient revenus à un meilleur avis ; ils insistaient pour qu'on renvoyât la députation avec une réponse favorable, et craignaient déjà de résister à la volonté du peuple : le seul Deiphobe appuyait Hécube, sans doute parce qu'il était épris de la même passion qu'Alexandre pour la beauté d'Hélène. Hécube, de son côté, s'adressait tantôt à Priam, tantôt à ses fils, et tantôt embrassant la princesse, elle jurait que rien ne pourrait l'en séparer. De cette manière, elle entraîna à son avis tous les assistants, et les caresses d'une mère triomphèrent enfin du bonheur public. Le jour suivant, Ménélas et ses collègues se rendirent à l'assemblée, redemandant Hélène, et avec elle tentes les richesses qui avaient été enlevées. Alors Priam, debout et entouré des princes ses fils, commande le silence; il prie Hélène, qui était présente, de choisir elle-même, et de déclarer si elle voulait retourner à Sparte ou demeurer à Troie. La princesse, dit-on, fit réponse qu'elle ne voulait ni revoir sa patrie, ni rester unie à Ménélas. Ainsi les princes sortent du conseil triomphants et joyeux de posséder
Hécube, qui connaissait son désir, et voyait en elle une parente, la tenait serrée contre son sein, et suppliait son époux de ne la point rendre. Cependant Priam et les princes étaient revenus à un meilleur avis ; ils insistaient pour qu'on renvoyât la députation avec une réponse favorable, et craignaient déjà de résister à la volonté du peuple : le seul Deiphobe appuyait Hécube, sans doute parce qu'il était épris de la même passion qu'Alexandre pour la beauté d'Hélène. Hécube, de son côté, s'adressait tantôt à Priam, tantôt à ses fils, et tantôt embrassant la princesse, elle jurait que rien ne pourrait l'en séparer. De cette manière, elle entraîna à son avis tous les assistants, et les caresses d'une mère triomphèrent enfin du bonheur public. Le jour suivant, Ménélas et ses collègues se rendirent à l'assemblée, redemandant Hélène, et avec elle tentes les richesses qui avaient été enlevées. Alors Priam, debout et entouré des princes ses fils, commande le silence; il prie Hélène, qui était présente, de choisir elle-même, et de déclarer si elle voulait retourner à Sparte ou demeurer à Troie. La princesse, dit-on, fit réponse qu'elle ne voulait ni revoir sa patrie, ni rester unie à Ménélas. Ainsi les princes sortent du conseil triomphants et joyeux de posséderHécube, qui connaissait son désir, et voyait en elle une parente, la tenait serrée contre son sein, et suppliait son époux de ne la point rendre. Cependant Priam et les princes étaient revenus à un meilleur avis ; ils insistaient pour qu'on renvoyât la députation avec une réponse favorable, et craignaient déjà de résister à la volonté du peuple : le seul Deiphobe appuyait Hécube, sans doute parce qu'il était épris de la même passion qu'Alexandre pour la beauté d'Hélène. Hécube, de son côté, s'adressait tantôt à Priam, tantôt à ses fils, et tantôt embrassant la princesse, elle jurait que rien ne pourrait l'en séparer. De cette manière, elle entraîna à son avis tous les assistants, et les caresses d'une mère triomphèrent enfin du bonheur public. Le jour suivant, Ménélas et ses collègues se rendirent à l'assemblée, redemandant Hélène, et avec elle tentes les richesses qui avaient été enlevées. Alors Priam, debout et entouré des princes ses fils, commande le silence; il prie Hélène, qui était présente, de choisir elle-même, et de déclarer si elle voulait retourner à Sparte ou demeurer à Troie. La princesse, dit-on, fit réponse qu'elle ne voulait ni revoir sa patrie, ni rester unie à Ménélas. Ainsi les princes sortent du conseil triomphants et joyeux de posséderHécube, qui connaissait son désir, et voyait en elle une parente, la tenait serrée contre son sein, et suppliait son époux de ne la point rendre. Cependant Priam et les princes étaient revenus à un meilleur avis ; ils insistaient pour qu'on renvoyât la députation avec une réponse favorable, et craignaient déjà de résister à la volonté du peuple : le seul Deiphobe appuyait Hécube, sans doute parce qu'il était épris de la même passion qu'Alexandre pour la beauté d'Hélène. Hécube, de son côté, s'adressait tantôt à Priam, tantôt à ses fils, et tantôt embrassant la princesse, elle jurait que rien ne pourrait l'en séparer. De cette manière, elle entraîna à son avis tous les assistants, et les caresses d'une mère triomphèrent enfin du bonheur public. Le jour suivant, Ménélas et ses collègues se rendirent à l'assemblée, redemandant Hélène, et avec elle tentes les richesses qui avaient été enlevées. Alors Priam, debout et entouré des princes ses fils, commande le silence; il prie Hélène, qui était présente, de choisir elle-même, et de déclarer si elle voulait retourner à Sparte ou demeurer à Troie. La princesse, dit-on, fit réponse qu'elle ne voulait ni revoir sa patrie, ni rester unie à Ménélas. Ainsi les princes sortent du conseil triomphants et joyeux de posséderHécube, qui connaissait son désir, et voyait en elle une parente, la tenait serrée contre son sein, et suppliait son époux de ne la point rendre. Cependant Priam et les princes étaient revenus à un meilleur avis ; ils insistaient pour qu'on renvoyât la députation avec une réponse favorable, et craignaient déjà de résister à la volonté du peuple : le seul Deiphobe appuyait Hécube, sans doute parce qu'il était épris de la même passion qu'Alexandre pour la beauté d'Hélène. Hécube, de son côté, s'adressait tantôt à Priam, tantôt à ses fils, et tantôt embrassant la princesse, elle jurait que rien ne pourrait l'en séparer. De cette manière, elle entraîna à son avis tous les assistants, et les caresses d'une mère triomphèrent enfin du bonheur public. Le jour suivant, Ménélas et ses collègues se rendirent à l'assemblée, redemandant Hélène, et avec elle tentes les richesses qui avaient été enlevées. Alors Priam, debout et entouré des princes ses fils, commande le silence; il prie Hélène, qui était présente, de choisir elle-même, et de déclarer si elle voulait retourner à Sparte ou demeurer à Troie. La princesse, dit-on, fit réponse qu'elle ne voulait ni revoir sa patrie, ni rester unie à Ménélas. Ainsi les princes sortent du conseil triomphants et joyeux de posséderHécube, qui connaissait son désir, et voyait en elle une parente, la tenait serrée contre son sein, et suppliait son époux de ne la point rendre. Cependant Priam et les princes étaient revenus à un meilleur avis ; ils insistaient pour qu'on renvoyât la députation avec une réponse favorable, et craignaient déjà de résister à la volonté du peuple : le seul Deiphobe appuyait Hécube, sans doute parce qu'il était épris de la même passion qu'Alexandre pour la beauté d'Hélène. Hécube, de son côté, s'adressait tantôt à Priam, tantôt à ses fils, et tantôt embrassant la princesse, elle jurait que rien ne pourrait l'en séparer. De cette manière, elle entraîna à son avis tous les assistants, et les caresses d'une mère triomphèrent enfin du bonheur public. Le jour suivant, Ménélas et ses collègues se rendirent à l'assemblée, redemandant Hélène, et avec elle tentes les richesses qui avaient été enlevées. Alors Priam, debout et entouré des princes ses fils, commande le silence; il prie Hélène, qui était présente, de choisir elle-même, et de déclarer si elle voulait retourner à Sparte ou demeurer à Troie. La princesse, dit-on, fit réponse qu'elle ne voulait ni revoir sa patrie, ni rester unie à Ménélas. Ainsi les princes sortent du conseil triomphants et joyeux de posséderHécube, qui connaissait son désir, et voyait en elle une parente, la tenait serrée contre son sein, et suppliait son époux de ne la point rendre. Cependant Priam et les princes étaient revenus à un meilleur avis ; ils insistaient pour qu'on renvoyât la députation avec une réponse favorable, et craignaient déjà de résister à la volonté du peuple : le seul Deiphobe appuyait Hécube, sans doute parce qu'il était épris de la même passion qu'Alexandre pour la beauté d'Hélène. Hécube, de son côté, s'adressait tantôt à Priam, tantôt à ses fils, et tantôt embrassant la princesse, elle jurait que rien ne pourrait l'en séparer. De cette manière, elle entraîna à son avis tous les assistants, et les caresses d'une mère triomphèrent enfin du bonheur public. Le jour suivant, Ménélas et ses collègues se rendirent à l'assemblée, redemandant Hélène, et avec elle tentes les richesses qui avaient été enlevées. Alors Priam, debout et entouré des princes ses fils, commande le silence; il prie Hélène, qui était présente, de choisir elle-même, et de déclarer si elle voulait retourner à Sparte ou demeurer à Troie. La princesse, dit-on, fit réponse qu'elle ne voulait ni revoir sa patrie, ni rester unie à Ménélas. Ainsi les princes sortent du conseil triomphants et joyeux de posséderHécube, qui connaissait son désir, et voyait en elle une parente, la tenait serrée contre son sein, et suppliait son époux de ne la point rendre. Cependant Priam et les princes étaient revenus à un meilleur avis ; ils insistaient pour qu'on renvoyât la députation avec une réponse favorable, et craignaient déjà de résister à la volonté du peuple : le seul Deiphobe appuyait Hécube, sans doute parce qu'il était épris de la même passion qu'Alexandre pour la beauté d'Hélène. Hécube, de son côté, s'adressait tantôt à Priam, tantôt à ses fils, et tantôt embrassant la princesse, elle jurait que rien ne pourrait l'en séparer. De cette manière, elle entraîna à son avis tous les assistants, et les caresses d'une mère triomphèrent enfin du bonheur public. Le jour suivant, Ménélas et ses collègues se rendirent à l'assemblée, redemandant Hélène, et avec elle tentes les richesses qui avaient été enlevées. Alors Priam, debout et entouré des princes ses fils, commande le silence; il prie Hélène, qui était présente, de choisir elle-même, et de déclarer si elle voulait retourner à Sparte ou demeurer à Troie. La princesse, dit-on, fit réponse qu'elle ne voulait ni revoir sa patrie, ni rester unie à Ménélas. Ainsi les princes sortent du conseil triomphants et joyeux de posséderHécube, qui connaissait son désir, et voyait en elle une parente, la tenait serrée contre son sein, et suppliait son époux de ne la point rendre. Cependant Priam et les princes étaient revenus à un meilleur avis ; ils insistaient pour qu'on renvoyât la députation avec une réponse favorable, et craignaient déjà de résister à la volonté du peuple : le seul Deiphobe appuyait Hécube, sans doute parce qu'il était épris de la même passion qu'Alexandre pour la beauté d'Hélène. Hécube, de son côté, s'adressait tantôt à Priam, tantôt à ses fils, et tantôt embrassant la princesse, elle jurait que rien ne pourrait l'en séparer. De cette manière, elle entraîna à son avis tous les assistants, et les caresses d'une mère triomphèrent enfin du bonheur public. Le jour suivant, Ménélas et ses collègues se rendirent à l'assemblée, redemandant Hélène, et avec elle tentes les richesses qui avaient été enlevées. Alors Priam, debout et entouré des princes ses fils, commande le silence; il prie Hélène, qui était présente, de choisir elle-même, et de déclarer si elle voulait retourner à Sparte ou demeurer à Troie. La princesse, dit-on, fit réponse qu'elle ne voulait ni revoir sa patrie, ni rester unie à Ménélas. Ainsi les princes sortent du conseil triomphants et joyeux de posséderHécube, qui connaissait son désir, et voyait en elle une parente, la tenait serrée contre son sein, et suppliait son époux de ne la point rendre. Cependant Priam et les princes étaient revenus à un meilleur avis ; ils insistaient pour qu'on renvoyât la députation avec une réponse favorable, et craignaient déjà de résister à la volonté du peuple : le seul Deiphobe appuyait Hécube, sans doute parce qu'il était épris de la même passion qu'Alexandre pour la beauté d'Hélène. Hécube, de son côté, s'adressait tantôt à Priam, tantôt à ses fils, et tantôt embrassant la princesse, elle jurait que rien ne pourrait l'en séparer. De cette manière, elle entraîna à son avis tous les assistants, et les caresses d'une mère triomphèrent enfin du bonheur public. Le jour suivant, Ménélas et ses collègues se rendirent à l'assemblée, redemandant Hélène, et avec elle tentes les richesses qui avaient été enlevées. Alors Priam, debout et entouré des princes ses fils, commande le silence; il prie Hélène, qui était présente, de choisir elle-même, et de déclarer si elle voulait retourner à Sparte ou demeurer à Troie. La princesse, dit-on, fit réponse qu'elle ne voulait ni revoir sa patrie, ni rester unie à Ménélas. Ainsi les princes sortent du conseil triomphants et joyeux de posséderHécube, qui connaissait son désir, et voyait en elle une parente, la tenait serrée contre son sein, et suppliait son époux de ne la point rendre. Cependant Priam et les princes étaient revenus à un meilleur avis ; ils insistaient pour qu'on renvoyât la députation avec une réponse favorable, et craignaient déjà de résister à la volonté du peuple : le seul Deiphobe appuyait Hécube, sans doute parce qu'il était épris de la même passion qu'Alexandre pour la beauté d'Hélène. Hécube, de son côté, s'adressait tantôt à Priam, tantôt à ses fils, et tantôt embrassant la princesse, elle jurait que rien ne pourrait l'en séparer. De cette manière, elle entraîna à son avis tous les assistants, et les caresses d'une mère triomphèrent enfin du bonheur public. Le jour suivant, Ménélas et ses collègues se rendirent à l'assemblée, redemandant Hélène, et avec elle tentes les richesses qui avaient été enlevées. Alors Priam, debout et entouré des princes ses fils, commande le silence; il prie Hélène, qui était présente, de choisir elle-même, et de déclarer si elle voulait retourner à Sparte ou demeurer à Troie. La princesse, dit-on, fit réponse qu'elle ne voulait ni revoir sa patrie, ni rester unie à Ménélas. Ainsi les princes sortent du conseil triomphants et joyeux de posséderHécube, qui connaissait son désir, et voyait en elle une parente, la tenait serrée contre son sein, et suppliait son époux de ne la point rendre. Cependant Priam et les princes étaient revenus à un meilleur avis ; ils insistaient pour qu'on renvoyât la députation avec une réponse favorable, et craignaient déjà de résister à la volonté du peuple : le seul Deiphobe appuyait Hécube, sans doute parce qu'il était épris de la même passion qu'Alexandre pour la beauté d'Hélène. Hécube, de son côté, s'adressait tantôt à Priam, tantôt à ses fils, et tantôt embrassant la princesse, elle jurait que rien ne pourrait l'en séparer. De cette manière, elle entraîna à son avis tous les assistants, et les caresses d'une mère triomphèrent enfin du bonheur public. Le jour suivant, Ménélas et ses collègues se rendirent à l'assemblée, redemandant Hélène, et avec elle tentes les richesses qui avaient été enlevées. Alors Priam, debout et entouré des princes ses fils, commande le silence; il prie Hélène, qui était présente, de choisir elle-même, et de déclarer si elle voulait retourner à Sparte ou demeurer à Troie. La princesse, dit-on, fit réponse qu'elle ne voulait ni revoir sa patrie, ni rester unie à Ménélas. Ainsi les princes sortent du conseil triomphants et joyeux de posséderHécube, qui connaissait son désir, et voyait en elle une parente, la tenait serrée contre son sein, et suppliait son époux de ne la point rendre. Cependant Priam et les princes étaient revenus à un meilleur avis ; ils insistaient pour qu'on renvoyât la députation avec une réponse favorable, et craignaient déjà de résister à la volonté du peuple : le seul Deiphobe appuyait Hécube, sans doute parce qu'il était épris de la même passion qu'Alexandre pour la beauté d'Hélène. Hécube, de son côté, s'adressait tantôt à Priam, tantôt à ses fils, et tantôt embrassant la princesse, elle jurait que rien ne pourrait l'en séparer. De cette manière, elle entraîna à son avis tous les assistants, et les caresses d'une mère triomphèrent enfin du bonheur public. Le jour suivant, Ménélas et ses collègues se rendirent à l'assemblée, redemandant Hélène, et avec elle tentes les richesses qui avaient été enlevées. Alors Priam, debout et entouré des princes ses fils, commande le silence; il prie Hélène, qui était présente, de choisir elle-même, et de déclarer si elle voulait retourner à Sparte ou demeurer à Troie. La princesse, dit-on, fit réponse qu'elle ne voulait ni revoir sa patrie, ni rester unie à Ménélas. Ainsi les princes sortent du conseil triomphants et joyeux de posséderHécube, qui connaissait son désir, et voyait en elle une parente, la tenait serrée contre son sein, et suppliait son époux de ne la point rendre. Cependant Priam et les princes étaient revenus à un meilleur avis ; ils insistaient pour qu'on renvoyât la députation avec une réponse favorable, et craignaient déjà de résister à la volonté du peuple : le seul Deiphobe appuyait Hécube, sans doute parce qu'il était épris de la même passion qu'Alexandre pour la beauté d'Hélène. Hécube, de son côté, s'adressait tantôt à Priam, tantôt à ses fils, et tantôt embrassant la princesse, elle jurait que rien ne pourrait l'en séparer. De cette manière, elle entraîna à son avis tous les assistants, et les caresses d'une mère triomphèrent enfin du bonheur public. Le jour suivant, Ménélas et ses collègues se rendirent à l'assemblée, redemandant Hélène, et avec elle tentes les richesses qui avaient été enlevées. Alors Priam, debout et entouré des princes ses fils, commande le silence; il prie Hélène, qui était présente, de choisir elle-même, et de déclarer si elle voulait retourner à Sparte ou demeurer à Troie. La princesse, dit-on, fit réponse qu'elle ne voulait ni revoir sa patrie, ni rester unie à Ménélas. Ainsi les princes sortent du conseil triomphants et joyeux de posséderHécube, qui connaissait son désir, et voyait en elle une parente, la tenait serrée contre son sein, et suppliait son époux de ne la point rendre. Cependant Priam et les princes étaient revenus à un meilleur avis ; ils insistaient pour qu'on renvoyât la députation avec une réponse favorable, et craignaient déjà de résister à la volonté du peuple : le seul Deiphobe appuyait Hécube, sans doute parce qu'il était épris de la même passion qu'Alexandre pour la beauté d'Hélène. Hécube, de son côté, s'adressait tantôt à Priam, tantôt à ses fils, et tantôt embrassant la princesse, elle jurait que rien ne pourrait l'en séparer. De cette manière, elle entraîna à son avis tous les assistants, et les caresses d'une mère triomphèrent enfin du bonheur public. Le jour suivant, Ménélas et ses collègues se rendirent à l'assemblée, redemandant Hélène, et avec elle tentes les richesses qui avaient été enlevées. Alors Priam, debout et entouré des princes ses fils, commande le silence; il prie Hélène, qui était présente, de choisir elle-même, et de déclarer si elle voulait retourner à Sparte ou demeurer à Troie. La princesse, dit-on, fit réponse qu'elle ne voulait ni revoir sa patrie, ni rester unie à Ménélas. Ainsi les princes sortent du conseil triomphants et joyeux de posséder

\endnumbering
\end{french}
\end{Leftside}

\begin{Rightside}
\beginnumbering
\autopar

Interim paucis post diebus, Alexander cum supradictis comitibus venit, Helenam secum habens. Cujus adventu in tota civitate quum partim exemplum facinoris exsecrarentur, alii injurias in Menelaum admissas dolerent, nullo omnium adprobante, postremo cunctis indignantibus, tumultus ortus est. Queisrebus anxius Priamus filios convocat, eosque quid super tali agendum negotio videretur, consulit : qui una voce, minime reddendam Helenam, respondent. Videbant quippe quantae opes cum ea advectae essent : quae universa, si Helena traderetur, necessario amitterent. Praeterea permoti forma mulierum quae cum Helena venerant, nuptias sibi singularum jam animo destinaverant. Quippe qui lingua moribusque barbari, nihil pensi aut consulti patientes, praeda ac libidine transversi agebantur.

Igitur Priamus, relictis his, senes conducit, sententiam filiorum aperit. Dein cunctos, quid agendum sit, consulit. Sed priusquam ex more sententiae dicerentur, reguli repente concilium irrumpunt, atque inconditis moribus malum singulis minitantur, si aliter quam ipsis videretur, decernerent. Interim omnis populus indigne admissam injuriam, atque in hunc modum multa alia, cum exsecratione reclamabant. Ob quae Alexander cupidine animi praeceps, veritus ne quid adversum se a popularibus oriretur, stipatus armatis fratribus impetum in multitudinem facit, multosque obtruncat : reliqui interventu procerum, qui in consilio fuerant, duce liberantur Antenore. Ita infectis rebus, populus contemptui habitus, non sine pernicie sua, domum discedit.

Dein secuta die rex hortatu Hecubae ad Helenam adit, eamque benigne salutans, animum bonum uti gereret hortatur; quae cujusque erset, requirit. Tum illa Alexandri se adfinem respondit, magisque ad Priamum et Hecubam, quam ad Plisthenis filios genere pertinere, repetens originem omnem majorum. Danaum enim atque Agenorem, et sui et Priami generis auctores esse. Namque ex Pleiona, Danai filia et Atlante, Electram natam, quam ex Jove gravidam Dardanum genuisse, ex quo Tros, et deinceps insecuti reges llii. Agenoris porro Taygetam; eam ex Jove habuisse Lacedaemonem, ex quo Amiclam natum, et ex eo Argalum patrem Oebali, quem Tyndari, ex quo ipsa genita videretur, patrem constaret. Repetebat etiam cum Hecuba materni generis adfinitatem. Agenoris quippe Phineum et Phoenicem, et inde patres Hecubae et Ledae consanguinitate originem divisisse. Postquam memoriter cuncta retexuit, ad postremum flens orare, ne se, quae semel in fldem illorum recepta esse, prodendam putarent. Ea secum domo Menelai apportata quae propria fuissent, nihil praeterea ablatum. Sed utrum immodico amore Alexandri, an poenarum metu, quas ob desertam domum a conjuge metuebat, ita sibi consulere maluerit, parum constabat.

Igitur Hecuba, cognita voluntate, simul ob generis conjunctionem, complexa Helenam, ne proderetur, summis opibus adnitebatur : quum jam Priamus et reliqui reguli non amplius differendos legatos dicerent, neque resistendum popularium voluntati; solo omnium Deiphobo Hecubae assenso : quem non aliter atque Alexandrum, Helenae desiderium a recto consilio praepediebat. Itaque quum obstinate Hecuba nunc Priamum, modo filios deprecaretur, modo complexu ejus nulla rations divelli posset, omnes qui aderant in voluntatem suam transduxit. Ita ad postremum bonum publicum materna gratia corruptum est. Dein postero die Menelaus cum suis in concionem venit, conjugem, et quae cum ea abrepta essent, repetens. Tunc Priamus inter regulos medius adstans, facto silentio, optionem Helenae, quae ob id in conspectu popularium venerat, offert, si ei videretur, dornum ed suos regredi. Quam ferunt dixisse, neque se patriam regredi velle, neque sibi cum Menelai matrimonio convenire. Ita reguli habentes Helenam, non sine exultatione ex concione discedunt.


\endnumbering
\end{Rightside}

\end{pages}
\Pages
\begin{pages}
\begin{Leftside}
\begin{french}
\beginnumbering
\autopar

Peu de jours après, le fils de Priam et ses compagnons arrivèrent, amenant avec eux la belle Hélène. Son retour mit la ville en mouvement. Les uns avaient l'action d'Alexandre en horreur ; les autres s'attendrissaient sur Ménélas, qui en était la victime. Tous étaient indignés, et personne ne cherchait à défendre le ravisseur. Priam, inquiet, appelle ses fils auprès de lui, les consulte sur ce qu'il doit faire dans une telle conjoncture : ils sont tous d'avis de ne point rendre Hélène. La vue des richesses qu'on avait enlevées avec elle les éblouissait, et ils n'ignoraient pas qu'il faudrait s'en dessaisir si on la rendait elle-même. Ils ne voyaient pas non plus avec indifférence les belles femmes de la suite d'Hélène, et se proposaient bien d'en faire leur conquête; car ces princes, dont les moeurs étaient aussi barbares que le langage, s'inquiétaient peu de ce qui était juste ou injuste, et ne voyaient dans cette affaire que deux objets qui partageaient également leur affection : le butin premièrement; ensuite le moyen d'assouvir leurs passions déréglées,

Priam, après cette réponse, les quitte, assemble les anciens, leur fait part de la résolution de ses fils et demande leur avis. Ceux-ci ne l'avaient pas encore donné, que les princes, sans garder aucune mesure, entrent tout-à-coup dans la salle du conseil, en menaçant chacun des assistants de leur vengeance, s'ils osent prendre le moindre arrêté contraitre à leurs intérêts. Cependant le peuple ne poutait retenir son indignation, et réclamait hautement contre l'injustice; il demandait satisfaction pour les députés, et pour lui-même la réparation des torts qu'il éprouvait journellement. Alexandre, toujours aveuglé par sa passion, et craignant tout d'un peuple irrité, sort accompagné de ses frères, les armes à la main, se jette au milieu de la multitude, et en fait un affreux carnage. Ce qui reste est sauvé par l'intervention des grands qui avaient assisté au conseil, et par Anténor, qui s'était mis â leur tête. Ainsi le peuple se retira méprisé, maltraité, et sans avoir rien obtenu.

Le lendemain, le roi, à la prière d'Hécube, se rend chez Hélène, la salue avec bonté, l'exhorte à prendre courage, et lui fait plusieurs questions sur son état et sur sa naissance. La princesse lui répondit que des liens de parenté l'unissaient â Alexandre, qu'elle appartenait plus à Priam et à Hécube qu'aux fils de Plisthène; et reprenant son origine de plus haut, elle dit que Danaüs et Agénor étaient leurs communs auteurs; que de Pléione, fille de Danaüs et d'Atlas, naquit Électre, qui, enceinte de Jupiter, avait mis au monde Dardanus, duquel sortirent Tros et les autres rois de Troie; que d'un autre côté, Taygète, fille d'Agénor, avait eu de Jupiter Lacédémon, père d'Amiclas ; que celui-ci donna le jour à Argalus, père d'Oebalus, qui engendra Tyndare, dont elle était la fille. Elle allégua aussi les liens qui l'unissaient à Hécube par Agénor, père de Phinée et de Phénice, aïeuls d'Hécube et de Léda, sa mère. Après avoir ainsi établi sa généalogie, elle conjura Priam et Hécube, les larmes aux yeux, de ne la point rendre aux Grecs après l'avoir prise sous leur protection. Elle ajouta que les richesses qui avaient été tirées du palais de Ménélas lui appartenaient, et qu'elle n'avait rien pris au-delà. On ne sait pas au juste si sa réponse lui fut inspirée par son amour pour Alexandre, ou par la crainte d'être punie un jour par son mari à cause de sa désertion.

Hécube, qui connaissait son désir, et voyait en elle une parente, la tenait serrée contre son sein, et suppliait son époux de ne la point rendre. Cependant Priam et les princes étaient revenus à un meilleur avis ; ils insistaient pour qu'on renvoyât la députation avec une réponse favorable, et craignaient déjà de résister à la volonté du peuple : le seul Deiphobe appuyait Hécube, sans doute parce qu'il était épris de la même passion qu'Alexandre pour la beauté d'Hélène. Hécube, de son côté, s'adressait tantôt à Priam, tantôt à ses fils, et tantôt embrassant la princesse, elle jurait que rien ne pourrait l'en séparer. De cette manière, elle entraîna à son avis tous les assistants, et les caresses d'une mère triomphèrent enfin du bonheur public. Le jour suivant, Ménélas et ses collègues se rendirent à l'assemblée, redemandant Hélène, et avec elle tentes les richesses qui avaient été enlevées. Alors Priam, debout et entouré des princes ses fils, commande le silence; il prie Hélène, qui était présente, de choisir elle-même, et de déclarer si elle voulait retourner à Sparte ou demeurer à Troie. La princesse, dit-on, fit réponse qu'elle ne voulait ni revoir sa patrie, ni rester unie à Ménélas. Ainsi les princes sortent du conseil triomphants et joyeux de posséder Hélène.

\endnumbering
\end{french}
\end{Leftside}

\begin{Rightside}
\beginnumbering
\autopar

Interim paucis post diebus, Alexander cum supradictis comitibus venit, Helenam secum habens. Cujus adventu in tota civitate quum partim exemplum facinoris exsecrarentur, alii injurias in Menelaum admissas dolerent, nullo omnium adprobante, postremo cunctis indignantibus, tumultus ortus est. Queisrebus anxius Priamus filios convocat, eosque quid super tali agendum negotio videretur, consulit : qui una voce, minime reddendam Helenam, respondent. Videbant quippe quantae opes cum ea advectae essent : quae universa, si Helena traderetur, necessario amitterent. Praeterea permoti forma mulierum quae cum Helena venerant, nuptias sibi singularum jam animo destinaverant. Quippe qui lingua moribusque barbari, nihil pensi aut consulti patientes, praeda ac libidine transversi agebantur.

Igitur Priamus, relictis his, senes conducit, sententiam filiorum aperit. Dein cunctos, quid agendum sit, consulit. Sed priusquam ex more sententiae dicerentur, reguli repente concilium irrumpunt, atque inconditis moribus malum singulis minitantur, si aliter quam ipsis videretur, decernerent. Interim omnis populus indigne admissam injuriam, atque in hunc modum multa alia, cum exsecratione reclamabant. Ob quae Alexander cupidine animi praeceps, veritus ne quid adversum se a popularibus oriretur, stipatus armatis fratribus impetum in multitudinem facit, multosque obtruncat : reliqui interventu procerum, qui in consilio fuerant, duce liberantur Antenore. Ita infectis rebus, populus contemptui habitus, non sine pernicie sua, domum discedit.

Dein secuta die rex hortatu Hecubae ad Helenam adit, eamque benigne salutans, animum bonum uti gereret hortatur; quae cujusque erset, requirit. Tum illa Alexandri se adfinem respondit, magisque ad Priamum et Hecubam, quam ad Plisthenis filios genere pertinere, repetens originem omnem majorum. Danaum enim atque Agenorem, et sui et Priami generis auctores esse. Namque ex Pleiona, Danai filia et Atlante, Electram natam, quam ex Jove gravidam Dardanum genuisse, ex quo Tros, et deinceps insecuti reges llii. Agenoris porro Taygetam; eam ex Jove habuisse Lacedaemonem, ex quo Amiclam natum, et ex eo Argalum patrem Oebali, quem Tyndari, ex quo ipsa genita videretur, patrem constaret. Repetebat etiam cum Hecuba materni generis adfinitatem. Agenoris quippe Phineum et Phoenicem, et inde patres Hecubae et Ledae consanguinitate originem divisisse. Postquam memoriter cuncta retexuit, ad postremum flens orare, ne se, quae semel in fldem illorum recepta esse, prodendam putarent. Ea secum domo Menelai apportata quae propria fuissent, nihil praeterea ablatum. Sed utrum immodico amore Alexandri, an poenarum metu, quas ob desertam domum a conjuge metuebat, ita sibi consulere maluerit, parum constabat.

Igitur Hecuba, cognita voluntate, simul ob generis conjunctionem, complexa Helenam, ne proderetur, summis opibus adnitebatur : quum jam Priamus et reliqui reguli non amplius differendos legatos dicerent, neque resistendum popularium voluntati; solo omnium Deiphobo Hecubae assenso : quem non aliter atque Alexandrum, Helenae desiderium a recto consilio praepediebat. Itaque quum obstinate Hecuba nunc Priamum, modo filios deprecaretur, modo complexu ejus nulla rations divelli posset, omnes qui aderant in voluntatem suam transduxit. Ita ad postremum bonum publicum materna gratia corruptum est. Dein postero die Menelaus cum suis in concionem venit, conjugem, et quae cum ea abrepta essent, repetens. Tunc Priamus inter regulos medius adstans, facto silentio, optionem Helenae, quae ob id in conspectu popularium venerat, offert, si ei videretur, dornum ed suos regredi. Quam ferunt dixisse, neque se patriam regredi velle, neque sibi cum Menelai matrimonio convenire. Ita reguli habentes Helenam, non sine exultatione ex concione discedunt.


\endnumbering
\end{Rightside}

\end{pages}
\Pages
\end{document}