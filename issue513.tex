
% !TeX program = xelatex
% !TeX encoding = UTF-8
% !TeX spellcheck = it_IT

\documentclass[b5paper]{book}

\usepackage{fontspec}
\setmainfont{Linux Libertine O}

\usepackage{polyglossia}
\setmainlanguage[babelshorthands=true]{italian}

\usepackage[noledgroup,noeledsec,series={A,B,C}]{reledmac}

\Xnotefontsize[A,B,C]{\footnotesize}
\renewcommand*{\ledlsnotefontsetup}%
    {\raggedleft\it\footnotesize}
\renewcommand*{\ledrsnotefontsetup}%
    {\raggedright\it\footnotesize}
\setsidenotesep{ \emph{|} }
\Xnonumber[A,B]
\Xnumberonlyfirstinline[C]
\Xnumberonlyfirstintwolines[C]
\Xinplaceofnumber[A,B,C]{0em}
\Xnolemmaseparator[A,B]
\Xnonbreakableafternumber[C]
\Xinplaceoflemmaseparator[A,B]{0em}
\Xinplaceoflemmaseparator[C]{.5em}
\addtolength{\skip\Afootins}{2em plus.4em minus.4em}
\Xbeforenotes[A]{2em plus.4em minus.4em}
\Xafternote[A,B,C]{2em plus.4em minus.4em}
\Xarrangement[A,C]{paragraph}
\Xarrangement[B]{normal}
\labelpstarttrue

\usepackage[widthliketwocolumns]{reledpar}
\setlength{\Lcolwidth}{0.48\textwidth}
\setlength{\Rcolwidth}{0.48\textwidth}
\setlength{\beforecolumnseparator}{0.02\textwidth}
\setlength{\aftercolumnseparator}{0.02\textwidth}
\columnsposition{J}
\firstlinenum*{1}
\linenumincrement*{1}
\AtBeginPairs{\sloppy}

\lineation*{page}
\begin{document}

\beginnumbering
\numberpstarttrue

\pstart
    Paragraph nr. 1] Quel ramo del lago di Como, che volge a mezzogiorno, tra due catene non interrotte di monti, tutto a seni e a golfi, a seconda dello sporgere e del rientrare di quelli, vien, quasi a un tratto, a ristringersi, e a prender corso e figura di fiume, tra un promontorio a destra, e un'ampia costiera dall'altra parte; e il ponte, che ivi congiunge le due rive, par che renda ancor più sensibile all'occhio questa trasformazione, e segni il punto in cui il lago cessa, e l'Adda rincomincia, per ripigliar poi nome di lago dove le rive, allontanandosi di nuovo, lascian l'acqua distendersi e rallentarsi in nuovi golfi e in nuovi seni.
\pend

\pstart%
    %
    Paragraph nr. 2%
\pend

\pstart%
    %
    Paragraph nr. 3%
\pend

\pausenumbering
\begin{pairs}
\begin{Leftside}
\resumenumbering
\numberpstartfalse
    \pstart\noindent Q. Mucius augur multa narrare de C. Laelio socero suo memoriter et iucunde solebat nec dubitare illum in omni sermone appellare sapientem; ego autem a patre ita eram deductus ad Scaevolam sumpta virili toga, ut, quoad possem et liceret, a senis latere numquam discederem; itaque multa ab eo prudenter disputata, multa etiam breviter et commode dicta memoriae mandabam fierique studebam eius prudentia doctior. \textbf{A}\pend
\pausenumbering
\end{Leftside}

\begin{Rightside}
\beginnumbering
\numberpstartfalse
    \pstart\noindent Quinto Mucio l'augure raccontava spesso, a memoria e in modo piacevole, molti episodi della vita di Caio Lelio, suo suocero, e in ogni discorso non esitava a chiamarlo "il Saggio". A Scevola ero stato affidato da mio padre, quando presi la toga virile, perché non mi staccassi mai dal fianco del vecchio, nei limiti del possibile e del consentito. Perciò, fissavo nella mente molti dei suoi accorti ragionamenti e anche molte delle sue massime secche e gustose, e cercavo di migliorare la mia educazione facendo tesoro della sua esperienza di vita. \textbf{B}\pend
\pausenumbering
\end{Rightside}
\end{pairs}
\Columns

\resumenumbering
\numberpstarttrue

\pstart \edlabel{itm:Ri33}\edtext{}{\Bfootnote{{\textbf{\pstartref{itm:Ri33}}}\enspace cf. Poll. 5.143, Hesych. σ 2293 (ex Cyr.), sed ex Platonis ipsius verbis quam e fontibus quibusdam sch. sumptum mihi videtur}}333a13 \textit{συμβόλαια Slings: ξυμβόλαια ADT}]\pend


\pausenumbering
\begin{pairs}
\begin{Leftside}
\resumenumbering
\numberpstartfalse
    \pstart\noindent \textbf{a)} σημείωσαι τίνα τὰ ξυμβόλαιά φησιν, ὅτι τὰ κοινωνήματα. \textbf{A}\pend
\pausenumbering
\end{Leftside}

\begin{Rightside}
\resumenumbering
\numberpstartfalse
    \pstart\noindent \textbf{b)} ὅτι συμβόλαια φησὶ τὰ κοινωνήματα. \textbf{T}\pend
\pausenumbering
\end{Rightside}
\end{pairs}
\Columns

\numberpstarttrue
\resumenumbering

\pstart%
    Paragraph nr. 5%
\pend


%%%%%%%%%%%%%%%%%%%%%%%%%%%%%%%%%%%%%%%%%%%%%%%%%%%%%%%%%%%%%%%%%%%%%%%%
\pausenumbering
\begin{pairs}
\begin{Leftside}
\resumenumbering
\numberpstartfalse
    \pstart
    Quel ramo del lago di Como, che volge a mezzogiorno, tra due catene non interrotte di monti, tutto a seni e a golfi, a seconda dello sporgere e del rientrare di quelli, vien, quasi a un tratto, a ristringersi, e a prender corso e figura di fiume, tra un promontorio a destra, e un'ampia costiera dall'altra parte; e il ponte, che ivi congiunge le due rive, par che renda ancor più sensibile all'occhio questa trasformazione, e segni il punto in cui il lago cessa, e l'Adda rincomincia, per ripigliar poi nome di lago dove le rive, allontanandosi di nuovo, lascian l'acqua distendersi e rallentarsi in nuovi golfi e in nuovi seni.
    \pend
\pausenumbering
\end{Leftside}

\begin{Rightside}
\resumenumbering
\numberpstartfalse
    \pstart
    Quel ramo del lago di Como, che volge a mezzogiorno, tra due catene non interrotte di monti, tutto a seni e a golfi, a seconda dello sporgere e del rientrare di quelli, vien, quasi a un tratto, a ristringersi, e a prender corso e figura di fiume, tra un promontorio a destra, e un'ampia costiera dall'altra parte; e il ponte, che ivi congiunge le due rive, par che renda ancor più sensibile all'occhio questa trasformazione, e segni il punto in cui il lago cessa, e l'Adda rincomincia, per ripigliar poi nome di lago dove le rive, allontanandosi di nuovo, lascian l'acqua distendersi e rallentarsi in nuovi golfi e in nuovi seni.
    \pend
\pausenumbering
\end{Rightside}
\end{pairs}
\Columns

\resumenumbering
\numberpstarttrue
%%%%%%%%%%%%%%%%%%%%%%%%%%%%%%%%%%%%%%%%%%%%%%%%%%%%%%%%%%%%%%%%%%%%%%%%


\pstart%
    Paragraph nr. 6%
\pend

\pstart%
    Paragraph nr. 7%
\pend

\pstart%
    Paragraph nr. 8%
\pend

\pstart%
    Paragraph nr. 9%
\pend

\pstart%
    Paragraph nr. 10%
\pend

\pstart%
    Paragraph nr. 11%
\pend

\pstart%
    Paragraph nr. 12%
\pend

\pstart%
    Paragraph nr. 13%
\pend

\pstart%
    Paragraph nr. 14%
\pend

\pstart%
    Paragraph nr. 15%
\pend

\pstart%
    Paragraph nr. 16%
\pend

\pstart%
    Paragraph nr. 17%
\pend

\pstart%
    Paragraph nr. 18%
\pend

\pstart%
    Paragraph nr. 19%
\pend

\pstart%
    Paragraph nr. 20%
\pend

\pstart%
    Paragraph nr. 21%
\pend

\pstart%
    Paragraph nr. 22%
\pend

\pstart%
    Paragraph nr. 23%
\pend

\pstart%
    Paragraph nr. 24%
\pend

\endnumbering

\end{document}
