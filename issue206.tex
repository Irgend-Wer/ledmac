% !TeX encoding = UTF-8
% !TeX TS-program = xelatex
% !TeX spellcheck = it_IT
% rubber: set program xelatex

\documentclass[11pt,b5paper,twoside]{book}
\usepackage{fontspec}
    \setmainfont{Linux Libertine O}
\usepackage{polyglossia}

\usepackage{eledmac}
    \lineation{page}
    \linenummargin{inner}
    \sidenotemargin{outer}
    \footparagraph{C}

\usepackage{hyperref}

\begin{document}

\chapter*{Praefatio}
%%%%%% PLEASE, SEE HERE AT \pstartref
\section{A question about par.~\pstartref{1}}

Cicero wrote the \textit{Tusculanae} in a period in which he was forced to abandon the political activities, as he himself tells us in par.~\pstartref{1}.

I. [1] Quintus hic dies, Brute, finem faciet Tusculanarum disputationum, quo die est a nobis ea de re, quam tu, ex omnibus maxime probas, disputatum. Placere enim tibi admodum sensi et ex eo libro, quem ad me accuratissime scripsisti, et ex multis sermonibus tuis virtutem ad beate vivendum se ipsa esse contentam. Quod etsi difficile est probatu propter tam varia et tam multa tormenta fortunae, tale tamen est, ut elaborandum sit, quo facilius probetur. Nihil est enim omnium quae in philosophia tractantur, quod gravius magnificentiusque dicatur.

[2] Nam cum ea causa impulerit eos qui primi se ad philosophiae studium contulerunt, ut omnibus rebus posthabitis totos se in optumo vitae statu exquirendo conlocarent, profecto spe beate vivendi tantam in eo studio curam operamque posuerunt. Quodsi ab is inventa et perfecta virtus est, et si praesidi ad beate vivendum in virtute satis est, quis est qui non praeclare et ab illis positam et a nobis susceptam operam philosophandi arbitretur? Sin autem virtus subiecta sub varios incertosque casus famula fortunae est nec tantarum virium est, ut se ipsa tueatur, vereor ne non tam virtutis fiducia nitendum nobis ad spem beate vivendi quam vota facienda videantur.

[3] Equidem eos casus, in quibus me fortuna vehementer exercuit, mecum ipse considerans huic incipio sententiae diffidere interdum et humani generis imbecillitatem fragilitatemque extimescere. Vereor enim ne natura, cum corpora nobis infirma dedisset isque et morbos insanabilis et dolores intolerabilis adiunxisset, animos quoque dederit et corporum doloribus congruentis et separatim suis angoribus et molestiis implicatos.

[4] Sed in hoc me ipse castigo, quod ex aliorum et ex nostra fortasse mollitia, non ex ipsi virtute de virtutis robore existumo. Illa enim, si modo est ulla virtus, quam dubitationem avunculus tuus, Brute, sustulit, omnia, quae cadere in hominem possunt, subter se habet eaque despiciens casus contemnit humanos culpaque omni carens praeter se ipsam nihil censet ad se pertinere. Nos autem omnia adversa cum venientia metu augentes, tum maerore praesentia rerum naturam quam errorem nostrum damnare malumus.

II. [5] Sed et huius culpae et ceterorum vitiorum peccatorumque nostrorum omnis a philosophia petenda correctio est. Cuius in sinum cum a primis temporibus aetatis nostra voluntas studiumque nos compulisset, his gravissimis casibus in eundem portum, ex quo eramus egressi, magna iactati tempestate confugimus. 0 vitae philosophia dux, o virtutis indagatrix expultrixque vitiorum! quid non modo nos, sed omnino vita hominum sine te esse potuisset? Tu urbis peperisti, tu dissipatos homines in societatem vitae convocasti, tu eos inter se primo domiciliis, deinde coniugiis, tum litterarum et vocum communione iunxisti, tu inventrix legum, tu magistra morum et disciplinae fuisti; ad te confugimus, a te opem petimus, tibi nos, ut antea magna ex parte, sic nunc penitus totosque tradimus. Est autem unus dies bene et ex praeceptis tuis actus peccanti inmortalitati anteponendus.

[6] Cuius igitur potius opibus utamur quam tuis, quae et vitae tranquillitatem largita nobis es et terrorem mortis sustulisti? Ac philosophia quidem tantum abest ut proinde ac de hominum est vita merita laudetur, ut a plerisque neglecta a multis etiam vituperetur. Vituperare quisquam vitae parentem et hoc parricidio se inquinare audet et tam impie ingratus esse, ut eam accuset, quam vereri deberet, etiamsi minus percipere potuisset? Sed, ut opinor, hic error et haec indoctorum animis offusa caligo est, quod tam longe retro respicere non possunt nec eos, a quibus vita hominum instructa primis sit, fuisse philosophos arbitrantur.

III. [7] Quam rem antiquissimam cum videamus, nomen tamen esse confitemur recens. Nam sapientiam quidem ipsam quis negare potest non modo re esse antiquam, verum etiam nomine? Quae divinarum humanarumque rerum, tum initiorum causarumque cuiusque rei cognitione hoc pulcherrimum nomen apud antiquos adsequebatur. Itaque et illos septem, qui a Graecis sofoi/, sapientes a nostris et habebantur et nominabantur, et multis ante saeculis Lycurgum, cuius temporibus Homerus etiam fuisse ante hanc urbem conditam traditur, et iam heroicis aetatibus Ulixem et Nestorem accepimus et fuisse et habitos esse sapientis.

[8] Nec vero Atlans sustinere caelum nec Prometheus adfixus Caucaso nec stellatus Cepheus cum uxore genero filia traderetur, nisi caelestium divina cognitio nomen eorum ad errorem fabulae traduxisset. A quibus ducti deinceps omnes, qui in rerum contemplatione studia ponebant, sapientes et habebantur et nominabantur, idque eorum nomen usque ad Pythagorae manavit aetatem. Quem, ut scribit auditor Platonis Ponticus Heraclides, vir doctus in primis, Phliuntem ferunt venisse, eumque cum Leonte, principe Phliasiorum, docte et copiose disseruisse quaedam. Cuius ingenium et eloquentiam cum admiratus esset Leon, quaesivisse ex eo, qua maxime arte confideret; at illum: artem quidem se scire nullam, sed esse philosophum. Admiratum Leontem novitatem nominis quaesivisse, quinam essent philosophi, et quid inter eos et reliquos interesset;

[9] Pythagoram autem respondisse similem sibi videri vitam hominum et mercatum eum, qui haberetur maxumo ludorum apparatu totius Graeciae celebritate; nam ut illic alii corporibus exercitatis gloriam et nobilitatem coronae peterent, alii emendi aut vendendi quaestu et lucro ducerentur, esset autem quoddam genus eorum, idque vel maxime ingenuum, qui nec plausum nec lucrum quaererent, sed visendi causa venirent studioseque perspicerent, quid ageretur et quo modo, item nos quasi in mercatus quandam celebritatem ex urbe aliqua sic in hanc vitam ex alia vita et natura profectos alios gloriae servire, alios pecuniae, raros esse quosdam, qui ceteris omnibus pro nihilo habitis rerum naturam studiose intuerentur; hos se appellare sapientiae studiosos—id est enim philosophos -; et ut illic liberalissimum esset spectare nihil sibi adquirentem, sic in vita longe omnibus studiis contemplationem rerum, cognitionemque praestare.

IV. [10] Nec vero Pythagoras nominis solum inventor, sed rerum etiam ipsarum amplificator fuit. Qui cum post hunc Phliasium sermonem in Italiam venisset, exornavit eam Graeciam, quae magna dicta est, et privatim et publice praestantissumis et institutis et artibus. Cuius de disciplina aliud tempus fuerit fortasse dicendi. Sed ab antiqua philosophia usque ad Socratem, qui Archelaum, Anaxagorae discipulum, audierat, numeri motusque tractabantur, et unde omnia orerentur quove reciderent, studioseque ab is siderum magnitudines intervalla cursus anquirebantur et cuncta caelestia. Socrates autem primus philosophiam devocavit e caelo et in urbibus conlocavit et in domus etiam introduxit et coegit de vita et moribus rebusque bonis et malis quaerere.

[11] Cuius multiplex ratio disputandi rerumque varietas et ingeni magnitudo Platonis memoria et litteris consecrata plura genera effecit dissentientium philosophorum, e quibus nos id potissimum consecuti sumus, quo Socratem usum arbitrabamur, ut nostram ipsi sententiam tegeremus, errore alios levaremus et in omni disputatione, quid esset simillimum veri, quaereremus. Quem morem cum Carneades acutissime copiosissimeque tenuisset, fecimus et alias saepe et nuper in Tusculano, ut ad eam consuetudinem disputaremus. Et quadridui quidem sermonem superioribus ad te perscriptum libris misimus, quinto autem die cum eodem in loco consedissemus, sic est propositum, de quo disputaremus:

V. [12] -Non mihi videtur ad beate vivendum satis posse virtutem. - At hercule Bruto meo videtur, cuius ego iudicium, pace tua dixerim, longe antepono tuo. - Non dubito, nec id nunc agitur, tu illum quantum ames, sed hoc, quod mihi dixi videri, quale sit, de quo a te disputari volo. - Nempe negas ad beate vivendum satis posse virtutem? - Prorsus nego. - Quid? ad recte honeste laudabiliter, postremo ad bene vivendum satisne est praesidi in virtute? - Certe satis. - Potes igitur aut, qui male vivat, non eum miserum dicere aut, quem bene fateare, eum negare beate vivere? - Quidni possim? nam etiam in tormentis recte honeste laudabiliter et ob eam rem bene vivi potest, dum modo intellegas, quid nunc dicam 'bene.' Dico enim constanter graviter sapienter fortiter. Haec etiam in eculeum coiciuntur, quo vita non adspirat beata.

[13] -Quid igitur? solane beata vita, quaeso, relinquitur extra ostium limenque carceris, cum constantia gravitas fortitudo sapientia reliquaeque virtutes rapiantur ad tortorem nullumque recusent nec supplicium nec dolorem? - Tu, si quid es facturus, nova aliqua conquiras oportet; ista me minime movent, non solum quia pervulgata sunt, sed multo magis, quia, tamquam levia quadam vina nihil valent in aqua, sic Stoicorum ista magis gustata quam potata delectant. Velut iste chorus virtutum in eculeum impositus imagines constituit ante oculos cum amplissima dignitate, ut ad eas cursim perrectura nec eas beata vita a se desertas passura videatur; cum autem animum ab ista pictura imaginibusque virtutum ad rem veritatemque traduxeris, hoc nudum relinquitur, possitne quis beatus esse, quam diu torqueatur.
\beginnumbering
\numberpstarttrue
\labelpstarttrue

\chapter*{Ciceronis Tusculanae Disputationes}

\pstart%
%%%%%% PLEASE, SEE HERE AT \pstartref
Cum defensionum \edtext{laboribus}{\Cfootnote{first note}} senatoriisque muneribus aut omnino aut magna ex parte essem aliquando liberatus, rettuli me, Brute, te hortante maxime ad ea studia, quae retenta animo, remissa temporibus, longo intervallo intermissa revocavi, et cum omnium artium, quae ad rectam vivendi viam pertinerent, \edtext{ratio}{\Cfootnote{second note}} et disciplina studio sapientiae, quae philosophia dicitur, contineretur, hoc mihi Latinis litteris \edtext{inlustrandum}{\Cfootnote{third note}} putavi, non quia \edtext{philosophia}{\Cfootnote{cf.~\pstartref{1}}} Graecis et litteris et doctoribus percipi non posset, sed meum semper iudicium fuit omnia nostros aut invenisse per se sapientius quam Graecos aut accepta ab illis fecisse meliora, quae quidem digna statuissent, in quibus elaborarent.
\pend

\pstart%
Cum defensionum \edtext{laboribus}{\Cfootnote{first note}} senatoriisque muneribus aut omnino aut magna ex parte essem aliquando liberatus, rettuli me, Brute, te hortante maxime ad ea studia, quae retenta animo, remissa temporibus, longo intervallo intermissa revocavi, et cum omnium artium, quae ad rectam vivendi viam pertinerent, \edtext{ratio}{\Cfootnote{second note}} et disciplina studio sapientiae, quae philosophia dicitur, contineretur, hoc mihi Latinis litteris \edtext{inlustrandum}{\Cfootnote{third note}} putavi, non quia \edtext{philosophia}{\Cfootnote{fourth note}} Graecis et litteris et doctoribus percipi non posset, sed meum semper iudicium fuit omnia nostros aut invenisse per se sapientius quam Graecos aut accepta ab illis fecisse meliora, quae quidem digna statuissent, in quibus elaborarent.
\pend

\pstart%
Cum defensionum \edtext{laboribus}{\Cfootnote{first note}} senatoriisque muneribus aut omnino aut magna ex parte essem aliquando liberatus, rettuli me, Brute, te hortante maxime ad ea studia, quae retenta animo, remissa temporibus, longo intervallo intermissa revocavi, et cum omnium artium, quae ad rectam vivendi viam pertinerent, \edtext{ratio}{\Cfootnote{second note}} et disciplina studio sapientiae, quae philosophia dicitur, contineretur, hoc mihi Latinis litteris \edtext{inlustrandum}{\Cfootnote{third note}} putavi, non quia \edtext{philosophia}{\Cfootnote{fourth note}} Graecis et litteris et doctoribus percipi non posset, sed meum semper iudicium fuit omnia nostros aut invenisse per se sapientius quam Graecos aut accepta ab illis fecisse meliora, quae quidem digna statuissent, in quibus elaborarent.
\pend

\pstart%
Cum defensionum \edtext{laboribus}{\Cfootnote{first note}} senatoriisque muneribus aut omnino aut magna ex parte essem aliquando liberatus, rettuli me, Brute, te hortante maxime ad ea studia, quae retenta animo, remissa temporibus, longo intervallo intermissa revocavi, et cum omnium artium, quae ad rectam vivendi viam pertinerent, \edtext{ratio}{\Cfootnote{second note}} et disciplina studio sapientiae, quae philosophia dicitur, contineretur, hoc mihi Latinis litteris \edtext{inlustrandum}{\Cfootnote{third note}} putavi, non quia \edtext{philosophia}{\Cfootnote{fourth note}} Graecis et litteris et doctoribus percipi non posset, sed meum semper iudicium fuit omnia nostros aut invenisse per se sapientius quam Graecos aut accepta ab illis fecisse meliora, quae quidem digna statuissent, in quibus elaborarent.
\pend

\pstart%
Cum defensionum \edtext{laboribus}{\Cfootnote{first note}} senatoriisque muneribus aut omnino aut magna ex parte essem aliquando liberatus, rettuli me, Brute, te hortante maxime ad ea studia, quae retenta animo, remissa temporibus, longo intervallo intermissa revocavi, et cum omnium artium, quae ad rectam vivendi viam pertinerent, \edtext{ratio}{\Cfootnote{second note}} et disciplina studio sapientiae, quae philosophia dicitur, contineretur, hoc mihi Latinis litteris \edtext{inlustrandum}{\Cfootnote{third note}} putavi, non quia \edtext{philosophia}{\Cfootnote{fourth note}} Graecis et litteris et doctoribus percipi non posset, sed meum semper iudicium fuit omnia nostros aut invenisse per se sapientius quam Graecos aut accepta ab illis fecisse meliora, quae quidem digna statuissent, in quibus elaborarent.
\pend

%%%%%% PLEASE, SEE HERE AT \edlabel
\pstart\edlabel{1}%
Cum defensionum \edtext{laboribus}{\Cfootnote{first note}} senatoriisque muneribus aut omnino aut magna ex parte essem aliquando liberatus, rettuli me, Brute, te hortante maxime ad ea studia, quae retenta animo, remissa temporibus, longo intervallo intermissa revocavi, et cum omnium artium, quae ad rectam vivendi viam pertinerent, \edtext{ratio}{\Cfootnote{second note}} et disciplina studio sapientiae, quae philosophia dicitur, contineretur, hoc mihi Latinis litteris \edtext{inlustrandum}{\Cfootnote{third note}} putavi, non quia \edtext{philosophia}{\Cfootnote{fourth note}} Graecis et litteris et doctoribus percipi non posset, sed meum semper iudicium fuit omnia nostros aut invenisse per se sapientius quam Graecos aut accepta ab illis fecisse meliora, quae quidem digna statuissent, in quibus elaborarent.
\pend

\labelpstartfalse
\numberpstartfalse
\endnumbering
\end{document}
