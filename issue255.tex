\documentclass{scrbook}
\usepackage{fontspec}

\usepackage{lipsum} 
\usepackage{eledmac,eledpar}

\newfontfamily\syriacfont
[Script=Syriac, Scale=1.2 ]{estre.otf}

\newcommand{\textsyriac}[1] % Syriac inside LTR
       {\bgroup\luatextextdir TRT\syriacfont #1\egroup}
\newcommand{\n}         [1] % for digits inside Arabic text
       {\bgroup\luatextextdir TLT #1\egroup}
\newenvironment{syriac}     % Syriac paragraph
       {\luatextextdir TRT\luatexpardir TRT\luatexbodydir TRT%
       \syriacfont}{}


\linenummargin{inner} %line numbers on inner margin

\firstlinenum{1}
\linenumincrement{1}
\setlength{\parindent}{0pt}

\begin{document}
\the\luatexbodydir
   \begin{syriac}
    \beginnumbering

\pstart 
        \lipsum[5]
    \pend
\begin{syriac}
    \pstart  %%% Here you can use either of the following three cases: 
%       \eledsection*{Section}
        \textsyriac{\eledsubsection*{Subsection}}
%       \eledsubsubsection*{Sububsection}
\pend
\end{syriac}
    \endnumbering
    \end{syriac}

\begin{pages}
    \begin{Leftside}
    \begin{syriac}
    \beginnumbering

\pstart 
        \lipsum[5]
        \lipsum[5]
    \pend
\begin{syriac}
    \pstart  %%% Here you can use either of the following three cases: 
%       \eledsection*{Section}
        \textsyriac{\eledsubsection*{Subsection}}
%       \eledsubsubsection*{Sububsection}
\pend
\end{syriac}
    \endnumbering
    \end{syriac}
    \end{Leftside}

    \begin{Rightside}
    \beginnumbering

\pstart
        Hallo
    \pend

\pstart
        Hallo 
    \pend

    \endnumbering
    \end{Rightside}
\end{pages} 
\Pages
\end{document}