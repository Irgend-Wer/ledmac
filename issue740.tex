\documentclass[10pt,b5paper,twoside]{memoir}
\usepackage[T1]{fontenc}             
\usepackage[utf8]{inputenc}
\usepackage{lmodern}
\usepackage[french,greek,german,italian]{babel}
%-------------------------------------------------------------------
\usepackage[series={A,B},noend,noeledsec,noledgroup]{reledmac}
\usepackage{reledpar} \setgoalfraction{.85}
\AtBeginDocument{\Xmaxhnotes{0.5\textheight}}
\setlength{\Lcolwidth}{0.47\textwidth}
\setlength{\Rcolwidth}{0.47\textwidth} 
\newcounter{Atest}
\def\Atest{\stepcounter{Atest}(\textbf{\theAtest\ - A}) }
\newcounter{Btest}
\def\Btest{\stepcounter{Btest}(\textbf{\theBtest\ - B}) }
\lineation*{page} \Xlineflag 

\newenvironment{DE}{\begin{otherlanguage*}{german}}{\end{otherlanguage*}}
\usepackage{hyperref}\hypersetup{colorlinks,linkcolor=black,citecolor=black,urlcolor=black}%

\begin{document}
\begin{center} 	{\Huge\bfseries{Titolo}} \end{center}	\setlength{\parindent}{0mm}

\begin{pairs}\begin{Leftside}\beginnumbering
\vspace{10mm} 
		
\pstart
%INIZIO TESTO TEDESCO
\begin{DE}
Dies ist der Brief, den Philipp Lord Chandos, j\"{u}nger Sohn des Earl of Bath, an Francis Bacon, sp\"{a}ter Lord Verulam und Viscount St. Albans, schrieb, um sich bei diesem Freunde wegen des g\"{a}nzlichen Verzichtes auf literarische Bet\"{a}tigung zu entschuldigen.
\end{DE}
\pend
		
\pstart
\begin{DE}
Die Wintersonne\edlabel{Menschen} stand nur als armer Schein, milchig und matt hinter Wolkenschichten \"{u}ber der engen Stadt...
\end{DE}
\pend
\endnumbering
\end{Leftside}%FINE TESTO TEDESCO
	
%INIZIO TESTO ITALIANO
\begin{Rightside}
\beginnumbering
\pstart%
Questa è la lettera che Lord Philip Chandos, il rampollo pù giovane dell'Earl di Bath, scrisse a Francesco Bacone, in seguito Lord Verulan e visconte di Sant'Alban, per spiegare all'amico la definitiva rinuncia all'attività letteraria.
\pend
		
\pstart
Il sole d'inverno stava solo come pallido chiarore bianco e latteo dietro uno strato di nubi\edtext{degli uomini non comuni, di quelli}{\Afootnote{\Atest\ \begin{DE}\textit{den nur gro\ss e Menschen},\end{DE} ln. \edlineref{Menschen}: <<quale [è] soltanto dei grandi uomini>>.}} sopra l'angusta città.....%
\pend%
\endnumbering
\end{Rightside}%FINE TESTO ITALIANO
\end{pairs} 
\Columns
\end{document}
