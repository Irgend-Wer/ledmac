\documentclass[a4paper, 11pt]{book}
\usepackage[utf8]{inputenc}
\usepackage[T1]{fontenc}
\usepackage[cyr]{aeguill}
\usepackage[polutonikogreek,english,francais]{babel}
\usepackage{etex,eledmac,eledpar}
\usepackage{epigraph}

% Epigraph perso
\newlength{\epigraphheight}
\setlength{\epigraphheight}{5cm}
\makeatletter
\newcommand{\ledepigraph}[2]{\vspace{\beforeepigraphskip}%on reprend le code d'epigraph, mais on ajoute juste une haute à la minipage pour ne pas tout décaler
  {\epigraphsize\begin{\epigraphflush}\begin{minipage}[t][\epigraphheight][t]{\epigraphwidth}
    \@epitext{#1}\\ \@episource{#2}
    \end{minipage}\end{\epigraphflush}
    \vspace{\afterepigraphskip}}}
\makeatother

\title{Titre}
\author{Auteur}
\date{Date}



\begin{document}
\makeatletter

        \setlength{\epigraphwidth}{0.5\textwidth}
            \begin{pages}

        \begin{Rightside}
          \beginnumbering
            \pstart[
               \ledepigraph{he shouted but not even the next
one in line noticed that something
terrible had happened to him.}{la source}
               ]
               \eledchapter*{PROLOGUE}         
            \pend
            \pstart
                texte anglais
            \pend
          \endnumbering
	 			\end{Rightside}
				\begin{Leftside}
				  \beginnumbering
             \pstart[\ledepigraph{he shouted but not even the next
one in line noticed that something
terrible had happened to him. he shouted but not even the next
one in line noticed that something
terrible had happened to him.}{la source}]

                \eledchapter*{PROLOGUE}         
            \pend
            \pstart
                texte français
            \pend
          \endnumbering
        \end{Leftside}
    \end{pages}
    \Pages
\end{document}
