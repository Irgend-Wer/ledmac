% !TeX program = xelatex
% !TeX encoding = UTF-8
% !TeX spellcheck = it_IT

\documentclass[11pt,a4paper]{book}

    \renewcommand{\thefootnote}{\normalfont\arabic{footnote}} %<<==

\usepackage[noeledsec,noledgroup,series={A,B,C}]{reledmac}
\usepackage{reledpar}


\begin{document}
\beginnumbering
\pstart
Quel ramo del lago di Como,\footnoteA{Lecco.} che volge a mezzogiorno, tra due catene non interrotte di monti, tutto a seni e a golfi, a seconda dello sporgere e del rientrare di quelli, vien, quasi a un tratto, a ristringersi, e a prender corso e figura di fiume, tra un promontorio a destra, e un’ampia costiera dall’altra parte; e il ponte, che ivi congiunge le due rive, par che renda ancor più sensibile all’occhio questa trasformazione, e segni il punto in cui il lago cessa, e l’Adda rincomincia, per ripigliar poi nome di lago dove le rive, allontanandosi di nuovo, lascian l’acqua distendersi e rallentarsi in nuovi golfi e in nuovi seni.\footnoteA{prova.} La costiera, formata dal deposito di tre grossi torrenti, scende appoggiata a due monti contigui, l’uno detto di san [p. 10 modifica]Martino, l’altro, con voce lombarda, il Resegone, dai molti suoi cocuzzoli in fila, che in vero lo fanno somigliare a una sega: talchè non è chi, al primo vederlo, purchè sia di fronte, come per esempio di su le mura di Milano che guardano a settentrione, non lo discerna tosto, a un tal contrassegno, in quella lunga e vasta giogaia, dagli altri monti di nome più oscuro e di forma più comune. Per un buon pezzo, la costa sale con un pendìo lento e continuo; poi si rompe in poggi e in valloncelli, in erte e in ispianate, secondo l’ossatura de’ due monti, e il lavoro dell’acque. Il lembo estremo, tagliato dalle foci de’ torrenti, è quasi tutto ghiaia e ciottoloni; il resto, campi e vigne, sparse di terre, di ville, di casali; in qualche parte boschi, che si prolungano su per la montagna. Lecco, la principale di quelle terre, e che dà nome al territorio, giace poco discosto dal ponte, alla riva del lago, anzi viene in parte a trovarsi nel lago stesso, quando questo ingrossa: un gran borgo al giorno d’oggi, e che s’incammina a diventar città. Ai tempi in cui accaddero i fatti che prendiamo a raccontare, quel borgo, già considerabile, era anche un castello, e aveva perciò l’onore d’alloggiare un comandante, e il vantaggio di possedere una stabile guarnigione di soldati spagnoli, che insegnavan la modestia alle fanciulle e alle donne del paese, accarezzavan di tempo in tempo le spalle a qualche marito, a qualche padre; e, sul finir dell’estate, non mancavan mai di spandersi nelle vigne, per diradar l’uve, e alleggerire a’ contadini le fatiche della vendemmia. Dall’una all’altra di quelle terre, dall’alture alla riva, da un poggio all’altro, correvano, e corrono tuttavia, strade e stradette, più o men ripide, o piane; ogni tanto affondate, sepolte tra due muri, donde, alzando lo sguardo, non iscoprite che un pezzo di cielo e qualche vetta di monte; ogni tanto elevate su terrapieni aperti: e da qui la vista spazia per prospetti più o meno estesi, ma ricchi sempre e sempre qualcosa nuovi, secondo che i diversi punti piglian più o meno della vasta scena circostante, e secondo che questa o quella parte campeggia o si scorcia, spunta o sparisce a vicenda. Dove un pezzo, dove un altro, dove una lunga distesa di quel vasto e variato specchio dell’acqua; di qua lago, chiuso all’estremità o piuttosto smarrito in un gruppo, in un andirivieni di montagne, e di mano in mano più allargato tra altri monti che si spiegano, a uno a uno, allo sguardo, e che l’acqua riflette capovolti, co’ paesetti posti sulle rive; di là braccio di fiume, poi lago, poi fiume ancora, che va a perdersi in lucido serpeggiamento pur tra’ [p. 11 modifica]monti che l’accompagnano, degradando via via, e perdendosi quasi anch’essi nell’orizzonte. Il luogo stesso da dove contemplate que’ vari spettacoli, vi fa spettacolo da ogni parte: il monte di cui passeggiate le falde, vi svolge, al di sopra, d’intorno, le sue cime e le balze, distinte, rilevate, mutabili quasi a ogni passo, aprendosi e contornandosi in gioghi ciò che v’era sembrato prima un sol giogo, e comparendo in vetta ciò che poco innanzi vi si rappresentava sulla costa: e l’ameno, il domestico di quelle falde tempera gradevolmente il selvaggio, e orna vie più il magnifico dell’altre vedute.
\pend
\endnumbering
\end{document}