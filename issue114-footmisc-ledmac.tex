\documentclass{book}
\usepackage{fontspec}
\raggedbottom

% Ça nous permet de voir si les notes sont vraiment en bas de page.
\usepackage[showframe]{geometry}
% On s’assure quand même que les notes sont tout en bas de la page.
\usepackage[bottom]{footmisc}
% On charge ensuite eledmac
\usepackage{eledmac}

\usepackage[nopar]{kantlipsum}

\begin{document}

\section{eledpar APRÈS footmisc}

\kant[1-3]


  \beginnumbering
  \pstart\kant[2]\footnote{A footnote.}\kant[3]\pend
  \endnumbering


\section{foo}
\section{foo}
\section{foo}
\section{foo}
\section{foo}
\section{foo}
\section{foo}
\section{foo}
\section{foo}

\bigskip

\Large\bf La note de bas de page est remontée au lieu de rester en bas: l’effet
de \textbackslash{}usepackage[bottom]{footmisc} est annulé par eledpar.


\end{document}