\documentclass{article}
\usepackage[osf,p]{libertinus}
\usepackage{microtype}
\usepackage[pdfusetitle,hidelinks]{hyperref}
\usepackage[english, main=latin]{babel}
\babeltags{english = english}

\usepackage[series={A,B},noend,noeledsec,noledgroup]{reledmac}
\sethangingsymbol{[\,}


\begin{document}
\begin{english}
\date{}

\title{Stanza}
\maketitle

\begin{abstract}
This file provides an example of typesetting verse with reledmac. 

The odd numbered verses are not indented, the even ones are. When a verse is too long for a line, the continuation is printed after a long indentation and a left bracket is inserted (\verb+\sethangingsymbol+).
Two levels of critical footnotes are used.

We also have an example of notes overlapping two lines of verse, using \verb+xxref+ and \verb+lemma+.
\end{abstract}
\end{english}

\beginnumbering
\setcounter{stanzaindentsrepetition}{2}
\setstanzaindents{8,0,1}
\stanza
\edlabel{begin:1}\edtext{Lorem}{\lemma{Lorem\ldots nisis}\xxref{begin:1}{end:1}\Afootnote{A note on two verses}} ipsum dolor sit amet, consectetur adipisicing elit,&
sed do eiusmod tempor incididunt ut labore et dolore&
magna aliqua. Ut enim ad minim veniam, quis nostrud&
exercitation ullamco laboris nisi\edlabel{end:1}&
\edtext{ut aliquip}{\Afootnote{ut aliliquip}} consequat ut aliquip consequat irure dolor in reprehenderit irure dolor in reprehenderit&
\edtext{Duis aute}{\Bfootnote{Some comments}} irure dolor in reprehenderit&
in voluptate velit esse cillum dolore eu ur. Excepteur sint occaecat&
cupidatat non proident, sunt in culpa qui officia deserunt&
\edlabel{begin:2}\edtext{Duis}{\xxref{begin:2}{end:2}\lemma{Duis\ldots occaecat}\Afootnote{Another note on two verses}} aute irure dolor in reprehenderit&
in voluptate velit esse cillum dolore eu fugiat nulla&
pariatur. Excepteur sint occaecat\edlabel{end:2}\&
\endnumbering

\end{document}
