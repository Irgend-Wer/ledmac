\documentclass{article}
\usepackage{polyglossia,fontspec,xunicode}
\usepackage[showframe]{geometry}
\usepackage{libertineotf}
\setmainlanguage{english}

\usepackage[series={},nocritical,noend,noeledsec,nofamiliar,noledgroup]{reledmac}
\usepackage{reledpar}

\linenummargin{left}
\setlength{\Lcolwidth}{.44\textwidth}
\setlength{\Rcolwidth}{.44\textwidth}
\setlength{\columnrulewidth}{0.4pt}

\begin{document}


\title{\texttt{\textbackslash Columns} alignment}
\date{}
\maketitle
\begin{abstract}
This file provides example of setting the alignment of \verb+\Columns+ on the page: right (default), left or center.

The column are 0.44\verb+\textwidth+, and we use \verb+\columnsposition+.

We also provide a last example, showing how we can manipulate space between colons, by redefining \verb+\beforecolumnseparator+ and \verb+\aftercolumnseparator+.
\end{abstract}


\section{Right (default)}

\columnsposition{R}
\begin{pairs}

\begin{Leftside}
\beginnumbering
\pstart
Left side left side left side left side left side left side left side left side left side left side 
\pend
\endnumbering
\end{Leftside}

\begin{Rightside}
\beginnumbering
\pstart
Right side right side right side right side right side right side right side right side right side right side 
\pend
\endnumbering
\end{Rightside}
\end{pairs}
\Columns

\section{Left}
\columnsposition{L}

\begin{pairs}

\begin{Leftside}
\beginnumbering
\pstart
Left side left side left side left side left side left side left side left side left side left side 
\pend
\endnumbering
\end{Leftside}

\begin{Rightside}
\beginnumbering
\pstart
Right side right side right side right side right side right side right side right side right side right side 
\pend
\endnumbering
\end{Rightside}
\end{pairs}
\Columns

\section{Center}

\columnsposition{C}
\begin{pairs}

\begin{Leftside}
\beginnumbering
\pstart
Left side left side left side left side left side left side left side left side left side left side 
\pend
\endnumbering
\end{Leftside}

\begin{Rightside}
\beginnumbering
\pstart
Right side right side right side right side right side right side right side right side right side right side 
\pend
\endnumbering
\end{Rightside}
\end{pairs}
\Columns

\section{Center, setting space between columns}

Now, we set the spaces between columns, to have all the line occupied.

\setlength{\beforecolumnseparator}{0.06\textwidth}
\setlength{\aftercolumnseparator}{0.06\textwidth}

\vspace{\baselineskip}

\begin{pairs}

\begin{Leftside}
\beginnumbering
\pstart
Left side left side left side left side left side left side left side left side left side left side 
\pend
\endnumbering
\end{Leftside}

\begin{Rightside}
\beginnumbering
\pstart
Right side right side right side right side right side right side right side right side right side right side 
\pend
\endnumbering
\end{Rightside}
\end{pairs}
\Columns

\end{document}