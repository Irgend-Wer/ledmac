\documentclass[12pt]{article}
\usepackage{libertineotf}
\usepackage[noend,nofamiliar,noeledsec,series={A}]{reledmac}
\usepackage{fontspec}
\usepackage{polyglossia}
\setmainlanguage{english}
\setotherlanguage{hebrew}

\newfontfamily{\hebrewfont}[Scale=0.9]{Ezra SIL}

\Xarrangement{paragraph}
\Xbeforeinserting{\LTR}
\Xlemmaseparator{[}%Will be reversed by Ezra SIL font
\Xafterlemmaseparator{0.5em}
\Xbeforelemmaseparator{0.25em}
\Xwrapcontent{\textenglish}
\Xwraplemma{\RL}

\linenumincrement{2}
\firstlinenum{1}

\title{Editing right-to-left text with left-to-right notes}
\date{}
\begin{document}
\maketitle
\begin{abstract}


In this file, we provide an example of an edition with right-to-left text and left-to-right notes, using \XeLaTeX.

\begin{itemize}
	\item The `hebrew' environment allows us to write Hebrew right-to-left.
	\item The \verb+\Xbeforeinserting{\LTR}+ makes the critical notes to be typeset left-to-right.
	\item The \verb+\Xwraplemma{\RL}+ assures the lemmas, which are in Hebrew, be typeset right-to-left.
	\item The \verb+\Xwrapcontent{\textenglish}+ assures the content of the note is marked as English text. 
	\item As the `Ezra SIL' font transforms a `]' to a `[', we use a `[' as lemma separator, that will be typeset as `]' by `Ezra SIL'. So the need for \verb+\Xlemmaseparator+ is not directly linked to reledmac.
\end{itemize}

\end{abstract}

\begin{hebrew}
\beginnumbering


\pstart
\edtext{בְּרֵאשִׁ֖ית בָּרָ֣א}{\Afootnote{Some comment}} 
אֱלֹהִ֑ים אֵ֥ת הַשָּׁמַ֖יִם וְאֵ֥ת הָאָֽרֶץ׃
\edtext{וְהָאָ֗רֶץ הָיְתָ֥ה}{\Afootnote{Some comment}}
\edtext{
 תֹ֨הוּ֙ וָבֹ֔הוּ וְחֹ֖שֶׁךְ עַל־פְּנֵ֣י תְהֹ֑ום וְר֣וּחַ אֱלֹהִ֔ים מְרַחֶ֖פֶת עַל־פְּנֵ֥י הַמָּֽיִם׃
וַיֹּ֥אמֶר אֱלֹהִ֖ים יְהִ֣י אֹ֑ור וַֽיְהִי־אֹֽור׃
}{\Afootnote{Some comment on a long lemma}}
וַיַּ֧רְא אֱלֹהִ֛ים אֶת־הָאֹ֖ור כִּי־טֹ֑וב וַיַּבְדֵּ֣ל אֱלֹהִ֔ים בֵּ֥ין הָאֹ֖ור וּבֵ֥ין הַחֹֽשֶׁךְ׃
\edtext{וַיִּקְרָ֨א אֱלֹהִ֤ים׀}{\Afootnote{Some comment}}
 לָאֹור֙ יֹ֔ום וְלַחֹ֖שֶׁךְ קָ֣רָא לָ֑יְלָה וַֽיְהִי־עֶ֥רֶב וַֽיְהִי־בֹ֖קֶר יֹ֥ום אֶחָֽד׃ פ
\edtext{וַיַּ֤רְא אֱלֹהִים֙}{\Afootnote{Some comment}}
 אֶת־כָּל־אֲשֶׁ֣ר עָשָׂ֔ה וְהִנֵּה־טֹ֖וב מְאֹ֑ד וַֽיְהִי־עֶ֥רֶב וַֽיְהִי־בֹ֖קֶר יֹ֥ום הַשִּׁשִּֽׁי׃ פ

\pend
\endnumbering

\end{hebrew}
\end{document}  
