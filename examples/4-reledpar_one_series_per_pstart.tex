\documentclass{article} 
\usepackage{lipsum}
\usepackage[series={A,B,C,D,E,F,G,H,I,J},noend,nofamiliar,noledgroup,noeledsec]{reledmac}%Declare series to be used
\usepackage{reledpar}
\usepackage{polyglossia}
\setmainlanguage{english}
\usepackage{libertine}


% Here is the loop system:
\makeatletter% We need to use command with @
  % Two counters: one for left and one for right
  \newcounter{mynotelevel}%
    \renewcommand{\themynotelevel}{\Alph{mynotelevel}}% Print the counter as a letter
  \newcounter{mynotelevelR}%
    \renewcommand{\themynotelevelR}{\Alph{mynotelevelR}}% Print the counter as a letter
  \setcounter{mynotelevelR}{5}

  % Following lines are called at each \pstart
  \AtEveryPstartCall{%
        \ifledRcol%If it's a right pstart
          \addtocounter{mynotelevelR}{1}%Step counter
          \ifnumequal{\arabic{mynotelevelR}}{11}{%No more than J series (10)
            \setcounter{mynotelevelR}{6}% If more than J => restart to F (6)
          }{}%
            \global\letcs{\mynote}{\themynotelevelR footnote}%  And let mynote equal to <X>footnote, where <X> is the current level \letcs is defined by etoolbox which is called by reledmac        
        \else% If it's a left page
          \addtocounter{mynotelevel}{1}%Step counter
          \ifnumequal{\arabic{mynotelevel}}{6}{%No more than E series (5)
            \setcounter{mynotelevel}{1}%If more than J => restart to A (1)
          }{}%
            \global\letcs{\mynote}{\themynotelevel footnote}%And let mynote equal to <X>footnote, where <X> is the current level \letcs is defined by etoolbox which called by reledmac        
        \fi%
  }
  % This code is called when beginnings of each pstart are printed
  \AtEveryPstart{%
    \ifodd\c@page%If we are on a left page
        \addtocounter{mynotelevelR}{1}%Step counter
        \ifnumequal{\arabic{mynotelevelR}}{11}{%No more than J series (10)
          \setcounter{mynotelevelR}{6}}{}% If more than J => restart to F (6)
                    \seriesatend{\themynotelevelR}% The series should be at the end
            \else%
        \addtocounter{mynotelevel}{1}%Step counter
        \ifnumequal{\arabic{mynotelevel}}{6}{%No more than E series (5)
          \setcounter{mynotelevel}{1}}{}%% If more than E => restart to A (1)
          \seriesatend{\themynotelevel}% The series should be at the end
    \fi%
   }

\makeatother
\begin{document}

\title{One series per \textbackslash pstart with reledmac}
\author{}
\maketitle
\begin{abstract}
This file provides examples of features that allow one series of notes by \verb+\pstart+…\verb+pend+, when typesetting parallel pages.

We suppose there is no more than five \verb+\pstart+…\verb+pend+ by page. The series A, B, C, D and E are for left pages, while the series F, G, H, I, J are for right notes.

\subsection*{In the left page}
Instead of using \verb+\Afootnote+, \verb+\Bfootnote+ etc., we use \verb+\mynote+, which we let equal to \verb+\Afootnote+ at the first \verb+pstart+, \verb+\Bfootnote+ at the second \verb+\pstart+, \verb+\Cfootnote+ at the third, \verb+\Dfootnote+ at the fourth, \verb+\Efootnote+ at the fifth, and we restart at \verb+\Afootnote+ at the sixth \verb+\pstart+.

To do that, we increment the counter \verb+mynotelevel+ at every \verb+pstart+ and reset it to~1 when it becomes equal to~6.



\subsection*{In the right page}

We do the same thing, but use the \verb+mynotelevelR+ counter, rotating its values between 6 and 10, and, consequently, rotating between F, G, H, I, J series.

\subsection*{Series position}
We use \verb+\seriesatend+ to ensure that a \verb+pstart+ series between two pages will be printed on the first page at the end of the series, but on the second page at the beginning of the series.
However, we need to do this when the \verb+pstart+s are printed, not when they are called. So we use:
\begin{description}
  \item[\textbackslash AtEveryPstartCall] to set the footnote series.
  \item[\textbackslash AtEveryPstart] to set the footnote position.
\end{description}
\end{abstract}



\begin{pages}
\begin{Leftside}
    \beginnumbering


    \pstart
    \edtext{\themynotelevel Here}{\mynote{First footnote in \themynotelevel}} is the 
    \edtext{\themynotelevel first}{\mynote{Second footnote in \themynotelevel.}} paragraph: 
    \lipsum[1]
    \pend

    \pstart
    \edtext{\themynotelevel Here}{\mynote{First footnote in \themynotelevel}} is the 
    \edtext{\themynotelevel second}{\mynote{Second footnote in \themynotelevel.}} paragraph: 
    \lipsum[2] 
    \pend

    \pstart
    \edtext{\themynotelevel Here}{\mynote{First footnote in \themynotelevel}} is the 
    \edtext{\themynotelevel second}{\mynote{Second footnote in \themynotelevel.}} paragraph: 
    \lipsum[2] 
    \pend

        \pstart
    \edtext{\themynotelevel Here}{\mynote{First footnote in \themynotelevel}} is the 
    \edtext{\themynotelevel second}{\mynote{Second footnote in \themynotelevel.}} paragraph: 
    \lipsum[2] 
    \pend
        \pstart
    \edtext{\themynotelevel Here}{\mynote{First footnote in \themynotelevel}} is the 
    \edtext{\themynotelevel second}{\mynote{Second footnote in \themynotelevel.}} paragraph: 
    \lipsum[2]\edtext{on therpage}{\mynote{on otherpage}} 
    \pend

        \pstart
    \edtext{\themynotelevel Here}{\mynote{First footnote in \themynotelevel}} is the 
    \edtext{\themynotelevel second}{\mynote{Second footnote in \themynotelevel.}} paragraph: 
    \lipsum[2] 
    \pend


    \pstart
    \edtext{\themynotelevel Here}{\mynote{First footnote in \themynotelevel}} is the 
    \edtext{\themynotelevel first}{\mynote{Second footnote in \themynotelevel.}} paragraph: 
    \lipsum[1]
    \pend

    \pstart
    \edtext{\themynotelevel Here}{\mynote{First footnote in \themynotelevel}} is the 
    \edtext{\themynotelevel second}{\mynote{Second footnote in \themynotelevel.}} paragraph: 
    \lipsum[2] 
    \pend

    \pstart
    \edtext{\themynotelevel Here}{\mynote{First footnote in \themynotelevel}} is the 
    \edtext{\themynotelevel second}{\mynote{Second footnote in \themynotelevel.}} paragraph: 
    \lipsum[2] 
    \pend

        \pstart
    \edtext{\themynotelevel Here}{\mynote{First footnote in \themynotelevel}} is the 
    \edtext{\themynotelevel second}{\mynote{Second footnote in \themynotelevel.}} paragraph: 
    \lipsum[2] 
    \pend
        \pstart
    \edtext{\themynotelevel Here}{\mynote{First footnote in \themynotelevel}} is the 
    \edtext{\themynotelevel second}{\mynote{Second footnote in \themynotelevel.}} paragraph: 
    \lipsum[2]\edtext{on therpage}{\mynote{on otherpage}} 
    \pend

        \pstart
    \edtext{\themynotelevel Here}{\mynote{First footnote in \themynotelevel}} is the 
    \edtext{\themynotelevel second}{\mynote{Second footnote in \themynotelevel.}} paragraph: 
    \lipsum[2] 
    \pend
    % Many more \pstart ... \pend to follow. 

    \endnumbering
\end{Leftside}

\begin{Rightside}
    \beginnumbering

    \pstart
    \edtext{\themynotelevelR Here}{\mynote{First footnote in \themynotelevelR}} is the 
    \edtext{\themynotelevelR first}{\mynote{Second footnote in \themynotelevelR.}} paragraph: 
    \lipsum[1]
    \pend

    \pstart
    \edtext{\themynotelevelR Here}{\mynote{First footnote in \themynotelevelR}} is the 
    \edtext{\themynotelevelR second}{\mynote{Second footnote in \themynotelevelR.}} paragraph: 
    \lipsum[2] 
    \pend

    \pstart
    \edtext{\themynotelevelR Here}{\mynote{First footnote in \themynotelevelR}} is the 
    \edtext{\themynotelevelR second}{\mynote{Second footnote in \themynotelevelR.}} paragraph: 
    \lipsum[2] 
    \pend

        \pstart
    \edtext{\themynotelevelR Here}{\mynote{First footnote in \themynotelevelR}} is the 
    \edtext{\themynotelevelR second}{\mynote{Second footnote in \themynotelevelR.}} paragraph: 
    \lipsum[2] 
    \pend
        \pstart
    \edtext{\themynotelevelR Here}{\mynote{First footnote in \themynotelevelR}} is the 
    \edtext{\themynotelevelR second}{\mynote{Second footnote in \themynotelevelR.}} paragraph: 
    \lipsum[2]\edtext{on therpage}{\mynote{on otherpage}} 
    \pend

        \pstart sss\themynotelevelR
    \edtext{\themynotelevelR Here}{\mynote{First footnote in \themynotelevelR}} is the 
    \edtext{\themynotelevelR second}{\mynote{Second footnote in \themynotelevelR.}} paragraph: 
    \lipsum[2] 
    \pend

    \pstart
    \edtext{\themynotelevelR Here}{\mynote{First footnote in \themynotelevelR}} is the 
    \edtext{\themynotelevelR first}{\mynote{Second footnote in \themynotelevelR.}} paragraph: 
    \lipsum[1]
    \pend

    \pstart
    \edtext{\themynotelevelR Here}{\mynote{First footnote in \themynotelevelR}} is the 
    \edtext{\themynotelevelR second}{\mynote{Second footnote in \themynotelevelR.}} paragraph: 
    \lipsum[2] 
    \pend

    \pstart
    \edtext{\themynotelevelR Here}{\mynote{First footnote in \themynotelevelR}} is the 
    \edtext{\themynotelevelR second}{\mynote{Second footnote in \themynotelevelR.}} paragraph: 
    \lipsum[2] 
    \pend

        \pstart
    \edtext{\themynotelevelR Here}{\mynote{First footnote in \themynotelevelR}} is the 
    \edtext{\themynotelevelR second}{\mynote{Second footnote in \themynotelevelR.}} paragraph: 
    \lipsum[2] 
    \pend
        \pstart
    \edtext{\themynotelevelR Here}{\mynote{First footnote in \themynotelevelR}} is the 
    \edtext{\themynotelevelR second}{\mynote{Second footnote in \themynotelevelR.}} paragraph: 
    \lipsum[2]\edtext{on therpage}{\mynote{on otherpage}} 
    \pend

        \pstart sss\themynotelevelR
    \edtext{\themynotelevelR Here}{\mynote{First footnote in \themynotelevelR}} is the 
    \edtext{\themynotelevelR second}{\mynote{Second footnote in \themynotelevelR.}} paragraph: 
    \lipsum[2] 
    \pend

    % Many more \pstart ... \pend to follow. 

    \endnumbering
\end{Rightside}
\end{pages}
\setcounter{mynotelevel}{0}%Restart the left counter
\setcounter{mynotelevelR}{5}% And restart the right counter
\Pages

\end{document}
