\documentclass{article}
\usepackage{polyglossia,fontspec}
\usepackage{libertineotf}
\setmainlanguage{latin}
\setotherlanguage{english}

\usepackage[series={},nocritical,noend,noeledsec,nofamiliar,noledgroup]{reledmac}
\begin{document}

\begin{english}
\date{}
\title{Sectioning with reledmac (not numbered)}
\maketitle
\begin{abstract}
This file provides examples of use of sectioning commands with reledmac.
The sections are outside of the numbered text. 

The optional argument of \verb+\pstart+ is used. The optional argument of \verb+\pend+ is also used.
\end{abstract}
\end{english}

\beginnumbering
\pstart[\section{Section title}]
Lorem ipsum dolor sit amet, consectetur adipisicing elit
sed do eiusmod tempor incididunt ut labore et dolore
magna aliqua. Ut enim ad minim veniam, quis nostrud
exercitation ullamco laboris nisi
Une note à la ligne 5 consequat ut aliquip consequat\pend[\vskip 2ex]
\pstart[\subsection{Subsection title}]
Duis aute irure dolor in reprehenderit
in voluptate velit esse cillum dolore eu ur. Excepteur sint occaecat
cupidatat non proident, sunt in culpa qui officia deserunt
Duis aute irure dolor in reprehenderit
in voluptate velit esse cillum dolore eu fugiat nulla
pariatur. Excepteur sint occaecat.
\pend[]
\pstart[\subsection{Subsection title}]
Lorem ipsum dolor sit amet, consectetur adipisicing elit
sed do eiusmod tempor incididunt ut labore et dolore
magna aliqua. Ut enim ad minim veniam, quis nostrud
exercitation ullamco laboris nisi
 consequat ut aliquip consequat
\pend[\vskip\baselineskip\noindent\hfill⁂\hfill\hbox{}]
\pstart[\section{Section title}]
Irure dolor in reprehenderit
in voluptate velit esse cillum dolore eu ur. Excepteur sint occaecat
cupidatat non proident, sunt in culpa qui officia deserunt
Duis aute irure dolor in reprehenderit
in voluptate velit esse cillum dolore eu fugiat nulla
pariatur. Excepteur sint occaecat.
\pend
\endnumbering

\end{document}
