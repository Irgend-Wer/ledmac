\documentclass[twoside]{scrartcl}
\usepackage[osf,p]{libertinus}
\usepackage[english, main=latin]{babel}
\babeltags{english = english}
\usepackage{blindtext}

\usepackage[series={A},nocritical,noend,noeledsec,noledgroup]{reledmac}
\usepackage[shiftedpstarts]{reledpar}

\begin{document}

\date{}
\begin{english}
\title{Parallel pages with notes called on the right page but printed on the left page}
\maketitle

\begin{abstract}
This file provides an example of calling familiar notes in parallel pages typesetting, with footmark on the right side but notes printed on the left side.

The notes of series \verb+A+ are called on the left side with \verb+footnoteAnomk+ and the footnote  mark are inserted on the right page with \verb+footnoteAmk+.
\end{abstract}
\end{english}

\begin{pages}

  \begin{Leftside}
    \beginnumbering
    \autopar

    \blindtext\footnoteAnomk{\blindtext} \blindtext 
    
  	\blindtext

  	\blindtext\footnoteAnomk{\blindtext} \blindtext 

  	\blindtext[2]

    \endnumbering
  \end{Leftside}

  \begin{Rightside}
    \beginnumbering
    \autopar
    \blindtext\footnoteAmk \blindtext
    
	  \blindtext

	  \blindtext\footnoteAmk \blindtext

	  \blindtext[2]
    
    \endnumbering
  \end{Rightside}


\end{pages}
  \Pages

\end{document}
