\documentclass{article}
\usepackage{polyglossia,fontspec,xunicode}
\usepackage{libertineotf}
\setmainlanguage{latin}
\setotherlanguage{english}

\usepackage[series={},nocritical,noend,noreledsec,nofamiliar,noledgroup]{reledmac}
\sidenotemargin{outer}
\renewcommand{\ledlsnotefontsetup}{\small\it}% left
\renewcommand{\ledrsnotefontsetup}{\small\it}% right
\leftnoteupfalse
\rightnoteupfalse
\renewcommand{\sidenotesep}{ $|$ }

\begin{document}

\begin{english}
\title{Side notes}
\maketitle
\begin{abstract}
This file provides examples of side notes with reledmac. 
We use :
\begin{itemize}
  \item \verb+\ledleftnote+ (left notes); 
  \item \verb+\ledrightnote+ (right notes); 
  \item \verb+\ledsidenote+ (configurable notes).
\end{itemize}

For the \verb+\ledsidenote+, we put them in the outer margin.
Notes are in small size, and italic. If there is more than one note by the line, they are separated$ | $. The notes start at the level of the line they are called, and continue after.
\end{abstract}
\end{english}

\beginnumbering
\pstart
Lorem\ledrightnote{1-R}\ledleftnote{1-L} \ledsidenote{1-M}\ledsidenote{2-M}ipsum\ledrightnote{2-R} dolor\ledleftnote{2-L} sit amet, consectetur adipiscing elit. Fusce sed dolor libero. Aenean rutrum vestibulum lacus ut pretium. Fusce et auctor lectus. Ut et commodo quam, quis gravida orci. Nullam at risus elementum, suscipit enim a, pellentesque mi. Morbi commodo, ligula vel consectetur accumsan, massa metus egestas velit, eu fringilla leo ante in turpis. Vivamus ut tellus sollicitudin, facilisis ipsum sit amet, tincidunt odio. Maecenas tincidunt dolor sed ante blandit tincidunt. Etiam vulputate ultricies facilisis.
\pend
\endnumbering


\end{document}