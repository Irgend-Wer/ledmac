\documentclass{article}
\usepackage{polyglossia,fontspec,xunicode}
\usepackage{libertineotf}
\usepackage[series={A,B},nofamiliar,noend,noeledsec,noledgroup,noreledmac]{eledmac}
\setmainlanguage{latin}
\setotherlanguage{english}
\begin{document}


\begin{english}
\date{}
\title{Lemma's disambiguation}
\maketitle

\footparagraph{A}
\firstlinenum{1}
\linenumincrement{1}

\begin{abstract}
This file provides an example of lemma's disambuigation. 

All word which can potentially be twice (or more) in a same line is marked by an \verb+\sameword+. \emph{Eledmac} print the word rank only if the word is effectively printed twice (or more) time in the same line. 

For use with \verb+\lemma+, please read the handbook.
 
\end{abstract}
\end{english}





\beginnumbering
\pstart
Leo \sameword{aut} ursus \sameword{aut} oryx \sameword{aut} ricinus \sameword{aut} equus \sameword{aut}
lupus \edtext{\sameword{aut}}{\Afootnote{et}\Bfootnote{monotone\ldots}} canis \sameword{aut} felix \sameword{aut} asinus \edtext{\sameword{aut}}{\Afootnote{et}} burricus.

\pend
\endnumbering







\end{document}
