\documentclass{article}
\usepackage{fontspec,xunicode}
\usepackage{libertineotf}

\usepackage[series={A,B,C},noend,noeledsec,noledgroup]{reledmac}
\usepackage{reledpar}
\renewcommand*{\goalfraction}{0.98}
\maxhXnotes{0.25\textheight}

\usepackage{polyglossia}
\setmainlanguage{latin}
\setotherlanguage{french}
\setotherlanguage{english}

%http://remacle.org/bloodwolf/historiens/dictys/troie.htm
\newcounter{Atest}
\def\Atest{\stepcounter{Atest}(\textbf{\theAtest\ - A}) }
\newcounter{Btest}
\def\Btest{\stepcounter{Btest}(\textbf{\theBtest\ - B}) }
\newcounter{Ctest}
\def\Ctest{\stepcounter{Ctest}(\textbf{\theCtest\ - C}) }
\onlyXside[B]{L}
\onlyXside[C]{R}


\begin{document}
\date{}
\begin{english}
\title{Long notes with reledpar in parallel pages}
\maketitle
\begin{abstract}
This file provides example of managing long notes with reledpar when typesetting parallel pages.
With \verb+\maxhXnotes+, we have set the maximum height of critical notes to \verb+0.25\textheight+. With \verb+\onlyXside+, we have say B notes would  print  on left pages only, and C notes would  print on right pages only .
\end{abstract}

\end{english}



\begin{pages}
\begin{Leftside}
\begin{french}
\beginnumbering
\autopar

\edtext{Tous}{\Afootnote{\Atest Afootnote is for both left and right page. \Atest Afootnote is for both left and right page. \Atest Afootnote is for both left and right page. \Atest Afootnote is for both left and right page. \Atest Afootnote is for both left and right page. \Atest Afootnote is for both left and right page. \Atest Afootnote is for both left and right page. \Atest Afootnote is for both left and right page. \Atest Afootnote is for both left and right page. \Atest Afootnote is for both left and right page. \Atest Afootnote is for both left and right page. \Atest Afootnote is for both left and right page. \Atest Afootnote is for both left and right page. \Atest Afootnote is for both left and right page. \Atest Afootnote is for both left and right page. \Atest Afootnote is for both left and right page. \Atest Afootnote is for both left and right page. \Atest Afootnote is for both left and right page. \Atest Afootnote is for both left and right page. \Atest Afootnote is for both left and right page. \Atest Afootnote is for both left and right page. \Atest Afootnote is for both left and right page. \Atest Afootnote is for both left and right page. \Atest Afootnote is for both left and right page. \Atest Afootnote is for both left and right page. \Atest Afootnote is for both left and right page. \Atest Afootnote is for both left and right page. \Atest Afootnote is for both left and right page. \Atest Afootnote is for both left and right page. \Atest Afootnote is for both left and right page. \Atest Afootnote is for both left and right page. \Atest Afootnote is for both left and right page. \Atest Afootnote is for both left and right page. \Atest Afootnote is for both left and right page. \Atest Afootnote is for both left and right page. \Atest Afootnote is for both left and right page. \Atest Afootnote is for both left and right page. \Atest Afootnote is for both left and right page. \Atest Afootnote is for both left and right page. \Atest Afootnote is for both left and right page. \Atest Afootnote is for both left and right page. \Atest Afootnote is for both left and right page. \Atest Afootnote is for both left and right page. \Atest Afootnote is for both left and right page. \Atest Afootnote is for both left and right page. \Atest Afootnote is for both left and right page. \Atest Afootnote is for both left and right page. \Atest Afootnote is for both left and right page. \Atest Afootnote is for both left and right page. \Atest Afootnote is for both left and right page. \Atest Afootnote is for both left and right page. \Atest Afootnote is for both left and right page. \Atest Afootnote is for both left and right page. \Atest Afootnote is for both left and right page. \Atest Afootnote is for both left and right page. \Atest Afootnote is for both left and right page. \Atest Afootnote is for both left and right page. \Atest Afootnote is for both left and right page. \Atest Afootnote is for both left and right page. \Atest Afootnote is for both left and right page. \Atest Afootnote is for both left and right page. \Atest Afootnote is for both left and right page. \Atest Afootnote is for both left and right page. \Atest Afootnote is for both left and right page. \Atest Afootnote is for both left and right page. \Atest Afootnote is for both left and right page. \Atest Afootnote is for both left and right page. \Atest Afootnote is for both left and right page. \Atest Afootnote is for both left and right page. \Atest Afootnote is for both left and right page.}}
les 
\edtext{rois}{\Bfootnote {\Btest xxx Bfootnote  is only for left side. \Btest xxx Bfootnote  is only for left side. \Btest xxx Bfootnote  is only for left side. \Btest xxx Bfootnote  is only for left side. \Btest xxx Bfootnote  is only for left side. \Btest xxx Bfootnote  is only for left side. \Btest xxx Bfootnote  is only for left side. \Btest xxx Bfootnote  is only for left side. \Btest xxx Bfootnote  is only for left side. \Btest xxx Bfootnote  is only for left side. \Btest xxx Bfootnote  is only for left side. \Btest xxx Bfootnote  is only for left side. \Btest xxx Bfootnote  is only for left side. \Btest xxx Bfootnote  is only for left side. \Btest xxx Bfootnote  is only for left side. \Btest xxx Bfootnote  is only for left side. \Btest xxx Bfootnote  is only for left side. \Btest xxx Bfootnote  is only for left side. \Btest xxx Bfootnote  is only for left side. \Btest xxx Bfootnote  is only for left side. \Btest xxx Bfootnote  is only for left side. \Btest xxx Bfootnote  is only for left side. \Btest xxx Bfootnote  is only for left side. \Btest xxx Bfootnote  is only for left side. \Btest xxx Bfootnote  is only for left side. \Btest xxx Bfootnote  is only for left side. \Btest xxx Bfootnote  is only for left side. \Btest xxx Bfootnote  is only for left side. \Btest xxx Bfootnote  is only for left side. \Btest xxx Bfootnote  is only for left side. \Btest xxx Bfootnote  is only for left side. \Btest xxx Bfootnote  is only for left side. \Btest xxx Bfootnote  is only for left side. \Btest xxx Bfootnote  is only for left side. \Btest xxx Bfootnote  is only for left side. \Btest xxx Bfootnote  is only for left side. \Btest xxx Bfootnote  is only for left side. \Btest xxx Bfootnote  is only for left side. \Btest xxx Bfootnote  is only for left side. \Btest xxx Bfootnote  is only for left side. \Btest xxx Bfootnote  is only for left side. \Btest xxx Bfootnote  is only for left side. \Btest xxx Bfootnote  is only for left side. \Btest xxx Bfootnote  is only for left side. \Btest xxx Bfootnote  is only for left side. \Btest xxx Bfootnote  is only for left side. \Btest xxx Bfootnote  is only for left side. \Btest xxx Bfootnote  is only for left side. \Btest xxx Bfootnote  is only for left side. \Btest xxx Bfootnote  is only for left side. \Btest xxx Bfootnote  is only for left side. \Btest xxx Bfootnote  is only for left side. \Btest xxx Bfootnote  is only for left side. \Btest xxx Bfootnote  is only for left side. \Btest xxx Bfootnote  is only for left side. \Btest xxx Bfootnote  is only for left side. \Btest xxx Bfootnote  is only for left side. \Btest xxx Bfootnote  is only for left side. \Btest xxx Bfootnote  is only for left side. \Btest xxx Bfootnote  is only for left side. \Btest xxx Bfootnote  is only for left side. \Btest xxx Bfootnote  is only for left side. \Btest xxx Bfootnote  is only for left side. \Btest xxx Bfootnote  is only for left side. \Btest xxx Bfootnote  is only for left side. \Btest xxx Bfootnote  is only for left side. \Btest xxx Bfootnote  is only for left side. \Btest xxx Bfootnote  is only for left side. \Btest xxx Bfootnote  is only for left side. \Btest xxx Bfootnote  is only for left side. \Btest xxx Bfootnote  is only for left side. \Btest xxx Bfootnote  is only for left side. \Btest xxx Bfootnote  is only for left side. \Btest xxx Bfootnote  is only for left side. \Btest xxx Bfootnote  is only for left side. \Btest xxx Bfootnote  is only for left side. \Btest xxx Bfootnote  is only for left side. \Btest xxx Bfootnote  is only for left side. \Btest xxx Bfootnote  is only for left side. \Btest xxx Bfootnote  is only for left side. \Btest xxx Bfootnote  is only for left side. \Btest xxx Bfootnote  is only for left side. \Btest xxx Bfootnote  is only for left side. \Btest xxx Bfootnote  is only for left side. \Btest xxx Bfootnote  is only for left side. \Btest xxx Bfootnote  is only for left side. \Btest xxx Bfootnote  is only for left side. \Btest xxx Bfootnote  is only for left side. \Btest xxx Bfootnote  is only for left side. \Btest xxx Bfootnote  is only for left side. \Btest xxx Bfootnote  is only for left side. \Btest xxx Bfootnote  is only for left side. \Btest xxx Bfootnote  is only for left side. \Btest xxx Bfootnote  is only for left side. \Btest xxx Bfootnote  is only for left side. \Btest xxx Bfootnote  is only for left side. \Btest xxx Bfootnote  is only for left side. \Btest xxx Bfootnote  is only for left side. \Btest xxx Bfootnote  is only for left side. \Btest xxx Bfootnote  is only for left side. \Btest xxx Bfootnote  is only for left side. \Btest xxx Bfootnote  is only for left side. \Btest xxx Bfootnote  is only for left side. \Btest xxx Bfootnote  is only for left side. \Btest xxx Bfootnote  is only for left side. \Btest xxx Bfootnote  is only for left side. \Btest xxx Bfootnote  is only for left side. \Btest xxx Bfootnote  is only for left side. \Btest xxx Bfootnote  is only for left side. \Btest xxx Bfootnote  is only for left side. \Btest xxx Bfootnote  is only for left side. \Btest xxx Bfootnote  is only for left side. \Btest xxx Bfootnote  is only for left side. \Btest xxx Bfootnote  is only for left side. \Btest xxx Bfootnote  is only for left side. \Btest xxx Bfootnote  is only for left side. \Btest xxx Bfootnote  is only for left side. \Btest xxx Bfootnote  is only for left side. \Btest xxx Bfootnote  is only for left side. \Btest xxx Bfootnote  is only for left side. \Btest xxx Bfootnote  is only for left side. \Btest xxx Bfootnote  is only for left side. \Btest xxx Bfootnote  is only for left side. \Btest xxx Bfootnote  is only for left side. \Btest xxx Bfootnote  is only for left side. \Btest xxx Bfootnote  is only for left side. \Btest xxx Bfootnote  is only for left side. \Btest xxx Bfootnote  is only for left side. \Btest xxx Bfootnote  is only for left side. \Btest xxx Bfootnote  is only for left side. \Btest xxx Bfootnote  is only for left side. \Btest xxx Bfootnote  is only for left side. \Btest xxx Bfootnote  is only for left side. \Btest xxx Bfootnote  is only for left side. \Btest xxx Bfootnote  is only for left side. \Btest xxx Bfootnote  is only for left side. \Btest xxx Bfootnote  is only for left side. \Btest xxx Bfootnote  is only for left side. \Btest xxx Bfootnote  is only for left side. \Btest xxx Bfootnote  is only for left side. \Btest xxx Bfootnote  is only for left side. \Btest xxx Bfootnote  is only for left side. \Btest xxx Bfootnote  is only for left side. \Btest xxx Bfootnote  is only for left side. \Btest xxx Bfootnote  is only for left side. \Btest xxx Bfootnote  is only for left side. \Btest xxx Bfootnote  is only for left side. \Btest xxx Bfootnote  is only for left side. \Btest xxx Bfootnote  is only for left side. \Btest xxx Bfootnote  is only for left side. \Btest xxx Bfootnote  is only for left side. \Btest xxx Bfootnote  is only for left side. \Btest xxx Bfootnote  is only for left side. \Btest xxx Bfootnote  is only for left side. \Btest xxx Bfootnote  is only for left side. \Btest xxx Bfootnote  is only for left side. \Btest xxx Bfootnote  is only for left side. \Btest xxx Bfootnote  is only for left side. \Btest xxx Bfootnote  is only for left side. \Btest xxx Bfootnote  is only for left side. \Btest xxx Bfootnote  is only for left side. \Btest xxx Bfootnote  is only for left side. \Btest xxx Bfootnote  is only for left side. \Btest xxx Bfootnote  is only for left side. \Btest xxx Bfootnote  is only for left side. \Btest xxx Bfootnote  is only for left side. \Btest xxx Bfootnote  is only for left side. \Btest xxx Bfootnote  is only for left side. \Btest xxx Bfootnote  is only for left side. \Btest xxx Bfootnote  is only for left side. \Btest xxx Bfootnote  is only for left side. \Btest xxx Bfootnote  is only for left side. \Btest xxx Bfootnote  is only for left side. \Btest xxx Bfootnote  is only for left side. \Btest xxx Bfootnote  is only for left side. \Btest xxx Bfootnote  is only for left side. \Btest xxx Bfootnote  is only for left side. \Btest xxx Bfootnote  is only for left side. \Btest xxx Bfootnote  is only for left side. \Btest xxx Bfootnote  is only for left side. \Btest xxx Bfootnote  is only for left side. \Btest xxx Bfootnote  is only for left side. \Btest xxx Bfootnote  is only for left side. \Btest xxx Bfootnote  is only for left side. \Btest xxx Bfootnote  is only for left side. \Btest xxx Bfootnote  is only for left side. \Btest xxx Bfootnote  is only for left side. \Btest xxx Bfootnote  is only for left side. \Btest xxx Bfootnote  is only for left side. \Btest xxx Bfootnote  is only for left side. \Btest xxx Bfootnote  is only for left side. \Btest xxx Bfootnote  is only for left side. \Btest xxx Bfootnote  is only for left side. \Btest xxx Bfootnote  is only for left side. \Btest xxx Bfootnote  is only for left side. \Btest xxx Bfootnote  is only for left side. \Btest xxx Bfootnote  is only for left side. \Btest xxx Bfootnote  is only for left side. \Btest xxx Bfootnote  is only for left side. \Btest xxx Bfootnote  is only for left side. \Btest xxx Bfootnote  is only for left side. \Btest xxx Bfootnote  is only for left side. \Btest xxx Bfootnote  is only for left side. \Btest xxx Bfootnote  is only for left side. \Btest xxx Bfootnote  is only for left side. \Btest xxx Bfootnote  is only for left side. \Btest xxx Bfootnote  is only for left side. \Btest xxx Bfootnote  is only for left side. \Btest xxx Bfootnote  is only for left side. \Btest xxx Bfootnote  is only for left side. \Btest xxx Bfootnote  is only for left side. \Btest xxx Bfootnote  is only for left side. \Btest xxx Bfootnote  is only for left side. \Btest xxx Bfootnote  is only for left side. \Btest xxx Bfootnote  is only for left side. \Btest xxx Bfootnote  is only for left side. \Btest xxx Bfootnote  is only for left side. \Btest xxx Bfootnote  is only for left side. \Btest xxx Bfootnote  is only for left side. \Btest xxx Bfootnote  is only for left side. \Btest xxx Bfootnote  is only for left side. \Btest xxx Bfootnote  is only for left side. \Btest xxx Bfootnote  is only for left side. \Btest xxx Bfootnote  is only for left side. \Btest xxx Bfootnote  is only for left side. \Btest xxx Bfootnote  is only for left side. \Btest xxx Bfootnote  is only for left side. \Btest xxx Bfootnote  is only for left side. \Btest xxx Bfootnote  is only for left side. \Btest xxx Bfootnote  is only for left side. \Btest xxx Bfootnote  is only for left side. \Btest xxx Bfootnote  is only for left side. \Btest xxx Bfootnote  is only for left side. \Btest xxx Bfootnote  is only for left side. \Btest xxx Bfootnote  is only for left side. \Btest xxx Bfootnote  is only for left side. \Btest xxx Bfootnote  is only for left side. \Btest xxx Bfootnote  is only for left side. \Btest xxx Bfootnote  is only for left side. \Btest xxx Bfootnote  is only for left side.
}} de la Grèce qui descendaient de Minos, fils de Jupiter, vinrent en Crète pour y recueillir la riche succession de Crétéus. Ce prince, fils de Minos, avait réglé par son testament qu'il serait fait un partage égal de tout ce qu'il possédait d'or, d'argent et de troupeaux, entre les enfants de ses filles ; et il laissait son empire à Idoménée, fils de Deucalion, son frère et à Mérion , fils de Molus, son neveu, qui devaient gouverner chacun sa part avec un pouvoir indépendant. Entre les princes présents au partage, on distinguait Palamède, fils de Clymène et de Nauplius, et Oeax, appelés Crétéides, avec Ménélas, fils d'Aerope et de Plisthène, qu'Anaxibie, sa soeur, épouse de Nestor, et Agamemnon, son frère aîné, avaient chargé de les représenter dans l'assemblée des héritiers. On connaissait moins ces derniers comme fils de Plisthène, mort à la fleur de son âge et sans avoir rien fait de mémorable, que comme petits-fils d'Atrée. Ce prince, en effet, touché de compassion pour la faiblesse de leur âge, les avait recueillis au-près de lui, et s'était chargé de leur donner une éducation conforme à leur naissance. Ils se conduisirent tous dans cette occasion avec la grandeur et la générosité qu'on devait attendre de personnes de leur rang.

A la nouvelle de leur arrivée, tons les descendants d'Europe, dont le nom était en grande vénération dans l'île, se rendirent auprès d'eux, les saluèrent avec bonté et les conduisirent au temple. Là, après un sacrifice solennel où furent immolées suivant l'usage, nombre de victimes on leur servit un repas splendide, et on les traita avec autant d'abondance que de délicatesse. Les fêtes continuèrent les jours suivants. Les rois reçurent les témoignages de l'affection de leurs amis avec joie et reconnaissance; mais ils furent encore plus frappés de la magnificence du temple d'Europe. Ils ne pouvaient se laisser d'examiner, dans le plus grand détail, les riches présents envoyés de Sidon à cette princesse par son père Phénicie  et par ses nobles compagnes, et qui faisaient l'ornement de ce bel édifice.

Dans le même temps, Alexandre de Phrygie, fils de Priam, accompagné d'Énée  et de plusieurs de ses parents, se rendait coupable d'un grand attentat à Sparte et dans le palais de Ménélas, où il avait été reçu comme hôte, et traité tomme ami. Aussitôt après le départ du roi, épris d'amour pour Hélène, qui surpassait en beauté toutes les femmes de la Grèce, il l'enleva, et avec elle tous les trésors qu'il put emporter. Cette princesse fut accompagnée dans sa fuite par Aetra et Clymène, parentes de Ménélas, attachées à son service. La nouvelle du crime commis par Alexandre contre la maison de Ménélas parvint bientôt en Crète; et la renommée, qui se plaît ordinairement à grossir les objets, publia que le palais du roi avait été détruit, son empire renversé, et répandit d'autres bruits aussi funestes.

Ménélas, à cette nouvelle, quoique vive. meut affecté de l'enlèvement de son épouse, fut encore plus irrité de la connivence perfide qu'il crut apercevoir entre le ravisseur et ses parentes. Palamède, voyant ce prince indigné et furieux sortir du conseil sans proférer un seul mot, fait approcher de terre les vaisseaux et dispose tout pour le départ. Après quelques paroles consolantes adressées au roi, il embarque à la hâte tout ce qui provenait du partage, fait monter Ménélas avec lui sur la flotte, et, secondés d'un vent favorable, ils arrivent en peu de jours à Sparte. Déjà Agamemnon, Nestor , et tous les rois descendants de Pélops, y étaient accourus. A l'arrivée de Ménélas, ils s'assemblent ; et quoique l'atrocité de l'action leur inspiràt une profonde horreur et les portât â une prompte vengeance, cependant, après avoir délibéré mûrement, ils résolurent d'envoyer d'abord à Troie, en qualité de députés, Palamède, Ulysse et Ménélas, avec ordre de se plaindre de l'injure, et de redemander Hélène ainsi que tous les trésors enlevés.

Les députés arrivèrent bientôt à Troie et n'y trouvèrent point Alexandre. Ce prince qui, dans sa fuite précipitée, avait peu consulté les vents, s'était vu forcé de relâcher en Chypre. De là, après s'être saisi de quelques vaisseaux, il avait abordé sur la côte de Phénicie. Toujours tourmenté par cette même avidité qui l'avait accompagné à Sparte, il égorge de nuit, par trahison, le roi des Sidoniens, qui lui avait fait un accueil favorable. Tout ce que renferme le palais est le prix de son crime; toutes les richesses accumulées dans ce lieu, monuments de la grandeur royale, sont par son ordre injustement enlevées et portées sur ses vaisseaux. Cependant, aux cris lamentables de ceux qui avaient échappé aux ravisseurs, le peuple se soulève, se porte en foule au palais, et, dans le moment où Alexandre, après avoir pris tout ce qui était à sa convenance, se préparait à mettre à la voile, une troupe, armée à la hâte, se présente; le combat s'engage et se poursuit avec acharnement; nombre de combattants tombent de part et d'autre; les uns s'opiniâtrent à venger la mort de leur roi, les autres à conserver leur butin. Enfin les Troyens, après avoir eu deux de leurs vaisseaux brûlés, furent assez heureux pour sauver le reste, et échappèrent ainsi à la vengeance des Sidoniens déjà fatigués du carnage.

Sur ces entrefaites, Palamède, un des députés qui s'étaient rendus à Troie, prince à qui sa valeur dans les combats et sa sagesse dans les conseils avaient mérité la plus grande confiance, se rend au palais de Priam. Là, devant le conseil assemblé, il se plaint du crime d'Alexandre, représente les droits de l'hospitalité indignement violés par lui, observe qu'une telle action est capable de réveiller la haine entre les deux nations, rappelle le souvenir des discordes qui, pour de semblables causes, divisèrent jadis les maisons d'llus et de Pélops, et d'autres familles encore, discordes qui ont entraîné les peuples dans des guerres désastreuses. Il met sous les yeux de Priam les dangers et l'incertitude des combats, les avantages et les douceurs de la paix, l'assure qu'un forfait aussi odieux ne manquera pas d'exciter l'indignation de toute la terre, de priver ses auteurs de tout secours humain, et de les conduire à une perte inévitable, digne récompense de leur détestable impiété. Il se préparait à continuer lorsque Priam l'interrompant, lui dit:
« Modérez-vous, je vous prie, Palamède; il n'est pas juste d'accuser un absent. Il peut bien arriver que ce grand crime dont on le charge soit suffisamment détruit dans sa réplique lorsqu'il sera présent. »
Sous ce prétexte et d'autres semblables, il ordonne de suspendre l'examen de l'affaire jusqu'à l'arrivée d'Alexandre. Il voyait bien, par l'effet du discours de Palamède sur chacun des conseillers, que l'on condamnait généralement, sans ces pendant oser rien dire, d'action de son fils. En effet, le prince grec avait exposé ses plaintes avec un art admirable; il avait répandu dans son discours un intérêt touchant bien capable de produire l'effet désiré. L'assemblée se sépara ainsi ce jour-là. Ensuite Anténor, homme généreux, et surtout ami de la justice et de la vertu, conduisit dans son palais les députés, qui l'y suivirent avec joie.

Peu de jours après, le fils de Priam et ses compagnons arrivèrent, amenant avec eux la belle Hélène. Son retour mit la ville en mouvement. Les uns avaient l'action d'Alexandre en horreur ; les autres s'attendrissaient sur Ménélas, qui en était la victime. Tous étaient indignés, et personne ne cherchait à défendre le ravisseur. Priam, inquiet, appelle ses fils auprès de lui, les consulte sur ce qu'il doit faire dans une telle conjoncture : ils sont tous d'avis de ne point rendre Hélène. La vue des richesses qu'on avait enlevées avec elle les éblouissait, et ils n'ignoraient pas qu'il faudrait s'en dessaisir si on la rendait elle-même. Ils ne voyaient pas non plus avec indifférence les belles femmes de la suite d'Hélène, et se proposaient bien d'en faire leur conquête; car ces princes, dont les moeurs étaient aussi barbares que le langage, s'inquiétaient peu de ce qui était juste ou injuste, et ne voyaient dans cette affaire que deux objets qui partageaient également leur affection : le butin premièrement; ensuite le moyen d'assouvir leurs passions déréglées,

Priam, après cette réponse, les quitte, assemble les anciens, leur fait part de la résolution de ses fils et demande leur avis. Ceux-ci ne l'avaient pas encore donné, que les princes, sans garder aucune mesure, entrent tout-à-coup dans la salle du conseil, en menaçant chacun des assistants de leur vengeance, s'ils osent prendre le moindre arrêté contraitre à leurs intérêts. Cependant le peuple ne poutait retenir son indignation, et réclamait hautement contre l'injustice; il demandait satisfaction pour les députés, et pour lui-même la réparation des torts qu'il éprouvait journellement. Alexandre, toujours aveuglé par sa passion, et craignant tout d'un peuple irrité, sort accompagné de ses frères, les armes à la main, se jette au milieu de la multitude, et en fait un affreux carnage. Ce qui reste est sauvé par l'intervention des grands qui avaient assisté au conseil, et par Anténor, qui s'était mis â leur tête. Ainsi le peuple se retira méprisé, maltraité, et sans avoir rien obtenu.

Le lendemain, le roi, à la prière d'Hécube, se rend chez Hélène, la salue avec bonté, l'exhorte à prendre courage, et lui fait plusieurs questions sur son état et sur sa naissance. La princesse lui répondit que des liens de parenté l'unissaient â Alexandre, qu'elle appartenait plus à Priam et à Hécube qu'aux fils de Plisthène; et reprenant son origine de plus haut, elle dit que Danaüs et Agénor étaient leurs communs auteurs; que de Pléione, fille de Danaüs et d'Atlas, naquit Électre, qui, enceinte de Jupiter, avait mis au monde Dardanus, duquel sortirent Tros et les autres rois de Troie; que d'un autre côté, Taygète, fille d'Agénor, avait eu de Jupiter Lacédémon, père d'Amiclas ; que celui-ci donna le jour à Argalus, père d'Oebalus, qui engendra Tyndare, dont elle était la fille. Elle allégua aussi les liens qui l'unissaient à Hécube par Agénor, père de Phinée et de Phénice, aïeuls d'Hécube et de Léda, sa mère. Après avoir ainsi établi sa généalogie, elle conjura Priam et Hécube, les larmes aux yeux, de ne la point rendre aux Grecs après l'avoir prise sous leur protection. Elle ajouta que les richesses qui avaient été tirées du palais de Ménélas lui appartenaient, et qu'elle n'avait rien pris au-delà. On ne sait pas au juste si sa réponse lui fut inspirée par son amour pour Alexandre, ou par la crainte d'être punie un jour par son mari à cause de sa désertion.

Hécube, qui connaissait son désir, et voyait en elle une parente, la tenait serrée contre son sein, et suppliait son époux de ne la point rendre. Cependant Priam et les princes étaient revenus à un meilleur avis ; ils insistaient pour qu'on renvoyât la députation avec une réponse favorable, et craignaient déjà de résister à la volonté du peuple : le seul Deiphobe appuyait Hécube, sans doute parce qu'il était épris de la même passion qu'Alexandre pour la beauté d'Hélène. Hécube, de son côté, s'adressait tantôt à Priam, tantôt à ses fils, et tantôt embrassant la princesse, elle jurait que rien ne pourrait l'en séparer. De cette manière, elle entraîna à son avis tous les assistants, et les caresses d'une mère triomphèrent enfin du bonheur public. Le jour suivant, Ménélas et ses collègues se rendirent à l'assemblée, redemandant Hélène, et avec elle tentes les richesses qui avaient été enlevées. Alors Priam, debout et entouré des princes ses fils, commande le silence; il prie Hélène, qui était présente, de choisir elle-même, et de déclarer si elle voulait retourner à Sparte ou demeurer à Troie. La princesse, dit-on, fit réponse qu'elle ne voulait ni revoir sa patrie, ni rester unie à Ménélas. Ainsi les princes sortent du conseil triomphants et joyeux de posséder Hélène.

\endnumbering
\end{french}
\end{Leftside}

\begin{Rightside}
\beginnumbering
\autopar

\edtext{Cuncti}{\Cfootnote{%
\Ctest xxx Cfootnote is only for right side. \Ctest xxx Cfootnote is only for right side. \Ctest xxx Cfootnote is only for right side. \Ctest xxx Cfootnote is only for right side. \Ctest xxx Cfootnote is only for right side. \Ctest xxx Cfootnote is only for right side. \Ctest xxx Cfootnote is only for right side. \Ctest xxx Cfootnote is only for right side. \Ctest xxx Cfootnote is only for right side. \Ctest xxx Cfootnote is only for right side. \Ctest xxx Cfootnote is only for right side. \Ctest xxx Cfootnote is only for right side. \Ctest xxx Cfootnote is only for right side. \Ctest xxx Cfootnote is only for right side. \Ctest xxx Cfootnote is only for right side. \Ctest xxx Cfootnote is only for right side. \Ctest xxx Cfootnote is only for right side. \Ctest xxx Cfootnote is only for right side. \Ctest xxx Cfootnote is only for right side. \Ctest xxx Cfootnote is only for right side. \Ctest xxx Cfootnote is only for right side. \Ctest xxx Cfootnote is only for right side. \Ctest xxx Cfootnote is only for right side. \Ctest xxx Cfootnote is only for right side. \Ctest xxx Cfootnote is only for right side. \Ctest xxx Cfootnote is only for right side. \Ctest xxx Cfootnote is only for right side. \Ctest xxx Cfootnote is only for right side. \Ctest xxx Cfootnote is only for right side. \Ctest xxx Cfootnote is only for right side. \Ctest xxx Cfootnote is only for right side. \Ctest xxx Cfootnote is only for right side. \Ctest xxx Cfootnote is only for right side. \Ctest xxx Cfootnote is only for right side. \Ctest xxx Cfootnote is only for right side. \Ctest xxx Cfootnote is only for right side. \Ctest xxx Cfootnote is only for right side. \Ctest xxx Cfootnote is only for right side. \Ctest xxx Cfootnote is only for right side. \Ctest xxx Cfootnote is only for right side. \Ctest xxx Cfootnote is only for right side. \Ctest xxx Cfootnote is only for right side. \Ctest xxx Cfootnote is only for right side. \Ctest xxx Cfootnote is only for right side. \Ctest xxx Cfootnote is only for right side. \Ctest xxx Cfootnote is only for right side. \Ctest xxx Cfootnote is only for right side. \Ctest xxx Cfootnote is only for right side. \Ctest xxx Cfootnote is only for right side. \Ctest xxx Cfootnote is only for right side. \Ctest xxx Cfootnote is only for right side. \Ctest xxx Cfootnote is only for right side. \Ctest xxx Cfootnote is only for right side. \Ctest xxx Cfootnote is only for right side. \Ctest xxx Cfootnote is only for right side. \Ctest xxx Cfootnote is only for right side. \Ctest xxx Cfootnote is only for right side. \Ctest xxx Cfootnote is only for right side. \Ctest xxx Cfootnote is only for right side. \Ctest xxx Cfootnote is only for right side. \Ctest xxx Cfootnote is only for right side. \Ctest xxx Cfootnote is only for right side. \Ctest xxx Cfootnote is only for right side. \Ctest xxx Cfootnote is only for right side. \Ctest xxx Cfootnote is only for right side. \Ctest xxx Cfootnote is only for right side. \Ctest xxx Cfootnote is only for right side. \Ctest xxx Cfootnote is only for right side. \Ctest xxx Cfootnote is only for right side. \Ctest xxx Cfootnote is only for right side. \Ctest xxx Cfootnote is only for right side. \Ctest xxx Cfootnote is only for right side. \Ctest xxx Cfootnote is only for right side. \Ctest xxx Cfootnote is only for right side. \Ctest xxx Cfootnote is only for right side. \Ctest xxx Cfootnote is only for right side. \Ctest xxx Cfootnote is only for right side. \Ctest xxx Cfootnote is only for right side. \Ctest xxx Cfootnote is only for right side. \Ctest xxx Cfootnote is only for right side. \Ctest xxx Cfootnote is only for right side. \Ctest xxx Cfootnote is only for right side. \Ctest xxx Cfootnote is only for right side. \Ctest xxx Cfootnote is only for right side. \Ctest xxx Cfootnote is only for right side. \Ctest xxx Cfootnote is only for right side. \Ctest xxx Cfootnote is only for right side. \Ctest xxx Cfootnote is only for right side. \Ctest xxx Cfootnote is only for right side. \Ctest xxx Cfootnote is only for right side. \Ctest xxx Cfootnote is only for right side. \Ctest xxx Cfootnote is only for right side. \Ctest xxx Cfootnote is only for right side. \Ctest xxx Cfootnote is only for right side. \Ctest xxx Cfootnote is only for right side. \Ctest xxx Cfootnote is only for right side. \Ctest xxx Cfootnote is only for right side. \Ctest xxx Cfootnote is only for right side. \Ctest xxx Cfootnote is only for right side. \Ctest xxx Cfootnote is only for right side. \Ctest xxx Cfootnote is only for right side. \Ctest xxx Cfootnote is only for right side. \Ctest xxx Cfootnote is only for right side. \Ctest xxx Cfootnote is only for right side. \Ctest xxx Cfootnote is only for right side. \Ctest xxx Cfootnote is only for right side. \Ctest xxx Cfootnote is only for right side. \Ctest xxx Cfootnote is only for right side. \Ctest xxx Cfootnote is only for right side. \Ctest xxx Cfootnote is only for right side. \Ctest xxx Cfootnote is only for right side. \Ctest xxx Cfootnote is only for right side. \Ctest xxx Cfootnote is only for right side. \Ctest xxx Cfootnote is only for right side. \Ctest xxx Cfootnote is only for right side. \Ctest xxx Cfootnote is only for right side. \Ctest xxx Cfootnote is only for right side. \Ctest xxx Cfootnote is only for right side. \Ctest xxx Cfootnote is only for right side. \Ctest xxx Cfootnote is only for right side.%
}} reges, qui Minois Jove geniti, pronepotes, Graeciae imperitabant, ad dividendas inter se Cretei opes, Cretam convenere; Creteus namque ex Minoe, postrema sua ordinans, quidquid auri atque argenti, pecorum etiam fuit, nepotibus, quos filiae genuerant, ex aequo dividendum reliquerat, excepto civitatum terrarumque imperio; haec quippe Idomeneus cum Merione, Deucalionis Idomeneus, alter Moli, jussu ejus seorsum habuere. Convenere autem Clymenae et Nauplii Palamedes, et Oeax, dicti Creteidae : item Menelaus, Aeropa et Plisthene genitus, a quo Anaxibia soror, quae eo tempore Nestori denupta erat, et Agamemnon major frater, ut vice sua in divisione uteretur, petiverant. Sed hi non Plisthenis, ut erant, magis quam Atrei dicebantur; ob eam causam, quod quum Plisthenes, admodum parvus ipse agens in primis annis vita functus, nihil dignum ad memoriam nominis reliquisset, Atreus miseratione aetatis secum eos habuerat, neque minus quam regios educaverat. In qua divisione singuli pro nominis celebritate inter se quisque magnifice transegere.

Ad eos re cognita omnes ex origine Europae, quae in ea insula summa religione colitur, confluunt benigneque salutatos in templum deducunt. Ibi multarum hostiarum immolatione celebrata, exhibitisque epulis, largiter magnificeque eos habuere : itemque insecutis diebus. At reges Graeciae, etsi ea quae exhibebantur cum laetitia accipiebant, tamen multo magis templi ejus magnifica pulchritudine, pretiosaque exstructione operum afficiebantur, inspicientes repetentesque memoria, singula, quae ex Sidone a Phoenice patre ejus, atque nobilibus matronis transmissa magno tum decori erant.

Per idem tempus Alexander Phrygius, Priami filius, cum Aenea aliisque ex consanguinitate comitibus, Spartae in domum Menelai hospitio receptus, indignissimum facinus perpetraverat. Is namque ubi animadvertit regem abesse, quod erat Helena praeter caeteras Graeciae faeminas miranda specie, amore ejus captus, ipsamque et multas opes domo ejus aufert, Aethram etiam et Clymenam Menelai adfines, quae ob necessitudinem cum Helena agebant. Postquam Cretam nuncius venit, et cuncta quae ab Alexandro adversus domum Menelai commissa erant, aperuit, per omnem insulam, sicut in tali re fieri amat, fama in majus divulgatur: expugnatam quippe domum regis, eversumque regnum, et alia in talem modum singuli disserebant.


Quibus cognitis Menelaus, etsi abstractio conjugis animum permoverat, multo amplius tamen ob injuriam adfinium, quas supra memoravimus, consternabatur. At ubi animadvertit Palamedes, regem ira atque indignatione stupefactum, concilio excidisse, ipso naves parat, atque omni instrumento compositas terrae applicat. Dein pro temporo regem breviter consolatus, positis etiam ex divisione, quae in tali negotio tempus patiebatur, navem ascendere facit : atque ita ventis ex sententia flantibus, paucis diebus Spartam pervenere. Eo jam Agamemnon et Nestor, omnesque qui ex origine Pelopis in Graecia regnabant, cognitis rebus confluxerant. Igitur postquam Menelaum advenisse sciunt, in unum coeunt. Et quanquam atrocitas facti ad indignationem, ultumque injurias rapiebat, tamen ex consilii sententia legantur prius ad Troiam Palamedes, Ulysses et Menelaus ; hisque mandatur, ut conquesti iniurias, Helenam, et quae cum ea abrepta erant, repeterent.

Legati paucis diebus ad Trojiam veniunt, neque tum Alexandrum in loco offendere. Eum namque properatione navigii inconsulte usum venti ad Cyprum appulere. Unde sumptis aliquot navibus, Phoenicem delapsus, Sidoniorum regem, qui eum amice susceperat, noctu per insidias necat : eademque qua apud Lacedaomonam, cupiditate, universam domum ejus in scelus proprium convertit. Ita omnia quae ad ostentationem regiae magnificentiae fuere, indigne rapta, ad naves deferri jubet. Sed ubi ex lamentatione eorum qui casum domini deflentes, reliqui praedae aufugerant, tumultus ortus est, populus omnis ad regiam concurrit. Inde quod jam Alexander, abreptis quae cupiebat, ascensionem properabat, pro tempore armati ad naves veniunt; ortoque inter eos acri proelio, cadunt utrinque plurimi, quum obstinate hi regis necem defenderent, hi ne amitterent partam praedam summis opibus adniterentur. Incensis dein duabus navibus, Trojani reliquas strenue defensas liberant, atque ita fatigatis jam proelio hostibus evadunt.

Interim apud Trojam, legatorum Palamedes, cujus maxime ea tempestate demi bellique consilium valuit, ad Priamum adit, conductoque concilio, primum de Alexandri injuria conqueritur, exponens conmunis hospitii eversionem. Dein monet, quantas ea res inter duo regna simultates concitatura esset, interjaciens memoriam discordiarum Ili et Pelopis, aliorumque qui ex causis similibus ad internecionem usque gentium pervenissent. Ad postremum belli difficultates, contraque pacis commoda adstruens, « Non se ignorare, ait, quantis mortalibus tam atrox facinus indignationem incuteret ; ex quo auctores injuriae ab omnibus derelictos, impietatis supplicia subituros. » Et quum plura dicere cuperet, Priamus medium ejus interrumpens sermonem, « Parcius, quaeso, Palamedes, inquit ; iniquum enim videtur, insimulari eum qui absit : maxime quum fieri possit, uti quae criminose objecta sunt, praesenti refutatione diluantur. » Haec atque alia hujusmodi inferens, differri querelas ad adventum Alexandri jubet. Videbat enim, ut singuli, qui in eo concilio aderant, Palamedis oratione moverentur; ut taciti vultu tamen admissum facinus condemnarent; quum singula miro genere orationis exponerentur, atque in sermone Graeci regis inesset quaedam permixta miserationi vis. Atque ita eo die consilium dimittitur. Sed legatos Antenor, vir hospitalis, et praeter caeteros boni hunestique sectator, domum ad se volentes deducit.

Interim paucis post diebus, Alexander cum supradictis comitibus venit, Helenam secum habens. Cujus adventu in tota civitate quum partim exemplum facinoris exsecrarentur, alii injurias in Menelaum admissas dolerent, nullo omnium adprobante, postremo cunctis indignantibus, tumultus ortus est. Queisrebus anxius Priamus filios convocat, eosque quid super tali agendum negotio videretur, consulit : qui una voce, minime reddendam Helenam, respondent. Videbant quippe quantae opes cum ea advectae essent : quae universa, si Helena traderetur, necessario amitterent. Praeterea permoti forma mulierum quae cum Helena venerant, nuptias sibi singularum jam animo destinaverant. Quippe qui lingua moribusque barbari, nihil pensi aut consulti patientes, praeda ac libidine transversi agebantur.

Igitur Priamus, relictis his, senes conducit, sententiam filiorum aperit. Dein cunctos, quid agendum sit, consulit. Sed priusquam ex more sententiae dicerentur, reguli repente concilium irrumpunt, atque inconditis moribus malum singulis minitantur, si aliter quam ipsis videretur, decernerent. Interim omnis populus indigne admissam injuriam, atque in hunc modum multa alia, cum exsecratione reclamabant. Ob quae Alexander cupidine animi praeceps, veritus ne quid adversum se a popularibus oriretur, stipatus armatis fratribus impetum in multitudinem facit, multosque obtruncat : reliqui interventu procerum, qui in consilio fuerant, duce liberantur Antenore. Ita infectis rebus, populus contemptui habitus, non sine pernicie sua, domum discedit.

Dein secuta die rex hortatu Hecubae ad Helenam adit, eamque benigne salutans, animum bonum uti gereret hortatur; quae cujusque erset, requirit. Tum illa Alexandri se adfinem respondit, magisque ad Priamum et Hecubam, quam ad Plisthenis filios genere pertinere, repetens originem omnem majorum. Danaum enim atque Agenorem, et sui et Priami generis auctores esse. Namque ex Pleiona, Danai filia et Atlante, Electram natam, quam ex Jove gravidam Dardanum genuisse, ex quo Tros, et deinceps insecuti reges llii. Agenoris porro Taygetam; eam ex Jove habuisse Lacedaemonem, ex quo Amiclam natum, et ex eo Argalum patrem Oebali, quem Tyndari, ex quo ipsa genita videretur, patrem constaret. Repetebat etiam cum Hecuba materni generis adfinitatem. Agenoris quippe Phineum et Phoenicem, et inde patres Hecubae et Ledae consanguinitate originem divisisse. Postquam memoriter cuncta retexuit, ad postremum flens orare, ne se, quae semel in fldem illorum recepta esse, prodendam putarent. Ea secum domo Menelai apportata quae propria fuissent, nihil praeterea ablatum. Sed utrum immodico amore Alexandri, an poenarum metu, quas ob desertam domum a conjuge metuebat, ita sibi consulere maluerit, parum constabat.

Igitur Hecuba, cognita voluntate, simul ob generis conjunctionem, complexa Helenam, ne proderetur, summis opibus adnitebatur : quum jam Priamus et reliqui reguli non amplius differendos legatos dicerent, neque resistendum popularium voluntati; solo omnium Deiphobo Hecubae assenso : quem non aliter atque Alexandrum, Helenae desiderium a recto consilio praepediebat. Itaque quum obstinate Hecuba nunc Priamum, modo filios deprecaretur, modo complexu ejus nulla rations divelli posset, omnes qui aderant in voluntatem suam transduxit. Ita ad postremum bonum publicum materna gratia corruptum est. Dein postero die Menelaus cum suis in concionem venit, conjugem, et quae cum ea abrepta essent, repetens. Tunc Priamus inter regulos medius adstans, facto silentio, optionem Helenae, quae ob id in conspectu popularium venerat, offert, si ei videretur, dornum ed suos regredi. Quam ferunt dixisse, neque se patriam regredi velle, neque sibi cum Menelai matrimonio convenire. Ita reguli habentes Helenam, non sine exultatione ex concione discedunt.


\endnumbering
\end{Rightside}

\end{pages}
\Pages
\end{document}