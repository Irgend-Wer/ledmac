\documentclass{book}
\usepackage[a4paper]{geometry}
\usepackage{fontspec}
\setmainfont[Ligatures=TeX]{Linux Libertine O}
\usepackage{xunicode}
\usepackage{polyglossia}

\usepackage{eledmac}
\usepackage{eledpar} 

\linenumincrement*{1}
\firstlinenum*{1}
\setlength{\Lcolwidth}{0.48\textwidth}
\setlength{\Rcolwidth}{0.48\textwidth} 

\setmainlanguage{english}
\setotherlanguage{russian}
\setotherlanguage{hebrew}    
\newfontfamily\hebrewfont[Script=Hebrew]{Linux Libertine O}
\newfontfamily\russianfont[Script=Cyrillic]{Linux Libertine O}    

\begin{document}


\title{Parallel columns in two different languages}



{\let\newpage\relax\maketitle}
{\small
This file provides example of typesetting parallel columns in two different language, using eledpar and polyglossia.

The main language is English. The left column is in Russian, the right column in Hebrew.
We have enlarged \verb+\Lcolwidth+ and  \verb+\Rcolwidth+ to avoid problem with hyphenations.

We also used \verb+\eledchapter+, which implies that chapters' titles are printed with line numbering beside. 
}



\begin{pairs}

\begin{Rightside} 
\begin{RTL}
\begin{hebrew}
\beginnumbering
\pstart
\eledchapter{המאמר השני}
\pend
\pstart
בפנות התוריות, ר״ל שהם יסודות ועמודים אשר בית אלהים נכון עליהם, ובמציאותם יציר מציאות התורה מסדרת ממנו יתברך, ואלו יציר העדר אחת מהם תפל התורה בכללה חלילה.
\pend    
\endnumbering
\end{hebrew}
\end{RTL}
\end{Rightside}




\begin{Leftside} 
\begin{russian}
\beginnumbering
\pstart
\eledchapter{Трактат Второй}   
\pend 
\pstart
О краеугольных [принципах] Торы, имеется ввиду, которые [есть] основы и столпы на которых дом Б-жий опирается/нахон, и с существованием их может быть представлено существование Торы упорядоченной от Него, благословенного, и если бы было представлено отсутствие одного из них — упадет Тора в общем, [Б-же] упаси.
\pend
\endnumbering
\end{russian}
\end{Leftside}
\Columns
\end{pairs}
\end{document}