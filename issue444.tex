\documentclass[10pt,b5paper,twoside,noinputencoding,]{memoir}
\usepackage[T1]{fontenc}             
\usepackage[utf8]{inputenc}
\usepackage{lmodern}
\usepackage[french,latin.classic,greek,english,italian]{babel}
\usepackage{teubner}
\usepackage{fixltx2e}
%...............................
\usepackage[series={A,B,C,D,E},noend,noeledsec,noledgroup]{reledmac}
\usepackage{reledpar}
%https://www.ctan.org/tex-archive/macros/latex/contrib/reledmac/examples
%----------------------------------------------------------
%\renewcommand*{\setgoalfraction}{0.9}
%\renewcommand*{\setgoalfraction}{1.5}
\setgoalfraction{0.95}
\Xmaxhnotes{0.45\textheight} % Spazio prima note. Era 0,25
\newcounter{Atest}
\def\Atest{\stepcounter{Atest}(\textbf{\theAtest\ - A}) }
\newcounter{Btest}
\def\Btest{\stepcounter{Btest}(\textbf{\theBtest\ - C}) }
\newcounter{Ctest}
\def\Ctest{\stepcounter{Ctest}(\textbf{\theCtest\ - B}) }
\Xonlyside[B]{L}
\Xonlyside[C]{R}
\firstlinenum*{1}
\linenumincrement*{1}
\begin{document}


\begin{pages}
  \begin{Leftside}
  \beginnumbering
  
\pstart\relax
[5]    \edtext{t~w| meg'ejei talika'utan e>~imen}{\Afootnote{\Atest: <<abbia tale grandezza che>>,   ln.  \SEref{magnitudo}. Il capitolo ........ }},  ...... 	
\pend
\endnumbering
\end{Leftside}

\begin{Rightside}
  \beginnumbering
\pstart\relax
[5]  Suppone questi infatti .... sia pure posto nel Sole,   \edlabelSE{magnitudo} abbia tale grandezza   che   il cerchio [descritto], lungo il quale egli suppone che la Terra rivolga ....
\pend
\endnumbering
\end{Rightside}

\end{pages}

\Pages

\end{document}
