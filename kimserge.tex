\documentclass[12pt,twoside]{book}
\usepackage[a4paper,top=3.5cm,bottom=3.5cm,inner=3.5cm,outer=3.5cm]{geometry}
\usepackage{fontspec}
\setmainfont[Scale=1,WordSpace=1,Mapping=tex-text]{Linux Libertine O}
%\setmainfont[Ligatures=TeX]{Linux Libertine O}
\newfontfamily\russianfont[Scale=1,WordSpace=1.5]{FreeSerif}
\usepackage{xltxtra,xunicode,amsmath}
\usepackage{polyglossia}


\setmainlanguage{french}
\setotherlanguage{latin}
\setotherlanguage{greek}
\setotherlanguage{russian}    

\usepackage{eledmac}
\noeledsec
\noendnotes
\usepackage[shiftedpstarts]{eledpar}

\symlinenum{$\parallel$}

\footparagraph{A} %все сноски последовательно
%\footparagraph{B}
\footparagraph{C}
\footparagraph{D}
\footparagraph{E}

\Xnotenumfont[A,B,C,D,E]{\bfseries}

\Xlemmadisablefontselection[A,B,C,D,E] 

\linenumincrement*{5}
\firstlinenum*{5}

\newcommand{\Dn}[2]{\edtext{#1}{\Dfootnote{#2}}}

\renewcommand{\Rlineflag}{*}

\let\oldDfootfmt\Dfootfmt
\renewcommand{\Dfootfmt}[3]{%
\let\printlines\printlinesR
\oldDfootfmt{#1}{#2}{#3}}

\newcommand{\Fol}[1]{\textbf{\textlatin{\{#1\}}}}

\newcommand{\Agg}{\ledsidenote{\textlatin{{\footnotesize \textit{Aggaeus}}}}}
\newcommand{\Mar}{\ledsidenote{\textlatin{{\footnotesize \textit{Marinus}}}}}
\newcommand{\Est}{\ledsidenote{\textlatin{{\footnotesize \textit{tres martyres}}}}} 
\newcommand{\Bac}{\ledsidenote{\textlatin{{\footnotesize \textit{Bacchus}}}}}
\newcommand{\The}{\ledsidenote{\textlatin{{\footnotesize \textit{Theophano}}}}}
\newcommand{\bog}{\ledsidenote{\textlatin{{\footnotesize \textit{theotokion}}}}}
\newcommand{\Eir}{\ledsidenote{\textlatin{{\footnotesize \textit{eirmos}}}}}
\newcommand{\Pro}{\ledsidenote{\textlatin{{\footnotesize \textit{prosomoion}}}}}


\nonumberinfootnote[A]

\newcommand{\trop}[2]{\edtext{\textlatin{\textbf{<#1.>}} }{\lemma{\textfrench{\textbf{#1:}} }\Afootnote[nosep]{\textfrench{\textbf{#2}}}}}

\newcommand{\app}[1]{\textlatin{\textit{#1}}\thinspace}

\newcommand{\mP}{\textlatin{\textbf{P}}\thinspace}
\newcommand{\mS}{\textlatin{\textbf{S}}\thinspace}
\newcommand{\mA}{\textlatin{\textbf{A}}\thinspace}
\newcommand{\mB}{\textlatin{\textbf{B}}\thinspace}

\AtEveryPend{\vskip\baselineskip} % space after stanza in parallel

\begin{document}
\maxhXnotes{0.6\textwidth}
%\maketitle

%\section{Introduction}

\section{Édition}\sloppy

\begin{pages}
\begin{Leftside}
\beginnumbering
\begin{russian}

\pstart
\begin{center}
\textbf{Canon.}\skipnumbering 
\textbf{Ode 1.}\skipnumbering
\end{center}
\pend

\pstart
\begin{center}
\textbf{უგალობსა}.
\textbf{ჴმაჲ ბ.}
\end{center}
\pend
\setcounter{stanzaindentsrepetition}{5}
\setstanzaindents{8,2,3,3,3,3}
\begin{astanza}\Eir
\textit{უგალობდეთ უფალსა რომელმან წ{}}.&
\noindent\skipnumbering\&
\end{astanza}
\begin{astanza}\Agg
შენ წინაჲსწარმეტყუელსა,&
წმიდასა და ანგელოწსა ღმრთისა და მოსლვისა,&
მისისა კაცთა მომართ ქადაგსა უბიწოსა,&
სარწმუნოებით გიგალობთ ყოვლად დიდებულო჻&
\noindent\skipnumbering\&
\end{astanza}
\begin{astanza}\Agg\trop{ii}{\mA ii, \mB ---}
რომელთა (ვიცანთ) ღმერთი,&
ყოვლისა-მპყრობელი და რომელნი განვერენით,&
ტყუეობისაგან სულისა, ტაძრად დიდებისა დღეს,& 
და უკლესიად ჭმრთისად თავნი \edtext{ჩუენნი}{\Bfootnote{Fruit de longues heures d’écriture, cette documentation est une somme de connaissances issue de la communauté SPIP. Tout ce travail est distribué sous licence libre Creative Commons - Paternité - Partage des Conditions Initiales à l’Identique (cc-by-sa). Vous pouvez utiliser ces textes quel que soit l’usage (y compris commercial), les modifier et les redistribuer à condition de laisser à vos lecteurs la même liberté de partage.
Cette œuvre fait l’objet de nombreuses relectures mais n’est certainement pas indemne de toute erreur. N’hésitez pas à proposer des améliorations ou signaler des coquilles en utilisant le formulaire de suggestion mis à disposition sur le site internet de la documentation. Vous pouvez aussi discuter de l’organisation (des contenus, de la technique) et des traductions sur la liste de discussion « spip-programmer » (sur abonnement).
Si vous êtes motivé par ce projet, vous pouvez proposer d’écrire un chapitre sur un sujet que vous maîtrisez ou refondre un chapitre existant pour le clarifier ou le compléter. Nous essaierons alors de vous accompagner et vous soutenir dans cette tâche.
Vous pouvez également participer à la traduction de cette documentation. L’espace privé du site permet de discuter des traductions en cours d’élaboration. Nous accueillerons très progressivement d’autres langues si des volontaires veulent s’atteler à cette immense et admirable tâche.
Par souci de compatibilité, les codes informatiques qui servent d’exemple ne contiennent généralement que des caractères du code ASCII. Cela signifie entre autre que vous ne trouverez que rarement des accents dans les commentaires accompagnant les exemples de code dans l’ensemble de la documentation. Ne soyez donc pas étonnés par cette absence.
Présentation et fonctionnement général. 
SPIP 3 est un logiciel libre développé sous licence GNU/GPL3. Utilisé comme un système de publication de contenu – sa vocation première – il est également une plateforme de développement permettant de créer des interfaces maintenables et extensibles quelle que soit la structure des données gérées.
SPIP est particulièrement adapté pour des portails éditoriaux mais peut aussi bien être utilisé comme système d’auto-publication (blog), de wiki, de réseau social ou pour gérer toute donnée issue de MySQL ou SQLite.
Il sait également sélectionner facilement des données formatées en XML, JSON, CSV ou YAML et de manière extensive, toute information transformée en tableau PHP.}} აღიფლბენნეთ჻&
\noindent\skipnumbering\&
\end{astanza}
\begin{astanza}\Mar\trop{iii}{\mA iii, \mB ---}
დიდებულმან უფალმან,& 
და დიდმან მღდელთ-მოძღუარემან და წინაჲსწარმეტყუელებისა,& 
სრულ მყოფელმან ღმრთივ-დიდებისა თჳსისა,&
ტაძარი სულიერი ქალწულისაგან იშენა჻&
\noindent\skipnumbering\&
\end{astanza}
\begin{astanza}\Mar\trop{iv}{\mA iv, \mB ---}
შეჭურვილი ახოვნად,&
მძჳნვარეთა მტერთა მიმართ და დამცემელი სიმჴნით,&
ზუავსა მას მძლავრებასა ულურებითა ჴორცთაჲთა,&
გჳრგჳნოსან ვყოთ ერნო მარინოს, ძლევისა ებნითა჻&
\noindent\skipnumbering\&
\end{astanza}
\begin{astanza}\Mar\trop{v}{\mA v, \mB ---}
გონებაჲ გონიერ,&
და სული წარმატეტებელი მოიგე სიყრმითებენვე,&
და ვითარცა რაჲ ჩჩჳლსა უგუნურსა ამხილებენ,&
დიდსა მას გონებასა ზუაობით სიქად(უ)ლი჻&
\noindent\skipnumbering\&
\end{astanza}
\begin{astanza}\Est\trop{vi}{\mA vi, \mB ---}
ქრისტესა ცხოველი ყვნეს,&
ევსტათი, ანატოლიოს და თესპესიოს ღირსნი,&
ძმანი სულითა უფროჲს ვიდრეღარა ჴორცთა ბრძენნი,&
რომელთა ერთობით წამეს ჭეშმარიტისათჳს჻&
\noindent\skipnumbering\&
\end{astanza}
\begin{astanza}\Est\trop{vii}{\mA vii, \mB ---}
\Fol{\mA237} 
ბაძვად ვნებათა შენთა,&
გულმან უთქ()ა სამთა მოწამეთა და გემსგავსნეს შენ,&
მოუთმენელთა მათ ტანჯვათა თავს-დებითა და ჴმა-ყვეს,& 
ერთსა სამებისა განსა ჴორც შესხმულსა გიგალობთ ქრისტე჻&
\noindent\skipnumbering\&
\end{astanza}
\begin{astanza}\Bac\trop{viii}{\mA viii, \mB ---}
წინა-მდებარე არს დღეს,&
საშუებელად უბურველად ბაქოს მადლით აღვივსნეთ,&
განუძღომელისა მის მოღუაწებისა მისისაგან,&
და წამებისა ღუაწლთა რამეთუ დიდებულ არს჻&
\noindent\skipnumbering\&
\end{astanza}
\begin{astanza}\The\trop{ix}{\mA ix, \mB ---}
გჳრგჳნოსან-მყოფელსა,&
თეოფანიაჲსსა განკითხვათა შენთა უფსკრ(ე)ლითა,&
და მეფობისა მიმცემელსა გიგალობთ კაცთ-მოყუარე,&
ქ(ე)ბისა გალობითა რამეთუ დიდებულ არს჻&
\noindent\skipnumbering\&
\end{astanza}
\begin{astanza}\The\trop{x}{\mA x, \mB ---}
გაქუნდა სახელი სეხნაჲ,&
ჭეშმარიტად საღმრთოჲ ნათლისა გამოცხადებისაჲ,&
წინამძღუარად დაეფარე მთავარსა ბნელისასა,& 
თეოფანია ღირსო ქრისტეს ღმთისა მსახურო჻&
\noindent\skipnumbering\&
\end{astanza}
\begin{astanza}\The\trop{xi}{\mA xi, \mB ---}
განაშუენე გჳრგჳნი,&
მეფპბისაჲ სათნოებათა შარავანდედითა,& 
და გამოშჱნდი სათნოდ მეფისა მეფეთაჲსა,&
ღმერთ შემოსილო ბრძენო ნეტარო თეოფანია჻&
\noindent\skipnumbering\&
\end{astanza}
\begin{astanza}\bog\trop{xii}{\mA xii, \mB ---}
მხოლოსა ღმრთის-მშობელსა,& 
რომელი თანა-წარჰჴდა შჯულსა ბუნებათასა,&
პატივ ვსცემდეთ მის მიერ ჴსნილნი პატიჟისაგან,&
ცოდვისა მამულისა შობითა მით მისითა.&
\noindent\skipnumbering\&
\end{astanza}

\end{russian}

\endnumbering
\end{Leftside}

\begin{Rightside}
\beginnumbering

\begin{greek}

\pstart
\begin{center}
\begin{latin}
\textbf{Canon.}\\ \skipnumbering
\textbf{Ode 1}\skipnumbering
\end{latin}
\end{center}
\pend

\pstart
\begin{center}
\textbf{\Dn{ᾨδὴ α´}{Ὁ κανὼν \mP}.} 
Ἦχος β´.
\end{center}
\pend
\setcounter{stanzaindentsrepetition}{6}
\setstanzaindents{10,6,7,7,7,7,7}
\begin{astanza}\Eir
\Fol{\mS~106r, \mP48v}
\textit{Τῷ μεταστρέψαντι,&
τὴν θάλασσαν εἰς ξηράν,&
καὶ διαγαγόντι τὸν Ἰσραὴλ,& 
\Dn{ἐκ μέσου αὐτῆς}{δι᾽ αὐτῆς \mP},&
ᾄσωμεν τῷ Χριστῷ,&
ὅτι \edtext{δε<δόξασται εἰς τοὺς αἰῶνας>.}{\Dfootnote{\app{suppleui e} \mP, \app{abbreviatum in} \mS}}}
\&
\end{astanza}
\begin{astanza}\Agg\trop{ii}{\mS ii, \mP ii}
Προφήτην ἅγιον,&
καὶ ἄγγελον τοῦ Θεοῦ&
καὶ τῆς πρὸς ἀνθρώπους,&
ἐπιδημίας αὐτοῦ,&
κύρηκα ἱερόν,&
ἀνευφημοῦμέν σε πιστοὶ Ἀγγαῖε.\&
\end{astanza}
\begin{astanza}\Agg\trop{iii}{\mS iii, \mP iii}
Τὸν παντοκράτορα,&
οἱ \Dn{ἐπιγνῶντες}{ἐπιγνόντες \mP} Θεόν,&
οἱ διαφυγόντες αἰχμαλωσίαν ψυχῆς,&
οἶκος δόξης αὐτῷ,&
ἁρμολογούμεθα τῇ ἐκκλησίᾳ.\&
\end{astanza}
\begin{astanza}\trop{iv}{\mS iv, \mP ---}
Ὁ μέγας Κύριος,&
ὁ μέγας ἀρχιερεύς,&
ὁ τῆς προφητείας τελειοτὴς Ἰησούς,&
ἔμψυχον ἑαυτῷ,&
ναὸν προσέλαβεν ἐκ τῆς παρθένου.
\&
\end{astanza}
\begin{astanza}\Mar\trop{v}{\mS v, \mP iv}
Τὸν ὁπλισάμενον,&
πρὸς μεγαλαύχους ἐχθρούς,&
τὸν καταβαλόντα ἐν ἀσθενείᾳ σαρκός,&
ὕψη τυραννικά,&
Μαρῖνον (?) στέψωμεν ἐπινικίοις.\&
\end{astanza}
\begin{astanza}\Mar\trop{vi}{\mS vi, \mP ---}
Ψυχὴν ἀκμάζουσαν,&
καὶ τελειότατον νοῦν,&
ἄρτι διελέγχεις,&
ἐν τῷ νεάζοντι ὤν,&
νήπιον παλαιόν,&
ἀνοητέναντα τὸν νοῦν τὸν μέγαν.
\&
\end{astanza}
\begin{astanza}\Est\trop{vii}{\mS ---, \mP v}
Χριστῷ Εὐστάθιος&
καὶ Ἀνατόλιος ζῇ (?),&
σὺν τῷ Θεσπεσίῳ ὡς ἀδελφοὶ ταῖς ψυχαῖς,&
μᾶλλον ἢ τῇ σαρκὶ,&
συμμαρτυρήσαντες τῇ ἀληθείᾳ.\&
\end{astanza}
\begin{astanza}
\noindent\skipnumbering&
\app{---}\skipnumbering
\&
\end{astanza}
\begin{astanza}\Bac\trop{viii}{\mS vii, \mP vi}
\Dn{Ὁ}{Β \mP} Βάκχος πρόκειται&
δαψιλετάτῇ τρυφῇ,&
ἀλλ᾽ ἐμφορηθῶμεν& 
τοῦ ἀκορέστου αὐτῆς,&
τῆς ἀσκήσεως γὰρ&
καὶ τῆς ἀθλήσεως, \Dn{τὸ νῖκος}{τὰς νίκας \mP} ἔχει.\&
\end{astanza}
\begin{astanza}\The\trop{ix}{\mS viii, \mP vii}
\Dn{Τῷ}{Τὰ \mP} σκήπτρα φέροντι&
ἐν ἀϊδίῳ χειρὶ,&
τῷ ἀ%
\Fol{\mS~106v}%
καταλήπτῳ,&
σοῖς ἐνκριμάτων (!) βυθοῖς,&
μέλπωμεν ὡς Θεῷ,&
ᾠδὴν αἰνέσεως εἰς τοῦς αἰῶνας.
\&
\end{astanza}
\begin{astanza}\The\trop{x}{\mS ix, \mP ---}
Τὴν κλήσιν ἔχουσα,&
φερώνυμον ἀληθῶς,&
τῆς θεοφανείας ὁδηγουμένη φωτί,&
ἄρχοντα ζοφερόν,&
τὸν τοῦ \Dn{ἀέρος}{ἀέρως \mS} διαλανθάνεις.
\&
\end{astanza}
\begin{astanza}\The\trop{xi}{\mS x, \mP ---}
Τὸ σὸν διάδημα,&
σεμνύνασα νουνεχῶς,&
τῷ πανακηράτῳ,&
στεφάνῳ τῶν ἀρετῶν,&
ὤφθης τῷ βασιλεῖ,& 
τῆς ἀφθαρσίας εὐαρεστοῦσα.
\&
\end{astanza}
\begin{astanza}\bog\trop{xii}{\mS xi, \mP viii}
Τὴν μόνην φύσεως,& 
λαθοῦσαν ἐν γυναιξίν,&
ὅρους 
\Fol{\mP49r} 
ἐν τῷ τίκτειν,& 
τιμῶμεν οἱ δι᾽ αὐτῆς,&
κλήρου προγονικοῦ,&
τῆς ἁμαρτίας λελυτρωμένοι.
\&
\end{astanza}

\end{greek}

\endnumbering
\end{Rightside} 
\Pages 
\end{pages} 
\end{document}