% !TEX encoding = UTF-8 Unicode

% !TEX program = xelatex
% !TEX spellcheck = it_IT
%---------------------------------------------------------------------------------------
% PACKAGES
%---------------------------------------------------------------------------------------
\documentclass[11pt,a4paper]{book}
\usepackage[outer=4.85cm, inner=4.85cm, top=5.8cm, bottom=5.8cm, headsep=4mm]{geometry}
\usepackage[libertine={Ligatures=TeX, Numbers=OldStyle, Scale=0.95}]{libertineotf}
\usepackage{polyglossia}    % LANGUAGE SUPPORT
\setmainlanguage[variant=polytonic]{greek}

\usepackage{eledmac}

\lineation{page}                %% numerazione per pagina
\linenummargin{inner}   %% Margine dei numeri di linea
\sidenotemargin{outer}  %% Margine dei marginalia

%Font testo apparato (default: footnotesize)
\Xnotefontsize[A,B,C]{\footnotesize}

%Formato marginalia
\renewcommand*{\ledlsnotefontsetup}{\raggedleft\it\footnotesize}    % left
\renewcommand*{\ledrsnotefontsetup}{\raggedright\it\footnotesize}   % right

%Togliamo numero di linea da A e B
\nonumberinfootnote[A,B]

%Numero solo sulla prima nota di una linea in C
\numberonlyfirstinline[C]

%Inseriamo un separatore per le note della stessa linea in C
%\symlinenum[C]{||}

% Spazio bianco dopo il separatore note stessa linea (default = 0.5em)
%\aftersymlinenum[C]{0em}

%Azzeriamo lo spazio residuo, una volta tolto il numero di linea
\inplaceofnumber[A,B,C]{0em}

%Togliamo separatore lemma da A e B
\nolemmaseparator[A,B]

%Azzeriamo lo spazio residuo, una volta tolto il separatore lemma
\inplaceoflemmaseparator[A,B]{0em}
\inplaceoflemmaseparator[C]{.5em}

%%% SPAZIO LIBERO SOPRA LE RIGHE SEPARATRICI
\addtolength{\skip\Afootins}{2em plus.4em minus.4em}

% SPAZIO BIANCO FRA TESTO ED APPARATO (def. = 5mm)
\setlength{\skip\Afootins}{2em plus.4em minus.4em}

% SPAZIO BIANCO FRA NOTE D'APPARATO (def. = 1em plus.4em minus.4em)
\afternote[A,B,C]{1em plus.4em minus.4em}

\footparagraph{A}
\footparagraph{B}
\footparagraph{C}

\makeatletter
    \renewcommand{\thepstart}{{\makebox[2mm][r]{\bf\@arabic\c@pstart}}\hspace{3mm}}
\makeatother

%% DEFINIZIONE NEWPARA
\newcounter{para}[chapter]\setcounter{para}{0}
    \newcommand{\newpara}{%
    \refstepcounter{para}}
    \newcommand{\oldpara}[1]{%
    \noindent\llap{\ref{#1}}}

\begin{document}

\beginnumbering

\setcounter{pstart}{1}
\numberpstarttrue
    \pstart\newpara \label{scolio1}\edtext{}{\Afootnote{\textbf{\ref{scolio1}} \textit{Anote} with numeric lemma in bold}}\edtext{}{\Bfootnote{\textbf{\ref{scolio1}} \textit{Bnote} with numeric lemma in bold}}scolio 1] testo \edtext{testo}{\lemma{\textit{prova C}}\Cfootnote{\textit{prova A}}} testo testo \textbf{testo} testo testo testo testo testo testo testo testo.\pend

    \pstart\newpara \label{scolio2}scolio 2] \ledsidenote{Σ \emph{|} D?}\edtext{Nel mezzo \edtext{del}{\Cfootnote{dello C}} cammin di nostra vita}{\Cfootnote{a 35 anni}} testo testo testo testo testo testo testo \textit{testo} testo testo testo testo testo testo testo.\pend

    \pstart\newpara \label{scolio3}\edtext{}{\Afootnote{\textbf{\ref{scolio3}} sch.3 Anote}}scolio 3] \ledsidenote{Σ \emph{|} D? \emph{|} Bk\textsuperscript{5}}\edtext{}{\lemma{\textbf{\ref{scolio3}}}\Cfootnote[nonum,nosep]{\textit{Cnote} with numeric lemma in bold}}testo testo \edtext{ἐρωμένων}{\Cfootnote{ἐρρωμένων sch. \textit{Min.}}} testo testo testo testo testo testo testo testo testo testo testo.\pend
    \newpage

    \pstart\newpara \label{itm:Clit5}\edtext{}{\Afootnote{{\textbf{\ref{itm:Clit5}}}\enspace aliter n. \ref{scolio3}}}\edtext{}{\Bfootnote{{\textbf{\ref{itm:Clit5}}}\enspace long Bnote with numeric lemma in bold}}scolio 4]\ledsidenote{Diog.?}\edtext{}{\lemma{\textbf{{\ref{itm:Clit5}}}}\Cfootnote[nonum,nosep]{sch. ab eodem ut vid. scriba et atramento magis diluto posterius additum (f.198r); signum πα. prope textum et prope scholium: παροιμία in init. Greene}}   testo testo testo testo testo \edtext{ὠφελείᾳ}{\lemma{ὠφελείᾳ τινὶ}\Cfootnote[nosep]{Diog.}} \edtext{ἢ σωτηρίᾳ}{\Cfootnote[nosep]{om. Diog. (sed cf. Diog.Vind. εἰς σωτηρίαν)}} \edtext{φαινομένων}{\lemma{ἀναφαινομένων}\Cfootnote[nosep]{Diog.}}. testo testo testo testo testo \edtext{ἐξ ἀφανοῦς θεοὶ}{\Cfootnote{θεοὶ ἐξ ἀφανοῦς Diog.}} testo testo testo testo testo testo testo testo testo testo. \textbf{T}\pend
    \newpage
    \setcounter{pstart}{41}
    \pstart\newpara \label{itm:Riv41}435c8 \textit{χαλεπὰ τὰ καλά}] \ledsidenote{Zen.}Περίανδρος ὁ Κορινθίων δυνάστης, κατ’ ἀρχὰς δημοτικὸς ὤν, ὕστερον εἰς τὸ \edtext{τυραννικὸς}{\Cfootnote{τύραννος sch. \textit{Hip.Mai.}}} εἶναι μετῆλθεν· \edtext{τοῦτο}{\Cfootnote[nosep]{revera A, sch. \textit{Hip.Mai.}: καὶ τοῦτο Greene}} \edtext{Πιττακὸν}{\lemma{Πίττακον}\Cfootnote[nosep]{Greene}} ἀκούσαντα, τότε \edtext{Μιτυληναίων}{\lemma{Μυ\super{ϊ}τυλ-}\Cfootnote[nosep]{sch. \textit{Hip.Mai.} cod. W}} δυναστεύοντα καὶ δείσαντα περὶ τῆς ἑαυτοῦ γνώμης, καθίσαι τε ἐπὶ τὸν βωμὸν ἱκέτην καὶ ἀπολυθῆναι τῆς ἀρχῆς ἀξιοῦν. τῶν δὲ Μιτυληναίων πυνθανομένων τὴν αἰτίαν, εἰπεῖν τὸν \edtext{Πιττακὸν}{\lemma{Πίττακον}\Cfootnote[nosep]{Greene}} ὡς “χαλεπὸν ἐσθλὸν ἔμμεναι”. τοῦτο δὲ μαθόντα Σόλωνα εἰπεῖν “χαλεπὰ τὰ καλά”, καὶ ἐντεῦθεν εἰς παροιμίαν ἐλθεῖν. \edtext{οἱ δ’ ἐπὶ}{\lemma{οἱ δὲ ἐπὶ}\Cfootnote[nosep]{(binis accentibus supra δὲ) sch. \textit{Hip.Mai.} cod. P}} τοῦ ἀδυνάτου τὸ χαλεπὸν ἀκούουσιν· \edtext{ἐπὶ πάντα}{\Cfootnote{ἐπὶ πάντων sch. \textit{Hip.Mai.}}} γὰρ γενέσθαι ἀγαθὸν ἀδύνατον (Sol. fr. 206b Martina). \textbf{A}\pend

\numberpstartfalse
\endnumbering
\end{document}