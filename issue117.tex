\documentclass{book}
\usepackage[T1]{fontenc}

\usepackage{fontspec}
%\usepackage{polyglossia}
%\setmainlanguage{italian}
\usepackage{eledmac}

\usepackage{imakeidx}
\makeindex[title=Indice dei nomi]
\usepackage{letltxmacro}
\makeatletter
\LetLtxMacro\orig@@index\index

\let\orig@@index\index
\newcommand\nindex[1]{\orig@@index{#1|innota}}
\newcommand\innota[1]{#1\textit{n}} 
\usepackage{etoolbox}
\AtBeginDocument{
\pretocmd\@footnotetext{\let\index\nindex}{}{}
}
\makeatother

\usepackage[backend=biber,style=philosophy-verbose,scauthors=true, latinemph=true, publocformat=locpubyear, inbeforejournal=true,annotation=true, volumeformat=romansc,origfields=edorig,origfieldsformat=parens,commacit=true,indexing,hyperref,backref]{biblatex}
\usepackage[babel,italian=guillemets]{csquotes}

\addbibresource{issue117.bib}

\begin{document}

\beginnumbering
\pstart
Paragrafo di prova per vedere se funziona anche con eledmac Paragrafo di prova per vedere se funziona anche con eledmac Paragrafo di prova per vedere se funziona anche con eledmac
Paragrafo di prova per vedere se funziona anche con eledmac
Paragrafo di \edtext{prova}{\Afootnote{Nota eledmac}} per vedere se funziona anche con eledmac
Paragrafo di prova per vedere se funziona anche con eledmac
Paragrafo di prova per vedere se funziona anche con eledmac
Paragrafo di prova per vedere se funziona anche con eledmac
Paragrafo di prova per vedere se funziona anche con eledmac
\pend
\endnumbering

Prova\footcite{bembo:donnini}

%\newpage
%Prova\footcite{bembo:donnini}
%Prova\footnote{\index{Tizio, Caio}}
%Prova\footnote{\nindex{Sempronio, Caio}}

\printbibliography
\printindex
\end{document}
