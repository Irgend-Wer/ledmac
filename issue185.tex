% !TeX encoding = UTF-8
% !TeX TS-program = xelatex
% !TeX spellcheck = it_IT
% rubber: set program xelatex

\documentclass[11pt,b5paper,twoside]{book}
\usepackage{fontspec}
\setmainfont{Linux Libertine O}
\usepackage{polyglossia}

\usepackage{eledmac}

%%%%%%%%%%%%%%%%%%%%%%%%%%%%%%%%%%%%%%%%%%%%%%%%%%%%%%%%%%%%%%
\lineation{page}
\firstlinenum{1}
\linenumincrement{1}
\linenummargin{inner}
\sidenotemargin{outer}

\Xnotefontsize[A,B,C]{\footnotesize}

\nonumberinfootnote[A,B]

\numberonlyfirstinline[C]

\numberonlyfirstintwolines[C]

\inplaceofnumber[A,B,C]{0em}

\nolemmaseparator[A,B]

\nonbreakableafternumber[C]

\inplaceoflemmaseparator[A,B]{0em}
\inplaceoflemmaseparator[C]{.5em}

\preXnotes{.5cm}

\afternote[A,B,C]{2em plus.4em minus.4em}

\footparagraph{A}
\footparagraph{B}
\footparagraph{C}
%%%%%%%%%%%%%%%%%%%%%%%%%%%%%%%%%%%%%%%%%%%%%%%%%%%%%%%%%%%%%%

\begin{document}

\section*{Ciceronis Tusculanae Disputationes}

\beginnumbering

\numberpstarttrue
\pstart%
Cum defensionum \edtext{laboribus}{\Cfootnote{first note}} senatoriisque muneribus aut omnino aut magna ex parte essem aliquando liberatus, rettuli me, Brute, te hortante maxime ad ea studia, quae retenta animo, remissa temporibus, longo intervallo intermissa revocavi, et cum omnium artium, quae ad rectam vivendi viam pertinerent, \edtext{ratio}{\Cfootnote{second note}} et disciplina studio sapientiae, quae philosophia dicitur, contineretur, hoc mihi Latinis litteris \edtext{inlustrandum}{\Cfootnote{third note}} putavi, non quia philosophia Graecis et litter
 is et do
 ctoribus percipi non posset, sed meum semper iudicium fuit omnia nostros aut invenisse per se sapientius quam Graecos aut accepta ab illis fecisse meliora, quae quidem digna statuissent, in quibus elaborarent.
\pend

\pstart%
Cum defensionum \edtext{laboribus}{\Cfootnote{first note}} senatoriisque muneribus aut omnino aut magna ex parte essem aliquando liberatus, rettuli me, Brute, te hortante maxime ad ea studia, quae retenta animo, remissa temporibus, longo intervallo intermissa revocavi, et cum omnium artium, quae ad rectam vivendi viam pertinerent, \edtext{ratio}{\Cfootnote{second note}} et disciplina studio sapientiae, quae philosophia dicitur, contineretur, hoc mihi Latinis litteris \edtext{inlustrandum}{\Cfootnote{third note}} putavi, non quia \edtext{philosophia}{\Cfootnote{fourth note}} Graecis et litteris et doctoribus percipi non posset, sed meum semper iudicium fuit omnia nostros aut invenisse per se sapientius quam Graecos aut accepta ab illis fecisse meliora, quae quidem digna statuissent, in quibus elaborarent.
\pend

\pstart%
Cum defensionum \edtext{laboribus}{\Cfootnote{first note}} senatoriisque muneribus aut omnino aut magna ex parte essem aliquando liberatus, rettuli me, Brute, te hortante maxime ad ea studia, quae retenta animo, remissa temporibus, longo intervallo intermissa revocavi, et cum omnium artium, quae ad rectam vivendi viam pertinerent, \edtext{ratio}{\Cfootnote{second note}} et disciplina studio sapientiae, quae philosophia dicitur, contineretur, hoc mihi Latinis litteris \edtext{inlustrandum}{\Cfootnote{third note}} putavi, non quia \edtext{philosophia}{\Cfootnote{fourth note}} Graecis et litteris et doctoribus percipi non posset, sed meum semper iudicium fuit omnia nostros aut invenisse per se sapientius quam Graecos aut accepta ab illis fecisse meliora, quae quidem digna statuissent, in quibus elaborarent.
\pend

\pstart%
%%%%%%% SEE HERE AT \edtext{aliquando liberatus}{\Cfootnote{prova}}
Cum defensionum \edtext{laboribus}{\Cfootnote{first note}} senatoriisque muneribus aut omnino aut magna ex parte essem \edtext{aliquando liberatus}{\Cfootnote{prova}}, rettuli me, Brute, te hortante maxime ad ea studia, quae retenta animo, remissa temporibus, \edtext{longo intervallo}{\Cfootnote{longo tempore}} intermissa revocavi, et cum omnium artium, quae ad rectam vivendi viam pertinerent, \edtext{ratio}{\Cfootnote{second note}}<
 /span> et disciplina studio sapientiae, quae philosophia dicitur, contineretur, hoc mihi Latinis litteris \edtext{inlustrandum}{\Cfootnote{third note}} putavi, non quia \edtext{philosophia}{\Cfootnote{fourth note}} Graecis et litteris et doctoribus percipi non posset, sed meum semper iudicium fuit omnia nostros aut invenisse per se sapientius quam Graecos aut accepta ab illis fecisse meliora, quae quidem digna statuissent, in quibus elaborarent.
\pend

\pstart%
Cum defensionum \edtext{laboribus}{\Cfootnote{first note}} senatoriisque muneribus aut omnino aut magna ex parte essem aliquando liberatus, rettuli me, Brute, te hortante maxime ad ea studia, quae retenta animo, remissa temporibus, longo intervallo intermissa revocavi, et cum omnium artium, quae ad rectam vivendi viam pertinerent, \edtext{ratio}{\Cfootnote{second note}} et disciplina studio sapientiae, quae philosophia dicitur, contineretur, hoc mihi Latinis litteris \edtext{inlustrandum}{\Cfootnote{third note}} putavi, non quia \edtext{philosophia}{\Cfootnote{fourth note}} Graecis et litteris et doctoribus percipi non posset, sed meum semper iudicium fuit omnia nostros aut invenisse per se sapientius quam Graecos aut accepta ab illis fecisse meliora, quae quidem digna statuissent, in quibus elaborarent.
\pend

\pstart
Cum defensionum \edtext{laboribus}{\Cfootnote{first note}} senatoriisque muneribus aut omnino aut magna ex parte essem aliquando liberatus, rettuli me, Brute, te hortante maxime ad ea studia, quae retenta animo, remissa temporibus, longo intervallo intermissa revocavi, et cum omnium artium, quae ad rectam vivendi viam pertinerent, \edtext{ratio}{\Cfootnote{second note}} et disciplina studio sapientiae, quae philosophia dicitur, contineretur, hoc mihi Latinis litteris \edtext{inlustrandum}{\Cfootnote{third note}} putavi, non quia \edtext{philosophia}{\Cfootnote{fourth note}} Graecis et litteris et doctoribus percipi non posset, sed meum semper iudicium fuit omnia nostros aut invenisse per se sapientius quam Graecos aut accepta ab illis fecisse meliora, quae quidem digna statuissent, in quibus elaborarent.
\pend

\labelpstartfalse
\numberpstartfalse
\endnumbering
\end{document}

