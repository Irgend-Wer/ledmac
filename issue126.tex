\documentclass[12pt,a4paper]{memoir}
\usepackage{subfiles}
\usepackage[french]{babel}
\usepackage{hyphenat}
\usepackage{scrdate,scrtime}
\usepackage[no-math]{fontspec}
\usepackage{xltxtra,xunicode,amsmath}

\defaultfontfeatures{Mapping=tex-text,RawFeature={+hlig;+onum}}
\setmainfont{Linux Libertine O}
\setsansfont{Linux Biolinum O}
\setmonofont{Linux Libertine Mono O}
\newfontfamily\uncial{P39}

\copypagestyle{mycompanion}{companion}
\chapterstyle{companion}
\pagestyle{mycompanion}
\aliaspagestyle{plain}{empty}

\makepagenote
\notepageref
\usepackage[punct-after]{fnpct}
\usepackage{afterpage}
\usepackage{csquotes}



%%%begin
\usepackage[final]{eledmac}
\usepackage{eledpar}
\lineation{page}
\footparagraph{A}
\txtbeforeXnotes[A]{\textsc{Test.}:\space}
\symlinenum[B]{$\parallel$}
\numberonlyfirstinline[B]
\numberonlyfirstintwolines[B]
\nolemmaseparator[B]
\inplaceoflemmaseparator[B]{.5em}
\nonbreakableafternumber
\footparagraph{B}
\footparagraphX{A}
\Xnotenumfont{\bfseries}
\renewcommand*{\Rlineflag}{}
\renewcommand*{\bodyfootmarkA}{%
\hbox{\textsuperscript{\thefootnoteA}}}
%\renewcommand*{\goalfraction}{0.8}% mais .825 passe aussi
%%%end






\newcommand{\hunayn}{Ḥunayn}

%%%
\usepackage{polyglossia}
\setdefaultlanguage{french}
\setotherlanguage{arabic}
\setotherlanguage{greek}

%\newfontfamily\arabicfont[Script=Arabic]{Amiri}
\usepackage[voc,fdf2noalif]{arabxetex}
\newcommand{\ta}{\textarab}
\newcommand{\tanv}[1]{\textarab[novoc]{#1}}
\catcode`_=11
\catcode`^=11


\let\aemph\veryundefinedcommand
%%%

\begin{document}
\begin{pages}
\begin{Leftside}
\ledsectnotoc

\footnote{S}
\beginnumbering

\firstlinenum{100}

\pstart\vspace{1.5\baselineskip} %pour ajuster l'alignement des paragraphes
\markboth{6.4}{}\textbf{6.4} Hippocrate dit: Dans les périodes,
lorsqu'une période arrive plus tôt qu'une autre période cela est un
augment,\footnoteA{\hunayn\
traduit ici littéralement un \emph{génitif absolu} grec qui est
une proposition participiale susceptible d'exprimer toute nuance
circonstancielle. C'est l'équivalent grec de \emph{l'ablatif
absolu} latin. Il faut donc comprendre ici: \enquote{cela est un
augment, même si la maladie se relâche}.} la maladie étant en repos\footnoteA{\hunayn\
traduit ici littéralement un \emph{génitif absolu} grec qui est
une proposition participiale susceptible d'exprimer toute nuance
circonstancielle. C'est l'équivalent grec de \emph{l'ablatif
absolu} latin. Il faut donc comprendre ici: \enquote{cela est un
augment, même si la maladie se relâche}.}.
\pend

\endnumbering

\end{Leftside}

\begin{Rightside}
\linenummargin{inner}
\lineation{page}

\beginnumbering
\pstart\markright{\ta{6--4}}
\begin{arab}
(6--4) qAla 'abuqrA.tu: wa-fI 'l-'adwAri 'i_dA kAna
\edtext{dawruN}{\Bfootnote{M: \tanv{dawraN} E1}} 'azyada
taqaddumaN min dawriN 'Axara fa-_d_alika tazayyuduN `inda sukUni
'l-mara.di.
\end{arab}
\pend

\endnumbering

\end{Rightside}

\Pages

\end{pages}
\end{document}