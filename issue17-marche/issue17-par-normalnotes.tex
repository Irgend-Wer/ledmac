\documentclass{article}
\usepackage{fontspec,xunicode}
\usepackage{eledmac}
\usepackage[shiftedpstarts]{eledpar}

\usepackage{polyglossia}
\setmainlanguage{latin}
\setotherlanguage{french}

\begin{document}
%\footparagraph{A}
\toggletrue{Xbothsides@A}
\maxhXnotes{0.25\textheight}
\afternote{0pt}
\Xbothsides
\begin{pages}

\begin{Leftside}
\begin{french}
\beginnumbering

\pstart


Tous les rois de la Grèce qui descendaient de Minos, fils de Jupiter, vinrent en Crète pour y recueillir la riche succession de Crétéus\edtext{titi}{\Afootnote{
Ce site contient la documentation pour SPIP 3.
Cette documentation est en cours de migration (mai 2012) et cela risque de prendre un peu de temps :)
Soyez indulgents ou bien participez !
Merci.
Nota : La documentation pour SPIP 2.1 se trouve à cette adresse : pour l’instant.}}. Ce prince, fils de Mi\-nos, avait réglé par son testament qu'il serait fait un partage égal de tout ce qu'il possédait d'or, d'argent et de troupeaux, entre les enfants de ses filles ; et il laissait son empire à Idoménée, fils de Deucalion, son frère et à Mérion\edtext{titi}{\Afootnote{Fruit de longues heures d’écriture, cette documentation est une somme de connaissances issue de la communauté SPIP. Tout ce travail est distribué sous licence libre Creative Commons - Paternité - Partage des Conditions Initiales à l’Identique (cc-by-sa). Vous pouvez utiliser ces textes quel que soit l’usage (y compris commercial), les modifier et les redistribuer à condition de laisser à vos lecteurs la même liberté de partage.
Cette œuvre fait l’objet de nombreuses relectures mais n’est certainement pas indemne de toute erreur. N’hésitez pas à proposer des améliorations ou signaler des coquilles en utilisant le formulaire de suggestion mis à disposition sur le site internet de la documentation. Vous pouvez aussi discuter de l’organisation (des contenus, de la technique) et des traductions sur la liste de discussion « spip-programmer » (sur abonnement).
Si vous êtes motivé par ce projet, vous pouvez proposer d’écrire un chapitre sur un sujet que vous maîtrisez ou refondre un chapitre existant pour le clarifier ou le compléter. Nous essaierons alors de vous accompagner et vous soutenir dans cette tâche.
Vous pouvez également participer à la traduction de cette documentation. L’espace privé du site permet de discuter des traductions en cours d’élaboration. Nous accueillerons très progressivement d’autres langues si des volontaires veulent s’atteler à cette immense et admirable tâche.
Par souci de compatibilité, les codes informatiques qui servent d’exemple ne contiennent généralement que des caractères du code ASCII. Cela signifie entre autre que vous ne trouverez que rarement des accents dans les commentaires accompagnant les exemples de code dans l’ensemble de la documentation. Ne soyez donc pas étonnés par cette absence.
Présentation et fonctionnement général. 
SPIP 3 est un logiciel libre développé sous licence GNU/GPL3. Utilisé comme un système de publication de contenu – sa vocation première – il est également une plateforme de développement permettant de créer des interfaces maintenables et extensibles quelle que soit la structure des données gérées.
SPIP est particulièrement adapté pour des portails éditoriaux mais peut aussi bien être utilisé comme système d’auto-publication (blog), de wiki, de réseau social ou pour gérer toute donnée issue de MySQL ou SQLite.
Il sait également sélectionner facilement des données formatées en XML, JSON, CSV ou YAML et de manière extensive, toute information transformée en tableau PHP.}}, fils de Molus, son neveu, qui devaient gouverner chacun sa part avec un pouvoir indépendant. Entre les princes présents au partage, on distinguait Palamède, fils de Clymène et de Nauplius, et Oeax, appelés Crétéides\edtext{titi}{\Afootnote{SPIP 3.0 nécessite a minima PHP 5.1 (et 16 Mo de mémoire pour PHP) ainsi qu’une base de données (MySQL ou SQLite).
Il possède une interface publique (front-office), visible de tous les visiteurs du site (ou en fonction d’autorisations particulières) et une interface privée (back-office) seulement accessible aux personnes autorisées et permettant d’administrer le logiciel et les contenus du site.}}, avec Ménélas, fils d'Ae\-ro\-pe et de Plisthène, qu'Anaxibie, sa soeur, épouse de Nestor, et Agamemnon, son frère aîné, avaient chargé de les représenter dans l'assemblée des héritiers. On connaissait moins ces derniers comme fils de Plisthène, mort à la fleur de son âge et sans avoir rien fait de mémorable, que comme petits-fils d'Atrée. Ce prince, en effet, touché de compassion pour la faiblesse de leur âge, les avait recueillis au-près de lui, et s'était chargé de leur donner une éducation conforme à leur naissance. Ils se conduisirent tous dans cette occasion avec la grandeur et la générosité qu'on devait attendre de personnes de leur rang.
 \pend
 \pstart

A la nouvelle de leur arrivée, tons les descendants d'Europe, dont le nom était en grande vénération dans l'île, se rendirent auprès d'eux, les saluèrent avec bonté et les conduisirent au temple. Là, après un sacrifice solennel où furent immolées suivant l'usage, nombre de victimes on leur servit un repas splendide, et on les traita avec autant d'abondance que de délicatesse. Les fêtes continuèrent les jours suivants. Les rois reçurent les témoignages de l'affection de leurs amis avec joie et reconnaissance; mais ils furent encore plus frappés de la magnificence du temple d'Europe. Ils ne pouvaient se laisser d'examiner, dans le plus grand détail, les riches présents envoyés de Sidon à cette princesse par son père Phénicie et par ses nobles compagnes, et qui faisaient l'ornement de ce bel édifice.
 \pend
 \pstart
Dans le même temps, Alexandre de Phrygie, fils de Priam, accompagné d'Énée et de plusieurs de ses parents, se rendait coupable d'un grand attentat à Sparte et dans le palais de Ménélas, où il avait été reçu comme hôte, et traité tomme ami. Aussitôt après le départ du roi, épris d'amour pour Hélène, qui surpassait en beauté toutes les femmes de la Grèce, il l'enleva, et avec elle tous les trésors qu'il put emporter. Cette princesse fut accompagnée dans sa fuite par Aetra et Clymène, parentes de Ménélas, attachées à son service. La nouvelle du crime commis par Alexandre contre la maison de Ménélas parvint bientôt en Crète; et la renommée, qui se plaît ordinairement à grossir les objets, publia que le palais du roi avait été détruit, son empire renversé, et répandit d'autres bruits aussi funestes.
 \pend
 \pstart
Ménélas, à cette nouvelle, quoique vive. meut affecté de l'enlèvement de son épouse, fut encore plus irrité de la connivence perfide qu'il crut apercevoir entre le ravisseur et ses parentes. Palamède, voyant ce prince indigné et furieux sortir du conseil sans proférer un seul mot, fait approcher de terre les vaisseaux et dispose tout pour le départ. Après quelques paroles consolantes adressées au roi, il embarque à la hâte tout ce qui provenait du partage, fait monter Ménélas avec lui sur la flotte, et, secondés d'un vent favorable, ils arrivent en peu de jours à Sparte. Déjà Agamemnon, Nestor , et tous les rois descendants de Pélops, y étaient accourus. A l'arrivée de Ménélas, ils s'assemblent ; et quoique l'atrocité de l'action leur inspiràt une profonde horreur et les portât â une prompte vengeance, cependant, après avoir délibéré mûrement, ils résolurent d'envoyer d'abord à Troie, en qualité de députés, Palamède, Ulysse et Ménélas, avec ordre de se plaindre de l'injure, et de redemander Hélène ainsi que tous les trésors enlevés.
\pend
\endnumbering
\end{french}
\end{Leftside}

\begin{Rightside}
\beginnumbering


\pstart

\edtext{Ad eos}{\Afootnote{Toute l’interface publique (dans le répertoire squelettes-dist) et l’interface privée (dans le répertoire prive) utilisent, pour s’afficher, des gabarits appelés « squelettes », mélange de code à produire (le plus souvent HTML) et de syntaxe SPIP.
Lorsqu’un visiteur demande à afficher la page d’accueil du site.}} cognita omnes ex origine Europae, quae in ea insula summa religione colitur, confluunt benigneque salutatos in templum deducunt. Ibi multarum hostiarum immolatione celebrata, exhibitisque epulis, largiter magnificeque eos habuere : itemque insecutis diebus. At reges Graeciae, etsi ea quae exhibebantur cum laetitia accipiebant, tamen multo magis templi ejus magnifica pulchritudine, pretiosaque exstructione operum afficiebantur, inspicientes repetentesque memoria, singula, quae ex Sidone a Phoenice patre ejus, atque nobilibus matronis transmissa magno tum decori erant.

\pend
\pstart
 
 Per idem tempus Alexander Phrygius, Priami filius, cum Aenea aliisque ex consanguinitate comitibus, Spartae in domum Menelai hospitio receptus, indignissimum facinus perpetraverat. Is namque ubi animadvertit regem abesse, quod erat Helena praeter caeteras Graeciae faeminas miranda specie, amore ejus captus, ipsamque et multas opes domo ejus aufert, Aethram etiam et Clymenam Menelai adfines, quae ob necessitudinem cum Helena agebant. Postquam Cretam nuncius venit, et cuncta quae ab Alexandro adversus domum Menelai commissa erant, aperuit, per omnem insulam, sicut in tali re fieri amat, fama in majus divulgatur: expugnatam quippe domum regis, eversumque regnum, et alia in talem modum singuli disserebant.
 
 \pend
 \pstart
 Quibus cognitis Menelaus, etsi abstractio conjugis animum permoverat, multo amplius tamen ob injuriam adfinium, quas supra memoravimus, consternabatur. At ubi animadvertit Palamedes, regem ira atque indignatione stupefactum, concilio excidisse, ipso naves parat, atque omni instrumento compositas terrae applicat. Dein pro temporo regem breviter consolatus, positis etiam ex divisione, quae in tali negotio tempus patiebatur, navem ascendere facit : atque ita ventis ex sententia flantibus, paucis diebus Spartam pervenere. Eo jam Agamemnon et Nestor, omnesque qui ex origine Pelopis in Graecia regnabant, cognitis rebus confluxerant. Igitur postquam Menelaum advenisse sciunt, in unum coeunt. Et quanquam atrocitas facti ad indignationem, ultumque injurias rapiebat, tamen ex consilii sententia legantur prius ad Troiam Palamedes, Ulysses et Menelaus ; hisque mandatur, ut conquesti iniurias, Helenam, et quae cum ea abrepta erant, repeterent.
 \pend
 \pstart
 
 Legati paucis diebus ad Trojiam veniunt, neque tum Alexandrum in loco offendere. Eum namque properatione navigii inconsulte usum venti ad Cyprum appulere. Unde sumptis aliquot navibus, Phoenicem delapsus, Sidoniorum regem, qui eum amice susceperat, noctu per insidias necat : eademque qua apud Lacedaomonam, cupiditate, universam domum ejus in scelus proprium convertit. Ita omnia quae ad ostentationem regiae magnificentiae fuere, indigne rapta, ad naves deferri jubet. Sed ubi ex lamentatione eorum qui casum domini deflentes, reliqui praedae aufugerant, tumultus ortus est, populus omnis ad regiam concurrit. Inde quod jam Alexander, abreptis quae cupiebat, ascensionem properabat, pro tempore armati ad naves veniunt; ortoque inter eos acri proelio, cadunt utrinque plurimi, quum obstinate hi regis necem defenderent, hi ne amitterent partam praedam summis opibus adniterentur. Incensis dein duabus navibus, Trojani reliquas strenue defensas liberant, atque ita fatigatis jam proelio hostibus evadunt.
 \pend
\endnumbering
\end{Rightside}

\Pages


%\unhbox25
\end{pages}
\end{document}