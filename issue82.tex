% !TEX TS-program = xelatex
% !TEX encoding = UTF-8
\documentclass[article,12pt]{memoir}
\usepackage[french]{babel}

\usepackage{fontspec,xltxtra,xunicode}
\defaultfontfeatures{Mapping=tex-text,RawFeature={+liga;+dlig;+onum}}


\chapterstyle{article}
\pagestyle{companion}

\usepackage{eledmac,eledpar}
\footparagraph{A}
\txtbeforeXnotes[A]{\textsc{Test.}~}
\symlinenum[B]{$\parallel$}
\numberonlyfirstinline[B]
\numberonlyfirstintwolines[B]
\nolemmaseparator[B]
\inplaceoflemmaseparator[B]{.5em}
\footparagraph{B}

\newfontface\arabicfont[Script=Arabic]{Amiri}
\usepackage[voc,fdf2noalif]{arabxetex}
\newcommand{\ta}{\textarab}


\begin{document}
\begin{pages}
\begin{Rightside}
\beginnumbering
\pstart
\ledchapter[Introduction]{\edtext{Introduction}{\Bfootnote{A Ar.: om. BE}}}
\edtext{Voici un essai d'�dition critique. Je vais bien voir comment �a marche, et surtout si �a marche bien. Je le fais en fran�ais. \edtext{C'est vraiment}{\Bfootnote{A: \emph{Cela est surtout} BE}} n'importe quoi, mais le but de l'op�ration est juste de g�n�rer quelques lignes de texte.}{\lemma{Voici... texte}\Afootnote{Cf. Untel, \emph{Lui-m�me}}} Je continue, car je voudrais \edtext{au moins}{\Bfootnote{A Ar.: om. BE}} arriver � la ligne 5 pour voir appara�tre le num�ro de la ligne dans la marge.
\pend

\pstart
\ledsection[Int�r�t de la chose]{\edtext{Int�r�t}{\Bfootnote{A Ar.: add. \emph{grand} ante \emph{int�r�t} BE}} de la chose}
Le grand int�r�t de la chose est de pouvoir d�sormais utiliser les commandes de sectionnement pour introduire des titres dans les �ditions critiques. Et, s'il vous pla�t, avec g�n�ration des titres dans l'en-t�te de page.
\pend

\pstart
Autre subtilit�: introduction automatique de \textsc{Test.} devant l'�tage des \emph{testimonia}!
\pend

\endnumbering
\end{Rightside}

\begin{Leftside}
\beginnumbering
\ledsectnotoc
\pstart
\ledchapter{Introduction}
Ceci est la traduction de l'introduction.
\pend

\pstart
\ledsection{Section}
Je continue...
\pend

\pstart
Et toujours ici...
\pend

\endnumbering

\end{Leftside}

\Pages

\end{pages}

\tableofcontents
\end{document}