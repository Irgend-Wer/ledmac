% !TeX encoding = UTF-8
% !TeX TS-program = xelatex
% !TeX spellcheck = it_IT
% rubber: set program xelatex

\documentclass[11pt,b5paper,twoside]{book}
\usepackage{fontspec}
    \setmainfont{Linux Libertine O}
\usepackage{polyglossia}

\usepackage[noeledsec,series={A,B,C}]{reledmac}
    \lineation{page}
    \linenummargin{inner}
    \sidenotemargin{outer}


%\Xendlemmaseparator[A]{\rbracket}

\Xendnonumber[A]
\usepackage{hyperref}

\begin{document}
\beginnumbering
\pstart
\begin{center}
    {\Large \ledinnernote{175r}Tusculanae Disputationes}
\end{center}
\pend
\bigskip

\setcounter{pstart}{1}
\numberpstarttrue

\pstart\edlabel{itm:Rv2}%
\edtext{}{\Aendnote[nosep]{{\textbf{\pstartref{itm:Rv2}}}\enspace nota lunga nota lunga nota lunga nota lunga nota lunga nota lunga nota lunga nota lunga nota lunga nota lunga nota lunga nota lunga nota lunga nota lunga nota lunga nota lunga nota lunga nota lunga nota lunga nota lunga nota lunga nota lunga nota lunga nota lunga nota lunga nota lunga nota lunga nota lunga nota lunga nota lunga nota lunga nota lunga nota lunga nota lunga nota lunga nota lunga nota lunga nota lunga nota lunga nota lunga nota lunga nota lunga nota lunga nota lunga nota lunga nota lunga nota lunga nota lunga nota lunga nota lunga nota lunga nota lunga nota lunga nota lunga nota lunga nota lunga nota lunga nota lunga nota lunga nota lunga nota lunga nota lunga nota lunga nota lunga nota lunga nota lunga nota lunga nota lunga nota lunga nota lunga nota lunga nota lunga nota lunga nota lunga nota lunga nota lunga nota lunga nota lunga nota lunga nota lunga nota lunga nota lunga nota lunga nota lunga}}
    Cum defensionum \edtext{laboribus}{\Cfootnote{first note}} senatoriisque muneribus aut omnino aut magna ex parte essem aliquando liberatus, rettuli me, Brute, te hortante maxime ad ea studia, quae retenta animo, remissa temporibus, longo intervallo intermissa revocavi, et cum omnium artium, quae ad rectam vivendi viam pertinerent, \edtext{ratio}{\Cfootnote{second note}} et disciplina studio sapientiae, quae philosophia dicitur, contineretur, hoc mihi Latinis litteris \edtext{inlustrandum}{\Cfootnote{third note}} putavi, non quia \edtext{philosophia}{\Cfootnote{blablabla}} Graecis et litteris et doctoribus percipi non posset, sed meum semper iudicium fuit omnia nostros aut invenisse per se sapientius quam Graecos aut accepta ab illis fecisse meliora, quae quidem digna statuissent, in quibus elaborarent.
\pend

\pstart%
\ledinnernote{175v}%
Cum defensionum \edtext{laboribus}{\Cfootnote{first note}} senatoriisque muneribus aut omnino aut magna ex parte essem aliquando liberatus, rettuli me, Brute, te hortante maxime ad ea studia, quae retenta animo, remissa temporibus, longo intervallo intermissa revocavi, et cum omnium artium, quae ad rectam vivendi viam pertinerent, \edtext{ratio}{\Cfootnote{second note}} et disciplina studio sapientiae, quae philosophia dicitur, contineretur, hoc mihi Latinis litteris \edtext{inlustrandum}{\Cfootnote{third note}} putavi, non quia \edtext{philosophia}{\Cfootnote{fourth note}} Graecis et litteris et doctoribus percipi non posset, sed meum semper iudicium fuit omnia nostros aut invenisse per se sapientius quam Graecos aut accepta ab illis fecisse meliora, quae quidem digna statuissent, in quibus elaborarent.
\pend

\endnumbering
\beginnumbering
\numberpstarttrue

\pstart%
Cum defensionum \edtext{laboribus}{\Cfootnote{first note}} senatoriisque muneribus aut omnino aut magna ex parte essem aliquando liberatus, rettuli me, Brute, te hortante maxime ad ea studia, quae retenta animo, remissa temporibus, longo intervallo intermissa revocavi, et cum omnium artium, quae ad rectam vivendi viam pertinerent, \edtext{ratio}{\Cfootnote{second note}} et disciplina studio sapientiae, quae philosophia dicitur, contineretur, hoc mihi Latinis litteris \edtext{inlustrandum}{\Cfootnote{third note}} putavi, non quia \edtext{philosophia}{\Cfootnote{fourth note}} Graecis et litteris et doctoribus percipi non posset, sed meum semper iudicium fuit omnia nostros aut invenisse per se sapientius quam Graecos aut accepta ab illis fecisse meliora, quae quidem digna statuissent, in quibus elaborarent.
\pend

\pstart%
Cum defensionum \edtext{laboribus}{\Cfootnote{first note}} senatoriisque muneribus aut omnino aut magna ex parte essem aliquando liberatus, rettuli me, Brute, te hortante maxime ad ea studia, quae retenta animo, remissa temporibus, longo intervallo intermissa revocavi, et cum omnium artium, quae ad rectam vivendi viam pertinerent, \edtext{ratio}{\Cfootnote{second note}} et disciplina studio sapientiae, quae philosophia dicitur, contineretur, hoc mihi Latinis litteris \edtext{inlustrandum}{\Cfootnote{third note}} putavi, non quia \edtext{philosophia}{\Cfootnote{fourth note}} Graecis et litteris et doctoribus percipi non posset, sed meum semper iudicium fuit omnia nostros aut invenisse per se sapientius quam Graecos aut accepta ab illis fecisse meliora, quae quidem digna statuissent, in quibus elaborarent.
\pend

\pstart%
Cum defensionum \edtext{laboribus}{\Cfootnote{first note}} senatoriisque muneribus aut omnino aut magna ex parte essem aliquando liberatus, rettuli me, Brute, te hortante maxime ad ea studia, quae retenta animo, remissa temporibus, longo intervallo intermissa revocavi, et cum omnium artium, quae ad rectam vivendi viam pertinerent, \edtext{ratio}{\Cfootnote{second note}} et disciplina studio sapientiae, quae philosophia dicitur, contineretur, hoc mihi Latinis litteris \edtext{inlustrandum}{\Cfootnote{third note}} putavi, non quia \edtext{philosophia}{\Cfootnote{fourth note}} Graecis et litteris et doctoribus percipi non posset, sed meum semper iudicium fuit omnia nostros aut invenisse per se sapientius quam Graecos aut accepta ab illis fecisse meliora, quae quidem digna statuissent, in quibus elaborarent.
\pend

\pstart\edlabel{x}
\edtext{}{\Aendnote[nosep]{{\textbf{\pstartref{x}}}\enspace nota lunga nota lunga nota lunga nota lunga nota lunga nota lunga nota lunga nota lunga nota lunga nota lunga nota lunga nota lunga nota lunga nota lunga nota lunga nota lunga nota lunga nota lunga nota lunga nota lunga nota lunga nota lunga nota lunga nota lunga nota lunga nota lunga nota lunga nota lunga nota lunga nota lunga nota lunga nota lunga nota lunga nota lunga nota lunga nota lunga nota lunga nota lunga nota lunga nota lunga nota lunga nota lunga nota lunga nota lunga nota lunga nota lunga nota lunga nota lunga nota lunga nota lunga nota lunga nota lunga nota lunga nota lunga nota lunga nota lunga nota lunga nota lunga nota lunga nota lunga nota lunga nota lunga nota lunga nota lunga nota lunga nota lunga nota lunga nota lunga nota lunga nota lunga nota lunga nota lunga nota lunga nota lunga nota lunga nota lunga nota lunga nota lunga nota lunga nota lunga nota lunga nota lunga nota lunga nota lunga}}
    Cum defensionum \edtext{laboribus}{\Cfootnote{first note}} senatoriisque muneribus aut omnino aut magna ex parte essem aliquando liberatus, rettuli me, Brute, te hortante maxime ad ea studia, quae retenta animo, remissa temporibus, longo intervallo intermissa revocavi, et cum omnium artium, quae ad rectam vivendi viam pertinerent, \edtext{ratio}{\Cfootnote{second note}} et disciplina studio sapientiae, quae philosophia dicitur, contineretur, hoc mihi Latinis litteris \edtext{inlustrandum}{\Cfootnote{third note}} putavi, non quia \edtext{philosophia}{\Cfootnote{fourth note}} Graecis et litteris et doctoribus percipi non posset, sed meum semper iudicium fuit omnia nostros aut invenisse per se sapientius quam Graecos aut accepta ab illis fecisse meliora, quae quidem digna statuissent, in quibus elaborarent.
\pend

\numberpstartfalse
\endnumbering

\newpage
\doendnotes{A}
\newpage
\section{Serie A}
\doendnotesbysection{A}
\section{Serie B}
\doendnotesbysection{A}
\end{document}
