\documentclass{book}
\usepackage[xetex,a4paper,landscape]{geometry}
\usepackage{fontspec}
\usepackage{xunicode}
\usepackage{polyglossia}

\usepackage{eledmac}
\usepackage{eledpar}  
\usepackage{bidi}  
\setdefaultlanguage{russian}
\setotherlanguage{latin}    
\setmainfont[Ligatures=TeX]{Linux Libertine O}
\newfontfamily\latinfont[Script=Latin]{Linux Libertine O}
\newfontfamily\russianfont[Script=Cyrillic]{Linux Libertine O}    

\begin{document}
\numberlinefalse
\numberpstarttrue 
\sidepstartnumtrue 
\beforeeledchapter

\begin{pairs}

\begin{Rightside} %

\begin{latin}%
\beginnumbering%
\pstart
\eledchapter{xxxx}
\pend
\pstart
Lorem ipsum dolor sit amet, consectetur adipiscing elit. Aenean at mauris elit. Nulla hendrerit condimentum sem id pharetra. Aliquam posuere urna massa, et ullamcorper purus faucibus non. Quisque bibendum enim in lectus ultricies ultrices. Nunc ullamcorper metus et tortor congue, ut luctus libero pulvinar. Mauris at sapien nec odio dictum blandit sed a massa. Mauris vel bibendum leo. Donec fermentum mattis molestie. Integer id velit eros.

\pend    
\endnumbering%
\end{latin}%


\end{Rightside}%



\begin{russian}%
\begin{Leftside} %
\beginnumbering%
\pstart
\eledchapter{Трактат Второй}   
\pend 
\pstart
О краеугольных [принципах] Торы, имеется ввиду, которые [есть] основы и столпы на которых дом Б-жий опирается/нахон, и с существованием их может быть представлено существование Торы упорядоченной от Него, благословенного, и если бы было представлено отсутствие одного из них — упадет Тора в общем, [Б-же] упаси.
\pend
\endnumbering%
\end{Leftside}%
\end{russian}%



\Columns
\end{pairs}
\end{document}