\documentclass{article}

\usepackage{reledmac}

\usepackage{reledpar}

\begin{document}

\setstanzaindents{0,0,0}

\begin{pairs}

\begin{Leftside} 

\beginnumbering
    \autopar
    
\newcommand{\choiceline}[2]{\linenum{|#1|#2||#1|#2}}
    \sublinenumberstyle{alph}
    \let\fullstop\relax

\noindent Arma virumque cano, Troiae qui \edtext{primus}{\Afootnote{A; primo B}} ab oris Italiam, fato profugus, Laviniaque venit litora, multum ille et terris iactatus et alto vi superum saevae memorem Iunonis ob iram; \edtext{multa}{\Afootnote{A; multo B}} quoque et bello passus, dum conderet urbem, inferretque deos Latio, genus unde Latinum, Albanique patres, atque altae moenia Romae.

\vspace{10pt}

\begin{stanza}
Musa, mihi causas \edtext{memora}{\choiceline{1}{1}\Afootnote{A; memorum B}}, quo numine laeso,&
quidve dolens, regina deum tot volvere casus\&
\end{stanza}

\vspace{10pt}

\begin{stanza}
\edtext{insignem}{\choiceline{2}{1}\Afootnote{A; insigni B}} pietate virum, tot adire labores&
impulerit. Tantaene animis caelestibus irae?\&
\end{stanza}

\end{Leftside}


\begin{Rightside} 

\setstanzaindents{0,0,0}

\beginnumbering
    \autopar
    
\newcommand{\choiceline}[2]{\linenum{|#1|#2||#1|#2}}
    \sublinenumberstyle{alph}
    \let\fullstop\relax

\noindent Arma virumque cano, Troiae qui \edtext{primus}{\choiceline{1}{3}\Afootnote{A; primo B}} ab oris Italiam, fato profugus, Laviniaque venit litora, multum ille et terris iactatus et alto vi superum saevae memorem Iunonis ob iram; \edtext{multa}{\Afootnote{A; multo B}} quoque et bello passus, dum conderet urbem, inferretque deos Latio, genus unde Latinum, Albanique patres, atque altae moenia Romae.

\vspace{10pt}

\begin{stanza}
Musa, mihi causas \edtext{memora}{\choiceline{1}{1}\Afootnote{A; memorum B}}, quo numine laeso,&
quidve dolens, regina deum tot volvere casus\&
\end{stanza}

\vspace{10pt}

\begin{stanza}
\edtext{insignem}{\choiceline{2}{1}\Afootnote{A; insigni B}} pietate virum, tot adire labores&
impulerit. Tantaene animis caelestibus irae?\&
\end{stanza}

\endnumbering

\end{Rightside}

\end{pairs}
\Columns

\end{document}
