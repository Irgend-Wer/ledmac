
\documentclass[a4paper]{book}
\usepackage[a4paper]{geometry}
\usepackage{polyglossia,fontspec,hyperref}
\setmainfont{Linux Libertine O}
\setmainlanguage{greek}

\makeatletter
\def\edtext#1{\Hy@raisedlink{}#1}
\makeatother


\begin{document}

\section{Bad}

6. Καὶ δὴ τῷ κράτει
	\edtext{τοῦ} \edtext{συγγενέος} \edtext{ἐπιθαρσήσας}, \edtext{οὐμενοῦν}
	\edtext{εἴ τι}
καὶ \edtext{δέοι}
παθεῖν
δείσας
καλὴν ἑαυτῷ
τῆς εἰς Χριστὸν πίστεως, τὴν ἐν
	\edtext{Γράμματι}
νομοθεσίαν,
ὑποβάθραν
	\edtext{προύθηκεν}.
	\edtext{Περιτυχὼν}
γὰρ
	\edtext{τῶν Μωσαικῶν}
	\edtext{βιβλίων},
καὶ
	\edtext{τῆς Ἐβραίων ἁπάσης Γραφῆς},
καὶ τούτοις ὡς μάλιστα φιλομαθῶς
	\edtext{ἐναπασχολήσας}τὸν νοῦν, καὶ
	\edtext{ὅλος}
	\edtext{ὅλῳ}
στοιχειωθεὶς,

\section{Good}

\def\edtext#1{#1}
6. Καὶ δὴ τῷ κράτει
	\edtext{τοῦ} \edtext{συγγενέος} \edtext{ἐπιθαρσήσας}, \edtext{οὐμενοῦν}
	\edtext{εἴ τι}
καὶ \edtext{δέοι}
παθεῖν
δείσας
καλὴν ἑαυτῷ
τῆς εἰς Χριστὸν πίστεως, τὴν ἐν
	\edtext{Γράμματι}
νομοθεσίαν,
ὑποβάθραν
	\edtext{προύθηκεν}.
	\edtext{Περιτυχὼν}
γὰρ
	\edtext{τῶν Μωσαικῶν}
	\edtext{βιβλίων},
καὶ
	\edtext{τῆς Ἐβραίων ἁπάσης Γραφῆς},
καὶ τούτοις ὡς μάλιστα φιλομαθῶς
	\edtext{ἐναπασχολήσας}τὸν νοῦν, καὶ
	\edtext{ὅλος}
	\edtext{ὅλῳ}
στοιχειωθεὶς,
\end{document}
