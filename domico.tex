% !TEX encoding = UTF-8 Unicode
% !TEX program = xelatex
% !TEX spellcheck = it_IT
%---------------------------------------------------------------------------------------
% PACKAGES
%---------------------------------------------------------------------------------------
\documentclass[11pt,a4paper]{book}
\usepackage[libertine={Ligatures=TeX,Numbers=OldStyle}]{libertineotf}
\usepackage{ledmac}

%%%%%%%%%%%%%%%%%%%%%%%%%%%%%%%%%%%%%%%%%%%%%%%%%%%%%%%%%%%%%%%%%%%%%%%%
%----------------------------------------------------------------------%
%-------------INIZIO EDIZIONE CRITICA (LEDMAC)-------------------------%
%----------------------------------------------------------------------%
%%%%%%%%%%%%%%%%%%%%%%%%%%%%%%%%%%%%%%%%%%%%%%%%%%%%%%%%%%%%%%%%%%%%%%%%

\lineation{page}                %% numerazione per pagina
\linenummargin{inner}   %% Margine dei numeri di linea
\sidenotemargin{outer}  %% Margine dei marginalia

\renewcommand*{\notenumfont}{\footnotesize} %%Font numeri linea note
\newcommand*{\notetextfont}{\footnotesize}      %%Font testo apparato

%Togliamo numero di linea da A e B
\nonumberinfootnote[A]
\nonumberinfootnote[B]

%Numero solo sulla prima nota di una linea in C
\numberonlyfirstinline[C]

%Azzeriamo lo spazio residuo, una volta tolto il numero di linea
\inplaceofnumber[A]{0em}
\inplaceofnumber[B]{0em}

%Togliamo separatore lemma da A e B
\nolemmaseparator[A]
\nolemmaseparator[B]

%Azzeriamo lo spazio residuo, una volta tolto il separatore lemma
\inplaceoflemmaseparator[A]{0em} %NOT IN 0.18
\inplaceoflemmaseparator[B]{0em} %NOT IN 0.18
\inplaceoflemmaseparator[C]{.5em} %NOT IN 0.18

%%% SPAZIO LIBERO SOPRA LE RIGHE SEPARATRICI
\addtolength{\skip\Afootins}{2em plus.4em minus.4em}

% SPAZIO BIANCO FRA TESTO ED APPARATO (def. = 5mm)
\setlength{\skip\Afootins}{2em plus.4em minus.4em}

% SPAZIO BIANCO FRA NOTE D'APPARATO
%\afternote[A]{1em plus.4em minus.4em}
%\afternote[B]{1em plus.4em minus.4em}
%\afternote[C]{1em plus.4em minus.4em}

\footparagraph{A}
\footparagraph{B}
\footparagraph{C}

%% DEFINIZIONE NEWPARA
\newcounter{para}[chapter]\setcounter{para}{0}
    \newcommand{\newpara}{%
    \refstepcounter{para}%
    \noindent\llap{\thepstart}}
    \newcommand{\oldpara}[1]{%
    \noindent\llap{\ref{#1}}}

%%%%%%%%%%%%%%%%%%%%%%%%%%%%%%%%%%%%%%%%%%%%%%%%%%%%%%%%%%%%%%%%%%%%%%%%
% ---------------------------------------------------------------------%
% ----------------------FINE EDIZIONE CRITICA--------------------------%
% ---------------------------------------------------------------------%
%%%%%%%%%%%%%%%%%%%%%%%%%%%%%%%%%%%%%%%%%%%%%%%%%%%%%%%%%%%%%%%%%%%%%%%%

\begin{document}

\beginnumbering

\pstart
\ledsidenote{57}ΕΧ. Αὐτός, ὦ \edtext{}{\Bfootnote{\emph{B note}}}Φαίδων, παρεγένου Σωκράτει ἐκείνῃ τῇ ἡμέρᾳ ᾗ τὸ φάρμακον ἔπιεν ἐν τῷ δεσμωτηρίῳ, ἢ ἄλλου του ἤκουσας;

ΦΑΙΔ. Αὐτός, ὦ \edtext{}{\Afootnote{\emph{A note}}}Ἐχέκρατες.

ΕΧ. Τί οὖν δή ἐστιν ἅττα εἶπεν ὁ ἀνὴρ \edtext{πρὸ}{\Cfootnote{note with rbraket}} τοῦ θανάτου; καὶ πῶς ἐτελεύτα; ἡδέως γὰρ ἂν ἐγὼ ἀκούσαιμι. καὶ γὰρ οὔτε [\edtext{τῶν πολιτῶν}{\Cfootnote[nosep]{\textbf{note without rbracket}}}] Φλειασίων οὐδεὶς πάνυ τι ἐπιχωριάζει τὰ νῦν Ἀθήναζε, οὔτε τις ξένος ἀφῖκται χρόνου συχνοῦ ἐκεῖθεν ὅστις ἂν ἡμῖν σαφές τι ἀγγεῖλαι οἷός τ' ἦν περὶ τούτων, πλήν γε δὴ ὅτι \edtext{φάρμακον}{\Cfootnote{note with rbracket and line number}} \edtext{πιὼν}{\Cfootnote{note with rbracket, but automatically without line number}} ἀποθάνοι· τῶν δὲ ἄλλων οὐδὲν εἶχεν \edtext{φράζειν}{\Cfootnote[nonum]{\textbf{note manually without} line number}}.

\ledsidenote{58}ΦΑΙΔ. Οὐδὲ τὰ περὶ τῆς δίκης ἄρα ἐπύθεσθε ὃν τρόπον ἐγένετο;

ΕΧ. Ναί, ταῦτα μὲν ἡμῖν ἤγγειλέ τις, καὶ ἐθαυμάζομέν γε ὅτι πάλαι γενομένης αὐτῆς πολλῷ ὕστερον φαίνεται ἀποθανών. τί οὖν ἦν τοῦτο, ὦ Φαίδων;

ΦΑΙΔ. Τύχη τις αὐτῷ, ὦ Ἐχέκρατες, συνέβη· \edtext{ἔτυχεν}{\Cfootnote[nonum,nosep]{note manually without line number and rbracket}} γὰρ τῇ προτεραίᾳ τῆς δίκης ἡ πρύμνα ἐστεμμένη τοῦ πλοίου ὃ εἰς Δῆλον Ἀθηναῖοι πέμπουσιν.

ΕΧ. Τοῦτο δὲ δὴ τί ἐστιν; 
\pend

\setcounter{pstart}{1}
\numberpstarttrue
    \pstart\newpara \label{scolio1}\edtext{}{\Afootnote{\textbf{\ref{scolio1}} Anote Anote Anote Anote Anote Anote Anote}}scolio 1] testo \edtext{testo}{\Cfootnote{\textit{prova A}}} testo testo testo testo testo testo testo testo testo testo testo.\pend

    \pstart\newpara scolio 2] testo testo testo testo testo \textit{testo} testo testo testo testo testo testo testo.\pend
\numberpstartfalse

\endnumbering
\end{document}