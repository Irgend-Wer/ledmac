% !TEX encoding = UTF-8
% !TEX program = xelatex
% !TEX spellcheck = it_IT
%---------------------------------------------------------------------------------------
% PACKAGES
%---------------------------------------------------------------------------------------
\documentclass[11pt,a4paper]{book}
\usepackage[libertine={Ligatures=TeX,Numbers=OldStyle}]{libertineotf}
\usepackage{ledmac}
%\usepackage{ledpar} %Necessario per \emph in ledsidenote, ma confligge con nosep/nonum
\newseries{A}\newseries{B}\newseries{C}
\lineation{page}                %% numerazione per pagina
\linenummargin{inner}   %% Margine dei numeri di linea
\sidenotemargin{outer}  %% Margine dei marginalia

\renewcommand*{\notenumfont}{\footnotesize} %%Font numeri linea note
\newcommand*{\notetextfont}{\footnotesize}      %%Font testo apparato

%Formato marginalia
\renewcommand*{\ledlsnotefontsetup}{\raggedleft\it\footnotesize}    % left
\renewcommand*{\ledrsnotefontsetup}{\raggedright\it\footnotesize}   % right

%Togliamo numero di linea da A e B
\nonumberinfootnote[A]
\nonumberinfootnote[B]

%Numero solo sulla prima nota di una linea in C
\numberonlyfirstinline[C]

%Azzeriamo lo spazio residuo, una volta tolto il numero di linea
\inplaceofnumber[A]{0em}
\inplaceofnumber[B]{0em}

%Togliamo separatore lemma da A e B
\nolemmaseparator[A]
\nolemmaseparator[B]

%Azzeriamo lo spazio residuo, una volta tolto il separatore lemma
\inplaceoflemmaseparator[A]{0em}
\inplaceoflemmaseparator[B]{0em}
\inplaceoflemmaseparator[C]{.5em}

%%% SPAZIO LIBERO SOPRA LE RIGHE SEPARATRICI
\addtolength{\skip\Afootins}{2em plus.4em minus.4em}

% SPAZIO BIANCO FRA TESTO ED APPARATO (def. = 5mm)
\setlength{\skip\Afootins}{2em plus.4em minus.4em}

% SPAZIO BIANCO FRA NOTE D'APPARATO
\afternote[A]{1em plus.4em minus.4em}
\afternote[B]{1em plus.4em minus.4em}
\afternote[C]{1em plus.4em minus.4em}

\footparagraph{A}
\footparagraph{B}
\footparagraph{C}

%% DEFINIZIONE NEWPARA
\newcounter{para}[chapter]\setcounter{para}{0}
    \newcommand{\newpara}{%
    \refstepcounter{para}%
    \noindent\llap{\thepstart}}
    \newcommand{\oldpara}[1]{%
    \noindent\llap{\ref{#1}}}

\begin{document}

\beginnumbering

\pstart
\ledsidenote{57}??. ?????, ? \edtext{}{\lemma{\textbf{LemmaBinBold}}\Bfootnote{\enspace Bnote with textual lemma in bold}}??????, �???????? ???????? ?????? ?? ????? ? ?? ???????? ?�??? ?? ?? ??????????, ? ????? ??? ????????

????. ?????, ? \edtext{}{\lemma{\textbf{LemmaAinBold}}\Afootnote{\enspace Anote with textual lemma in bold}}?????????.

??. ?? ??? ?? ????? ???? ??�?? ? ???? \edtext{�??}{\Cfootnote{note with \textit{rbraket}}} ??? ???????? ??? �?? ????????? ????? ??? ?? ??? ?????????. ??? ??? ???? [\edtext{??? �??????}{\Cfootnote[nosep]{note with \textit{nosep}}}] ????????? ?????? �??? ?? ?�????????? ?? ??? ???????, ???? ??? ????? ??????? ?????? ?????? ??????? ????? ?? ???? ????? ?? ???????? ???? ?' ?? �??? ??????, �??? ?? ?? ??? \edtext{????????}{\Cfootnote{note with \textit{rbracket} and line number}} \edtext{�???}{\Cfootnote{note with \textit{rbracket}, but automatically without line number}} ?�??????? ??? ?? ????? ????? ????? \edtext{???????}{\Cfootnote[nonum]{note with nonum}}.

\ledsidenote{58}????. ???? ?? �??? ??? ????? ??? ?�?????? ?? ???�?? ????????

??. ???, ????? ??? ???? ??????? ???, ??? ??????????? ?? ??? �???? ????????? ????? �???? ??????? ???????? ?�??????. ?? ??? ?? ?????, ? ???????

????. ???? ??? ????, ? ?????????, ??????? \edtext{??????}{\lemma{\textit{blablabla}}\Cfootnote[nonum,nosep]{note with nonum+nosep}} ??? ?? �???????? ??? ????? ? �????? ????????? ??? �????? ? ??? ????? ???????? �??�?????.

??. ????? ?? ?? ?? ?????? 
\pend

\setcounter{pstart}{1}
\numberpstarttrue
    \pstart\newpara \label{scolio1}\edtext{}{\Afootnote{\textbf{\ref{scolio1}} \textit{Anote} with numeric lemma in bold}}\edtext{}{\Bfootnote{\textbf{\ref{scolio1}} \textit{Bnote} with numeric lemma in bold}}scolio 1] testo \edtext{testo}{\lemma{\textit{prova C}}\Cfootnote{\textit{prova A}}} testo testo \textbf{testo} testo testo testo testo testo testo testo testo.\pend

    \pstart\newpara \label{scolio2}scolio 2] \ledsidenote{? \emph{|} D?}\edtext{Nel mezzo \edtext{del}{\Cfootnote{dello C}} cammin di nostra vita}{\Cfootnote{a 35 anni}} testo testo testo testo testo testo testo \textit{testo} testo testo testo testo testo testo testo.
    \pend

    \pstart\newpara \label{scolio3}\edtext{}{\Afootnote{\textbf{\ref{scolio3}} sch.3 Anote}}scolio 3] \ledsidenote{? \emph{|} D? \emph{|} Bk\textsuperscript{5}}\edtext{}{\lemma{\textbf{\ref{scolio3}}}\Cfootnote[nonum,nosep]{\textit{Cnote} with numeric lemma in bold}}testo testo \edtext{????????}{\Cfootnote{????????? sch. \textit{Min.}}} testo testo testo testo testo testo testo testo testo testo testo.\pend

    \pstart\newpara \label{itm:Clit5}\edtext{}{\Afootnote{{\textbf{\ref{itm:Clit5}}}\enspace aliter n. \ref{scolio3}}}\edtext{}{\Bfootnote{{\textbf{\ref{itm:Clit5}}}\enspace long Bnote with numeric lemma in bold}}scolio 4]\ledsidenote{Diog.?}\edtext{}{\lemma{\textbf{{\ref{itm:Clit5}}}}\Cfootnote[nonum,nosep]{sch. ab eodem ut vid. scriba et atramento magis diluto posterius additum (f.198r); signum �?. prope textum et prope scholium: �??????? in init. Greene}}   testo testo testo testo testo \edtext{???????}{\lemma{??????? ????}\Cfootnote[nosep]{Diog.}} \edtext{? ???????}{\Cfootnote[nosep]{om. Diog. (sed cf. Diog.Vind. ??? ????????)}} \edtext{??????????}{\lemma{?????????????}\Cfootnote[nosep]{Diog.}}. testo testo testo testo testo \edtext{?? ??????? ????}{\Cfootnote{???? ?? ??????? Diog.}} testo testo testo testo testo testo testo testo testo testo. \textbf{T}\pend

\numberpstartfalse

\endnumbering
\end{document}